% author: Fabian Bosshard % © CC BY 4.0
\documentclass[9pt,headings=standardclasses,parskip=half]{scrartcl}

% ===================== basics =====================
\usepackage{ifthen,iftex,csquotes}
\usepackage[T1]{fontenc}
\usepackage[english]{babel}

% ===================== page style =====================
\usepackage[automark]{scrlayer-scrpage}
\pagestyle{scrheadings}

% ===================== graphics / color =====================
\usepackage{graphicx}
\usepackage[dvipsnames]{xcolor}
\usepackage{caption,subcaption}
\usepackage[left=38mm,right=38mm,top=20mm,bottom=30mm]{geometry}

% ===================== math =====================
\usepackage{amsmath,amssymb,amsthm,mathtools,mathdots,accents}
\numberwithin{equation}{section}

% ===================== lists =====================
\usepackage{enumitem}
\renewcommand{\labelitemi}{\textbullet}
\renewcommand{\labelitemii}{\raisebox{0.1ex}{\scalebox{0.8}{\textbullet}}}
\renewcommand{\labelitemiii}{\raisebox{0.2ex}{\scalebox{0.6}{\textbullet}}}
\renewcommand{\labelitemiv}{\raisebox{0.3ex}{\scalebox{0.4}{\textbullet}}}

% ===================== bibliography =====================
\usepackage[backend=biber,style=numeric]{biblatex}
\addbibresource{\jobname.bib}

% ===================== symbols =====================
\usepackage{pifont}

% ===================== TikZ / plots =====================
\usepackage{tikz}
\usetikzlibrary{arrows,arrows.meta,shapes,positioning,calc,fit,patterns,intersections,math,3d,tikzmark,decorations.pathreplacing,decorations.markings}
\usepackage{tikz-dependency,tikz-qtree,tikz-qtree-compat,tikz-3dplot,tikzpagenodes}
\usepackage{pgfplots}

% ===================== highlighting (SOUL + math-safe \hl) =====================
\usepackage{soulutf8}
\usepackage{xparse}

\ExplSyntaxOn
\tl_new:N \__l_SOUL_argument_tl
\cs_set_eq:Nc \SOUL_start:n { SOUL@start }
\cs_generate_variant:Nn \SOUL_start:n { V }
\cs_set_protected:cpn {SOUL@start} #1 {
  \tl_set:Nn \__l_SOUL_argument_tl { #1 }
  \regex_replace_all:nnN { \c{\(} (.*?) \c{\)} } { \cM\$ \1 \cM\$ } \__l_SOUL_argument_tl
  \SOUL_start:V \__l_SOUL_argument_tl
}
\ExplSyntaxOff

\let\SOULhl\hl
\renewcommand{\hl}[1]{\SOULhl{#1}}

\definecolor{HLgreen}{HTML}{77DD77}
\definecolor{HLyellow}{HTML}{FFFF66}
\definecolor{HLorange}{HTML}{FFB347}
\definecolor{HLred}{HTML}{FF6961}
\definecolor{HLpink}{HTML}{FFB6C1}
\definecolor{HLturquoise}{HTML}{40E0D0}

\RenewDocumentCommand{\hl}{O{1} m}{%
  \begingroup
  \IfEqCase{#1}{%
    {1}{\sethlcolor{HLgreen}\SOULhl{#2}}%
    {2}{\sethlcolor{HLyellow}\SOULhl{#2}}%
    {3}{\sethlcolor{HLorange}\SOULhl{#2}}%
    {4}{\sethlcolor{HLred}\SOULhl{#2}}%
    {5}{\sethlcolor{red}\SOULhl{#2}}%
    {6}{\sethlcolor{HLturquoise}\SOULhl{#2}}%
  }[\PackageError{hl}{Undefined highlight level: #1}{}]%
  \endgroup
}

% ===================== colors / emphasis =====================
\definecolor{funblue}{rgb}{0.10,0.35,0.66}
\definecolor{alizarincrimsonred}{rgb}{0.85,0.17,0.11}
\definecolor{amethyst}{rgb}{0.6,0.4,0.8}
\definecolor{mypurple}{rgb}{0.5,0,0.5}
\definecolor{highlightpurple}{rgb}{0.6,0,0.6}
\renewcommand{\emph}[1]{\textcolor{highlightpurple}{\textsl{#1}}}

% ===================== theorem environments =====================
\usepackage{thmtools}

\newlength{\thmspace}\setlength{\thmspace}{3pt plus 1pt minus 1pt}

\declaretheoremstyle[headfont=\bfseries,bodyfont=\normalfont,spaceabove=\thmspace,spacebelow=\thmspace,qed=\ensuremath{\vartriangleleft},postheadspace=1em]{assertionstyle}
\declaretheorem[style=assertionstyle,name=Theorem,numberwithin=section]{theorem}
\declaretheorem[style=assertionstyle,name=Lemma,sibling=theorem]{lemma}
\declaretheorem[style=assertionstyle,name=Corollary,sibling=theorem]{corollary}
\declaretheorem[style=assertionstyle,name=Proposition,sibling=theorem]{proposition}
\declaretheorem[style=assertionstyle,name=Conjecture,sibling=theorem]{conjecture}
\declaretheorem[style=assertionstyle,name=Claim,sibling=theorem]{claim}
\declaretheorem[style=assertionstyle,name=Fact,sibling=theorem]{fact}

\declaretheoremstyle[headfont=\bfseries,bodyfont=\normalfont,spaceabove=\thmspace,spacebelow=\thmspace,qed=\ensuremath{\blacktriangleleft},postheadspace=1em]{definitionstyle}
\declaretheorem[style=definitionstyle,name=Definition,numberwithin=section]{definition}
\declaretheorem[style=definitionstyle,name=Problem,sibling=definition]{problem}

\declaretheoremstyle[headfont=\bfseries,bodyfont=\normalfont,spaceabove=\thmspace,spacebelow=\thmspace,qed=\ding{45},postheadspace=1em]{exercisestyle}
\declaretheorem[style=exercisestyle,name=Exercise,numberwithin=section]{exercise}
\declaretheoremstyle[headfont=\bfseries\color{red},bodyfont=\normalfont,spaceabove=\thmspace,spacebelow=\thmspace,qed=\ensuremath{\color{red}\blacktriangleleft},postheadspace=1em]{solutionstyle}
\declaretheorem[style=solutionstyle,name=Solution,numbered=no]{solution}

\declaretheoremstyle[headfont=\bfseries,bodyfont=\normalfont,spaceabove=6pt,spacebelow=6pt,qed=\ensuremath{\square},postheadspace=1em]{proofstyle}
\let\proof\relax \let\endproof\relax
\declaretheorem[style=proofstyle,name=Proof,numbered=no]{proof}

\declaretheoremstyle[headfont=\bfseries,bodyfont=\normalfont\normalsize,spaceabove=\thmspace,spacebelow=\thmspace,qed=\ensuremath{\blacktriangleleft},postheadspace=1em]{remarkstyle}
\declaretheorem[style=remarkstyle,name=Remark,numberwithin=section]{remark}

\declaretheoremstyle[headfont=\bfseries\color{funblue},bodyfont=\normalfont\normalsize,spaceabove=\thmspace,spacebelow=\thmspace,qed=\ensuremath{\color{funblue}\blacktriangleleft},postheadspace=1em]{examplestyle}
\declaretheorem[style=examplestyle,name=Example,sibling=remark]{example}

\declaretheoremstyle[headfont=\color{alizarincrimsonred}\bfseries,bodyfont=\normalfont\normalsize,spaceabove=\thmspace,spacebelow=\thmspace,qed=\ensuremath{\color{alizarincrimsonred}\blacktriangleleft},postheadspace=1em]{cautionstyle}
\declaretheorem[style=cautionstyle,name=Caution,sibling=remark]{caution}

\declaretheoremstyle[headfont=\bfseries,bodyfont=\normalfont\footnotesize,spaceabove=\thmspace,spacebelow=\thmspace,postheadspace=1em]{smallremarkstyle}
\declaretheorem[style=smallremarkstyle,name=Remark,sibling=remark]{smallremark}

\declaretheoremstyle[headfont=\bfseries\color{amethyst},bodyfont=\normalfont,spaceabove=6pt,spacebelow=6pt,qed=\ensuremath{\color{amethyst}\blacktriangleleft},postheadspace=1em]{digressionstyle}
\declaretheorem[style=digressionstyle,name=Digression,sibling=remark]{digression}

% ===================== misc helpers =====================
\newenvironment{verticalhack}{\begin{array}[b]{@{}c@{}}\displaystyle}{\\\noalign{\hrule height0pt}\end{array}}

% ===================== algorithms =====================
\usepackage{algorithm,algorithmicx}
\usepackage[italicComments=false]{algpseudocodex}
\newcommand*{\algorithmautorefname}{Algorithm}

\algnewcommand{\TO}{, \ldots ,}
\algnewcommand{\DOWNTO}{, \ldots ,}
\algnewcommand{\OR}{\vee}
\algnewcommand{\AND}{\wedge}
\algnewcommand{\NOT}{\neg}
\algnewcommand{\LEN}{\operatorname{len}}
\algnewcommand{\tru}{\ensuremath{\mathrm{\texttt{true}}}}
\algnewcommand{\fals}{\ensuremath{\mathrm{\texttt{false}}}}
\algnewcommand{\append}{\circ}
\algnewcommand{\nil}{\ensuremath{\mathrm{\textsc{nil}}}}
\algnewcommand{\red}{\ensuremath{\mathrm{\textsc{red}}}}
\algnewcommand{\black}{\ensuremath{\mathrm{\textsc{black}}}}
\algnewcommand{\gray}{\ensuremath{\mathrm{\textsc{gray}}}}
\algnewcommand{\white}{\ensuremath{\mathrm{\textsc{white}}}}

% ===================== operators / notation =====================
\DeclareMathOperator*{\argmax}{arg\,max}
\DeclareMathOperator*{\argmin}{arg\,min}
\DeclareMathOperator{\Span}{span}
\DeclareMathOperator{\Kern}{kern}
\DeclareMathOperator{\Trace}{trace}
\DeclareMathOperator{\Rank}{rank}
\newcommand{\im}{\operatorname{Im}}
\newcommand{\re}{\operatorname{Re}}

\newcommand*\matrspace{0.8mu}
\newcommand{\matr}[1]{\mspace{\matrspace}\underline{\mspace{-\matrspace}\smash[b]{\boldsymbol{#1}}\mspace{-\matrspace}}\mspace{\matrspace}}
\newcommand{\vect}[1]{\vec{\boldsymbol{#1}}}

\newcommand{\dif}{\mathrm{d}}
\newcommand{\eu}{\mathrm{e}}
\newcommand{\iu}{\mathrm{i}}
\newcommand{\sign}{\operatorname{sgn}}

\newcommand{\R}{\mathbb{R}}
\newcommand{\N}{\mathbb{N}}
\newcommand{\Z}{\mathbb{Z}}
\newcommand{\Q}{\mathbb{Q}}
\newcommand{\C}{\mathbb{C}}
\newcommand{\K}{\mathbb{K}}
\newcommand{\F}{\mathbb{F}}

\newcommand{\Var}{\operatorname{Var}}
\newcommand{\Cov}{\operatorname{Cov}}
\newcommand{\Exp}{\operatorname{E}}
\newcommand{\Prob}{\operatorname{P}}
\newcommand{\numof}{\ensuremath{\#\,}}
\newcommand{\blackheight}{\operatorname{bh}}

\newcommand{\attrib}[2]{\ensuremath{#1\mathtt{.}\mathtt{#2}}}
\newcommand{\attribnormal}[2]{\ensuremath{#1\mathtt{.}#2}}

% ===================== metadata / hyperlinks =====================
\title{Graph Deep Learning}
\author{Fabian Bosshard}
\date{\today}

\usepackage[
  linktoc=none,
  pdfauthor={Fabian Bosshard},
  pdftitle={USI - Graph Deep Learning - Course Notes},
  pdfkeywords={USI, Graph Deep Learning, course notes, informatics},
  colorlinks=false,
  pdfborder={0 0 0},
  linkbordercolor={0 0.6 1},
  urlbordercolor={0 0.6 1},
  citebordercolor={0 0.6 1}
]{hyperref}

% ToC entries: dotted leaders + full-width clickable line
\DeclareTOCStyleEntry[linefill=\dotfill]{tocline}{section}
\DeclareTOCStyleEntry[linefill=\dotfill]{tocline}{subsection}
\DeclareTOCStyleEntry[linefill=\dotfill]{tocline}{subsubsection}

\makeatletter
\newlength\FB@toclinkht \newlength\FB@toclinkdp
\setlength\FB@toclinkht{.80\ht\strutbox}
\setlength\FB@toclinkdp{.40\dp\strutbox}

\let\FB@orig@contentsline\contentsline
\renewcommand*\contentsline[4]{%
  \begingroup
    \Hy@safe@activestrue
    \edef\Hy@tocdestname{#4}%
    \FB@orig@contentsline{#1}{%
      \Hy@raisedlink{\hyper@anchorstart{toc:\Hy@tocdestname}\hyper@anchorend}%
      \leavevmode
      \rlap{%
        \hyper@linkstart{link}{\Hy@tocdestname}%
          \raisebox{0pt}[\FB@toclinkht][\FB@toclinkdp]{%
            \hbox to \dimexpr\hsize-\parindent\relax{\hfil}%
          }%
        \hyper@linkend
      }%
      #2%
    }{#3}{#4}%
  \endgroup
}

\let\FB@orig@sectionlinesformat\sectionlinesformat
\renewcommand{\sectionlinesformat}[4]{%
  \ifx\@currentHref\@empty
    \FB@orig@sectionlinesformat{#1}{#2}{#3}{#4}%
  \else
    \leavevmode
    \hyper@linkstart{link}{toc:\@currentHref}%
      \@hangfrom{\hskip #2\relax #3}{#4}%
    \hyper@linkend
    \par
  \fi
}
\makeatother

% hyperref anchor uniqueness for section-relative numbering
\makeatletter
\renewcommand{\theHequation}{\thesection.\arabic{equation}}
\renewcommand{\theHtheorem}{\thesection.\arabic{theorem}}
\renewcommand{\theHlemma}{\theHtheorem}
\renewcommand{\theHcorollary}{\theHtheorem}
\renewcommand{\theHproposition}{\theHtheorem}
\renewcommand{\theHconjecture}{\theHtheorem}
\renewcommand{\theHclaim}{\theHtheorem}
\renewcommand{\theHfact}{\theHtheorem}
\renewcommand{\theHdefinition}{\thesection.\arabic{definition}}
\renewcommand{\theHproblem}{\theHdefinition}
\renewcommand{\theHexercise}{\thesection.\arabic{exercise}}
\renewcommand{\theHremark}{\thesection.\arabic{remark}}
\renewcommand{\theHexample}{\theHremark}
\renewcommand{\theHcaution}{\theHremark}
\renewcommand{\theHsmallremark}{\theHremark}
\renewcommand{\theHdigression}{\theHremark}
\makeatother

% ===================== license / cleveref =====================
\usepackage[type={CC},modifier={by},version={4.0}]{doclicense}
\usepackage{cleveref}

% ===================== acronyms =====================
\usepackage[acronym,nomain,toc,nonumberlist]{glossaries-extra}
\setabbreviationstyle[acronym]{long-short}
\makeglossaries
\newacronym{rbf}{RBF}{radial basis function}
\newacronym{rkhs}{RKHS}{reproducing kernel Hilbert space}

\begin{document}

\maketitle

\tableofcontents


% table of acronyms (remember everything is in build folder)
% \printglossary[type=\acronymtype, title={List of Acronyms}]



\section{Background}\label{sec:background}

Let $G = (V, E)$ be an undirected graph with the
adjacency matrix $\matr{A} \in \R^{n\times n}$
\begin{equation}\label{eq:adjacency_matrix}
    \matr{A}_{uv}
    =
    \begin{cases}
        1 & (u,v) \in E\\
        0 & (u,v) \notin E
    \end{cases}
\end{equation}

The \emph{diagonal degree matrix} $\matr{D}\in\R^{n\times n}$ is defined by
\begin{equation}\label{eq:degree_matrix}
    \matr{D}_{uv}
    =
    \begin{cases}
        d_u &  u = v\\
        0 & u \ne v
    \end{cases}
\end{equation}
i.e. $\matr{D}$ simply places all node degrees on the diagonal.




\subsection{normalized adjacency and multi-hop propagation}


\begin{definition}\label{def:normalized_adjacency}
The \emph{symmetrically normalized adjacency matrix} is
\begin{equation}\label{eq:normalized_adjacency}
    {\color{Red}\boxed{\color{black}
    \hat{\matr{A}}
    =
    \matr{D}^{-1/2}\matr{A}\matr{D}^{-1/2}
    }}
\end{equation}
or, entrywise,
\[
\begin{verticalhack}
    \hat{\matr{A}}_{uv}
    =
    \begin{cases}
        \dfrac{1}{\sqrt{d_u d_v}} & (u,v) \in E\\[0.5ex]
        0 & (u,v) \notin E
    \end{cases}
    \end{verticalhack}
    \qedhere
\]
\end{definition}

\begin{fact}[multi-hop propagation]\label{prop:multi_hop_propagation}
The entry $(\hat{\matr{A}}^{k})_{vu}$ can be computed explicitly as follows:
\begin{equation}\label{eq:multi_hop_propagation}
    (\hat{\matr{A}}^{k})_{vu}
    =
    \sum_{\pi}
    \prod_{(x, y) \in E_\pi} \frac{1}{\sqrt{d_x d_y}}
\end{equation}
the sum is over all walks $\pi = (v, \ldots, u)$ of length $k$ from $v$ to $u$ and the product is over the edges \(E_\pi = \{(v, u_1), \ldots, (u_{k-1}, u)\}\) on the walk.
\end{fact}



\begin{corollary}\label{cor:multi_hop_propagation_single_path}
Let $v,u \in V$ with $r = d_{G}(v,u)$, where $d_{G}(\cdot,\cdot)$ denotes the shortest-path distance.
Assume there is exactly one path
\[
    (v, u_{1}, \ldots, u_{r-1}, u)
\]
of length $r$ between $v$ and $u$:
\[
\begin{tikzpicture}[
  font=\footnotesize,
  scale=0.4,
  vertex/.style={
    circle,
    draw,
    fill=green!40,
    minimum size=10pt,
    inner sep=0,
    label={
        [anchor=base, label distance=5pt]
        below:{#1} % parentheses necessary, otherwise the '=' breaks TikZ parsing
    }
  },
  measure/.style={|-|}
]

\node[vertex={$v$}] (v) at (0, 0) {};
\node[vertex={$u_{1}$}] (u1) at (3, 0) {};
\node[vertex={$u_{r-1}$}] (urm1) at (9, 0) {};
\node[vertex={$u$}] (u) at (12, 0) {};


\draw[] (v) -- (u1);
\draw[] (u1) -- ($(u1)!1/3!(urm1)$);
\draw[dotted] ($(u1)!1/3!(urm1)$) -- ($(u1)!2/3!(urm1)$);
\draw[] ($(u1)!2/3!(urm1)$) -- (urm1);
\draw[] (urm1) -- (u);


\def\offset{-0.5cm};
\draw[measure] ($(v) + (0, -1) + (0, \offset)$) -- node[below, outer sep = 2pt, inner sep = 0] {$d_{G}(v,u) = r$} ($(u) + (0, -1) + (0, \offset)$);


\end{tikzpicture}
\]
Then
\begin{equation}\label{eq:multi_hop_propagation_single_path}
    % \begin{aligned}
    % (\hat{\matr{A}}^{r})_{vu}
    % &=
    % \frac{1}{\sqrt{d_v d_{u_{1}}}} \cdot
    % \prod_{i=1}^{r-2} \frac{1}{\sqrt{d_{u_{i}} d_{u_{i+1}}}} \cdot
    % \frac{1}{\sqrt{d_{u_{r-1}} d_{u}}} \\
    % &=
    % \frac{1}{\sqrt{d_v d_u}} \prod_{i=1}^{r-1} \frac{1}{d_{u_i}}
    % \end{aligned}
    (\hat{\matr{A}}^{r})_{vu}
    =
    \frac{1}{\sqrt{d_v d_{u_{1}}}} \cdot
    \prod_{i=1}^{r-2} \frac{1}{\sqrt{d_{u_{i}} d_{u_{i+1}}}} \cdot
    \frac{1}{\sqrt{d_{u_{r-1}} d_{u}}} 
    =
    \frac{1}{\sqrt{d_v d_u}} \prod_{i=1}^{r-1} \frac{1}{d_{u_i}}
\end{equation}
\end{corollary}



\subsubsection{distance layers and layer degrees}

% For \gls{laser} and related methods it is useful to separate connections by graph distance.

\begin{definition}\label{def:distance_layer_adjacency}
For $\ell \in \N_0$, we define the \emph{distance-($\ell+1$) adjacency matrix} $\matr{A}_{\ell} \in \R^{n\times n}$ by
\begin{equation}\label{eq:distance_layer_adjacency}
    \bigl(\matr{A}_{\ell}\bigr)_{uv}
    =
    \begin{cases}
        1 & d_{G}(u,v) = \ell+1\\
        0 & \text{otherwise}
    \end{cases}
\end{equation}
where $d_{G}(u,v)$ is the shortest-path distance.
The corresponding \emph{layer degree} of a node $v$ at distance level $\ell$ is
\begin{equation}\label{eq:layer_degree}
    d_{v,\ell}
    =
    \sum_{u \in V} \bigl(\matr{A}_{\ell}\bigr)_{vu},
\end{equation}
i.e.\ the number of nodes at graph distance $\ell+1$ from $v$.
Let $\matr{D}_{\ell}$ be the diagonal matrix with $(\matr{D}_{\ell})_{vv} = d_{v,\ell}$.
The \emph{normalized distance-$(\ell+1)$ adjacency} is
\begin{equation}\label{eq:normalized_distance_layer_adjacency}
    {\color{Red}\boxed{\color{black}
    \hat{\matr{A}}_{\ell}
    =
    \matr{D}_{\ell}^{-1/2} \matr{A}_{\ell} \matr{D}_{\ell}^{-1/2}
    }}
\end{equation}
so that
\[
\begin{verticalhack}
\bigl(\hat{\matr{A}}_{\ell}\bigr)_{uv}
=
\begin{cases}
    \frac{1}{\sqrt{d_{u,\ell} d_{v,\ell}}} & d_{G}(u,v) = \ell+1 \\
    0 & \text{otherwise}
\end{cases}
\end{verticalhack}
\qedhere
\]
\end{definition}

% Finally, we denote by
% \begin{equation}\label{eq:minimum_degree}
%     d_{\min}
%     =
%     \min_{v \in V} d_v
% \end{equation}
% the \emph{minimum node degree} in the graph.







\subsection{graph Laplacian}

\begin{definition}\label{def:graph_laplacian}
The \emph{combinatorial} graph Laplacian is
% \begin{equation}\label{eq:graph_laplacian}
%     \matr{L} = \matr{D} - \matr{A}
% \end{equation}
\(\matr{L} = \matr{D} - \matr{A}\)
and the \emph{normalized} graph Laplacian is
\begin{equation}\label{eq:normalized_graph_laplacian}
    {\color{Red}\boxed{\color{black}
    \hat{\matr{L}}
    =
    \matr{D}^{-1/2}\matr{L}\matr{D}^{-1/2}
    % \overset{\text{\eqref{eq:graph_laplacian}}}{=}
    =
    \matr{D}^{-1/2}(\matr{D} - \matr{A})\matr{D}^{-1/2}
    \overset{\text{\eqref{eq:normalized_adjacency}}}{=}
    \matr{I}_{n} - \hat{\matr{A}}
    }}
\end{equation}
It is symmetric and positive semidefinite, and its eigenvalues satisfy
\[
    0 = \lambda_{0} \le \lambda_{1} \le \dots \le \lambda_{n-1}
\]
$\lambda_1$ is called the \emph{spectral gap}.
The number of zero eigenvalues (i.e., the multiplicity of the \(0\) eigenvalue) equals the number of connected components of the graph.
\end{definition}

To understand \autoref{def:graph_laplacian}, consider a function $f\colon V \to \R$.
Denote by $\vect{f} \in \R^{n}$ the vector whose $v$-th entry is $f(v)$.
Then
% \begin{equation}\label{eq:laplacian_action_on_function}
% (\matr{L}\vect{f})_v
% =
% d_v f(v) - \sum_{(u,v) \in E} f(u)
% \end{equation}
% i.e., $(\matr{L}\vect{f})_v$ is (up to the factor $d_v$) the difference between the value at $v$ and the sum of the values at its neighbors.
% For the normalized Laplacian, we have
\begin{equation}\label{eq:normalized_laplacian_action_on_function}
(\hat{\matr{L}}\vect{f})_v
=
f(v)
-
\frac{1}{\sqrt{d_v}}
\sum_{(u,v) \in E}
\frac{f(u)}{\sqrt{d_u}}
\end{equation}
i.e., $(\hat{\matr{L}}\vect{f})_v$ is the value at $v$ minus a degree-normalized average of the neighbors.
This is why the Laplacian is often viewed as a \emph{discrete second derivative} on the graph:
\hl[2]{it measures how much $f$ at $v$ deviates from its neighborhood}.
Another important identity is the quadratic form
\begin{equation}\label{eq:laplacian_quadratic_form}
    \vect{f}^{\top}\matr{L}\vect{f}
    =
    \frac{1}{2}
    \sum_{(u,v) \in E}
    \bigl(f(u) - f(v)\bigr)^{2}
\end{equation}
which shows that \(\matr{L}\) (and hence also \(\hat{\matr{L}}\)) is positive semidefinite, since the right-hand side is always nonnegative.
Moreover, \eqref{eq:laplacian_quadratic_form} is small exactly when $f$ varies slowly across edges, so the Laplacian encodes the \emph{smoothness} of functions on the graph.




\subsection{Cheeger inequality}

The \emph{Cheeger inequality} relates the spectral gap $\lambda_1$ to the \emph{Cheeger constant} $h(G)$, which measures how difficult it is to separate the graph into two large pieces.  It states, in particular, that
\[
    \tfrac{1}{2}h(G)^2
    \le
    \lambda_1
    \le
    2 h(G),
\]
so a larger spectral gap implies that the graph is more ``well-connected''.


\subsection{effective resistance}
\begin{definition}[effective resistance]\label{def:effective_resistance}
View each edge $(u,v)\in E$ as an electrical resistor of resistance $1\,$\(\Omega\).
The resulting network has a well-defined resistance between any two nodes.

For two nodes $s,t \in V$, the \emph{effective resistance} $R(s,t)$ is defined as the voltage difference needed to send one unit of electrical current from $s$ to $t$.
It can be computed as
\begin{equation}\label{eq:effective_resistance}
{\color{Red}\boxed{\color{black}
    R(s,t)
    =
    (\vect{e}_s - \vect{e}_t)^{\top}
    \matr{L}^{\dagger}
    (\vect{e}_s - \vect{e}_t)
}}
\end{equation}
where $\matr{L}^{\dagger}$ is the Moore--Penrose pseudoinverse of $\matr{L}$ and $\vect{e}_v$ is the standard basis vector of vertex $v$.
\end{definition}


\subsubsection{Interpretation}
If the graph offers many short, parallel paths between $s$ and $t$, then current can flow easily, so $R(s,t)$ is small.
If there are few or long paths, the current is ``bottlenecked'' and $R(s,t)$ is large.
Thus, effective resistance measures how ``well-connected'' two nodes are inside the global geometry of the graph.

\subsubsection{Connection to random walks}
A \emph{random walk} on $G$ is the Markov chain that, from a node $v$, moves to a uniformly random neighbor of $v$.  Its transition matrix is
\begin{equation}\label{eq:random_walk_transition_matrix}
\matr{P} = \matr{D}^{-1}\matr{A}
\end{equation}
so $\matr{P}_{vu} = 1/d_v$ if $(v,u)\in E$.

\medskip
For two nodes $u,v$, the \emph{commute time} $\texttt{CT}(u,v)$ is the expected number of steps for the random walk to start at $u$, reach $v$, and return to $u$ again.  
It can be related to the \nameref{def:effective_resistance} via
\begin{equation}\label{eq:commute_time_effective_resistance}
    \texttt{CT}(u,v)
    =
    2|E|
    R(u,v)
    % \text{,}
\end{equation}
giving a geometric interpretation of how ``far apart'' two nodes are in terms of random-walk behavior,
i.e. two nodes have small commute time exactly when they have small effective resistance.










\end{document}
