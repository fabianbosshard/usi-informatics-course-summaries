% author: Fabian Bosshard
% © CC BY 4.0

% \documentclass[twocolumn, a3paper, fontsize=9pt, headings=standardclasses, parskip=never]{scrartcl}
\documentclass[a4paper, fontsize=9pt, headings=standardclasses, parskip=half, titlepage]{scrartcl}
% \documentclass[a4paper, fontsize=9pt, headings=standardclasses]{scrartcl}

\usepackage{csquotes}



\usepackage[automark]{scrlayer-scrpage}
\clearpairofpagestyles
\ofoot{\pagemark} % für ein einseitiges Dokument: \ofoot platziert den Inhalt in der äußeren Fußzeile (unten rechts)
% \pagestyle{scrheadings}

% \renewcommand{\familydefault}{\sfdefault} % sans serif font for text (math font is still serif)

\usepackage{graphicx}
\usepackage[dvipsnames, table]{xcolor}
\usepackage{colortbl}
\usepackage{soulutf8}
\usepackage{xparse}





\ExplSyntaxOn
\tl_new:N \__l_SOUL_argument_tl
\cs_set_eq:Nc \SOUL_start:n { SOUL@start }
\cs_generate_variant:Nn \SOUL_start:n { V }
\cs_set_protected:cpn {SOUL@start} #1
 {
  \tl_set:Nn \__l_SOUL_argument_tl { #1 }
  \regex_replace_all:nnN
   { \c{\(} (.*?) \c{\)} } % look for \(...\) (lazily)
   { \cM\$ \1 \cM\$ }      % replace with $...$
   \__l_SOUL_argument_tl
  \SOUL_start:V \__l_SOUL_argument_tl % do the usual
 }
\ExplSyntaxOff

% save soul's \hl under a private name
\let\SOULhl\hl
\renewcommand{\hl}[1]{\SOULhl{#1}} % keep plain \hl working if you want

% punchy highlighter colors
\definecolor{HLgreen}{HTML}{77DD77}
\definecolor{HLyellow}{HTML}{FFFF66}
\definecolor{HLorange}{HTML}{FFB347}
\definecolor{HLred}{HTML}{FF6961}
\definecolor{HLpink}{HTML}{FFB6C1}
\definecolor{HLturquoise}{HTML}{40E0D0}

% master highlight macro: \hl[level]{text}, default = green
\RenewDocumentCommand{\hl}{O{1} m}{%
  \begingroup
  \IfEqCase{#1}{%
    {1}{\sethlcolor{HLgreen}\SOULhl{#2}}%
    {2}{\sethlcolor{HLyellow}\SOULhl{#2}}%
    {3}{\sethlcolor{HLorange}\SOULhl{#2}}%
    {4}{\sethlcolor{HLred}\SOULhl{#2}}%
    {5}{\sethlcolor{red}\SOULhl{#2}}%
    {6}{\sethlcolor{HLturquoise}\SOULhl{#2}}%
  }[\PackageError{hl}{Undefined highlight level: #1}{}]%
  \endgroup
}

% --- Soul + hyperlinks, references, citations etc. ---

% --- hyperref core ---
\soulregister\href7
\soulregister\url7
\soulregister\hyperref7
\soulregister\nameref7
\soulregister\Nameref7
\soulregister\autoref7

% --- cleveref / varioref ---
\soulregister\cref7
\soulregister\Cref7
\soulregister\cpageref7
\soulregister\Cpageref7
\soulregister\vref7
\soulregister\Vref7

% --- standard LaTeX referencing commands ---
\soulregister\ref7
\soulregister\pageref7
\soulregister\eqref7
\soulregister\label7   % rarely needed but safe

% --- biblatex / natbib ---
\soulregister\cite7
\soulregister\parencite7
\soulregister\footcite7
\soulregister\textcite7
\soulregister\autocite7
\soulregister\Cite7
\soulregister\Citeauthor7
\soulregister\Citeyear7
\soulregister\citeauthor7
\soulregister\citeyear7

% --- glossaries, acronyms, etc. ---
\soulregister\gls7
\soulregister\Gls7
\soulregister\glspl7
\soulregister\Glspl7
\soulregister\acrshort7
\soulregister\acrlong7
\soulregister\acrfull7
\soulregister\Acrshort7
\soulregister\Acrlong7
\soulregister\Acrfull7

% --- misc. text-formatting macros ---
\soulregister\textbf7
\soulregister\textit7
\soulregister\emph7
\soulregister\underline7
\soulregister\textcolor7
\soulregister\color7
\soulregister\foreignlanguage7



\usepackage[left=40mm, right=40mm, top=20mm, bottom=30mm]{geometry} % for A4
% \usepackage[left=25mm, right=25mm, top=25mm, bottom=35mm]{geometry} % for A3
% \usepackage{showframe}
\usepackage{changepage}


% float management ––––––––––––––––––––––––––––––––––––––––––––––––
\usepackage{float}
\usepackage{placeins} 


% \makeatletter
%   % single-column floats
%   \def\fps@figure {htb} 
%   \def\fps@table  {htb}

%   % double-column floats
%   \def\fps@figure*{htb} 
%   \def\fps@table* {htb}
% \makeatother

\preto\section{\FloatBarrier}
\preto\subsection{\FloatBarrier}
\preto\subsubsection{\FloatBarrier} 

\setcounter{topnumber}{8}
\setcounter{bottomnumber}{8}
\setcounter{totalnumber}{20}
\renewcommand{\textfraction}{0.0}
\renewcommand{\topfraction}{1.0}
\renewcommand{\bottomfraction}{1.0}
\renewcommand{\floatpagefraction}{1.0}
\renewcommand{\dblfloatpagefraction}{1.0}

% On a float‐only page, kill the default “centered” glue
\makeatletter
  \setlength{\@fptop}{0pt} % no extra space above
  \setlength{\@fpsep}{\floatsep} % between floats: same as \floatsep
  \setlength{\@fpbot}{0pt plus 1fil} % infinite stretch below, so floats are pushed up
\makeatother
% ––––––––––––––––––––––––––––––––––––––––––––––––––––––––––––––––


% \newlength\tindent
% \setlength{\tindent}{\parindent}
% \setlength{\parindent}{0pt}
% \renewcommand{\indent}{\hspace*{\tindent}}

\newlength{\savedparindent}
\setlength{\savedparindent}{\parindent} % remember the class's default
\setlength{\parindent}{0pt}             % turn off automatic indenting
\renewcommand{\indent}{\hspace*{\savedparindent}} % manual indent on demand

\usepackage{amsmath}
\usepackage{amssymb}
\usepackage{amsfonts}
\usepackage{amsthm}
\usepackage{mathtools}
\usepackage{mathdots} % without this package, only \vdots and \ddots are taken from the text font (not the math font), which looks bad if the text font is different from the math font
\usepackage{cancel} % for \cancel, \bcancel, \xcancel, \cancelto


\usepackage{enumitem}
\renewcommand{\labelitemi}{\textbullet}
\renewcommand{\labelitemii}{\raisebox{0.1ex}{\scalebox{0.8}{\textbullet}}}
\renewcommand{\labelitemiii}{\raisebox{0.2ex}{\scalebox{0.6}{\textbullet}}}
\renewcommand{\labelitemiv}{\raisebox{0.3ex}{\scalebox{0.4}{\textbullet}}}
% \setlist{nosep} % = no itemsep, no parsep, minimal topsep



\usepackage{pifont}
\usepackage{marvosym}

\usepackage{booktabs}
% \usepackage{arydshln}
\usepackage{tabularx}
\usepackage{array}
\usepackage{hhline} % for double lines in tables
\usepackage{ragged2e}
\usepackage{makecell}

\usepackage{tikz}
\usetikzlibrary{arrows, arrows.meta, shapes, positioning, calc, fit, patterns, intersections, math, 3d, tikzmark, decorations.pathreplacing, backgrounds, chains, svg.path, hobby}
\usepackage{forest}
\usepackage{tikz-cd}
\usepackage{tikz-3dplot}
\usepackage{pgfplots}
\pgfplotsset{compat=1.18}


% define colors
\definecolor{funblue}{rgb}{0.10, 0.35, 0.66}
\definecolor{alizarincrimsonred}{rgb}{0.85, 0.17, 0.11}
\definecolor{amethyst}{rgb}{0.6, 0.4, 0.8}
\definecolor{mypurple}{rgb}{128, 0, 128} 

\definecolor{AccentBlue}{HTML}{013399}
\definecolor{AccentRed}{HTML}{CC0100}  
\definecolor{AccentPurple}{HTML}{660066}
\definecolor{AccentGray}{HTML}{666666}


\definecolor{highlightpurple}{rgb}{0.5, 0.0, 0.5}
% \renewcommand{\emph}[1]{\textcolor{highlightpurple}{#1}}
% \renewcommand{\emph}[1]{\textsl{#1}}
\renewcommand{\emph}[1]{\textcolor{highlightpurple}{\textsl{#1}}}

% \renewcommand{\emph}[1]{\textbf{#1}}


\usepackage{thmtools}

% 717 \providecommand\thmcontinues[1]{%
% 718 \ifcsname hyperref\endcsname
% 719 \hyperref[#1]{continuing}
% 720 \else
% 721 continuing
% 722 \fi
% 723 from p.\,\pageref{#1}%
% 724 }
\renewcommand\thmcontinues[1]{%
  \ifcsname hyperref\endcsname
    \hyperref[#1]{continuing}%
  \else
    continuing%
  \fi
}

\declaretheoremstyle[headfont=\bfseries,bodyfont=\normalfont,spaceabove=6pt,spacebelow=6pt,qed=\ensuremath{\vartriangleleft},postheadspace=1em]{assertionstyle}
\declaretheorem[style=assertionstyle,name=Theorem, numberwithin=section]{theorem}
\declaretheorem[style=assertionstyle,name=Lemma,sibling=theorem]{lemma}
\declaretheorem[style=assertionstyle,name=Lemma, numbered=no]{lemma*} % unnumbered lemma
\declaretheorem[style=assertionstyle,name=Corollary,sibling=theorem]{corollary}
\declaretheorem[style=assertionstyle,name=Proposition,sibling=theorem]{proposition}
\declaretheorem[style=assertionstyle,name=Conjecture,sibling=theorem]{conjecture}
\declaretheorem[style=assertionstyle,name=Claim,sibling=theorem]{claim}
\declaretheorem[style=assertionstyle,name=Claim, numbered=no]{claim*} % unnumbered claim
\declaretheorem[style=assertionstyle,name=Fact,sibling=theorem]{fact}
\declaretheorem[style=assertionstyle,name=Property,sibling=theorem]{property}

\declaretheoremstyle[headfont=\bfseries,bodyfont=\normalfont,spaceabove=6pt,spacebelow=6pt,qed=\ding{45},postheadspace=1em]{definitionstyle}
\declaretheorem[style=definitionstyle,name=Definition, numberwithin=section]{definition}
\declaretheorem[style=definitionstyle,name=Axiom,sibling=definition]{axiom}
\declaretheorem[style=definitionstyle, name=Problem, sibling=definition]{problem}
\declaretheorem[style=definitionstyle, name=Notation, sibling=definition]{notation}
\declaretheorem[style=definitionstyle, name=Notation, numbered=no]{notation*} % unnumbered notation


\declaretheoremstyle[headfont=\bfseries\color{funblue},bodyfont=\normalfont\normalsize,spaceabove=6pt,spacebelow=6pt,qed=\ensuremath{\color{funblue}\blacktriangleleft},postheadspace=1em]{examplestyle}
\declaretheorem[style=examplestyle,name=Example, numberwithin=section]{example}

\declaretheoremstyle[headfont=\bfseries,bodyfont=\normalfont\normalsize,spaceabove=6pt,spacebelow=6pt,qed=\ensuremath{\blacktriangleleft},postheadspace=1em]{remarkstyle}
\declaretheorem[style=remarkstyle,name=Remark, numberwithin=section]{remark}
\declaretheorem[style=remarkstyle,name=Observation, sibling=remark]{observation}
\declaretheorem[style=remarkstyle,name=Invariant, sibling=remark]{invariant}
\declaretheorem[style=remarkstyle,name=Assumption, sibling=remark]{assumption}
\declaretheorem[style=remarkstyle,name=Convention, sibling=remark]{convention}
\declaretheorem[style=remarkstyle,name=Convention, numbered=no]{convention*} % unnumbered convention

\declaretheoremstyle[headfont=\color{alizarincrimsonred}\bfseries,bodyfont=\normalfont\normalsize,spaceabove=6pt,spacebelow=6pt,qed=\ensuremath{\color{alizarincrimsonred}\blacktriangleleft},postheadspace=1em]{cautionstyle}
\declaretheorem[style=cautionstyle,name=Caution,sibling=remark]{caution}
\declaretheoremstyle[headfont=\bfseries,bodyfont=\normalfont\footnotesize,spaceabove=6pt,spacebelow=6pt,postheadspace=1em]{smallremarkstyle}
\declaretheorem[style=smallremarkstyle,name=Remark,sibling=remark]{smallremark}

\declaretheoremstyle[headfont=\bfseries\color{amethyst},bodyfont=\normalfont,spaceabove=6pt,spacebelow=6pt,qed=\ensuremath{\color{amethyst}\blacktriangleleft},postheadspace=1em]{digressionstyle}
\declaretheorem[style=digressionstyle,name=Digression, numberwithin=section]{digression}
\let\proof\relax
\let\endproof\relax
\declaretheoremstyle[
  headfont=\bfseries,
  bodyfont=\normalfont,
  spaceabove=6pt,
  spacebelow=6pt,
  qed=\ensuremath{\square},
  postheadspace=1em
]{proofstyle}
\declaretheorem[
  style=proofstyle,
  name=Proof,
  numbered=no
]{proof}
\newenvironment{verticalhack}
  {\begin{array}[b]{@{}c@{}}\displaystyle}
  {\\\noalign{\hrule height0pt}\end{array}} % to make qed symbol aligned




% =========================
% Meta / research statements
% =========================

% --- style (pick one): looks like your assertion boxes, but no qed symbol ---
\declaretheoremstyle[
  headfont=\bfseries,
  bodyfont=\normalfont,
  spaceabove=6pt,
  spacebelow=6pt,
  postheadspace=1em,
  qed=\ensuremath{\vartriangleleft}
]{metastyle}

% --- numbered (shares counter with theorem) ---
\declaretheorem[style=metastyle,name=Question,sibling=theorem]{question}
\declaretheorem[style=metastyle,name=Open Question,sibling=theorem]{openquestion}
\declaretheorem[style=metastyle,name=Open Problem,sibling=theorem]{openproblem}
\declaretheorem[style=metastyle,name=Challenge,sibling=theorem]{challenge}
\declaretheorem[style=metastyle,name=Research Direction,sibling=theorem]{researchdirection}
\declaretheorem[style=metastyle,name=Fundamental Question,sibling=theorem]{fundamentalquestion} 

% --- unnumbered (starred) variants ---
\declaretheorem[style=metastyle,name=Question,numbered=no]{question*}
\declaretheorem[style=metastyle,name=Open Question,numbered=no]{openquestion*}
\declaretheorem[style=metastyle,name=Open Problem,numbered=no]{openproblem*}
\declaretheorem[style=metastyle,name=Challenge,numbered=no]{challenge*}
\declaretheorem[style=metastyle,name=Research Direction,numbered=no]{researchdirection*}
\declaretheorem[style=metastyle,name=Fundamental Question,numbered=no]{fundamentalquestion*}





\usepackage{algorithm}
\usepackage[italicComments=false]{algpseudocodex}



% numbering within sections ~~~~~~~~~~~~~~~~~~~~~~~~~~~~~~~~

\numberwithin{equation}{section}



\usepackage{chngcntr}
\counterwithin{figure}{section}
\counterwithin{table}{section}
\counterwithin{algorithm}{section} % works with the 'algorithm' package



% notation makros ~~~~~~~~~~~~~~~~~~~~~~~~~~~~~~~~

\DeclareMathOperator*{\argmax}{arg\,max}
\DeclareMathOperator*{\argmin}{arg\,min}

% commands for functions etc.
\newcommand{\im}{\operatorname{Im}}

% commands for vectors and matrices

\newcommand*\matrspace{0.8mu}
\newcommand{\matr}[1]{%
  \mspace{\matrspace}%
  \underline{%
    \mspace{-\matrspace}%
    \smash[b]{\boldsymbol{#1}}%
    \mspace{-\matrspace}%
  }%
  \mspace{\matrspace}%
}
\newcommand{\vect}[1]{\vec{\boldsymbol{#1}}}
% \newcommand{\vect}[1]{\vec{{#1}}\,}

% commands for derivatives
\newcommand{\dif}{\mathrm{d}}

% commands for special constants
\newcommand{\eu}{\mathrm{e}}
\newcommand{\iu}{\mathrm{i}}

% commands for number sets
\newcommand{\R}{\mathbb{R}}
\newcommand{\N}{\mathbb{N}}
\newcommand{\Z}{\mathbb{Z}}
\newcommand{\Q}{\mathbb{Q}}
\newcommand{\C}{\mathbb{C}}

% commands for probability
\newcommand{\Var}{\operatorname{Var}}
\newcommand{\Cov}{\operatorname{Cov}}
\newcommand{\Exp}{\operatorname{E}}
% \newcommand{\P}{\operatorname{P}} % this is already defined in amsmath/amsopn
\newcommand{\Prob}{\operatorname{P}}
\newcommand{\numof}{\ensuremath{\# \,}} % number of elements in a set
\newcommand{\blackheight}{\operatorname{bh}}

% \algnewcommand{\LeftComment}[1]{\(\triangleright\) #1}
\algnewcommand{\TO}{, \ldots ,}
\algnewcommand{\DOWNTO}{, \ldots ,}
\algnewcommand{\OR}{\vee}
\algnewcommand{\AND}{\wedge}
\algnewcommand{\NOT}{\neg}
\algnewcommand{\LEN}{\operatorname{len}}
\algnewcommand{\tru}{\ensuremath{\mathrm{\texttt{true}}}}
\algnewcommand{\fals}{\ensuremath{\mathrm{\texttt{false}}}}
\algnewcommand{\append}{\circ}

\algnewcommand{\nil}{\ensuremath{\mathrm{\textsc{nil}}}}
\algnewcommand{\red}{\ensuremath{\mathrm{\textsc{red}}}}
\algnewcommand{\black}{\ensuremath{\mathrm{\textsc{black}}}}
\algnewcommand{\gray}{\ensuremath{\mathrm{\textsc{gray}}}}
\algnewcommand{\white}{\ensuremath{\mathrm{\textsc{white}}}}

\algnewcommand{\parent}{\operatorname{parent}}
\algnewcommand{\sign}{\operatorname{sign}}

\newcommand{\attrib}[2]{\ensuremath{#1\mathtt{.}\mathtt{#2}}} % object.field with field in math typewriter (like code)
\newcommand{\attribnormal}[2]{\ensuremath{#1\mathtt{.}#2}} % object.field with field in normal math font
\newcommand{\attribute}[1]{\ensuremath{\mathtt{#1}}} 
\newcommand{\attributenormal}[1]{\ensuremath{#1}}




% ~~~~~~~~~~~~~~~~~~~~~~~~~~~~~~~~~~~~~~~


\usepackage{caption, subcaption}

\usepackage[backend=biber,style=numeric]{biblatex}
\addbibresource{\jobname.bib}




\usepackage[
  % linktoc=all,      % make the whole entry (text + pagenumber) a link
  linktoc=none,
  pdfauthor={Fabian Bosshard},
  pdftitle={USI - Algorithms and Complexity - Course Notes},
  pdfkeywords={USI, Algorithms and Complexity, course notes, informatics},
  colorlinks=false,  
  % pdfborder={0.0 0.0 0.3},  % debug
  pdfborder={0 0 0},        % no visible borders
  linkbordercolor={0 0.6 1},      % internal links
  urlbordercolor={0 0.6 1},       % URLs
  citebordercolor={0 0.6 1}       % citations
]{hyperref}
\usepackage{etoolbox}

\makeatletter
\newlength\FB@toclinkht
\newlength\FB@toclinkdp
\setlength\FB@toclinkht{.80\ht\strutbox}
\setlength\FB@toclinkdp{.40\dp\strutbox}

% --- ToC: add a target at the entry + one full-width clickable line ---
\let\FB@orig@contentsline\contentsline
\renewcommand*\contentsline[4]{%
  \begingroup
    \Hy@safe@activestrue
    \edef\Hy@tocdestname{#4}%
    \FB@orig@contentsline{#1}{%
      % target for "back to ToC"
      \Hy@raisedlink{\hyper@anchorstart{toc:\Hy@tocdestname}\hyper@anchorend}%
      \leavevmode
      \rlap{%
        \hyper@linkstart{link}{\Hy@tocdestname}%
          \raisebox{0pt}[\FB@toclinkht][\FB@toclinkdp]{%
            \hbox to \dimexpr\hsize-\parindent\relax{\hfil}%
          }%
        \hyper@linkend
      }%
      #2%
    }{#3}{#4}%
  \endgroup
}

% --- Headings: make NUMBER + TITLE one link back to the ToC entry ---
\let\FB@orig@sectionlinesformat\sectionlinesformat
\renewcommand{\sectionlinesformat}[4]{%
  \ifx\@currentHref\@empty
    \FB@orig@sectionlinesformat{#1}{#2}{#3}{#4}%
  \else
    \leavevmode
    \hyper@linkstart{link}{toc:\@currentHref}%
      % IMPORTANT: \hskip needs a DIMENSION only; #3 is text -> separate them
      \@hangfrom{\hskip #2\relax #3}{#4}%
    \hyper@linkend
    \par
  \fi
}
\makeatother


\usepackage[
  type     = {CC},
  modifier = {by},
  version  = {4.0},
]{doclicense}






% autoref names ~~~~~~~~~~~~~~~~~~~~~~~~~~~~~~~~

\renewcommand{\definitionautorefname}{Definition}
\renewcommand{\notationautorefname}{Notation}
\renewcommand{\lemmaautorefname}{Lemma}
\renewcommand{\theoremautorefname}{Theorem}
\renewcommand{\corollaryautorefname}{Corollary}
\renewcommand{\propositionautorefname}{Proposition}
\renewcommand{\conjectureautorefname}{Conjecture}
\renewcommand{\claimautorefname}{Claim}
\renewcommand{\exampleautorefname}{Example}
\renewcommand{\remarkautorefname}{Remark}
\renewcommand{\invariantautorefname}{Invariant}
\renewcommand{\assumptionautorefname}{Assumption}
\renewcommand{\conventionautorefname}{Convention}

\renewcommand{\observationautorefname}{Observation}
\renewcommand{\cautionautorefname}{Caution}
\renewcommand{\digressionautorefname}{Digression}

\renewcommand{\questionautorefname}{Question}
\renewcommand{\openquestionautorefname}{Open Question}
\renewcommand{\openproblemautorefname}{Open Problem}
\renewcommand{\challengeautorefname}{Challenge}
\renewcommand{\researchdirectionautorefname}{Research Direction}
\renewcommand{\fundamentalquestionautorefname}{Fundamental Question}

\makeatletter
\def\thmt@dummyctrautorefname~#1\null{Proof\null}
\makeatother


\makeatletter
\patchcmd{\ALG@step}{\addtocounter{ALG@line}{1}}{\refstepcounter{ALG@line}}{}{}
\newcommand{\ALG@lineautorefname}{Line}
\makeatother
\newcommand{\algorithmautorefname}{Algorithm}



\renewcommand{\sectionautorefname}{Section}
\renewcommand{\subsectionautorefname}{Section}
\renewcommand{\subsubsectionautorefname}{Section}





\makeatletter

% --------------------
% equations + floats
% --------------------
\renewcommand*{\theHequation}{\thesection.\arabic{equation}}

\renewcommand*{\theHfigure}{\thesection.\arabic{figure}}
\renewcommand*{\theHtable}{\thesection.\arabic{table}}




% --------------------
% algorithms
% --------------------
\renewcommand*{\theHalgorithm}{\thesection.\arabic{algorithm}}

% Make hyperref anchor for algorithmic line numbers unique:
% include the current algorithm anchor (\theHalgorithm) + the line number.
\providecommand*{\theHALG@line}{\theHalgorithm.\arabic{ALG@line}}
\renewcommand*{\theHALG@line}{\theHalgorithm.\arabic{ALG@line}}



% --------------------
% theorem-like: assertion family
% --------------------
\renewcommand*{\theHtheorem}{\thesection.\arabic{theorem}}
\renewcommand*{\theHlemma}{\thesection.\arabic{lemma}}
\renewcommand*{\theHcorollary}{\thesection.\arabic{corollary}}
\renewcommand*{\theHproposition}{\thesection.\arabic{proposition}}
\renewcommand*{\theHconjecture}{\thesection.\arabic{conjecture}}
\renewcommand*{\theHclaim}{\thesection.\arabic{claim}}
\renewcommand*{\theHfact}{\thesection.\arabic{fact}}
\renewcommand*{\theHproperty}{\thesection.\arabic{property}}

% --------------------
% theorem-like: definition family
% --------------------
\renewcommand*{\theHdefinition}{\thesection.\arabic{definition}}
\renewcommand*{\theHaxiom}{\thesection.\arabic{axiom}}
\renewcommand*{\theHproblem}{\thesection.\arabic{problem}}
\renewcommand*{\theHnotation}{\thesection.\arabic{notation}}

% --------------------
% theorem-like: other families
% --------------------
\renewcommand*{\theHexample}{\thesection.\arabic{example}}

\renewcommand*{\theHremark}{\thesection.\arabic{remark}}
\renewcommand*{\theHobservation}{\thesection.\arabic{observation}}
\renewcommand*{\theHinvariant}{\thesection.\arabic{invariant}}
\renewcommand*{\theHassumption}{\thesection.\arabic{assumption}}
\renewcommand*{\theHconvention}{\thesection.\arabic{convention}}


\renewcommand*{\theHcaution}{\thesection.\arabic{caution}}
\renewcommand*{\theHsmallremark}{\thesection.\arabic{smallremark}}

\renewcommand*{\theHdigression}{\thesection.\arabic{digression}}


\renewcommand*{\theHquestion}{\thesection.\arabic{question}}
\renewcommand*{\theHopenquestion}{\thesection.\arabic{openquestion}}
\renewcommand*{\theHopenproblem}{\thesection.\arabic{openproblem}}
\renewcommand*{\theHchallenge}{\thesection.\arabic{challenge}}
\renewcommand*{\theHresearchdirection}{\thesection.\arabic{researchdirection}}
\renewcommand*{\theHfundamentalquestion}{\thesection.\arabic{fundamentalquestion}}



\makeatother





% % this is almost what i want:
% \makeatletter
% \RedeclareSectionCommand[
%   toclinefill=\hyper@linkstart{link}{\Hy@tocdestname}\TOCLineLeaderFill\hyper@linkend,
% ]{section}
% \RedeclareSectionCommand[
%   toclinefill=\hyper@linkstart{link}{\Hy@tocdestname}\TOCLineLeaderFill\hyper@linkend,
% ]{subsection}
% \RedeclareSectionCommand[
%   toclinefill=\hyper@linkstart{link}{\Hy@tocdestname}\TOCLineLeaderFill\hyper@linkend,
% ]{subsubsection}
% \makeatother
% % we just need to make the links fill the entire width (from beginning of section name to end of page number)




% \title{Algorithms \& Complexity -- Summary}
\title{Algorithms \& Complexity}
\author{Fabian Bosshard}
\date{\today}




\begin{document}

\begin{titlepage}
\thispagestyle{empty}
\centering

\vspace*{5cm}
{\Huge\bfseries \href{https://search.usi.ch/courses/35275486/algorithms-complexity}{Algorithms \& Complexity}\par}
\vspace{0.5cm}
{\large\color{AccentGray}\href{https://www.usi.ch/}{\textsc{Università della Svizzera italiana}}\par}
\vspace{1.5cm}

{\Large \href{https://fabianbosshard.github.io/}{Fabian Bosshard}\par}
\vfill
{ \today\par}

\end{titlepage}

\pagenumbering{roman}



\thispagestyle{scrheadings}
\tableofcontents

\section*{Preface}
\addcontentsline{toc}{section}{Preface}

This document is an unofficial student-made summary of the course
\href{https://search.usi.ch/courses/35275486/algorithms-complexity}{Algorithms \& Complexity} taught by \href{http://usi.to/i3c}{Evanthia Papadopoulou} in Winter 2025/2026 at the \href{https://www.usi.ch/en}{Università della Svizzera italiana}.
It is based on the lecture slides and other course materials.
If you find any errors, please report them to \href{mailto:fabianlucasbosshard@gmail.com}{{fabianlucasbosshard@gmail.com}}.
The \LaTeX{} source code is available on \url{https://github.com/fabianbosshard/usi-informatics-course-summaries}.

\doclicenseThis

% ----- Keep the following commands unchanged -------------------
% \renewcommand{\emph}[1]{\textcolor{black}{#1}}
% \renewcommand{\emph}[1]{\textbf{#1}}

\clearpage
\pagenumbering{arabic}

\section{Introduction}

typically,
\begin{itemize}%[nolistsep, noitemsep]
\item numerical data \(\rightarrow\) numerical analysis 
\item discrete data \(\rightarrow\) algorithm design and analysis
\end{itemize} 

\medskip

two fundamental issues to consider: \emph{correctness}, \emph{efficiency}




\section{Asymptotic Notation and Recurrences}

\begin{theorem}[Simplified Master Theorem]\label{thm:simpl_master_theorem}
Let $a \geq 1, b>1$ be constants and let $T(n)$ be the recurrence
\begin{equation}\label{eq:master_theorem}
T(n)=a T(n / b)+c n^k
\end{equation}
defined for $n \geq 0$.
\begin{itemize}%[nolistsep, noitemsep]
\item Case 1: $a>b^k$ then $T(n)$ is $\Theta(n^{\log _b a})$
\item Case 2: $a=b^k$ then $T(n)$ is $\Theta(n^k \log n)$
\item Case 3: $a<b^k$ then $T(n)$ is $\Theta(n^k)$
\qedhere
\end{itemize}
\end{theorem}

\bigskip

Summation with general bounds:
\[
\sum_{i=a}^b f(i)=\sum_{i=0}^b f(i)-\sum_{i=0}^{(a-1)} f(i)
\]

\medskip

% Linearity of summation:
% \[
% \sum((f(i)+g(i)) h(i))=\sum(f(i) h(i))+\sum(g(i) h(i))
% \]

% \medskip

Approximation using integrals:
\[
\int_0^n f(x) \dif x \leq \sum_{i=1}^n f(i) \leq \int_1^{n+1} f(x) \dif x
\]
(if $f(x)$ is monotonically increasing)

\clearpage

\section{Stable Matching}

motivation: set up pairings between different entities, where each side has a notion of preference 

% \medskip

\begin{definition}[Matching]\label{def:matching}
Given a pair of sets $M$ and $W$, a matching \(S\) is a collection of pairs \((m,w)\), where \(m \in M\) and \(w \in W\), and each element from either set appears in at most one pair. 
A matching is perfect if no element remains unmatched.
\end{definition}

\begin{definition}[Stability]\label{def:stability}
Given sets $M$ and $W$ of equal size and a preference ordering for each element of each set, a stable matching is a perfect matching where no unstable pair exists.
An unstable pair is a pair \((m,w)\) that is \emph{not} in the matching such that \(m\) and \(w\) prefer each other to their current partners in the matching.
\end{definition}

given two sets \(M\) and \(W\) of equal size \(n\), where every \(m\in M\) and every \(w\in W\) has a strict and complete preference list over the elements of the other set
% \(2n\) strict and complete preference lists, each consisting of \(n\) elements (one list for every \(m\in M\) and every \(w\in W\)), where \(n = |M| = |W|\)

\begin{algorithm}[h]
\caption{Propose-and-Reject (Gale-Shapley, 1962)}
\label{alg:gale_shapley}
\begin{algorithmic}[1]
  \Require \(2n\) strict and complete preference lists, each consisting of \(n\) elements
  \Ensure a matching \(S^*\) that pairs each \(m\in M\) with each \(w\in W\)
  \State initialize all \(m\in M\) and \(w\in W\) as \attribute{free}
  \While{there exists a \attribute{free} \(m\in M\) who has not proposed to every \(w\in W\)}
    \State choose such an \(m\)
    \State \(w\) \(\gets\) highest ranked element on \(m\)'s list to whom \(m\) has not yet proposed
    \State \(m\) proposes to \(w\)
    \If{\(w\) is \attribute{free}}
      \State match \(m\) and \(w\)
    \ElsIf{\(w\) prefers \(m\) to her current partner \(m'\)}
      \State match \(m\) and \(w\)
      \State mark \(m'\) as \attribute{free}
    \Else
      \State \(w\) rejects \(m\)'s proposal
    \EndIf
  \EndWhile
  \State \Return \(S^*\)
\end{algorithmic}
\end{algorithm}


\begin{observation}\label{obs:gale_shapley:men_propose_in_decreasing_order}
\(M\) propose to \(W\) in decreasing order of preference.
\end{observation}

\begin{observation}\label{obs:gale_shapley:women_only_trade_up}
Once a \(w\in W\) is matched, it never becomes unmatched, it only trades up.
\end{observation}


\subsection{Correctness}

\begin{claim}[Correctness: Termination]
\label{clm:gale_shapley:termination}
\autoref{alg:gale_shapley} terminates after at most \(n^2\) iterations of the while-loop.
\end{claim}

\begin{proof}\label{proof:gale_shapley:termination}
\leavevmode
% \begin{itemize}[nolistsep, noitemsep]
% \item 
There is one proposal per iteration.
% \item 
By \autoref{obs:gale_shapley:men_propose_in_decreasing_order}, once a \(m\in M\) has proposed to a \(w\in W\), it never proposes to \(w\) again. 
So each \(m\in M\) does \(\leq n\) proposals.
% \item 
There are \(n\) elements in \(M\), and each does \(\leq n\) proposals. 
After \(\leq n^2\) iterations, no one is left to propose to. 
Thus, \autoref{alg:gale_shapley} must terminate after \(\leq n^2\) iterations.
% \qedhere
% \end{itemize}
\end{proof}


\begin{claim}[Correctness: Perfect Matching]\label{clm:gale_shapley:perfection}
Every \(m\in M\) and every \(w\in W\) gets matched by \autoref{alg:gale_shapley}.
\end{claim}

\begin{proof}[Contradiction]\label{proof:gale_shapley:perfection}
Assume there is a \(m\in M\) who is unmatched.
Since \(|M| = |W|\), there must also be a \(w\in W\) who is unmatched.
By \autoref{obs:gale_shapley:women_only_trade_up}, \(w\) was never proposed to.
But for \(m\) to remain unmatched after termination, \(m\) must have proposed to every \(w\in W\), including this \(w\).
\end{proof}


\begin{claim}[Correctness: Stability]\label{clm:gale_shapley:stability}
The matching created by \autoref{alg:gale_shapley} is stable.
\end{claim}

\begin{proof}[Contradiction]\label{proof:gale_shapley:stability}
Assume there is an unstable pair \((m, w)\), where \(m \in M\) and \(w \in W\) after termination of \autoref{alg:gale_shapley}.


\medskip

We can distinguish two cases:\tikzmark{distinguish-cases}%
\begin{tikzpicture}[>=latex, remember picture, overlay, scale=1]
\coordinate (distinguishcases-y) at (pic cs:distinguish-cases);
\coordinate (distinguishcases-border) at (current page.east |- distinguishcases-y);
\coordinate (distinguishcases-margin) at ($(distinguishcases-border) + 0.5*(\textwidth,0) + 0.5*(-\paperwidth,0)$);
\begin{scope}[shift={(distinguishcases-margin)}, scale=0.8]
\begin{scope}[shift={(-2,-0.75)}]
\def\matchingellipse{(0,0) ellipse (2 and 1)}
\filldraw[fill=gray!20, draw=gray!60]\matchingellipse;
\node[left] at (-1.6,0.8) {\footnotesize matching};
\node (m) at (-1.0,0.5) {\footnotesize $m$};
\node (w) at (1.0,0.5) {\footnotesize $w$};
\node (mp) at (-1.0,-0.5) {\footnotesize $m'$};
\node (wp) at (1.0,-0.5) {\footnotesize $w'$};
\draw[densely dotted, latex -latex] (m) -- node[pos=0.5, above]{\footnotesize ?} (w);
\draw[latex -latex] (m) -- (wp);
\draw[latex -latex] (mp) -- (w);
\end{scope}
\end{scope}
\end{tikzpicture}

\begin{enumerate}[label=Case \arabic*:, ref=Case \arabic*, leftmargin=*, labelindent=1em, itemsep=\smallskipamount, topsep=\medskipamount]
\item \(m\) never proposed to \(w\) \label{proof:gale_shapley:stability:case:never_proposed}
\item \(m\) proposed to \(w\) \label{proof:gale_shapley:stability:case:proposed}
\end{enumerate}

In \ref{proof:gale_shapley:stability:case:never_proposed}, by \autoref{obs:gale_shapley:men_propose_in_decreasing_order}, \(m\) prefers his final partner \(w'\) to \(w\), contradicting the assumption that \((m,w)\) is an unstable pair. \medskip

In \ref{proof:gale_shapley:stability:case:proposed}, \(w\) either rejected \(m\) or traded up from \(m\) to her final partner \(m'\).
So, by \autoref{obs:gale_shapley:women_only_trade_up}, \(w\) prefers her final partner \(m'\) to \(m\), contradicting the assumption that \((m,w)\) is an unstable pair.
\end{proof}

\subsection{Male Optimality}

For a given problem instance, there may be multiple stable matchings.

\begin{definition}[Valid Partner]\label{def:valid_partner}
An element \(m \in M\) is a valid partner of an element \(w \in W\) if there exists a stable matching in which they are paired.
\end{definition}

\begin{theorem}[\(M\)-optimality]\label{clm:gale_shapley:x_optimality}
The matching created by \autoref{alg:gale_shapley} is \(M\)-optimal, i.e., each \(m \in M\) is matched with their best valid partner.
The order that \(M\) propose does not matter.
\end{theorem}

\begin{proof}[Contradiction]\label{proof:gale_shapley:x_optimality}
Assume there are men that are matched with someone other than their best valid partner.
By \autoref{obs:gale_shapley:men_propose_in_decreasing_order}, those men must have been rejected or dumped by valid partners during \autoref{alg:gale_shapley}.
Let \(y\) be the first such man and let \(a\) be the first valid woman who rejects/dumps him.
When \(y\) is dumped/rejected by \(a\) in \autoref{alg:gale_shapley}, \(a\) forms/reaffirms engagement with a man, say \(z\).
So \(a\) prefers \(z\) to \(y\).

\[
\begin{tikzpicture}[>=latex, scale=1.0]
\def\matchingellipse{(0,0) ellipse (2 and 1)}
\filldraw[fill=gray!20, draw=gray!60]\matchingellipse;
\node[left] at (-1.6,0.8) {$S$};
\node (y) at (-1.0,0.5) {$y$};
\node (a) at (1.0,0.5) {$a$};
\node (z) at (-1.0,-0.5) {$z$};
\node (b) at (1.0,-0.5) {$b$};
\draw[latex -latex] (y) -- (a);
\draw[latex -latex] (z) -- (b);
% \draw[densely dotted, latex-latex] (z) .. controls (-0.4,-1.3) and (0.4,1.3) .. (a);
\draw[densely dotted, latex-latex] (z) -- node[pos=0.5]{\textcolor{red}{\Lightning}} (a);
\end{tikzpicture}
\]

Let \(S\) be a stable matching in which \(y\) is matched with \(a\) (it exists, since \(a\) is a valid partner of \(y\)).
Let \(b\) be the partner of \(z\) in \(S\).
\(z\) has not been rejected/dumped by any valid partner (including \(b\)) at the point when \(y\) is rejected/dumped by \(a\), because the latter is the first such event during \autoref{alg:gale_shapley}.
That means \(z\) has not yet proposed to \(b\) when he proposes to \(a\), so \(z\) prefers \(a\) to \(b\).
Thus \((z,a)\) is an unstable pair in \(S\), contradicting the stability of \(S\).
\end{proof}

It can be proven that the \(M\)-optimal matching produced by \autoref{alg:gale_shapley} is unique and \(W\)-pessimal, i.e., each \(w \in W\) is matched with their worst valid partner.



\subsection{Implementation}

We want to implement \autoref{alg:gale_shapley} in \(O(n^2)\) time.
This means one iteration of the while-loop must take \(O(1)\) time.

\medskip

To that end, we need to be able to answer the question ``Does \(w\) prefer \(m\) to her current partner \(m'\)?'' in \(O(1)\) time!
Let's say that each \(w \in W\) has a preference array \(\texttt{pref}_w = [m_1, \ldots, m_n]\), where \(m_i\) is the \(i\)-th most preferred man for \(w\).
To answer the above question in constant time, we build an inverse preference array \(\texttt{inverse}_w\) in \(O(n)\) time once for each \(w \in W\) as a preprocessing step.

\begin{algorithm}[h]
\caption{Build Inverse Preference Arrays}
\label{alg:build_inverse_preference_arrays}
\begin{algorithmic}[1]
\Require for each \(w \in W\), a preference array \(\texttt{pref}_w\) such that \(\texttt{pref}_w[i]\) gives the \(i\)-th most preferred man for \(w\)
\Ensure for each \(w \in W\), an inverse preference array \(\texttt{inverse}_w\) such that \(\texttt{inverse}_w[m]\) gives the rank of \(m\) in \(w\)'s preference list
\For{each \(w \in W\)}
  \State initialize the array \(\texttt{inverse}_w\) of size \(n\)
  \For{\(i \gets 1 \TO n\)}
    \State \(\texttt{inverse}_w[\texttt{pref}_w[i]] \gets i\) 
  \EndFor
\EndFor
\end{algorithmic}
\end{algorithm}


\subsection{Other Properties}\label{sec:other_properties}

\begin{claim}\label{clm:gale_shapley:women_any_pairing}
Suppose all men use the same preference list \(l: w_1 \succ \cdots \succ w_n\).
If the women know about \(l\), they can force \autoref{alg:gale_shapley} to return any desired pairing \(P\).
\end{claim}

\begin{proof}[Induction]\label{proof:gale_shapley:women_any_pairing}
Consider the desired pairing \(P=\{(w_i,m_i)\}_{i=1}^n\).
Let every woman \(w_i\) submit a {fake} preference list that places \(m_i\) at the very top
(the order of the remaining men is arbitrary).

\begin{enumerate}[partopsep=0em, label=(\roman*)]
\item Base case:
When it is \(m_1\)'s time to propose, he will first propose to \(w_1\), since all men share the list \(l\).
Because \(m_1\) is ranked first by \(w_1\), she accepts and will never dump him (\autoref{obs:gale_shapley:women_only_trade_up}).
So \((w_1,m_1)\) is permanently formed.
\textcolor{Green}{\ding{52}}

\item Induction hypothesis:
Assume for some \(k\ge 2\) that \((w_i,m_i)\) are permanently matched for all \(i<k\). 
\label{proof:gale_shapley:women_any_pairing:induction_hypothesis}

\item Induction step:
Consider \(m_k\).
% By \autoref{obs:gale_shapley:men_propose_in_decreasing_order}, \(m_k\) proposed (in sequence) to \(w_1,w_2,\ldots,w_{k-1}\) before \(w_k\).
% Each \(w_i\) with \(i<k\) rejects \(m_k\), because she is already engaged to her top choice \(m_i\).
Because of \ref{proof:gale_shapley:women_any_pairing:induction_hypothesis}, each \(w_i\) with \(i<k\) is already permanently matched with her top choice \(m_i\) and thus rejected/dumped \(m_k\) at some point.
So \(m_k\) eventually proposes to \(w_k\), who ranks \(m_k\) first and accepts.
By \autoref{obs:gale_shapley:women_only_trade_up}, \(w_k\) never drops \(m_k\), so \((w_k,m_k)\) is permanent.
\textcolor{Green}{\ding{52}}
\end{enumerate}

By induction, all pairs \((w_i,m_i)\) form and remain, so the algorithm returns \(P\).
\end{proof}

\begin{claim}
There is at most one man \(m^* \in M\) such that the matching created by \autoref{alg:gale_shapley} matches \(m^*\) with his last-choice woman \(w^*\).
Moreover, at the moment \(m^*\) is matched with \(w^*\), \autoref{alg:gale_shapley} terminates.
\end{claim}

\begin{proof}
Let \(m^*\) be the first man to propose to his last-choice woman \(w^*\) in \autoref{alg:gale_shapley}.
By \autoref{obs:gale_shapley:men_propose_in_decreasing_order}, he has proposed to every \(w\in W\setminus\{w^*\}\) and been rejected/dumped and by \autoref{obs:gale_shapley:women_only_trade_up}, each such \(w\) is (and stays) engaged. 
Hence exactly \(n-1\) women (and thus \(n-1\) men) are engaged, so \(m^*\) is the unique free man.
If \(w^*\) were engaged, there would be \(n\) engaged men, contradicting that \(m^*\) is free. 
Therefore \(w^*\) is free and accepts \(m^*\), yielding \(n\) engaged pairs and immediate termination.
%Since the algorithm stops here, no second man can ever reach his last choice.
\end{proof}

\clearpage

\section{Graphs}\label{sec:graphs}


\subsection{Representations of Graphs}

\newcommand{\mixcolor}{70}

\definecolor{Red}{rgb}{1.0, 0.0, 0.0}
\definecolor{Orange}{rgb}{1.0, 0.5, 0.0}
\definecolor{Yellow}{rgb}{1.0, 1.0, 0.0}
\definecolor{Green}{rgb}{0.0, 1.0, 0.0}
\definecolor{BlueGreen}{rgb}{0.0, 1.0, 0.5}
\definecolor{Blue}{rgb}{0.0, 1.0, 1.0}

\newcolumntype{C}{>{\centering\arraybackslash}m{35mm}}
\setlength{\tabcolsep}{4pt}
\renewcommand{\arraystretch}{1.15}
\begin{table}[h]
\centering
\begin{tabular}{l C @{\hspace{10pt}} C}
\toprule
\textbf{Operation}                                             & \multicolumn{2}{c}{\textbf{Representation}}                                                                     \\
&                             {Adjacency List} &                                                              {Adjacency Matrix}                                                 \\
\midrule
% accessing vertex $u$ &        \cellcolor{Blue!\mixcolor}\makecell{$O(1)$\\optimal} &                          \cellcolor{Blue!\mixcolor}\makecell{$O(1)$\\optimal}             \\
% \addlinespace[2pt]
% iteration through $V$ &       \cellcolor{Green!\mixcolor}\makecell{$\Theta(n)$\\optimal} &                  \cellcolor{Green!\mixcolor}\makecell{$\Theta(n)$\\optimal}         \\
% \addlinespace[2pt]
space complexity &            \cellcolor{Green!\mixcolor}\makecell{$\Theta(n+m)$\\optimal} &              \cellcolor{Red!\mixcolor}\makecell{$\Theta(n^{2})$\\possibly very bad} \\
\addlinespace[2pt]
checking $(u,v)\in E$ &       \cellcolor{Orange!\mixcolor}\makecell{$O(\deg(u))$\\bad} &                             \cellcolor{Blue!\mixcolor}\makecell{$\Theta(1)$\\optimal}        \\
\addlinespace[2pt]
iteration through $E$ &       \cellcolor{Yellow!\mixcolor}\makecell{$\Theta(n+m)$\\okay (not optimal)} &  \cellcolor{Red!\mixcolor}\makecell{$\Theta(n^{2})$\\possibly very bad} \\
\bottomrule
\end{tabular}
\end{table}

in adjacency list, we often keep crosslinks between the same edges to avoid traversing lists unnecessarily

\begin{digression}[Sparse Matrix Representation]
We can use a multilist structure to store sparse matrices.
For that, we create \(2 n\) linked lists, one for each row and one for each column.
Each non-zero entry \(a_{ij}\) is stored in a node that contains the value \(a_{ij}\), the row index \(i\), the column index \(j\), a pointer to the next non-zero entry in row \(i\), and a pointer to the next non-zero entry in column \(j\).
With this representation, all standard matrix operations (such as multiplication, transposition) can be performed efficiently.
\end{digression}



\subsection{Basic Definitions}
a graph \(G=(V,E)\) consists of a set \(V\) of \(n=|V|\) vertices and a set \(E\) of \(m=|E|\) edges.
If pairs in \(E\) are ordered, the graph is \emph{directed}, otherwise it is \emph{undirected}.
In a digraph, self-loops \((v,v)\) are allowed.

\medskip 

in a graph:\\
\indent number of edges: \(0 \leq m \leq \binom{n}{2} = \frac{n(n-1)}{2} \in O(n^2)\) \\
\indent sum of degrees: \(\sum_{v\in V} \deg(v) = 2m\) 

\smallskip

in a digraph:\\
\indent number of edges: \(0 \leq m \leq n^2\) \\
\indent sum of degrees: \(\sum_{v\in V} \deg^+(v) = \sum_{v\in V} \deg^-(v) = m\)

\medskip

\emph{sparse} if \(m\) is \(O(n)\), else \emph{dense}




\begin{definition}[Path]\label{def:path}
  A path \(P\) in an undirected graph is a sequence of nodes \(v_1, \ldots, v_k\) with the property that each consecutive pair \(v_i, v_{i+1}\) is joined by an edge in \(E\).
\end{definition}

\begin{definition}[Simple Path]\label{def:simple_path}
  A path is simple if all nodes are distinct.
\end{definition}

\begin{definition}[Connected Graph]\label{def:connected_graph}
  An undirected graph is connected if for every pair of nodes \(u\) and \(v\), there is a path between \(u\) and \(v\).
\end{definition}

\begin{definition}[Cycle]\label{def:cycle}
  A cycle is a path \(v_1, \ldots, v_k\) in which \(v_1 = v_k\), \(k > 2\), and all other nodes are distinct, i.e., all nodes are distinct except the first and last.
\end{definition}

\begin{definition}[Tree]\label{def:tree}
  An undirected graph is a \emph{tree} if it is connected and does not contain a cycle.
\end{definition}

\begin{theorem}
  Let \(G\) be an undirected graph of \(n\) nodes.
  If we pick any two of the following statements, they imply the remaining one:
  \begin{itemize}[itemsep=0.05em]
    \item \(G\) is connected.
    \item \(G\) does not contain a cycle.
    \item \(G\) has \(n-1\) edges. \qedhere
  \end{itemize}
\end{theorem}

An acyclic undirected graph (which need not be connected) is a collection of free trees; it is called a forest.

\smallskip

An acyclic digraph is called a directed acyclic graph; a DAG.

\begin{definition}[Rooted tree]
  Given a tree \(T\), choose a root node \(r\) and orient each edge away from \(r\).
  Can be used to model hierarchical structure.
  \begin{itemize}%[nolistsep, noitemsep]
    \item \emph{depth} of a node: \(\#\) edges from the root to the node
    \item \emph{height} of a node: \(\#\) edges from the node to the deepest leaf
    \item \emph{height} of a tree: height of the root \qedhere
  \end{itemize}
\end{definition}




\subsection{Breadth-First Search}
\label{sec:bfs}

intuition: explore outward from \(s\) in all possible directions, adding nodes one `layer' at a time
\[
\begin{tikzpicture}[line cap=round, scale=0.8]
  \usetikzlibrary{shapes.symbols}
  \definecolor{blob}{RGB}{190,190,190}

  % location of the layers
  \coordinate (s) at (0,0);
  \coordinate (L1) at (3,0);
  \coordinate (L2) at (6,0);
  \coordinate (dots) at (9,0);
  \coordinate (Lk) at (12,0);

  % edges
  \draw[-] (s) -- ($(L1) + (0, 0.6)$);
  \draw[-] (s) -- ($(L1) - (0, 0.8)$);
  % edges between L1 and L2
  \draw[-] ($(L1) + (0, 0.6)$) -- ($(L2) + (0, 1.0)$);
  \draw[-] ($(L1) - (0, 0.3)$) -- ($(L2) + (0, 0.4)$);
  \draw[-] ($(L1) - (0, 0.9)$) -- ($(L2) - (0, 0.7)$);
  % edges between L2 and dots
  \draw[-] ($(L2) + (0, 1.0)$) -- ($(dots) + (-1.5, 0.8)$);
  \draw[-] ($(L2) + (0, -0.1)$) -- ($(dots) + (-1.5, 0.2)$);
  \draw[-] ($(L2) + (0, -0.7)$) -- ($(dots) + (-1.5, -0.9)$);
  % edges between dots and Lk
  \draw[-] ($(dots) + (1.5, 0.8)$) -- ($(Lk) + (0, 1.1)$);
  \draw[-] ($(dots) + (1.5, -0.2)$) -- ($(Lk) + (0, -0.6)$);

  % layers
  \tikzset{
    fixedcloud/.style={
      cloud, cloud puffs=13, cloud puff arc=120,
      minimum width=1.5cm, minimum height=2.5cm,
      inner sep=0pt, draw=none, fill=blob, aspect=1
    }
  }
\node[circle, inner sep=0pt, minimum size=0.6cm, draw=black, fill=white] (s_layer) at (s) {$s$};
\node[fixedcloud] (L1_layer) at (L1) {};
\node[align=center, inner sep=0pt] at (L1_layer.center) {$L_1$\\{\footnotesize $\delta=1$}};
\node[fixedcloud] (L2_layer) at (L2) {};
\node[align=center, inner sep=0pt] at (L2_layer.center) {$L_2$\\{\footnotesize $\delta=2$}};
\node at (dots) {\huge $\cdots$};
\node[fixedcloud] (Lk_layer) at (Lk) {};
\node[align=center, inner sep=0pt] at (Lk_layer.center) {$L_{l_s}$\\{\footnotesize $\delta=l_s$}};
\end{tikzpicture}
\]

\nameref{alg:bfs} discovers vertices in increasing order of edge distance from \(s\).

Distance of \(v\): \(\delta(s,v) = \text{min \# edges in any path from \(s\) to \(v\)}\)

Layer \(L_i\) consists of all nodes \(v\) at distance exactly \(i\) from \(s\): \(L_i = \{v \in V \mid \delta(s,v) = i\}\)

the number of layers is \(l_s = \max_{v \in V} \delta(s,v)\) and depends on \(s\)

to implement \nameref{alg:bfs}, we need to know which vertices have been visited and which haven't
\begin{itemize}
\item keep a ``frontier'' of verices that have been discovered but not yet processed (a FIFO queue)
\item initially all vertices (except \(s\)) are marked as undiscovered (\(\white\))
\item when a vertex is discovered, it is marked as discovered (\(\gray\)) and added to the frontier
\item after processing a discovered vertex, it is marked as finished (\(\black\))
\end{itemize}



\begin{algorithm}[h]
\caption{BFS}\label{alg:bfs}
\begin{algorithmic}[1]
\Function{BFS}{$G,\,s$} \Comment{\(s\) is the source}
  \ForAll{$u\in V \setminus \{s\}$}
    \State $\attrib{u}{color},\attribnormal{u}{\delta},\attrib{u}{\pi}\gets \white,\infty,\nil$ \Comment{initialization}
  \EndFor
  % \State Initialize $\attrib{v}{color}\gets\white$, $\attribnormal{v}{\delta}\gets\infty$, $\attrib{v}{\pi}\gets\nil$ for all $v\in V$
  \State $\attrib{s}{color},\attribnormal{s}{\delta},\attrib{s}{\pi}\gets \gray,0,\nil$ \Comment{initialize source \(s\)}
  \State $Q\gets\emptyset$ \Comment{initialize empty queue for vertices to visit}
  \State \Call{Enqueue}{$Q,s$}
  \While{$Q\neq\emptyset$} \label{alg:bfs:dequeue-while}
    \State $u\gets$ \Call{Dequeue}{$Q$} \Comment{get next vertex from the frontier}
    \ForAll{$v\in\Gamma(u)$} \label{alg:bfs:inner-loop}
      \If{$\attrib{v}{color}=\white$} \Comment{first time we have seen \(v\)?}
        \State $\attrib{v}{color}\gets\gray$ \Comment{mark it discovered}
        \State $\attribnormal{v}{\delta}\gets\attribnormal{u}{\delta}+1$ \Comment{set its distance from \(s\)}
        \State $\attrib{v}{\pi}\gets u$ \Comment{set its parent}
        \State \Call{Enqueue}{$Q,v$} 
      \EndIf
    \EndFor
    \State $\attrib{u}{color}\gets\black$ \Comment{we are done with \(u\)}
  \EndWhile
\EndFunction
\end{algorithmic}
\end{algorithm}

Predecessor pointers define an inverted tree (root = \(s\))\\
if we reverse these edges, we get a rooted tree called \nameref{alg:bfs} tree\\
\indent \(\exists\) many \nameref{alg:bfs} trees (depends on the order of vertices placed on the queue)

tree edges: edges of \nameref{alg:bfs} tree \\ 
cross edges: remaining edges

We define the \nameref{alg:bfs} tree as \(T_{\text{\nameref{alg:bfs}}}=(V_\pi,E_\pi)\) where
\[
V_\pi = \{v\in V\mid v.\pi \neq \nil \} \cup \{s\}
\quad
\text{and}
\quad
E_\pi = \{(v.\pi,v)\mid v\in V_\pi \setminus \{s\}\}
\]

\begin{property}
Let \(T_{\text{\nameref{alg:bfs}}}\) be a \nameref{alg:bfs} tree of \(G = (V, E)\).
Let \((x, y)\) be an edge of \(G\).
Then \hl{the level of \(x\) and \(y\) differ by at most \(1\)} (same layer or one apart):
\[
(x, y) \in E \quad \Longrightarrow \quad | \delta(s,x) - \delta(s,y) | \leq 1
\qedhere
\]
\end{property}

\textcolor{AccentBlue}{Running time of \nameref{alg:bfs}}:

We enqueue a vertex only if is $\white$, and we immediately color it $\gray$; thus, we enqueue every vertex at most once.

So the \emph{outer loop} at Line \ref{alg:bfs:dequeue-while} executes at most \(n\) times (once for each vertex).

For each vertex \(u\), the \emph{inner loop} at Line \ref{alg:bfs:inner-loop} executes \(1 + \deg(u)\) times.

Total time:
\[
\begin{aligned}
  T(n) &= n + \sum_{u \in V} (\deg(u) + 1) = n + \underbrace{\sum_{u \in V} \deg(u)}_{=2m} + \underbrace{\sum_{u \in V} 1}_{=n} \\[-12pt]
  &= 2n + 2m \in O(n + m)
\end{aligned}
\]

Cross edges are not arbitrary.
They connect nodes on the same layer or in neighboring layers.

% \clearpage

\subsection{Bipartite Graphs}

\begin{definition}[Bipartite]\label{def:bipartite}
An undirected graph \(G=(V,E)\) is bipartite if the nodes can be colored red or blue such that every edge has one red and one blue end.
\begin{itemize}
\item a bipartite graph is \(2\)-colorable 
\item a \(2\)-colorable graph is bipartite 
\item \(\Rightarrow\) a bipartite graph is a \(2\)-colorable graph
\qedhere
\end{itemize}
\end{definition}

many graph problems become:
\begin{itemize}
\item easier if the underlying graph is bipartite (matching)
\item tractable if the underlying graph is bipartite (independent set)
\end{itemize}

\begin{theorem}\label{thm:bipartite_odd_cycle}
  \hl[2]{a graph \(G\) is bipartite iff it contains no odd length cycle}.
\end{theorem}


\begin{algorithm}[h]
\caption{Find cross edges and odd cycles in an undirected graph using \nameref{alg:bfs}}\label{alg:bfs_detect_cross_edges_odd_cycles}
\label{alg:bfs_detect_cross_edges_odd_cycles}
\begin{algorithmic}[1]
\Function{BFS}{$G,\,s$} \Comment{\(s\) is the source}
  \ForAll{$u\in V \setminus \{s\}$}
    \State $\attrib{u}{color},\attribnormal{u}{\delta},\attrib{u}{\pi}\gets \white,\infty,\nil$ \Comment{initialization}
  \EndFor
  % \State Initialize $\attrib{v}{color}\gets\white$, $\attribnormal{v}{\delta}\gets\infty$, $\attrib{v}{\pi}\gets\nil$ for all $v\in V$
  \State $\attrib{s}{color},\attribnormal{s}{\delta},\attrib{s}{\pi}\gets \gray,0,\nil$ \Comment{initialize source \(s\)}
  \State $Q\gets\emptyset$ \Comment{initialize empty queue for vertices to visit}
  \State \Call{Enqueue}{$Q,s$}
  \While{$Q\neq\emptyset$} 
    \State $u\gets$ \Call{Dequeue}{$Q$} \Comment{get next vertex from the frontier}
    \ForAll{$v\in\Gamma(u)$} 
      \If{$\attrib{v}{color}=\white$} \Comment{first time we have seen \(v\)?}
        \State $\attrib{v}{color}\gets\gray$ \Comment{mark it discovered}
        \State $\attribnormal{v}{\delta}\gets\attribnormal{u}{\delta}+1$ \Comment{set its distance from \(s\)}
        \State $\attrib{v}{\pi}\gets u$ \Comment{set its parent}
        \State \Call{Enqueue}{$Q,v$}
      \BeginBox[draw=red]
      \Else
        \If{$\attrib{v}{color} = \black \AND v \neq \attrib{u}{\pi}$} \label{line:detect_cross_edge} \Comment{\(2^\text{nd}\) time we see \((u, v)\)}
          \State add \((u,v)\) to the list of cross edges
        \EndIf
        \If{$\attribnormal{v}{\delta} = \attribnormal{u}{\delta}$} \Comment{\(u\) and \(v\) are in the same layer}
          \State odd cycle detected!
        \EndIf
      \EndIf
      \EndBox
    \EndFor
    \State $\attrib{u}{color}\gets\black$ \Comment{we are done with \(u\)}
  \EndWhile
\EndFunction
\end{algorithmic}
\end{algorithm}

\autoref{alg:bfs_detect_cross_edges_odd_cycles} shows how to detect cross edges and odd cycles using \nameref{alg:bfs}, with only a few lines added (highlighted in red).

The reason we check only for \(\attrib{v}{color} = \black\) at Line~\ref{line:detect_cross_edge} is that if \((u,v)\) is a cross edge, we will encounter it twice: once from \(u\) to \(v\) and once from \(v\) to \(u\).
When we first encounter it (say, wlog, from \(u\)), both vertices will be \(\gray\), so we do not count it yet.
When we encounter it the second time (from \(v\)), \(u\) will already be \(\black\), so we count it then.
This way, we ensure that each cross edge is counted exactly once.



\begin{proof}[`\(\Rightarrow\)' of \autoref{thm:bipartite_odd_cycle}]
not possible to \(2\)-color an odd cycle, let alone \(G\).
\end{proof}
\begin{corollary}
  any cycle in a bipartite graph has an even \(\numof\)edges.
\end{corollary}

\begin{lemma}\label{lem:layers_bipartite}
For `\(\Leftarrow\)' of \autoref{thm:bipartite_odd_cycle}, we use the layers of \nameref{alg:bfs}: Let \(G\) be a connected graph and let \(L_0, \ldots, L_k\) be the layers produced by \nameref{alg:bfs} starting at node \(s\).
\begin{enumerate}
\item if no edge of \(G\) joins 2 nodes of the same layer, then \(G\) is bipartite \label{item:no_edge_same_layer_bipartite}
\item if an edge of \(G\) joins 2 nodes of the same layer, then \(G\) contains an odd length cycle, and hence is not bipartite \label{item:edge_same_layer_odd_cycle}
\qedhere
\end{enumerate}
\end{lemma}

\begin{corollary}[of \autoref{lem:layers_bipartite}]\label{cor:layers_bipartite}
  \leavevmode
  \begin{itemize}
  \item \(G\) has no odd cycle iff no edge joins two nodes of the same \nameref{alg:bfs} layer
  \item \hl[2]{\(G\) is bipartite iff no edge joins two nodes of the same \nameref{alg:bfs} layer}
  \qedhere
  \end{itemize}
\end{corollary}

\begin{proof}[of \autoref{lem:layers_bipartite}]
Suppose no edge joins two nodes in the same layer.
Then we can 2-color the graph, e.g. red for odd layers and blue for even layers.
So \ref{lem:layers_bipartite}.\ref{item:no_edge_same_layer_bipartite} holds.

Suppose \((x, y)\) is an edge joining two nodes in the same layer \(L_j\).
Let 
\[
z = \operatorname{lca}(x, y)
\]
be the lowest level common ancestor in the \nameref{alg:bfs} tree and let \(L_i\) be the layer of \(z\).

\[
\begin{tikzpicture}[every node/.style={circle,draw,minimum size=3.4mm,inner sep=0pt},>=stealth, scale=0.55, font=\footnotesize]
  % layers
  \usetikzlibrary{shapes.symbols}
  \definecolor{blob}{RGB}{190,190,190}
  \tikzset{
    fixedcloud/.style={
      cloud, cloud puffs=16, cloud puff arc=120,
      minimum width=3.4cm, minimum height=1cm,
      inner sep=0pt, draw=none, fill=blob
    }
  }
  \coordinate (Li) at (0, 3);
  \node[fixedcloud] (Li_layer) at (Li) {};
  \coordinate (Lj) at (0, 0);
  \node[fixedcloud] (Lj_layer) at (Lj) {};

  % nodes
  \node (s) at (0,6) {$s$};
  \node (z) at (1,3) {$z$};
  \node (x) at (0,0) {$x$};
  \node (y) at (2,0) {$y$};

  % Edges
  \draw[rounded corners=2pt] (s) 
    -- ($($(s)!0.45!(z)$) + ($(s)!0.05!90:(z)$) - (s)$)
    -- ($($(s)!0.55!(z)$) + ($(s)!0.05!270:(z)$)  - (s)$)
    -- (z);
  \draw (s)  -- ++(-0.1,-0.8);
  \draw (s)  -- ++(-1,-0.7);
\draw[rounded corners=2pt] (x)
  -- ($($(x)!0.5!(z)$) + ($(x)!0.05!270:(z)$) - (x)$)
  -- ($($(x)!0.6!(z)$) + ($(x)!0.05!90:(z)$)  - (x)$)
  -- (z);
  \draw[rounded corners=2pt] (y) 
    -- ($($(y)!0.4!(z)$) + ($(y)!0.05!90:(z)$) - (y)$)
    -- ($($(y)!0.5!(z)$) + ($(y)!0.05!270:(z)$) - (y)$)
    -- (z);
  \draw[red] (x) -- (y);

  % Layer labels
  \node[draw=none,fill=none,left=8mm of Li] {$L_i$};
  \node[draw=none,fill=none,left=8mm of Lj] {$L_j$};
\end{tikzpicture}
\]

Consider a cycle that takes edge \((x, y)\), then the path in the \nameref{alg:bfs} tree from \(y\) to \(z\), and then the path in the \nameref{alg:bfs} tree from \(z\) to \(x\).
The length of this cycle is
\[
\underbrace{1}_{(x, y)} + \underbrace{(j - i)}_{y \leadsto z} + \underbrace{(j - i)}_{z \leadsto x} = 1 + 2(j - i) 
\]
which is odd.
So \ref{lem:layers_bipartite}.\ref{item:edge_same_layer_odd_cycle} holds.
\end{proof}

\begin{proof}[`\(\Leftarrow\)' of \autoref{thm:bipartite_odd_cycle}]
\autoref{lem:layers_bipartite}.
\end{proof}





% \clearpage

\pagebreak[3]

\subsection{Depth-First Search}
\label{sec:dfs}

\nameref{alg:dfs} is \hl{recursive} process.

we maintain four auxiliary arrays:
\begin{itemize}
\item \(\attrib{u}{color}\): undiscovered (\(\white\)), discovered (\(\gray\)), finished (\(\black\)).
\item \(\attribnormal{u}{d}\) (discovery time): time when \nameref{alg:dfs} started at vertex \(u\).
\item \(\attribnormal{u}{f}\) (finish time): time when \(u\) is finished processing (all neighbors have been visited).
\item \(\attribnormal{u}{\pi}\) (predecessor pointer): vertex which discovered \(u\). the edge \((\attribnormal{u}{\pi}, u)\) is a tree edge in the \nameref{alg:dfs} recursion tree.
\end{itemize}

% \medskip

\textcolor{AccentBlue}{Running time of \nameref{alg:dfs}}:

the call \Call{DFSvisit}{} is made exactly once for each vertex

ignoring time for recursive calls, each vertex \(u\) is processed in \(O(1 + \deg(u))\) time

Total time:
\[
\begin{aligned}
  T(n) &= n + \sum_{u \in V} (\deg(u) + 1) = n + \underbrace{\sum_{u \in V} \deg(u)}_{=2m} + \underbrace{\sum_{u \in V} 1}_{=n} \\[-12pt]
  &= 2n + 2m \in O(n + m)
\end{aligned}
\]


Since \nameref{alg:dfs} always explores all vertices, we define the predecessor subgraph as \(G_\pi=(V,E_\pi)\), where
\[
E_\pi = \{(v.\pi,v)\mid v\in V \AND v.\pi \neq \nil\}
\]
It is a \nameref{alg:dfs} \emph{forest}, comprising several {\nameref{alg:dfs} trees}.


\begin{algorithm}[h]
\caption{DFS}\label{alg:dfs}
\begin{algorithmic}[1]
\Function{DFS}{$G$}
  \ForAll{$u\in V$}
    \State $\attrib{u}{color},\attrib{u}{\pi}\gets \white,\nil$ \Comment{initialization}
  \EndFor
  \State $T\gets 0$ 
  \ForAll{$u\in V$} \label{alg:dfs:for_all_vertices}
    \If{$\attrib{u}{color}=\white$} \Comment{undiscovered vertex?}
      \State \Call{DFSvisit}{$u$} \Comment{...start a new search here} \label{alg:dfs:new_tree}
    \EndIf
  \EndFor
\EndFunction

\Function{DFSvisit}{$u$} \Comment{new \nameref{alg:dfs} search at \(u\)} \label{alg:dfs:dfsvisit}
  \State $\attrib{u}{color}\gets\gray$ \Comment{\(u\) has been discovered} \label{alg:dfs:discover}
  \State $T\gets T+1$
  \State $\attribnormal{u}{d}\gets T$ \label{alg:dfs:discover_time}
  \ForAll{$v\in\Gamma(u)$}  \label{alg:dfs:explore_edges}
    \If{$\attrib{v}{color}=\white$} \Comment{first time we have seen \(v\)?}
      \State $\attrib{v}{\pi}\gets u$
      \State \Call{DFSvisit}{$v$} \Comment{...visit it} \label{alg:dfs:recursive_call}
    \EndIf
  \EndFor
  \State $\attrib{u}{color}\gets\black$ \Comment{we are done with \(u\)} \label{alg:dfs:finish}
  \State $T\gets T+1$
  \State $\attribnormal{u}{f}\gets T$ \label{alg:dfs:finish_time}
\EndFunction
\end{algorithmic}
\end{algorithm}


\subsubsection{Edge Classification}

for undirected graphs, there are two types of edges: 
\begin{itemize}
\item \emph{tree edges} 
\item \emph{back edges}: \(\attribnormal{v}{d} < \attribnormal{u}{d}\) AND \(\attribnormal{u}{\pi} \neq v\) 
\end{itemize}
Observation: for each edge in an undirected graph, either \(u\) is a proper ancestor (\(\attribnormal{u}{d} < \attribnormal{v}{d}\)) or a proper descendant (\(\attribnormal{u}{d} > \attribnormal{v}{d}\)) of \(v\) (by \autoref{lem:parenthesis_dfs}).

{for directed graphs, there are three types of non-tree edges:}
% Each edge \((u,v) \in E\) is classified as one of the following:
\nopagebreak
\begin{itemize}
  %\item \emph{tree edges} are edges \((u,v) \in E_\pi\) in the \nameref{alg:dfs} forest \(G_\pi\). \((u,v)\) is a tree edge if \(v\) was first discovered by exploring the edge \((u,v)\) in Line~\ref{alg:dfs:explore_edges}.
  \item \emph{back edges} \((u,v)\) where  \(v\) is an ancestor (not necessarily proper) of \(u\) 
  \item \emph{forward edges} \((u,v)\) where \(v\) is a proper descendant of \(u\)
  \item \emph{cross edges} \((u,v)\) where \(u\) and \(v\) are neither ancestor nor descendant of one another
\end{itemize}

\subsubsection{Properties of the \nameref{alg:dfs} Forest \(G_\pi\)} \label{sec:dfs_properties}

  % \(v\) is a descendant of \(u\) \(\Longleftrightarrow\) \(v\) is discovered during the time in which \(u\) is gray
  \nameref{alg:dfs} imposes a nesting structure on the discovery-finish time intervals (no overlapping intervals!):
  \begin{lemma}[parenthesis]\label{lem:parenthesis_dfs}
     for any two vertices \(u_1\) and \(u_2\):
    \begin{itemize}
      \item \([\attribnormal{u_1}{d}, \attribnormal{u_1}{f}] \subset [\attribnormal{u_2}{d}, \attribnormal{u_2}{f}]\) \(\Longleftrightarrow\) \(u_1\) is a descendant of \(u_2\)
      \item \([\attribnormal{u_1}{d}, \attribnormal{u_1}{f}] \supset [\attribnormal{u_2}{d}, \attribnormal{u_2}{f}]\) \(\Longleftrightarrow\) \(u_2\) is a descendant of \(u_1\)
      \item \([\attribnormal{u_1}{d}, \attribnormal{u_1}{f}] \cap [\attribnormal{u_2}{d}, \attribnormal{u_2}{f}] = \emptyset\) \(\Longleftrightarrow\) neither is a descendant of the other
      \qedhere
    \end{itemize}
  \end{lemma}
  % \(v\) is a descendant of \(u\) \(\Longleftrightarrow\) at the time \attribnormal{u}{d} that the search discovers \(u\), there is a path from \(u\) to \(v\) consisting entirely of white vertices



\begin{lemma}\label{lem:dfs-edges-finish-times}
Let \((u, v)\) be any graph edge.
If it is a tree, forward, or cross edge, then \(\attribnormal{u}{f} > \attribnormal{v}{f}\).
If it is a back edge, then \(\attribnormal{u}{f} \le \attribnormal{v}{f}\).
\end{lemma}
\begin{proof}
\autoref{lem:parenthesis_dfs}.
\end{proof}

\begin{lemma}\label{lem:digraph-cycle-back-edge}
\hl[2]{A directed graph has a cycle iff the \nameref{alg:dfs} forest has a back edge}.
\end{lemma}
\begin{proof}
`\(\Leftarrow\)' easy: If there is a back edge then there is a cycle.
`\(\Rightarrow\)': 
Consider a cycle. 
In a cycle there must be an edge \((u,v)\) with \(\attribnormal{u}{f} < \attribnormal{v}{f}\).
By \autoref{lem:dfs-edges-finish-times}, \((u,v)\) must be a back edge.
\end{proof}


When during an \nameref{alg:dfs} an edge \((u,v)\) is first explored, the color of \(v\) yields information about the edge:
\begin{itemize}%[before={\parskip=0pt},nosep]
  \item if \(v\) is \(\white\), then \((u,v)\) is a tree edge
  \item if \(v\) is \(\gray\), then \((u,v)\) is a back edge
  \item if \(v\) is \(\black\), then \((u,v)\) is a
\begin{itemize}%[before={\parskip=0pt},nosep]
    \item forward edge if \(\attribnormal{u}{d} < \attribnormal{v}{d}\)
    \item cross edge if \(\attribnormal{u}{d} > \attribnormal{v}{d}\)
\end{itemize}
\end{itemize}
Observe that the gray vertices always form a linear chain of descendants corresponding to the stack of active \textsc{DFSvisit} invocations.

\begin{example}[Edge Classification]
\[
\begin{tikzpicture}[
  scale = 0.5,
  >=Stealth,
  vertex/.style={draw, ellipse, minimum width=1cm, minimum height=0.6cm, line width=0.6pt, inner sep=0pt, font=\footnotesize},
  tree/.style   ={black, ->},
  fwd/.style    ={densely dotted, red, ->},
  back/.style   ={dashed, blue, ->},
  cross/.style  ={densely dotted, green!60!black, ->}
]

%--- Nodes ----------------------------------------------------------
\node[vertex, label=left:$a$] (a) at (0, 6) {1/10};
\node[vertex, label=left:$b$ ] (b) at (-3, 3) {2/5};
\node[vertex, label=left:$c$ ] (c) at (-3, 0) {3/4};
\node[vertex, label=above:$f$] (f) at (3, 3) {6/9};
\node[vertex, label=right:$g$] (g) at (3, 0) {7/8};
\node[vertex, label=right:$d$] (d) at (7, 6) {11/14};
\node[vertex, label=right:$e$] (e) at (7, 3) {12/13};

%--- Tree edges (solid black) --------------------------------------
\draw[tree] (a) -- (b);
\draw[tree] (b) -- (c);
\draw[tree] (a) -- (f);
\draw[tree] (f) -- (g);
\draw[tree] (d) -- (e);

%--- Forward edge(s) (red dotted) ----------------------------------
\draw[fwd] (a) -- (c);

%--- Back edge(s) (blue dashed) ------------------------------------
\draw[back] (g) -- (a);

%--- Cross edges (green dotted) ------------------------------------
\draw[cross] (d) -- (a);
\draw[cross] (e) -- (f);
\draw[cross] (g) -- (c);

%--- Legend ---------------------------------------------------------
\begin{scope}[shift={(6,1.5)}]
  \draw[tree]  (0,0) -- +(1.2,0); \node[anchor=west, font=\footnotesize] at (1.4,0) {tree};
  \draw[fwd]   (0,-0.6) -- +(1.2,0); \node[anchor=west, font=\footnotesize] at (1.4,-0.6) {forward};
  \draw[back]  (0,-1.2) -- +(1.2,0); \node[anchor=west, font=\footnotesize] at (1.4,-1.2) {back};
  \draw[cross] (0,-1.8) -- +(1.2,0); \node[anchor=west, font=\footnotesize] at (1.4,-1.8) {cross};
\end{scope}

\end{tikzpicture}
\qedhere
\]
\end{example}



\subsection{Connectivity in Directed Graphs} \label{sec:connectivity_digraphs}

\begin{definition}[strongly connected]\label{def:strongly_connected}
A directed graph is \emph{strongly connected} if for each \(u\) and \(v\), there is a path from \(u\) to \(v\) and a path from \(v\) to \(u\).
\end{definition}

\begin{lemma}\label{lem:source-strongly_connected}
Let \(s\) be any node. 
\(G\) is strongly connected iff every node is reachable from \(s\) AND \(s\) is reachable from every node.
\end{lemma}

\begin{proof}
`\(\Rightarrow\)' follows from definition.
`\(\Leftarrow\)':
Consider two nodes \(u\) and \(v\).
\[
\begin{tikzpicture}[>=Stealth, scale=0.4, font=\footnotesize]
\useasboundingbox (-3,-2) rectangle (5,2);
% background blob
\fill[black!20] plot[smooth cycle,tension=0.7] coordinates {(-3.4,1.5) (-2.0,2.3) (-0.4,2.0) (1.6,2.2) (3.6,1.6) (4.5,0.7) (5.1,-0.2) (4.6,-1.3) (3.2,-2.1) (1.3,-2.3) (-0.6,-2.0) (-2.2,-1.7) (-3.2,-0.8)};
% nodes
\tikzset{state/.style={circle,draw,fill=white,minimum size=9pt,inner sep=0pt}}
\node[state] (s) at (-1.5,0.0) {\(s\)};
\node[state] (u) at (3.3,0.5)  {\(u\)};
\node[state] (v) at (2.4,-1.5) {\(v\)};
% paths
\draw[<-, rounded corners=3pt] (s) -- (-0.4, 1.2) -- (0.8, 1.4) -- (1.9, 1.1) -- (u); % u -> s
\draw[->, rounded corners=3pt] (s) -- (-0.2, 0.2) -- (0.9, 0.4) -- (2.0, 0.2) -- (u); % s -> u
\draw[<-, rounded corners=3pt] (s) -- (-0.6,-1.3) -- (0.9,-1.9) -- (1.7,-1.8) -- (v); % v -> s
\draw[->, rounded corners=3pt] (s) -- (-0.5,-0.8) -- (0.5,-0.2) -- (1.5,-0.6) -- (v); % s -> v
% \draw (current bounding box.north east) -- (current bounding box.north west) -- (current bounding box.south west) -- (current bounding box.south east) -- cycle; % debugging
\end{tikzpicture} 
\]
Path from \(u\) to \(v\): Concatenate \(u \leadsto s\) and \(s \leadsto v\).
Path from \(v\) to \(u\): Concatenate \(v \leadsto s\) and \(s \leadsto u\).
\end{proof}

\begin{theorem}\label{thm:strongly_connected_test}
We can test if \(G\) is strongly connected in \(O(n + m)\) time.
\end{theorem}
\begin{proof}
  Pick any node \(s\).
  Run \nameref{alg:bfs} (or \nameref{alg:dfs}) from \(s\) in \(G\).
  Run \nameref{alg:bfs} (or \nameref{alg:dfs}) from \(s\) in \(G^{\top}\).
  If all nodes are reached in both searches, then \(G\) is strongly connected (by \autoref{lem:source-strongly_connected}).
\end{proof}

\begin{theorem}[Tarjan, 1972]\label{thm:tarjan_strongly_connected_components}
The strongly connected components of a directed graph can be computed in \(O(n + m)\)
\end{theorem}






\subsection{Topological Sort}\label{sec:topological_sort}
\begin{definition}[topological order]\label{def:topological_order}
  A \emph{topological order} of a directed \(G=(V,E)\) is a ordering of its vertices \(v_1, \ldots, v_n\) such that for every edge \((v_i, v_j)\), we have \(i < j\).
\end{definition}
\begin{lemma}\label{lem:dag_topological_order}
If \(G\) has a topological order, then \(G\) is a DAG.
\end{lemma}

\begin{proof}[Contradiction]
  Suppose \(G\) has a topological order \(v_1, \ldots, v_n\) and also has a directed cycle \(C\).
  Let \(v_i\) be the lowest-indexed node in \(C\) and \(v_j\) be the node just before \(v_i\) in \(C\):
  \[
\begin{tikzpicture}[
  font=\footnotesize,
  scale=0.6,
  vertex/.style={circle,draw,fill=gray!40,minimum size=14pt,inner sep=0}
]
\node[vertex] (v1) at (0, 0) {$v_1$};
\node[] at (1, 0) {$\hdots$};
\node[vertex] (vi) at (2, 0) {$v_i$};
\node[] at (3, 0) {$\hdots$};
\node[vertex] (ui) at (4, 0) {};
\node[] at (5, 0) {$\hdots$};
\node[vertex] (uj) at (6, 0) {};
\node[] at (7, 0) {$\hdots$};
\node[vertex] (vj) at (8, 0) {$v_j$};
\node[] at (9, 0) {$\hdots$};
\node[vertex] (vn) at (10, 0) {$v_n$};

\def\Hbig{0.6}
\draw[blue,->] (vj.north) -- ($(vj.north)+(0,\Hbig)$) -- ($(vi.north)+(0,\Hbig)$) -- (vi.north);

\draw[blue,->] (vi) to[out=-45, in=225] (ui);
\draw[ dotted, blue,->] (ui) to[out=-45, in=225] (uj);
\draw[blue,->] (uj) to[out=-45, in=225] (vj);

\node[blue, anchor=south east] at ($(vj.north)+(0,\Hbig)$) {the directed cycle $C$};
\end{tikzpicture}
\]
Then \((v_j, v_i)\) is an edge with \(j > i\), by our choice of \(v_i\).
This contradicts the definition of a \nameref{def:topological_order}, which requires \(i < j\) for every edge \((v_i, v_j)\).
\end{proof}

\begin{lemma}\label{lem:dag_has_source}
If \(G\) is a DAG, then \(G\) has a node with no incoming edges (a \emph{source}).
\end{lemma}
\begin{proof}[Contradiction]
  Suppose \(G\) is a DAG and every node has at least one incoming edge.
  Pick any node \(v\), and begin following edges backwards from \(v\).
  Since every node has an incoming edge, we can repeat this process and if \(G\) is finite, we must eventually revisit a node, say \(w\).
  Let \(C\) denote the sequence of nodes encountered between successive visits to \(w\).
  Then \(C\) is a cycle.
\end{proof}
\begin{caution}
The converse of \autoref{lem:dag_has_source} does not hold!
\end{caution}

\begin{lemma}\label{lem:dag_has_topological_order}
If \(G\) is a DAG, then \(G\) has a topological order.
\end{lemma}
\begin{proof}[Induction]\label{proof:dag_has_topological_order}
The inductive proof spells out a \hyperref[alg:topological_sort_recursive]{recursive algorithm}.
\begin{enumerate}[partopsep=0em, label=(\roman*)]
\item Base case:
For \(n=1\), we just have one node, which trivially has a topological order.~\textcolor{Green}{\ding{52}}
\item Induction hypothesis:
Assume a DAG of \(n \ge 2\) nodes has a topological order. 
\label{proof:dag_has_topological_order:induction_hypothesis}
\item Induction step:
Let \(G\) be a DAG with \(n+1\) nodes.
By \autoref{lem:dag_has_source}, \(G\) has a source node, say \(v\).
\(G \setminus \{v\}\) is a DAG (removing \(v\) cannot create cycles) with \(n\) nodes.
By~\ref{proof:dag_has_topological_order:induction_hypothesis}, \(G \setminus \{v\}\) has a TO.
Append \(v\) to the front of this TO.
This is a TO for \(G\) since \(v\) has no incoming edges.~\textcolor{Green}{\ding{52}}
\qedhere
\end{enumerate}
\end{proof}
\begin{algorithm}[h]
\caption{Topological Sort (Recursive)}\label{alg:topological_sort_recursive}
\begin{algorithmic}[1]
\Function{TopologicalSort}{$G$}
  \If{\(G\) is empty}
    \State \Return empty list \Comment{recursion bottoms out here}
  \EndIf
  \State Find a node \(v\) with no incoming edges
  \State Remove \(v\) from \(G\)
  \State \(L \gets [v]\) concatenated  with \Call{TopologicalSort}{\(G \setminus \{v\}\)}
  \State \Return \(L\)
\EndFunction
\end{algorithmic}
\end{algorithm}

\begin{theorem}
\autoref{alg:topological_sort_recursive} finds a TO in \hl[2]{\(O(n + m)\)} time.
\end{theorem}
\begin{proof}
Maintain the following information:
\begin{itemize}
  \item \(\texttt{in\_degree}[w]\): number of incoming edges to node \(w\)
  \item \(S\): set of nodes with no incoming edges
\end{itemize}
Initialization: \(O(n + m)\), one scan through \(G\) to find out the in-degrees of all nodes and populate \(S\).

When removing a node \(v\) from \(S\), we decrement \(\texttt{in\_degree}[w]\) for each outgoing edge \((v, w)\).
If \(\texttt{in\_degree}[w]\) hits zero, we add \(w\) to \(S\).
This is \(O(1)\) work per edge incident to \(v\).
\end{proof}

\autoref{alg:topological_sort} is an alternative way to find a TO of a DAG using \nameref{alg:dfs}.

\begin{algorithm}[h]
\caption{Topological Sort (DFS)}\label{alg:topological_sort}
\begin{algorithmic}[1]
\Function{topSort}{$G$}
  \ForAll{$u\in V$}
    \State $\attrib{u}{color} \gets \white$ \Comment{initialization}
  \EndFor
  \State $S \gets \emptyset$ \Comment{empty stack}
  \ForAll{$u\in V$}
    \If{$\attrib{u}{color}=\white$}
      \State \Call{topVisit}{$u$}
    \EndIf
  \EndFor
  \While{$S \neq \emptyset$} \Comment{while stack not empty}
    \State output \Call{pop}{$S$} \Comment{pop stack for final TO}
  \EndWhile
\EndFunction

\Function{topVisit}{$u$}
  \State $\attrib{u}{color}\gets\gray$ \Comment{mark \(u\) visited} 
  \ForAll{$v\in\Gamma(u)$}
    \If{$\attrib{v}{color}=\white$}
      \State \Call{topVisit}{$v$} 
    \EndIf
  \EndFor
  \State push \(u\) onto \(S\) \Comment{last to finish is top of stack}
\EndFunction
\end{algorithmic}
\end{algorithm}

In \autoref{lem:dfs-edges-finish-times} we have seen that if \((u,v)\) is a tree/forward/cross edge, then \(\attribnormal{u}{f} > \attribnormal{v}{f}\).
Since a DAG has no back edges (it is acyclic), we have \(\attribnormal{u}{f} > \attribnormal{v}{f}\) for every directed edge \((u,v)\).
Thus, ordering vertices in decreasing order of finish times when running \autoref{alg:dfs} yields a TO.
Hence, \autoref{alg:topological_sort} is correct.

\begin{remark}
Vertex with highest finish time in \nameref{alg:dfs} is a source of the DAG.
\end{remark}




\subsection{Applications of DFS}

\subsubsection{Longest Path in a DAG}\label{sec:longest_path_dag}
Given a DAG \(G=(V,E)\), where each vertex \(u\) is thought of as a task that takes \(\attrib{u}{time}\) time units to complete,
and each edge \((u,v)\) represents a precedence constraint (task \(u\) must be completed before task \(v\) can begin),
we want to know the minimum time required to complete all tasks, assuming maximum parallelism.
This is equivalent to computing the maximum cost of any path in the DAG, where cost is defined as the sum of \(\attrib{u}{time}\) values of the vertices along the path.

We can solve this in \(O(n + m)\) time through \nameref{alg:dfs}.
The idea is to associate each vertex \(u\) with the \hl{maximum cost of any path that starts at this vertex}, denoted by \(\attrib{u}{cost}\).
We let \(\texttt{max\_cost}\) be the maximum cost of all \(u\)'s neighbors.% and set \(\attrib{u}{cost} = \texttt{max\_cost} + \attrib{u}{time}\).

\begin{algorithm}[h]
\caption{Longest Path in a DAG}\label{alg:longest_path_dag}
\begin{algorithmic}[1]
\Function{LongPathVisit}{$u$}
  \State $\attrib{u}{color}\gets\gray$ \Comment{mark \(u\) visited} 
  \State $\texttt{max\_cost} \gets 0$ \Comment{initialize max outgoing cost}
  \ForAll{$v\in\Gamma(u)$}
    \If{$\attrib{v}{color}=\white$}
      \State \Call{LongPathVisit}{$v$} \Comment{process \(v\) if undiscovered}
    \EndIf
    \State $\texttt{max\_cost} \gets \max (\texttt{max\_cost}, \attrib{v}{cost})$ \Comment{update maximum cost}
  \EndFor
  \State \(\attrib{u}{cost} \gets \texttt{max\_cost} + \attrib{u}{time}\) \Comment{save final cost}
\EndFunction
\end{algorithmic}
\end{algorithm}

Because the graph is acyclic, every edge \((u,v)\) goes from \(u\) to a vertex \(v\) whose finish time is greater than \(u\)'s (by \autoref{lem:dfs-edges-finish-times}).
Therefore, \(\attrib{v}{cost}\) is fully defined before it is accessed by \(u\).
The longest path in the entire DAG is the largest value of \(\attrib{u}{cost}\) among all vertices \(u\).



% \clearpage

\subsubsection{Biconnected Components}\label{sec:biconnected_components}
Let \(G=(V,E)\) be a connected, undirected graph.
\begin{definition}[cut vertex]\label{def:cut_vertex}
any vertex whose removal (along with incident edges) disconnects the graph
\end{definition}
\begin{definition}[bridge]\label{def:bridge}
any edge whose removal disconnects the graph
\end{definition}
\begin{definition}[biconnected]\label{def:biconnected}
a graph is biconnected if it contains no cut vertices
\end{definition}
a graph is \(k\)-connected if it remains connected after removing any \(k-1\) vertices.
\begin{definition}[biconnected component]\label{def:biconnected_component}
partition \(E\) into maximal subgraphs that are biconnected.
\end{definition}
Biconnected components are the equivalence classes of the co-cyclicity relation.
Two edges \(e_1, e_2 \in E\) are co-cyclic if there is a simple cycle that contains both, or if \(e_1 = e_2\).

\(G\) is biconnected iff it consists of one biconnected component.

\begin{lemma}\label{lem:cut_vertex_root}
 The root \(r\) of the \nameref{alg:dfs} tree is a cut vertex iff it has \(\ge 2\) children.
\end{lemma}
\begin{lemma}\label{lem:cut_vertex_internal}
An internal (i.e. not a leaf and not the root) vertex \(u\) of the \nameref{alg:dfs} tree is a cut vertex 
iff there exists a subtree rooted at a child \(v\) of \(u\) such that 
there is no back edge from any vertex in this subtree to a proper ancestor of \(u\).
\end{lemma}

We can exploit the structure of the \nameref{alg:dfs} tree to determine cut vertices.

Keeping track of all back edges from each subtree is expensive.

Instead we can just keep track of one back edge from each subtree, the one that goes highest (closest to the root) in the tree.

Observe: discovery times decrease as we go up the tree.

Idea: keep track of back edge \((x, w)\) with smallest discovery time \(\attribnormal{w}{d}\)


\[
\begin{tikzpicture}[
  % line cap=round,
  % line join=round,
  >=stealth, 
  font=\footnotesize,
  scale=0.8,
]
\begin{scope} % left figure
  % trapezoid-like block 
  \coordinate (A) at (-1.2,0);
  \coordinate (B) at ( 1.2,0);
  \coordinate (C) at ( 0.4,2);
  \coordinate (D) at (-0.4,2);
  \draw[thick] (A)--(B)--(C)--(D)--cycle;

  % vertices v, u, w
  \coordinate (vcoor) at (0,2);
  \coordinate (ucoor) at (0.3,2.6);
  \coordinate (wcoor) at ($(ucoor)!2!180:(vcoor)$);
  \node[circle,draw,thick,fill=white,inner sep=0pt, minimum size=10] (v) at (vcoor) {$v$};
  \node[circle,draw,thick,fill=white,inner sep=0pt, minimum size=10] (u) at (ucoor) {$u$};
  \node[circle,draw,thick,fill=white,inner sep=0pt, minimum size=10] (w) at (wcoor) {$w$};

  % tree edges
  \draw[thick] (v)--(u);
  \draw[thick] (u)--($(u)!0.35!0:(w)$);
  \draw[thick, densely dotted] ($(u)!0.35!0:(w)$)--($(u)!0.65!0:(w)$);
  \draw[thick] ($(u)!0.65!0:(w)$)--(w);

  % dotted back edges
  \draw[thick, densely dotted, blue] (-0.3,1.5) to[bend left=40] (u);
  \draw[thick, densely dotted, blue] (-0.7,0.25) to[bend left=55] (-0.35,1.3);
  \draw[thick, densely dotted, blue] (0.7, 0.25) to[bend right=55] (v);
  \draw[thick, densely dotted, blue] (-0.5,1.0) to[bend left=45] (w);

  % annotations
  \node[right=0.1 of u] {$u$ is \textbf{not} a cut vertex};
  \node[right=0.3 of v] {$\attrib{v}{low} = \attribnormal{w}{d} < \attribnormal{u}{d}$};
\end{scope}

\begin{scope}[xshift=5cm] % mid figure (cut vertex)
  % trapezoid-like block 
  \coordinate (A) at (-1.2,0);
  \coordinate (B) at ( 1.2,0);
  \coordinate (C) at ( 0.4,2);
  \coordinate (D) at (-0.4,2);
  \draw[thick] (A)--(B)--(C)--(D)--cycle;

  % vertices v and u on the side arm
  \coordinate (vcoor) at (0,2);
  \coordinate (ucoor) at (0.3,2.6);
  \node[circle,draw,thick,fill=white,inner sep=0pt, minimum size=10] (v) at (vcoor) {$v$};
  \node[circle,draw,thick,fill=white,inner sep=0pt, minimum size=10] (u) at (ucoor) {$u$};

  % the tree edge (v--u) and its upward continuation
  \draw[thick] (v)--(u);
  \draw[thick] (u)--($(u)!0.6!180:(v)$);
  \draw[thick,densely dotted] ($(u)!0.6!180:(v)$)--($(u)!1.2!180:(v)$);

  % dotted back edges / paths 
  \draw[thick, densely dotted, blue] (-0.3,1.5) to[bend left=45] (u);
  \draw[thick, densely dotted, blue] (-0.7,0.25) to[bend left=55] (-0.35,1.3);
  \draw[thick, densely dotted, blue] (0.7, 0.25) to[bend right=55] (v);

  % annotations
  \node[right=0.1 of u] {$u$ is a cut vertex};
  \node[right=0.3 of v] {$\attrib{v}{low} \ge \attribnormal{u}{d}$};
\end{scope}

\begin{scope}[xshift=10cm] % right figure (bridge)
  % trapezoid-like block 
  \coordinate (A) at (-1.2,0);
  \coordinate (B) at ( 1.2,0);
  \coordinate (C) at ( 0.4,2);
  \coordinate (D) at (-0.4,2);
  \draw[thick] (A)--(B)--(C)--(D)--cycle;

  % vertices v and u on the side arm
  \coordinate (vcoor) at (0,2);
  \coordinate (ucoor) at (0.3,2.6);
  \node[circle,draw,thick,fill=white,inner sep=0pt, minimum size=10] (v) at (vcoor) {$v$};
  \node[circle,draw,thick,fill=white,inner sep=0pt, minimum size=10] (u) at (ucoor) {$u$};

  % the tree edge (v--u) and its upward continuation
  \draw[thick] (v)--(u);
  \draw[thick] (u)--($(u)!0.6!180:(v)$);
  \draw[thick,densely dotted] ($(u)!0.6!180:(v)$)--($(u)!1.2!180:(v)$);

  % dotted back edges / paths 
  \draw[thick, densely dotted, blue] (-0.5,1.0) to[bend left=45] (v);
  \draw[thick, densely dotted, blue] (-0.7,0.25) to[bend left=55] (-0.35,1.3);
  \draw[thick, densely dotted, blue] (0.7, 0.25) to[bend right=55] (v);

  % annotations
  \node[right=0.1 of u] {$(u,v)$ is a bridge};
  \node[right=0.3 of v] {$\attrib{v}{low} > \attribnormal{u}{d}$};
\end{scope}
\end{tikzpicture}
\]

\begin{definition}[low value]\label{def:low_value}
highest up we can reach from the subtree rooted at \(v\) (via back edges):
\vspace{-1em}
\[
  % {\color{Red}
  % % \setlength{\fboxrule}{1pt}
  % \boxed{ 
  % \color{black}
\attrib{v}{low}
:= \min\biggl(
\{\attribnormal{v}{d}\} \cup
\Bigl\{\attribnormal{w}{d} \ \Bigm|\ 
\begin{aligned}
& (x,w)\ \text{is a \textcolor{blue}{back edge} for some}\\[-4pt]
& \text{(nonproper) descendant } x \text{ of } v
\end{aligned}
\Bigr\}
\biggr)
\]
where ``nonproper'' means that \(x\) can be \(v\) itself.
\end{definition}

Once \(\attrib{v}{low}\) is computed for all vertices \(v\), we can test wether a given non-root vertex \(u\) is a cut vertex by \autoref{lem:cut_vertex_internal} as follows:
\(u\) is a \hl{cut vertex} iff it has a child \(v\) in the \nameref{alg:dfs} tree such that \(\attrib{v}{low} \ge \attribnormal{u}{d}\).
Moreover, \((u,v)\) is a \hl{bridge} iff it is a tree edge (i.e. \(u = \attrib{v}{\pi}\)) and \(\attrib{v}{low} > \attribnormal{u}{d}\).


\begin{algorithm}
\caption{Modification of \hyperref[alg:dfs:dfsvisit]{\textsc{DFSvisit}} for \attribute{low} computation}\label{alg:low_computation}
\begin{algorithmic}[1]
\Function{LowComputation}{$u$} 
  \State $\attrib{u}{color}\gets\gray$ \Comment{\(u\) has been discovered} \label{alg:low_computation:discover}
  \State $T\gets T+1$
  \State $\attrib{u}{low} \gets \attribnormal{u}{d} \gets T$ \Comment{set discovery time and init low} \label{alg:low_computation:discover_time}
  \ForAll{$v\in\Gamma(u)$}  \label{alg:low_computation:explore_edges}
    \If{$\attrib{v}{color}=\white$} \Comment{\((u,v)\) is a tree edge}
      \State $\attrib{v}{\pi}\gets u$ \Comment{\(v\)'s parent is \(u\)}
      \State \Call{LowComputation}{$v$} \label{alg:low_computation:recursive_call}
      \State $\attrib{u}{low} \gets \min(\attrib{u}{low}, \attrib{v}{low})$ \Comment{update \(\attrib{u}{low}\) (propagate \(\attrib{v}{low}\) up)} \label{alg:low_computation:update_low_tree}
    \ElsIf{$v \neq \attrib{u}{\pi}$} \Comment{\((u,v)\) is a back edge}
      \State $\attrib{u}{low} \gets \min(\attrib{u}{low}, \attribnormal{v}{d})$ \Comment{update \(\attrib{u}{low}\) (consider back edge to \(v\))} \label{alg:low_computation:update_low_back}
    \EndIf
  \EndFor
\EndFunction
\end{algorithmic}
\end{algorithm}


\begin{algorithm}
\caption{Modification of \hyperref[alg:dfs:dfsvisit]{\textsc{DFSvisit}} to find cut vertices}\label{alg:find_cut_vertices}
\begin{algorithmic}[1]
\Function{findCutVertices}{$u$} 
  \State $\attrib{u}{color}\gets\gray$ \Comment{\(u\) has been discovered} \label{alg:find_cut_vertices:discover}
  \State $T\gets T+1$
  \State $\attrib{u}{low} \gets \attribnormal{u}{d} \gets T$ \Comment{set discovery time and init low} \label{alg:find_cut_vertices:discover_time}
  \ForAll{$v\in\Gamma(u)$}  \label{alg:find_cut_vertices:explore_edges}
    \If{$\attrib{v}{color}=\white$} \Comment{\((u,v)\) is a tree edge}
      \State $\attrib{v}{\pi}\gets u$ \Comment{\(v\)'s parent is \(u\)}
      \State \Call{findCutVertices}{$v$} \label{alg:find_cut_vertices:recursive_call}
      \State $\attrib{u}{low} \gets \min(\attrib{u}{low}, \attrib{v}{low})$ \Comment{update \(\attrib{u}{low}\) (propagate \(\attrib{v}{low}\) up)} \label{alg:find_cut_vertices:update_low_tree}
      \If{$\attrib{u}{\pi} = \nil$} \Comment{\(u\) is root: apply \autoref{lem:cut_vertex_root}}
        \If{this is \(u\)'s second child} 
          \State label \(u\) as a cut vertex 
        \EndIf
      \ElsIf{$\attrib{v}{low} \ge \attribnormal{u}{d}$} \Comment{\(u\) is internal: apply \autoref{lem:cut_vertex_internal}}
        \State label \(u\) as a cut vertex
      \EndIf
    \ElsIf{$v \neq \attrib{u}{\pi}$} \Comment{\((u,v)\) is a back edge}
      \State $\attrib{u}{low} \gets \min(\attrib{u}{low}, \attribnormal{v}{d})$ \Comment{update \(\attrib{u}{low}\) (consider back edge to \(v\))} \label{alg:find_cut_vertices:update_low_back}
    \EndIf
  \EndFor
\EndFunction
\end{algorithmic}
\end{algorithm}

\begin{algorithm}
\caption{Modification of \hyperref[alg:dfs:dfsvisit]{\textsc{DFSvisit}} to find bridges}\label{alg:find_bridges}
\begin{algorithmic}[1]
\Function{findBridges}{$u$}
  \State $\attrib{u}{color}\gets\gray$
  \State $T\gets T+1$
  \State $\attrib{u}{low}\gets\attribnormal{u}{d}\gets T$  \Comment{discover $u$}
  \ForAll{$v\in\Gamma(u)$}
    \If{$\attrib{v}{color}=\white$} \Comment{\((u,v)\) is a tree edge}
      \State $\attrib{v}{\pi}\gets u$
      \State \Call{findBridges}{$v$}
      \State $\attrib{u}{low}\gets \min(\attrib{u}{low},\attrib{v}{low})$
      \If{$\attrib{v}{low}>\attribnormal{u}{d}$} \Comment{bridge test}
        \State label edge \((u,v)\) as a bridge
      \EndIf
    \ElsIf{$v \neq \attrib{u}{\pi}$} \Comment{\((u,v)\) is a back edge}
      \State $\attrib{u}{low}\gets \min(\attrib{u}{low},\attribnormal{v}{d})$
    \EndIf
  \EndFor
\EndFunction
\end{algorithmic}
\end{algorithm}

If we want to to find the \nameref{def:biconnected_component}s of \(G\), we can store the edges in a stack while performing \autoref{alg:find_cut_vertices}.
Whenever we reach a cut vertex, edges in a biconnected component will be consecutive on the stack.
See \autoref{alg:find_bicc}.

\begin{algorithm}
\caption{Modification of \hyperref[alg:dfs:dfsvisit]{\textsc{DFSvisit}} to find biconnected components (edge stack)}\label{alg:find_bicc}
\begin{algorithmic}[1]
\Function{findBiCC}{$u$}
  \State $\attrib{u}{color}\gets\gray$ \Comment{\(u\) has been discovered} \label{alg:find_bicc:discover}
  \State $T\gets T+1$
  \State $\attrib{u}{low}\gets \attribnormal{u}{d}\gets T$ \Comment{set discovery time and init low} \label{alg:find_bicc:discover_time}
  \ForAll{$v\in\Gamma(u)$} \label{alg:find_bicc:explore_edges}
    \If{$\attrib{v}{color}=\white$} \Comment{\((u,v)\) is a tree edge} \label{alg:find_bicc:tree_edge}
      \State $\attrib{v}{\pi}\gets u$ \Comment{\(v\)'s parent is \(u\)}
      \State push $(u,v)$ onto $\mathcal{S}$ \Comment{store tree edge on the edge stack} \label{alg:find_bicc:push_tree}
      \State \Call{findBiCC}{$v$} \label{alg:find_bicc:recursive_call}
      \State $\attrib{u}{low} \gets \min(\attrib{u}{low}, \attrib{v}{low})$ \Comment{propagate \(\attrib{v}{low}\) up} \label{alg:find_bicc:update_low_tree}
      \If{$\attrib{v}{low}\ge \attribnormal{u}{d}$} \Comment{biconnected-component boundary at \((u,v)\)} \label{alg:find_bicc:boundary}
        % \State $C\gets\emptyset$
        % \Repeat
        %   \State $e\gets \mathcal{S}.\text{pop}()$
        %   \State $C\gets C\cup\{e\}$
        % \Until{$e=(u,v)$} \Comment{pop until the tree edge \((u,v)\)} \label{alg:find_bicc:pop_until}
        % \State output \(C\) as a biconnected component \label{alg:find_bicc:output}
        \State pop from \(\mathcal{S}\) until tree edge \((u,v)\) is popped and output them as a BiCC
      \EndIf
    \ElsIf{$v\neq \attrib{u}{\pi}\ \textbf{and}\ \attribnormal{v}{d}<\attribnormal{u}{d}$} \Comment{\((u,v)\) is a back edge to an ancestor} \label{alg:find_bicc:back_edge}
      \State push $(u,v)$ onto $\mathcal{S}$ \Comment{store back edge on the edge stack} \label{alg:find_bicc:push_back}
      \State $\attrib{u}{low} \gets \min(\attrib{u}{low}, \attribnormal{v}{d})$ \Comment{consider back edge to \(v\)} \label{alg:find_bicc:update_low_back}
    \EndIf
  \EndFor
  % \State $\attrib{u}{color}\gets\black$ \Comment{\(u\) is finished}
  % \State $T\gets T+1$
  % \State $\attribnormal{u}{f}\gets T$
\EndFunction
\end{algorithmic}
\end{algorithm}

{When do we output a biconnected component?}
We do \emph{not} only output components ``when we reach a cut vertex''.
Instead, we output a component \emph{after returning from a recursive call} at a child \(v\) of \(u\), precisely when
\(
  \attrib{v}{low} \ge \attribnormal{u}{d}
\).
This condition means that the subtree rooted at \(v\) has no back edge to a \emph{proper} ancestor of \(u\) (``proper'' means excluding \(u\)).
Consequently, all edges pushed onto the edge stack \(\mathcal{S}\) since traversing the tree edge \((u,v)\) belong to a single biconnected component, and they appear \emph{consecutively} on \(\mathcal{S}\).
Hence we can pop edges until \((u,v)\) to obtain exactly that component.

For internal vertices \(u\), the same inequality \(\attrib{v}{low}\ge \attribnormal{u}{d}\) is also the cut-vertex test.
The root requires a separate cut-vertex criterion (\autoref{lem:cut_vertex_root}), while biconnected components must still be output even if the root is \emph{not} a cut vertex (e.g. when the entire graph is biconnected).





\subsubsection{Strongly Connected Components}\label{sec:strongly_connected_components}
\begin{definition}\label{def:strongly_connected_v2}
A di-graph is \textcolor{red}{strongly connected} if every pair of nodes \(u, v\) is mutually reachable, i.e., there is a path from \(u\) to \(v\) and also a path from \(v\) to \(u\).
\end{definition}

mutual reachability relation between vertices is an equivalence relation.

\medskip

partitions \(V\) into equivalence classes: \emph{strongly connected components} of \(G\)

\medskip

if we collapse the vertices within each strong component into a single vertex, we get the \emph{component digraph}. it is a DAG.
\begin{itemize}
\item if there is an edge (or path) from component \(C\) to component \(C'\), then there cannot be an edge (or path) from \(C'\) to \(C\) (or else \(C \cup C'\) would be one strong component).
\item thus, there is no cycle in the component digraph
\end{itemize}

\medskip

strong components can be computed in \(O(n + m)\) time using two \nameref{alg:dfs}:
\begin{itemize}
\item \hyperref[thm:strongly_connected_test]{recall} that we can answer if a digraph is strongly connected by 2 \nameref{alg:bfs}/\nameref{alg:dfs} 
\end{itemize}

\medskip

If we run our usual \nameref{alg:dfs} and record the finish times we make the following observations:

\begin{observation}\label{obs:highest_finish_time_source_of_component_digraph}
 Node of highest finish time must be a source (in-degree \(0\)) of the component digraph.
\end{observation}

\begin{observation}\label{obs:edge_between_strong_components_finish_times}
 More generally, if \(C\) and \(C'\) are two strong components such that there is an edge from a vertex in \(C\) to a vertex in \(C'\), 
 then 
 \begin{equation}\label{eq:connected_components_finish_times}
  \max_{u \in C} \attribnormal{u}{f} > \max_{v \in C'} \attribnormal{v}{f}
 \end{equation}
 i.e.\tikzmark{strong_components_mark} the \hl{highest finish time in \(C\) is greater than the highest finish time in \(C'\)}.
\begin{tikzpicture}[remember picture,overlay, font=\footnotesize]
\coordinate (strong_components) at (pic cs:strong_components_mark);
\coordinate (first_component) at ($(strong_components)+(0.4,0.85)$);
\node[draw, fill=gray!20, circle, minimum size=0.65cm, inner sep=0cm, densely dotted] (c) at (first_component) {\(C\)};
\node[draw, fill=gray!20, circle, minimum size=0.65cm, inner sep=0cm, densely dotted] (cp) at ($(first_component)+(1.6,0)$) {\(C'\)};
\draw[->] (c) -- (cp);
\end{tikzpicture}
\end{observation}

\begin{proof}
We distinguish two cases:
\begin{itemize}
\item 
Suppose \nameref{alg:dfs} first encounters a vertex \(u\) in \(C\).
Then it will visit all vertices in \(C'\) and \(C\) before \(u\) can finish.
Thus, \(u\) will have higher finish time than every vertex in \(C \cup C'\).
\item
Suppose \nameref{alg:dfs} first encounters a vertex \(v\) in \(C'\).
Then it will ``get stuck'' in \(C'\) (since there are no edges from \(C'\) to \(C\)).
So, all vertices in \(C\) will have higher discovery times than those in \(C'\); so they will also have higher finish times.
\qedhere
\end{itemize}
\end{proof}

\begin{caution}\label{caution:lowest_finish_time_not_sink}
It may be tempting at this point to try to conclude from \autoref{obs:highest_finish_time_source_of_component_digraph} and \autoref{obs:edge_between_strong_components_finish_times} that the node of lowest finish time must be in a sink (out-degree \(0\)) of the component digraph.
However, this is \emph{not} true in general, as one can see from the following counterexample:
\(V = \{a,b,c\}\), \(E = \{(a,b), (b,a), (a,c)\}\).
If we run \nameref{alg:dfs} starting from \(a\), exploring neighbors in alphabetical order, we get \(\attribnormal{b}{f} < \attribnormal{c}{f} < \attribnormal{a}{f}\), even though \(b\) is not in a sink component.
\end{caution}

Using DFS we can identify the vertex of highest finish time.
It is a vertex in a source node of the component DAG \(G^{\mathrm{SCC}}\).



but we want to identify a vertex in a \hl[2]{sink of the component DAG}:
\begin{itemize}
  \item if we can do this, then we can start a \nameref{alg:dfs} at a vertex in the sink component, and identify all vertices of this component
  \item (no other strong component will be visited at this DFS call, as there is no edge from the sink component to another one)
  \item repeat (after ignoring the vertices of this component)
\end{itemize}

\hl[3]{trick}:
\begin{itemize}
  \item reverse the edges of \(G\) to get the transpose graph \(G^{\top}\)
  \item strong component of \(G^{\top}\) are the same as those of \(G\)
  \item the edges of the component DAG of \(G^{\top}\) are the reversed edges of the component DAG of \(G\)
  \item the vertex of highest finish time in \(G^{\top}\) must be in the sink node of the original component DAG of \(G\) (by \autoref{obs:highest_finish_time_source_of_component_digraph})
\end{itemize} 

\begin{algorithm}[h]
\caption{Strongly Connected Components}\label{alg:strongly_connected_components}
\begin{algorithmic}[1]
  \Function{StronglyConnectedComponents}{$G$}
  \State create the transpose graph \(G^{\top}\)
  \State call \Call{DFS}{$G^{\top}$} to compute the finish time \(\attribnormal{v}{f}\) for each \(v\in V\) \label{alg:strongly_connected_components:first_dfs}
  \State as the vertices finish (Line~\ref{alg:dfs:finish}), push them on a stack % {thus, the vertices are automatically sorted by decreasing finish time}
  \State call \Call{DFS}{$G$}, but in Line~\ref{alg:dfs:for_all_vertices}, consider the vertices in order of decreasing \(\attribnormal{u}{f}\) \label{alg:strongly_connected_components:second_dfs}
  \State \Return the depth-first trees of \(G_\pi\) formed in Line~\ref{alg:strongly_connected_components:second_dfs}
  \EndFunction
\end{algorithmic}
\end{algorithm}

By considering vertices in the \(2^{\text{nd}}\) \nameref{alg:dfs} in decreasing order of finish times from the \(1^{\text{st}}\) \nameref{alg:dfs}, we are visiting the vertices of the component DAG in topologically sorted (reverse) order.

\clearpage
% !TEX root = ../algo-summary.tex
\section{Greedy Algorithms}\label{sec:greedy_algorithms}

\subsection{Interval Scheduling}
Given \(n\) jobs, we want to schedule a maximum number of non-overlapping jobs, i.e.
\begin{itemize}
  \item Job \(j\) starts at \(s_j\) and finishes at \(f_j\)
  \item Two jobs are \hl[2]{compatible} if they don't overlap
  \item Goal: find a \emph{maximum cardinality} subset of mutually compatible jobs
\end{itemize}


greedy template:
Consider jobs in some order.
Take a job if it is compatible with all previously selected jobs.

What order?
\begin{itemize}
\item (Earliest start time) ascending order of start time ${s}_{{j}}$.
\item (Shortest interval) ascending order of interval length $f_j-s_j$.
\item (Fewest conflicts) 
for each job, count the number of conflicting jobs ${c}_{{j}}$. 
ascending order of conflicts ${c}_{{j}}$.
\item (Earliest finish time) ascending order of finish time $f_j$.
\end{itemize}
all of the above are valid thoughts, but can easily find counterexample for the first three of them:
\[
\begin{tikzpicture}[scale=0.25]
\definecolor{L}{gray}{0.85}
\definecolor{D}{gray}{0.35}
\def\h{0.35}

% ------------------ breaks earliest start time ------------------
\begin{scope}[shift={(0,5)}]
  \fill[L] (0,0.15) rectangle ++(2,\h);
  \fill[L] (3,0.15) rectangle ++(2,\h);
  \fill[L] (6,0.15) rectangle ++(2,\h);
  \fill[L] (9,0.15) rectangle ++(2,\h);
  \fill[D] (-0.5,-0.15) rectangle ++(12,-\h);
  \node[anchor=west] at (12,0) {breaks earliest start time \textcolor{red}{\Lightning}};
\end{scope}

% ------------------ breaks shortest interval ------------------
\begin{scope}[shift={(0,2.5)}]
  \fill[L] (0,0.15) rectangle ++(5,\h);
  \fill[L] (6,0.15) rectangle ++(5,\h);
  \fill[D] (4,-0.15) rectangle ++(3,-\h);
  \node[anchor=west] at (12,0) {breaks shortest interval \textcolor{red}{\Lightning}};
\end{scope}

% ------------------ breaks fewest conflicts ------------------
\begin{scope}[shift={(0,0)}]
  % top row
  \fill[L] (0,0.15) rectangle ++(2,\h);
  \fill[L] (3,0.15) rectangle ++(2,\h);
  \fill[L] (6,0.15) rectangle ++(2,\h);
  \fill[L] (9,0.15) rectangle ++(2,\h);
  % second row
  \fill[L] (1.5,-0.15) rectangle ++(2,-\h);
  \fill[D] (4.5,-0.15) rectangle ++(2,-\h);
  \fill[L] (7.5,-0.15) rectangle ++(2,-\h);
  % third row
  \fill[L] (1.5,-0.8) rectangle ++(2,-\h);
  \fill[L] (7.5,-0.8) rectangle ++(2,-\h);
  % fourth row
  \fill[L] (1.5,-1.45) rectangle ++(2,-\h);
  \fill[L] (7.5,-1.45) rectangle ++(2,-\h);
  \node[anchor=west] at (12,0) {breaks fewest conflicts \textcolor{red}{\Lightning}};
\end{scope}

\end{tikzpicture}
\]


(a correct) greedy strategy: 
increasing order of finish time \textcolor{Green}{\ding{52}}
% Take a job if it is compatible with all previously selected jobs.

or symmetrically:
decreasing order of start time \textcolor{Green}{\ding{52}}

\begin{algorithm}[h]
\caption{Greedy Interval Scheduling}\label{alg:greedy_interval_scheduling}
\begin{algorithmic}[1]
\State sort jobs by finish time: \(f_1 \le f_2 \le \ldots \le f_n\)
\State $A\gets\emptyset$ \Comment{selected jobs}
\For{$j=1 \TO n$}
  \If{job \(j\) is compatible with (last) job in \(A\)}
    \State $A\gets A\cup\{j\}$
  \EndIf
\EndFor
\State \Return $A$
\end{algorithmic}
\end{algorithm}


Implementation. \(O(n\log n)\) for sorting
\begin{itemize}
\item remember job \(j^*\) that was added last to \(A\)
\item job \(j\) is compatible with \(A\) if \(s_j \geq f_{j^*}\)
\end{itemize}


\begin{theorem}[Optimality]\label{thm:greedy_interval_scheduling_optimality}
\autoref{alg:greedy_interval_scheduling} is optimal.
\end{theorem}
\begin{proof}
Consider an optimal schedule \(O\), and let \(G\) be the greedy schedule.
Let \(O = [x_1, x_2, \ldots, x_k]\) the activities of \(O\) listed in increasing order of finish time,
and let \(G = [g_1, g_2, \ldots, g_{k'}]\) be the corresponding greedy schedule.
If \(G = O\), we are done.
Otherwise, let \(j\) be the first index where \(x_j \neq g_j\), i.e.
\[
\begin{aligned}
O &= [x_1, x_2, \ldots, x_{j-1}, {x_j}, x_{j+1}, \ldots, x_k] \\
G &= [x_1, x_2, \ldots, x_{j-1}, {g_j}, g_{j+1}, \ldots, g_{k'}]
\end{aligned}
\]
where \(g_j \neq x_j\).
Note that \(k \geq j\) since otherwise \(G\) would have more activities than \(O\), contradictory.
\autoref{alg:greedy_interval_scheduling} selects the the activity with the earliest finish time that does not conflicct with any earlier activity.
Thus, we know that \(g_j\) does not conflict with any earlier activity, and it finishes no later than \(x_j\) finishes:
\[
\begin{tikzpicture}[>=Stealth, scale=0.5, font=\footnotesize]
  \tikzset{
    box/.style={draw, minimum width=0.9cm, minimum height=0.28cm,
                inner sep=0pt, align=center, anchor=west},
    gbox/.style={box, fill=gray!30},
    boxsmall/.style={draw, minimum width=0.6cm, minimum height=0.28cm,
                     inner sep=0pt, align=center, anchor=west},
    gboxsmall/.style={boxsmall, fill=gray!30}
  }

  % left-edge x positions of columns
  \def\xA{1.0}
  \def\xB{3.0}
  \def\xC{5.35}
  \def\xD{6.2}
  \def\xE{8.2}
  \def\xF{10.2}
  \def\xG{12.2}
  \def\xH{14.45}

  % y-positions for the rows
  \def\yO{0.0}
  \def\yG{-0.8}
  \def\yLine{-1.4}
  \def\yOp{-2.0}

  % Row labels
  \node at (0,\yO) {$O:$};
  \node at (0,\yG) {$G:$};
  \node at (0,\yOp) {$O':$};

  % --- Row O ---------------------------------------------------------------
  \node[box] at (\xA,\yO) {$x_1$};
  \node[box] at (\xB,\yO) {$x_2$};
  \node      at (\xC+0.2,\yO) {$\cdots$};
  \node[box] at (\xD,\yO) {$x_{j-1}$};
  \node[box] at (\xE,\yO) {$x_j$};
  \node[box] at (\xF,\yO) {$x_{j+1}$};
  \node[box] at (\xG,\yO) {$x_{j+2}$};
  \node      at (\xH,\yO) {$\cdots$};

  % --- Row G ---------------------------------------------------------------
  \node[box]       at (\xA,\yG) {$x_1$};
  \node[box]       at (\xB,\yG) {$x_2$};
  \node            at (\xC+0.2,\yG) {$\cdots$};
  \node[box]       at (\xD,\yG) {$x_{j-1}$};
  \node[gboxsmall] at (\xE,\yG) {$g_j$};      % left edges aligned with x_j column
  \node[gbox]      at (\xF-0.6,\yG) {$g_{j+1}$};
  \node[gbox]      at (\xG-0.6,\yG) {$g_{j+2}$};
  \node            at (\xH-0.6,\yG) {$\cdots$};

  % Divider
  \draw (0.0,\yLine) -- (\xH+0.6,\yLine);

  % --- Row O' --------------------------------------------------------------
  \node[box]  at (\xA,\yOp) {$x_1$};
  \node[box]  at (\xB,\yOp) {$x_2$};
  \node       at (\xC+0.2,\yOp) {$\cdots$};
  \node[box]  at (\xD,\yOp) {$x_{j-1}$};
  \node[gboxsmall] at (\xE,\yOp) {$g_j$};          % same left edge as above
  \node[box]  at (\xF,\yOp) {$x_{j+1}$};
  \node[box]  at (\xG,\yOp) {$x_{j+2}$};
  \node       at (\xH,\yOp) {$\cdots$};

  % Curved arrow on the left
  \draw[->, line cap=round] (-0.6,\yO) .. controls ($(-1,-0.5)$) and ($(-1,-1.5)$) .. (-0.6,\yOp);
\end{tikzpicture}
\]

Consider the the modified `greedier' schedule \(O'\) obtained by replacing \(x_j\) with \(g_j\) in \(O\).
That is, 
\[
O' = [x_1, x_2, \ldots, x_{j-1}, {g_j}, x_{j+1}, \ldots, x_k]
\]
it is a valid schedule, because \(g_j\) finishes no later than \(x_j\) and therefore cannot create any new conflicts.
It has the same \(\numof\)activities as \(O\), so it is at least as good.
By repeating this process, we will eventually convert \(O\) into \(G\) without ever decreasing the number of activities.
Thus, \(G\) is optimal.
\end{proof}






\subsection{Interval Partitioning}
Given \(n\) jobs, we want to schedule them all using a minimum number of resources (e.g., classrooms), i.e.
\begin{itemize}
  \item Job \(j\) starts at \(s_j\) and finishes at \(f_j\)
  \item Goal: find a \emph{minimum number} of resources (e.g., classrooms) to schedule all jobs
\end{itemize}

\begin{definition}[depth]\label{def:depth}
The \emph{depth} of a set of open intervals is the maximum number that contain any given time (vertical line), i.e.
\[
\text{depth} = \max_{t} A(t)
\]
where \(A(t)\) is the number of intervals that contain (i.e. that are \emph{active} at time \(t\)).
\end{definition}
Key observation: \(d\) \(=\) \(\numof\)resources \(\geq\) depth

Greedy strategy: 
Consider jobs in increasing order of start time.
Assign job to any compatible resource, or open a new resource.


\begin{algorithm}[h]
\caption{Greedy Interval Partitioning}\label{alg:greedy_interval_partitioning}
\begin{algorithmic}[1]
\State sort jobs by starting time: \(s_1 \le s_2 \le \ldots \le s_n\)
\State \(d \gets 0\) \Comment{number of allocated resources}
\For{$j=1 \TO n$}
  \If{job \(j\) is compatible some available resource \(k\)}
    \State schedule job \(j\) on resource \(k\)
  \Else
    \State allocate a new resource \(d + 1\)
    \State schedule job \(j\) on resource \(d + 1\)
    \State \(d \gets d + 1\)
  \EndIf
\EndFor
\end{algorithmic}
\end{algorithm}


Implementation. \(O(n\log n)\) 
\begin{itemize}
\item sort takes \(O(n\log n)\) 
\item for each ressource \(k\), maintain the finish time of the last job added 
\item avoid quadratic algorithm (scanning all allocated \(d\) resources every single time) 
\item first idea: priority queue (min-heap) of ressources, keyed by time they become free
\item interact with the heap in \(O(\log d)\) time per job: get-min, del-min, insert
\end{itemize}

But actually, we can do better (for the part after sorting): 
\begin{itemize}
\item maintain a stack/queue \(Q\) of available resources 
\item initially \(Q\) is empty 
\item create a sorted list of \(2n\) \hl{job events}: 
turn every interval \([s_i, f_i)\) into two events  
\((s_i, \texttt{start})\) and \((f_i, \texttt{finish})\). 
finish events come before start events at the same time 
(lets a job ending at time \(t\) free a room that a job starting at time \(t\) can immediately reuse).  
\item at a start event a ressource is allocated from \(Q\) (or a new one (\(+\!+\!d\)) is created if \(Q\) is empty) 
\item at a finish event, a ressource is freed and put back onto \(Q\) 
\item \textcolor[HTML]{013399}{principle of 1D plane sweep by vertical line from left to right}
\end{itemize}

\begin{theorem}[Optimality]\label{thm:greedy_interval_partitioning_optimality}
\autoref{alg:greedy_interval_partitioning} is optimal.
\end{theorem}
\begin{proof}
We can show that ${d} \leq \text{depth}$ (enough to prove optimality):
\begin{itemize}
\item Ressource $d$ is allocated because we needed to schedule a job $j$ that is incompatible with all the current ${d}-1$ ressources, which hold jobs that started before ${s}_{j}$ (no later than ${s}_{j}$) and none has finished yet.
\item Consider time ${s}_{j}+\varepsilon$. At that time $({d}-1)+1={d}$ intervals are overlapping and since since depth is the maximum number of overlapping intervals at any time, we have ${d} \leq \text{depth}$.
\qedhere
\end{itemize}
\end{proof}



\subsection{Minimizing Lateness}\label{sec:minimizing_lateness}
Given \(n\) jobs, where
\begin{itemize}
  \item job \(j\) requires \(t_j\) units processing time and is due at \(d_j\) (deadline)
  \item if job \(j\) starts at time \(s_j\), it finishes at \(f_j = s_j + t_j\)
  \item lateness: \(\ell_j = \max(0, f_j - d_j)\) \hspace{1em} \textcolor{green}{finish time - deadline}
  \item goal: schedule all jobs to \emph{minimize} \hl[2]{maximum lateness} \(L = \max_j \ell_j\)
\end{itemize}

greedy template:
Consider jobs in some order.
\begin{itemize}
\item (Shortest processing time) ascending order of processing time ${t}_{j}$.
\item (Earliest deadline) ascending order of deadlines ${d}_{j}$.
\item (Smallest slack) ascending order of slack time $s_j = d_j - t_j$
\end{itemize}

but there are problems for first and last:
\begin{itemize}
\item short job may have a long deadline, e.g. \([(t_1,d_1)=(1,100),(t_2,d_2)=(10,10)]\) \textcolor{red}{\Lightning}
\item long job may have \(0\) slack, e.g. \([(t_1,d_1)=(1, 2),(t_2,d_2)=(10,10)]\) \textcolor{red}{\Lightning}
\end{itemize}

the correct greedy strategy is: earliest deadline first \textcolor{Green}{\ding{52}}
\begin{algorithm}[h]
% \begin{aligned}
% &\text { Sort } \mathrm{n} \text { jobs by deadline so that } \mathrm{d}_1 \leq \mathrm{d}_2 \leq \ldots \leq \mathrm{d}_{\mathrm{n}}\\
% &\mathrm{t} \leftarrow 0\\
% &\text { for j = } 1 \text { to n }\\
% &\text { Assign job j to interval [ } t, t+t_j \text { ] }\\
% &\mathbf{s}_{\mathbf{j}} \leftarrow \mathbf{t}, \mathbf{f}_{\mathbf{j}} \leftarrow \mathbf{t}+\mathbf{t}_{\mathbf{j}}\\
% &\mathrm{t} \leftarrow \mathrm{t}+\mathrm{t}_{\mathrm{j}}\\
% &\text { output intervals }\left[\mathrm{s}_{\mathrm{j}}, \mathrm{f}_{\mathrm{j}}\right]
% \end{aligned}
\caption{Greedy Minimizing Lateness}\label{alg:greedy_minimizing_lateness}
\begin{algorithmic}[1]
\State sort jobs by deadline: \(d_1 \le d_2 \le \ldots \le d_n\)
\State $t \gets 0$ \Comment{current time}
\For{$j=1 \TO n$}
  \State assign job \(j\) to interval \([t, t+t_j]\)
  \State $s_j \gets t$
  \State $f_j \gets t+t_j$
  \State $t \gets t+t_j$
\EndFor
\Return intervals \([s_j, f_j]\)
\end{algorithmic}
\end{algorithm}



\begin{observation}\label{obs:exists_optimal_no_idle}
There exists an optimal schedule with no \textcolor[HTML]{CC0100}{idle time} (idle time gives no benefit).
\end{observation}

\begin{observation}\label{obs:greedy_no_idle}
The \hyperref[alg:greedy_minimizing_lateness]{greedy schedule} has no idle time (by construction).
\end{observation}

\begin{definition}[inversion]\label{def:inversion}
An inversion in schedule \(S\) is a pair of jobs \(i,j\) such that \(d_i < d_j\) but \(j\) is scheduled before \(i\).
\end{definition}

\begin{observation}\label{obs:greedy_no_inversion}
The \hyperref[alg:greedy_minimizing_lateness]{greedy schedule} has no inversions (by construction - ascending order of deadlines).
\end{observation}

\begin{observation}\label{obs:inversion_implies_consecutive_iversion}
If a schedule (with no idle time) has an inversion, there is an inversion with two inverted jobs scheduled consecutively.
\end{observation}

\begin{claim}
\label{claim:swap_does_not_increase_lateness}
Swapping two adjacent inverted jobs reduces number of inversions by \(1\) and does not increase the max lateness \(\ell_{\max} := \max \{\ell_1, \ldots, \ell_n\}\).
\end{claim}

\begin{proof}
Let \(\ell\) be the lateness before the swap and let \(\ell'\) be it after the swap.
\begin{itemize}
  \item \(\ell_k' = \ell_k\) for all \(k \neq i,j\) (no change)
  \item \(\ell_i' \leq \ell_i\) because \(i\) finishes earlier after the swap
  \item if job \(j\) was late before the swap: 
  \[
  \begin{aligned}
    \ell_j' &= f_j' - d_j &&\text{(definition)} \\
            &= f_i - d_j &&\text{(\(j\) now finishes when \(i\) used to finish)} \\
            &\leq f_i - d_i &&\text{(since \(d_i < d_j\))} \\
            &= \ell_i &&\text{(definition)}
  \end{aligned}
  \]
  \end{itemize}
\[
\begin{tikzpicture}[scale=0.5,
  >={Latex[length=1mm, width=1mm]},
  lab/.style={font=\footnotesize, text height=1ex, text depth=0pt, baseline}
]

\def\labelpos{0.9}
\def\dipos{1}
\def\djpos{2}
\def\sjpos{3.5}
\def\mpos{5}
\def\fipos{7}

\def\blockh{0.7}     % block height


\node[lab, anchor=east] at (\labelpos,\blockh/2) {before swap:};

% jobs
\node at (\sjpos-0.5,\blockh/2) {$\cdots$};
\draw[draw=black, fill=gray!35] (\sjpos,0) rectangle (\mpos,\blockh);
\node[lab] at ({(\sjpos+\mpos)/2},\blockh/2) {$j$};
\draw[draw=black, fill=gray!35] (\mpos,0) rectangle (\fipos,\blockh);
\node[lab] at ({(\mpos+\fipos)/2},\blockh/2) {$i$};
\node at (\fipos+0.5,\blockh/2) {$\cdots$};
\draw[<->] (\sjpos,\blockh+0.2) -- (\mpos,\blockh+0.2) node[midway,above,lab] {$t_j$};
\draw[<->] (\mpos,\blockh+0.2) -- (\fipos,\blockh+0.2) node[midway,above,lab] {$t_i$};

% deadlines (d_i < d_j)
\draw[densely dotted] (\dipos,-2) -- (\dipos,1);
\draw[] (\dipos,1) -- (\dipos,1.2) node[lab, above] {$d_i$};
\draw[densely dotted] (\djpos,-1.2) -- (\djpos,1);
\draw[] (\djpos,1) -- (\djpos,1.2) node[lab, above] {$d_j$};

% finish times
\draw[densely dotted] (\mpos,-1.2) -- (\mpos,0);
\draw[densely dotted] (\fipos,-2) -- (\fipos,0);

% lateness
\draw[<->] (\djpos,-1) -- (\mpos,-1) node[lab, midway, above] {$\ell_j$};
\draw[<->] (\dipos,-1.8) -- (\fipos,-1.8) node[lab, midway, above] {$\ell_i$};


\begin{scope}[shift={(12,0)}]
\def\labelpos{0.9}
\def\dipos{1}
\def\djpos{2}
\def\sipos{3.5}
\def\mpos{5.5}
\def\fjpos{7}

\def\blockh{0.7}     % block height

\node[lab, anchor=east] at (\labelpos,\blockh/2) {after swap:};

% jobs
\node at (\sipos-0.5,\blockh/2) {$\cdots$};
\draw[draw=black, fill=gray!35] (\sipos,0) rectangle (\mpos,\blockh);
\node[lab] at ({(\sipos+\mpos)/2},\blockh/2) {$i$};
\draw[draw=black, fill=gray!35] (\mpos,0) rectangle (\fjpos,\blockh);
\node[lab] at ({(\mpos+\fjpos)/2},\blockh/2) {$j$};
\node at (\fjpos+0.5,\blockh/2) {$\cdots$};
\draw[<->] (\sipos,\blockh+0.2) -- (\mpos,\blockh+0.2) node[midway,above,lab] {$t_i$};
\draw[<->] (\mpos,\blockh+0.2) -- (\fjpos,\blockh+0.2) node[midway,above,lab] {$t_j$};

% deadlines (d_i < d_j)
\draw[densely dotted] (\dipos,-2) -- (\dipos,1);
\draw[] (\dipos,1) -- (\dipos,1.2) node[lab, above] {$d_i$};
\draw[densely dotted] (\djpos,-1.2) -- (\djpos,1);
\draw[] (\djpos,1) -- (\djpos,1.2) node[lab, above] {$d_j$};

% finish times
\draw[densely dotted] (\fjpos,-1.2) -- (\fjpos,0);
\draw[densely dotted] (\mpos,-2) -- (\mpos,0);

% lateness
\draw[<->] (\djpos,-1) -- (\fjpos,-1) node[lab, midway, above] {$\ell_j'$};
\draw[<->] (\dipos,-1.8) -- (\mpos,-1.8) node[lab, midway, above] {$\ell_i'$};
\end{scope}

\end{tikzpicture}
\]

Thus we have \(
 \ell_{\max}' = \max{\{\ell_i', \ell_j', \text{others}\}} \leq \max{\{\ell_i, \text{others}\}} = \ell_{\max}
\).
\end{proof}

\begin{theorem}[Optimality]\label{thm:greedy_minimizing_lateness_optimality}
\autoref{alg:greedy_minimizing_lateness} is optimal.
\end{theorem}
\begin{proof}
Let \(G\) be the greedy schedule produced by \autoref{alg:greedy_minimizing_lateness}.
Let \(O\) be an optimal schedule with the fewest number of inversions.
We can assume \(O\) has no idle time.
If \(O\) has no inversions, then \(O = G\).
Otherwise, let \(i,j\) be two adjacent inverted jobs in \(O\).
By \autoref{claim:swap_does_not_increase_lateness}, swapping \(i\) and \(j\) does not increase the maximum lateness and strictly decreases the number of inversions.
This contradicts the definition of \(O\). \textcolor{red}{\Lightning}
\end{proof}

Alternatively, we can prove \autoref{thm:greedy_minimizing_lateness_optimality} using an exchange argument:
We can take \(O\) and transform into \(G\) by removing inversions 1 by 1 with no change in max lateness.


\subsection{Analysis Strategies and Approximation}

\subsubsection{Correctness Proofs}
\textcolor[HTML]{013399}{Greedy algorithm stays ahead}.
Show that after each step of the greedy algorithm, its solution is at least as good as any other algorithm's.

\textcolor[HTML]{013399}{Exchange argument}.
Gradually transform any solution to the one found by the greedy algorithm without hurting its quality.

\textcolor[HTML]{013399}{Structural}.
Discover a simple "structural" bound asserting that every possible solution must have a certain value. 
Then show that your algorithm always achieves this bound.

\textcolor[HTML]{013399}{Counterexample}.
To show that a greedy algorithms does \emph{not} work, all we need is a small counterexample.


\subsubsection{Greedy Approximation for NP-Hard Problems}

The greedy approach is not very powerful as an algorithm design technique.
\begin{itemize}
  \item One of most common applications of greedy algorithms is to produce approximation solutions to NP-hard problems.
  \item NP-hard optimization problems are challenging computational problems with no known exact solution of worst-case polynomial-time running time.
  \item Given an NP-hard problem, there are no ideal solutions. Compromise between optimality or running time.
  \item Instances of such problems where simple greedy algorithms produce solutions that are not far from optimal. E.g., clustering and Facility Location.
\end{itemize}


\subsection{Shortest Paths}\label{sec:shortest_paths}

\textcolor[HTML]{013399}{Shortest path network}.
\begin{itemize}
  \item directed of undirected graph \(G=(V,E)\)
  \item source \(s\), destination/target \(t\)
  \item length/weight function \(w:E\to\R\) (assigns cost/length to each edge)
\end{itemize}

\textcolor[HTML]{013399}{Shortest path problem}: find shortest (directed) path from \(s\) to \(t\).






\subsubsection*{\nameref{alg:dijkstra}'s algorithm}
%  is a greedy, label-setting method for nonnegative edge weights.  
% At each step it finalizes the closest unreached vertex and relaxes its outgoing edges, yielding a time complexity of \(O(( n+m)\log  n)\) with a binary-heap implementation.

\begin{itemize}
  \item assume \hypertarget{non_negative_assumption}{\(w(e) \geq 0\)} for all edges \(e\) (non-negative weights)
  \item maintain a set of \textcolor[HTML]{CC0100}{explored nodes \(S\)} for which we have determined the the \hl{shortest path distance \(\attribnormal{u}{\delta}\)} from \(s\) to \(u\)
  \item initialize \(S = \{s\}\) and \(\attribnormal{s}{\delta} = 0\)
  \item repeatedly choose unexplored node \(v\) which minimizes 
  \begin{equation}\label{eq:dijkstra_ds}
  d_S(v) := \min_{e=(u,v), u\in S} \left(\attrib{u}{\delta} + w(e)\right)
  \end{equation}
  where \(d_S(v)\) is the shortest path to some \(u\) in explored part \(S\), followed by a single edge \((u,v)\), i.e.
  it is the shortest path from \(s\) to \(v\) using only vertices in \(S\).
  Add \(v\) to \(S\) and set \(\attrib{v}{\delta} = d_S(v)\).
\end{itemize}

two sets of vertices: \(S\) (explored nodes) and \(V\setminus S\) (unexplored)
\begin{itemize}
  \item \(V \setminus S\) organized in a priority queue \(Q\) (heap of min \(d\) value)
  \item vertices in \(Q\) have weight \(d_S(v)\): length of shortest path to \(v\) using only vertices in \(S\)
  \item vertices in \(S\) have weight \(\attrib{u}{\delta}\): length of shortest \(s\)-\(u\) path
  \item every time a vertex \(u\) gets explored (joins \(S\)), we \hl[2]{``relax'' all incident edges \((u,v)\)}:
        for all vertices \(v\) adjacent to \(u\):
        update \(d_S(v)\) to \(\min(d_S(v), \attrib{u}{\delta} + w(u,v))\)
  \item next node to explore = node with minimum \(d_S(v)\) 
\end{itemize}

\textcolor[HTML]{013399}{Efficient implementation}.
Maintain a priority queue of unexplored nodes, prioritized by \(d_S(v)\).


\begin{table}[ht]
  \centering
  \renewcommand{\arraystretch}{1.1}
  \begin{tabular}{
    p{0.18\textwidth}  % a bit wider first column
    |  
    >{\centering\arraybackslash}p{0.12\textwidth}
    >{\centering\arraybackslash}p{0.14\textwidth}
    >{\centering\arraybackslash}p{0.14\textwidth}
    >{\centering\arraybackslash}p{0.14\textwidth}
    >{\centering\arraybackslash}p{0.14\textwidth}
  }
    \toprule
    {PQ Operation} & {Dijkstra} & {Array} & {Binary heap} & {d-way Heap} & {Fib heap}$^{\hyperlink{table_footnote}{\dagger}}$ \\
    \midrule
    Insert     & $n$ & $n$ & $\log n$ & $d \log_{d} n$ & $1$ \\
    ExtractMin & $n$ & $n$ & $\log n$ & $d \log_{d} n$ & $\log n$ \\
    ChangeKey  & $m$ & $1$ & $\log n$ & $\log_{d} n$ & $1$ \\
    IsEmpty    & $n$ & $1$ & $1$ & $1$ & $1$ \\
    \midrule
    \textbf{Total} &  & $n^{2}$ & $m \log n$ & $m \log_{m/n} n$ & $m + n \log n$ \\
    \bottomrule
  \end{tabular}
  \vspace{0.5em}
  {\footnotesize \hypertarget{table_footnote}{$^{\dagger}$ Individual operations are amortized bounds.}}
\end{table}


\begin{algorithm}[h]
\caption{Dijkstra}\label{alg:dijkstra}
\begin{algorithmic}[1]
\Function{Dijkstra}{$G, w:E\to\R_{\textcolor{red}{\ge 0}}, s$} 
  \ForAll{$v\in V$} \Comment{initialization}
    \State $\attrib{v}{\delta}\gets\infty$ \Comment{best known cost from \(s\) to \(v\)}
    \State $\attrib{v}{color} \gets \text{undiscovered}$
    \State $\attrib{v}{\pi}\gets\nil$ \Comment{node preceding \(v\) on the least-cost path from \(s\)}
  \EndFor
  \State $\attrib{s}{\delta}\gets 0$ \Comment{\(s\) is the source}
  \State $Q \gets $ priority queue of all nodes \(u \in V\) sorted by \(\attrib{u}{\delta}\)
  \While{$Q \neq \emptyset$} \Comment{until all vertices are processed}
    \State extract \(u\) with minimal \(\attrib{u}{\delta}\) from \(Q\) \label{line:dijkstra_extract_min}
    \ForAll{$v\in\Gamma(u)$}
      \If{$\attrib{u}{\delta} + w(u,v) < \attrib{v}{\delta}$} \Comment{relax edge \((u,v)\), i.e. update \hyperref[eq:dijkstra_ds]{\(d_S(v)\)}}
        \State $\attrib{v}{\delta} \gets \attrib{u}{\delta} + w(u,v)$
        \State decrease key of \(v\) in \(Q\) to \(\attrib{v}{\delta}\)
        \State $\attrib{v}{\pi} \gets u$
      \EndIf
    \EndFor
    \State $\attrib{u}{color} \gets \text{finished}$
  \EndWhile
  \LComment{the \(\attribute{\pi}\) pointers define an `inverted' shortest path tree}
\EndFunction
\end{algorithmic}
\end{algorithm}



\textcolor[HTML]{013399}{Time analysis (using a binary heap)}.
\begin{itemize}
  \item extract a vertex \(u\) from priority queue \(Q\): \(O(\log n)\)
  \item for each incidnt edge \((u,v)\), relax it in \(O(1)\) and spend \(O(\log n)\) if we decrease the key of \(v\) in \(Q\) 
\end{itemize}
Thus, time to process vertex \(u\) is \(O(\log n + \deg(u) \log n)\) and total time is
\[
\begin{aligned}
T(n, m) &= \sum_{u \in V} (\log n + \deg(u) \log n) \\
&= \log n \sum_{u \in V} (1 + \deg(u)) \\
&= (\log n) (n + 2m) \\
&= O((n + m) \log n)
\end{aligned}
\]


\begin{observation}\label{obs:attribute_d}
For all \(u \in V \setminus S\) we have
\[
\attrib{u}{\delta} = d_S(u)
\]
i.e. the key in the PQ is exactly \(d_S(u)\).
\end{observation}

\begin{caution}
We use the same variable/attribute \(\attrib{u}{\delta}\) to store both \(d_S(u)\) for the \(u \in V \setminus S\) and `later' \(\delta(s, u)\) for the \(u \in S\).

Only once a vertex \(u\) is extracted from the queue \(Q\) in Line \ref{line:dijkstra_extract_min} (which means it is added to \(S\)), we are guaranteed that \(\attrib{u}{\delta} = \delta(s, u)\).
\end{caution}

\begin{theorem}\label{thm:dijkstra_correctness}
For all \(u \in S\) we have
\[
\attrib{u}{\delta} = \delta(s, u)
\]
i.e. it is the length of the shortest \(s\)-\(u\) path.
\end{theorem}

\begin{proof}[Induction on \(|S|\)]\label{proof:dijkstra_correctness}
\leavevmode
\begin{enumerate}[partopsep=0em, label=(\roman*)]
\item Base case:
\(|S|=1\) is trivial: \(\attrib{u}{\delta} = \attrib{s}{\delta} = 0\) \textcolor{Green}{\ding{52}}
\item Induction hypothesis:
Assume true for \(|S| = k \geq 1\). 
\label{proof:dijkstra_correctness:induction_hypothesis}
\item Induction step:
Prove true for \(|S| = k+1\).

Let \(v\) be the \((k+1)^{\text{th}}\) node added to \(S\), and let \((u,v)\) be the chosen edge, and therefore \(\attrib{v}{\delta} = \attrib{u}{\delta} + w(u,v)\).
% We need to show that in that case \(\attrib{v}{\delta} = \delta(s,v)\).

Consider any \(s\)-\(v\) path \(P\):
  \[
\begin{tikzpicture}[>=Stealth, scale=0.4, font=\footnotesize]
\useasboundingbox (-3,-2.5) rectangle (5,2.5);
% background blob
\fill[black!20] plot[smooth cycle,tension=0.7] coordinates {(-3.4,1.5) (-2.0,2.3) (-0.4,2.0) (1.6,2.2) (3.6,1.6) (4.5,0.7) (5.1,-0.2) (4.6,-1.3) (3.2,-2.1) (1.3,-2.3) (-0.6,-2.0) (-2.2,-1.7) (-3.2,-0.8)};
\node at (-2.75,1.5) {\(S\)};
% nodes
\tikzset{state/.style={circle,draw,fill=white,minimum size=9pt,inner sep=0pt}}
\node[state] (s) at (-1.5,0.0) {\(s\)};
\node[state] (x) at (3.3,0.5)  {\(x\)};
\node[state] (y) at (5.5,1.5)  {\(y\)};
\node[state] (u) at (2.4,-1.5) {\(u\)};
\node[state] (v) at (5.2,-2) {\(v\)};
% paths
\draw[-, rounded corners=3pt, blue] (s) -- (-0.2, 1.2) -- (0.9, 0.4) node[midway, right, yshift=0.08cm] {$P'$} -- (2.0, 1.2) -- (x) ; % s -> x (P')
\draw[-, rounded corners=3pt] (s) -- (-0.5,-0.8) -- (0.5,-0.2) -- (1.5,-0.6) -- (u); % s -> u
\draw[-, blue] (x) -- (y); % edge (x,y)
\draw[->] (u) -- (v); % edge (u,v)
\draw[->, rounded corners=3pt, blue] (y) -- (5.5, 0) -- (6.2, -0.8) -- (6.6, -1.1) node[midway, right, blue] {$P$} -- (v); % y -> v
% \draw (current bounding box.north east) -- (current bounding box.north west) -- (current bounding box.south west) -- (current bounding box.south east) -- cycle; % debugging
\end{tikzpicture} 
\]

Let \((x, y)\) be the first edge in \(P\) that leaves \(S\), and let \(P'\) be the subpath to \(x\).
For the length \(\ell\) of \(P\), we have
\[
\ell(P) 
\overset{\hyperlink{non_negative_assumption}{w(e) \geq 0}}{\geq} 
\ell(P') + w(x,y) 
\overset{\text{\ref{proof:dijkstra_correctness:induction_hypothesis}}}{\geq} 
\attrib{x}{\delta} + w(x,y) 
\overset{\text{\eqref{eq:dijkstra_ds}}}{\geq} 
d_S(y) 
% \overset{\text{Line \ref{line:dijkstra_extract_min}}}{\geq} 
% d_S(v)
\overset{\text{Line \ref{line:dijkstra_extract_min}}}{\geq} 
\attrib{v}{\delta}
\]
Since \(P\) is arbitrary, this in particular holds for the shortest path, hence \(\delta(s, v) \ge \attrib{v}{\delta}\).

By \ref{proof:dijkstra_correctness:induction_hypothesis}, \(\attrib{u}{\delta} = \delta(s,u)\), 
so the path consisting of the shortest \(s\)-\(u\) path followed by \((u,v)\) 
has length \(\delta(s,u) + w(u,v) = \attrib{v}{\delta}\).
Hence, \(\delta(s,v) \le \attrib{v}{\delta}\).

Thus, \(\attrib{v}{\delta} = \delta(s,v)\) shortest path from \(s\) to \(v\).
\textcolor{Green}{\ding{52}}
\qedhere
\end{enumerate}
\end{proof}







\subsection{Minimum Spanning Tree}\label{sec:minimum_spanning_tree}

\hl{\nameref{alg:prim} similar to \nameref{alg:dijkstra}}

\hl{\nameref{alg:kruskal} similar to \nameref{alg:strongly_connected_components}}


\begin{definition}[MST]\label{def:mst}
Given a connected graph \(G=(V,E)\) with real-valued edge weights \(c_e\),
an MST is a subset of the edges \(T\subseteq E\) such that \(T\) is a spanning tree whose sum of edge weights is minimized.
\end{definition}
From Cayley's theorem we know that we do not want to find an MST by brute force.

\textcolor[HTML]{013399}{\nameref{alg:kruskal}'s algorithm}. Start with $T=\emptyset$. Consider edges in ascending order of cost. Insert edge \(e\) in \(T\) unless doing so would create a cycle.

\textcolor[HTML]{013399}{Reverse-Delete algorithm}. Start with ${T}={E}$. Consider edges in descending order of cost. Delete edge \(e\) from \(T\) unless doing so would disconnect \(T\) .

\textcolor[HTML]{013399}{\nameref{alg:prim}'s algorithm}. Start with some root node \(s\) and greedily grow a tree \(T\) from \(s\) outward. At each step, add the cheapest edge \(e\) to \(T\) that has exactly one endpoint in $T$.

\begin{remark} All three algorithms produce an MST.\end{remark}

\textcolor[HTML]{013399}{Simplifying assumption}. All edge costs \(c_e\) are distinct. (unique MST)

\begin{property}[Cut]\label{property:cut}
Let \(S\) be any subset of nodes, and let \(e\) be the min cost edge with exactly one endpoint in \(S\).
Then the MST contains \(e\).  
\end{property}
\begin{proof}[Exchange argument]
  Suppose \(e\) does not belong to the MST \(T^*\).
  Since \(T^*\) is a spanning tree there must be an edge \(f\) in \(T^*\) connecting \(S\) and \(V \setminus S\).
  But \(c_e < c_f\).
  Since \(T^*\) is a tree, \(f\) is the only such edge.
  Now let \(T' = T^* \cup \{e\} \setminus \{f\}\). 
  It is also a spanning tree.
  But \(\operatorname{cost}(T') < \operatorname{cost}(T^*)\), which is a contradiction. \textcolor{red}{\Lightning}
\end{proof}

\begin{property}[Cycle]\label{property:cycle}
Let \(C\) be any cycle, and let \(f\) be the max cost edge belonging to \(C\).
Then the MST does not contain \(f\).  
\end{property}
\begin{proof}[Exchange argument]
  Suppose \(f\) does belong to the MST \(T^*\).
  Deleting \(f\) from \(T^*\) disconnects \(S\) and \(V \setminus S\) in \(T^*\).
  Since \(C\) is a cycle there is another edge \(e\) incident to \(S\) and \(V \smallsetminus S\) and \(c_e < c_f\).
  Now let \(T' = T^* \cup \{e\} \setminus \{f\}\). 
  It is also a spanning tree.
  But \(\operatorname{cost}(T') < \operatorname{cost}(T^*)\), which is a contradiction. \textcolor{red}{\Lightning}
\end{proof}



\subsubsection*{\nameref{alg:prim}'s algorithm}

\begin{algorithm}[h]
\caption{Prim}\label{alg:prim}
\begin{algorithmic}[1]
\Function{Prim}{$G, w:E\to\R, s$} 
  \ForAll{$v\in V$} \Comment{initialization}
    \State $\attrib{v}{key}\gets\infty$
    \State $\attrib{v}{color} \gets \text{undiscovered}$
    \EndFor
    \State $\attrib{s}{key}\gets 0$ \Comment{start at root \(s\)}
    \State $\attrib{s}{\pi}\gets\nil$
  \State $Q \gets $ priority queue of all nodes \(u \in V\) sorted by \(\attrib{u}{key}\)
  \While{$Q \neq \emptyset$} \Comment{until all vertices are processed}
    \State extract \(u\) with minimal \(\attrib{u}{key}\) from \(Q\) \label{line:prim_extract_min}
    \ForAll{$v\in\Gamma(u)$}
      \If{$\attrib{v}{color} = \text{undiscovered} \AND  w(u,v) < \attrib{v}{key}$} 
        \State $\attrib{v}{key} \gets w(u,v)$  \Comment{new lighter edge for \(v\)}
        \State decrease key of \(v\) in \(Q\) to \(\attrib{v}{key}\)
        \State $\attrib{v}{\pi} \gets u$
      \EndIf
    \EndFor
    \State $\attrib{u}{color} \gets \text{finished}$
  \EndWhile
  \LComment{the \(\attribute{\pi}\) pointers define the `inverted' MST rooted at \(s\)}
\EndFunction
\end{algorithmic}
\end{algorithm}


\pagebreak[3]
Correctness:
\vspace{-\parskip}
\begin{itemize}[nosep]
  \item initialize \(S = \{\text{any node}\}\), \(T = \emptyset\)
  \item apply \hyperref[property:cut]{\hl{cut property}} to \(S\) (and repeat)
  \begin{itemize}[nosep]
    \item choose cheapest edge out of \(S\)
    \item add it to \(T\)
    \item update \(S\) (add one new explored node \(u\) to \(S\))
  \end{itemize}
  \item repeat
\end{itemize}

use a queue \(Q\) to organize the unexplored nodes 

weight of unexplored nodes: \hl[2]{cost of cheapest edge connecting it to a node in \(S\)}



% \begin{algorithm}[h]
% \caption{Prim}\label{alg:prim}
% \begin{algorithmic}[1]
% \Function{Prim}{$G, w:E\to\R, r$} 
%   \ForAll{$v\in V$} \Comment{initialization}
%     \State $\attribnormal{v}{a}\gets\infty$ 
%   \EndFor
%   \State $\attribnormal{s}{a}\gets 0$ \Comment{\(s\) is the source}
%   \State $Q \gets $ priority queue of all nodes \(u \in V\) sorted by \(\attribnormal{u}{a}\)
%   \State $S \gets \emptyset$ \Comment{initialize set of unexplored nodes}
%   \While{$Q \neq \emptyset$} \Comment{until all vertices are processed}
%     \State extract \(u\) with minimal \(\attribnormal{u}{a}\) from \(Q\) \label{line:prim_extract_min}
%     \State $S \gets S \cup \{u\}$
%     \ForAll{$v\in\Gamma(u)$}
%       \If{$v \notin S \AND w(u,v) < \attribnormal{v}{a}$} 
%         \State $\attribnormal{v}{a} \gets w(u,v)$
%         \State decrease key of \(v\) in \(Q\) to \(\attribnormal{v}{a}\)
%       \EndIf
%     \EndFor
%   \EndWhile
% \EndFunction
% \end{algorithmic}
% \end{algorithm}



The runtime analysis of \autoref{alg:prim} is exactly the same as that of \autoref{alg:dijkstra}













\subsubsection*{\nameref{alg:kruskal}'s algorithm}

\begin{algorithm}[h]
\caption{Kruskal}\label{alg:kruskal}
\begin{algorithmic}[1]
\Function{Kruskal}{$G,w:E\to\R$}
  \State $A\gets\emptyset$ \Comment{initially \(A\) is empty}
  \ForAll{$v\in V$}
  \State \Call{MakeSet}{$v$} \Comment{create set containing singleton \(v\)}
  \EndFor
  \State sort $E$ in non-decreasing order by $w$ \label{alg:kruskal:sort}
  \ForAll{$(u,v)\in E$ in non-decreasing $w$-order}
    \If{$\text{\Call{Find}{$u$}} \neq \text{\Call{Find}{$v$}}$} \Comment{are \(u\) and \(v\) in different trees?}
      \State $A\gets A\cup\{(u,v)\}$ \Comment{join subtrees together}
      \State \Call{Union}{$u,v$} \Comment{merge the connected components of \(u\) and \(v\)}
    \EndIf
  \EndFor
  % \State \Return $A$ \Comment{edges of the minimum spanning tree}
\EndFunction
\end{algorithmic}
\end{algorithm}


Correctness:
\vspace{-\parskip}
\begin{itemize}[nosep]
  \item consider edges in ascending order of weight
  \item initialize \(T = \emptyset\)
  \item Case 1: if adding \(e\) to \(T\) creates a cycle, discard \(e\) (according to \hyperref[property:cycle]{\hl{cycle property}})
  \item Case 2: insert \(e = (u, v)\) in \(T\) (according to \hyperref[property:cut]{\hl{cut property}}). unite the connected components of \(u\) and \(v\)
\end{itemize}

\(T\) is a forest (a colleciton of trees) (some trees may be \(1\) node)

to detect if \(e=(u,v)\) creates a cycle, we need to check the component \(T_u\) where \(u\) belongs; if \(v \in T_u\), we have a cycle, else no.

to check wether \(v \in T_u\), we could use e.g. DFS, which would be \(O(|T_u|)\) each time we check an edge, leading to a quadratic algorithm for Kruskal's method.

when adding \(e\) to \(T\), we merge two trees into one

\textcolor[HTML]{013399}{Disjoint-set data structure}:
\begin{itemize}
\item $\operatorname{create}(u)$: Create a set containing a single item $u$.
\item $\operatorname{find}(u)$: Find the set that contains a given item $u$.
\item $\operatorname{union}(u, v)$: Merge the set containing $u$ and the set containing $v$ into a common set
\end{itemize}

\begin{theorem}
  Given \(n\) elements, each initially in its own set, the union-find data structure can perform any sequence of up to \(n\) union and find operations in total \(O(n \, \alpha(n))\) time, where \(\alpha(n)\) is the (extremely slowly growing) inverse Ackermann function.
\end{theorem}

Union-find: \(O(\alpha(n))\) \emph{amortized time} per operation (essentially constant time)


\noindent
\textcolor{funblue}{Implementation.}
Use the \textcolor{alizarincrimsonred}{union-find} data structure.

\begin{itemize}
  \item Build the set $T$ of edges in the MST.
  \item Maintain one set for each connected component.
  \item $O(m \log \tikzmark{sort} n)$ for sorting and
        $O(m \, \tikzmark{ufL} \alpha(n)\tikzmark{ufR})$ for union-find.
\end{itemize}
\newcommand{\annotdrop}{0.5cm}  % how far below the text the note sits
\newcommand{\arrowpad}{0.1cm}   % small offset between mark and arrow target
\begin{tikzpicture}[remember picture,overlay, font=\tiny]

  % left note + arrow
  \node[anchor = north west] (note1) at ($(pic cs:sort)+(0.4,-\annotdrop)$) {$m \le n^{2} \Rightarrow \log m \text{ is } O(\log n)$}; % depth arrow
  \draw[->, outer sep=0cm, inner sep=0cm] (note1.west) -- ($(pic cs:sort)+(0,-\arrowpad)$);  % offset label
  % use variable .e.g 0.5 + 0.1 to calculate total height of the picture...

  % brace under O(m α(n)) + text
  \draw[
    decorate,
    decoration={brace,amplitude=4pt,mirror}
  ]
    ($(pic cs:ufL)+(0.01,-0.1cm)$) -- ($(pic cs:ufR)+(-0.01,-0.1cm)$)
    node[midway,below=2pt]{essentially a constant};

\end{tikzpicture}
\vspace{\dimexpr\annotdrop-0.1cm\relax}

\clearpage

Ties in line \ref{alg:kruskal:sort} can be broken by small perturbations to the weights or implicitly the index, i.e.,:
\begin{algorithm}[h]
\caption{Boolean Less}\label{alg:boolean_less}
\begin{algorithmic}[1]
\Function{BooleanLess}{$e_i, e_j$}
  \If{$w(e_i) < w(e_j)$} \Return true
  \ElsIf{$w(e_i) > w(e_j)$} \Return false
  \ElsIf{$i < j$} \Return true \Comment{break ties by index}
  \Else \ \Return false
  \EndIf
\EndFunction
\end{algorithmic}
\end{algorithm}


\begin{example}[Clustering]\label{ex:clustering_kruskal}
Given a set \(U\) of \(n\) objects labeled \(p_1, \ldots, p_n\) classify into coherent groups.

Distance function \(d: U \times U \to \R_{\geq 0}\) satisfying:
\begin{itemize}
  \item \(d(p_i, p_j) = 0 \iff i = j\) \hfill (identity or indiscernibility)
  \item \(d(p_i, p_j) \geq 0\) \hfill (non-negativity)
  \item \(d(p_i, p_j) = d(p_j, p_i)\) \hfill (symmetry)
\end{itemize}

\medskip

\hl{Spacing}. Min distance between any pair of points in different clusters:
\begin{equation}\label{eq:clustering_spacing}
\operatorname{spacing}(C) = \min \{\operatorname{spacing}(C_s, C_t) \mid \text{for any two clusters in \(C\)}\}
\end{equation}
where
\[
\operatorname{spacing}(C_s, C_t) = \min \{d(p_i, p_j) \mid p_i \in C_s, p_j \in C_t\}
\]

Given an integer \(k\), find a \(k\)-clustering maximizing the spacing, i.e., maximize the (min) distance between any two clusters.

can be solved using \nameref{alg:kruskal}'s algorithm:
\begin{itemize}
  \item consider a graph, nodes are objects, edges weighted by distance between objects
  \item run \nameref{alg:kruskal} but stop when we are left with \(k\) sets
  \item equivalently: find MST and remove the \(k-1\) most expensive edges
  \item spacing: length \(d\) of the \((k-1)^{\text{th}}\) most expensive MST edge
\end{itemize}

\begin{theorem}
  The MST clustering \(C^* = \{C_1^*, \ldots, C_k^*\}\) has max spacing, where the spacing (i.e. the minimum distance between any two clusters) \(d^*\) is the length of the \((k-1)^{\text{th}}\) most espensive MST edge.
\end{theorem}

\begin{proof}
  Let \(C = \{C_1, \ldots, C_k\}\) be any other \(k\)-clustering with spacing \(d\).
  Let \(p_i, p_j\) be two points in two different clusters of \(C\), say \(C_s\) and \(C_t\), but the same cluster in \(C^*\).
  Let \((p, q)\) be the MST edge connecting \(C_s\) and \(C_t\) (on the \(p_i\)-\(p_j\) MST path).
  The spacing between \(C_s\) and \(C_t\) is \(d \leq |(p, q)|\).
  But \(|(p, q)| \leq d^*\) (by \nameref{alg:kruskal}).
  Thus, \(d \leq |(p, q)| \leq d^*\).
\end{proof}
\vspace{-\parskip}
\end{example}







\subsection{Center Selection}
% given
% \begin{itemize}
%   \item sites / points: \(P = \{s_1, \ldots, s_n\}\)
%   \item distance function \(\operatorname{dist} : P \times P \to \R_{\ge 0}\)
% \end{itemize}

\begin{problem}[\(k\)-select]\label{problem:k_selection}
select \(k\) centers \(C = \{c_1, \ldots, c_k\}\) so that the maximum distance from a site its nearest center is minimized.
\end{problem}

\begin{problem}[\(k\)-center]\label{problem:k_center}
select \(k\) centers \(C \subseteq P\) so that the maximum distance from a site its nearest center is minimized.
(\hl{centers are located on the input sites})
\end{problem}

to call \(\delta(\cdot, \cdot)\) a distance function, we require:
\begin{itemize}
  \item \(\delta(u, v) \geq 0\) and \(\delta(u, v) = 0 \iff u = v\) \hfill (non-negativity and identity)
  \item \(\delta(u, v) = \delta(v, u)\) \hfill (symmetry)
  \item \(\delta(u, w) \leq \delta(u, v) + \delta(v, w)\) \hfill (triangle inequality)
\end{itemize}



\begin{definition}[Covering radius]\label{def:covering_radius}
Given a set of sites/points \(P = \{s_1, \ldots, s_n\}\) and a distance function \(\delta\), for a set of centers \(C\) define:
\begin{itemize}
  \item \(\delta(s_i, C) := \min_{c \in C} \delta(s_i, c)\): distance of site \(s_i\) to nearest center in \(C\)
  \item \(r(C) 
  := \max_{i} \delta(s_i, C)
  = \max_{i} \min_{c \in C} \delta(s_i, c)
  \): \hl[2]{covering radius}
  \qedhere
\end{itemize}
\end{definition}
\[
\begin{tikzpicture}[scale=0.38,font=\footnotesize]

% background
\fill[gray!15] (0,0) rectangle (16.6,9.9);

% styles
\tikzset{
  site/.style   ={fill=black, draw, minimum size=0.1cm, inner sep=0pt},
  center/.style ={fill=blue!70!black,draw=blue!30!black,  circle,
                  minimum size=0.1cm, inner sep=0pt}
}

% helper
\newcommand{\siteat}[2]{\node[site] at (#1,#2) {};}

% radius (found ≈ 1.99 units from the image)
\def\R{2}

%------------------------------------------------
% centers (from CV)
%------------------------------------------------
\coordinate (C1) at (3.71,7.19); % top-left disk
\coordinate (C2) at (12.18,6.20); % top-right disk
\coordinate (C3) at (13.38,3.46); % bottom-right disk
\coordinate (C4) at (7.52,2.65); % bottom-center disk
\coordinate (C5) at (1.45,2.30); % legend center

%------------------------------------------------
% disks + crosshair radii
%------------------------------------------------
\foreach \C in {C1,C2,C3,C4}{
  \fill[gray!35] (\C) circle (\R);
}
\foreach \C in {C1,C2,C3,C4}{
  \draw[densely dotted] (\C) circle (\R);
  \draw (\C) -- ++(0,\R);
  \draw (\C) -- ++(\R,0);
  \node[center] at (\C) {};
}

%------------------------------------------------
% sites (all coordinates from CV)
%------------------------------------------------

% around C1 (top-left)
\siteat{2.71}{8.92} % C1 + 2 ([1,0] cos(2/3 * pi * i) + [0,1] sin(2/3 * pi * i))
\siteat{4.71}{5.46} % C1 + 2 ([1,0] cos(5/3 * pi * i) + [0,1] sin(5/3 * pi * i))
\siteat{4.40}{7.89}
\siteat{2.69}{6.83}
\siteat{3.92}{6.21}
\siteat{3.17}{5.98}


% around C2 (top-right)
\siteat{11.88}{7.51}
\siteat{11.02}{7.38}
\siteat{12.76}{7.12}
\siteat{11.49}{6.63}
\siteat{11.79}{5.79}
\siteat{12.94}{5.72}
\siteat{12.55}{4.83}

% around C3 (bottom-right)
\siteat{13.87}{4.39}
\siteat{12.68}{3.89}
\siteat{11.99}{3.76}
\siteat{14.09}{2.65}

% around C4 (bottom-center)
\siteat{8.21}{3.35}
\siteat{6.50}{2.29}
\siteat{7.74}{1.67}


%------------------------------------------------
% r(C) arrow and label (for the left disk)
%------------------------------------------------
\coordinate (Rtip) at ($(C1)+(0,2/3 *\R)$);
\node[inner sep=0pt, outer sep=2pt] (label) at (7.4,9.0) {$r(C)$};
\draw[->] (label) -- (Rtip);

%------------------------------------------------
% legend box
%------------------------------------------------
\draw (0.8,0.7) rectangle (4.5,2.5);

\node[center] (cc) at (1.4,2.0) {};
\node[right=0.05cm of cc] {center};

\node[site] (ss) at (1.4,1.2) {};
\node[right=0.05cm of ss] {site};
\end{tikzpicture}
\]


\begin{remark}
For \autoref{problem:k_selection} centers can be anywhere in the space \(\Rightarrow\) search space infinite.
For (discrete) \autoref{problem:k_center} centers must be in \(P\) \(\Rightarrow\) search space finite.
% Same greedy approximation algorithm works for both; it always chooses sites as centers.
\end{remark}

both \nameref{problem:k_center} and \nameref{problem:k_selection} problem is NP-hard, so we look for a greedy approximation algorithm\\
\(\rightarrow\) approximation to \nameref{problem:k_center} also useful for the \nameref{problem:k_selection} problem




\textcolor[HTML]{660066}{Naive greedy: a false start}
\begin{itemize}
  \item idea: place first center optimally for a single center
  \item then repeatedly add a new center to reduce \(r(C)\) as much as possible
\end{itemize}
\begin{caution}\label{caution:greedy_k_center_arbitrarily_bad}
This locally optimal improvement strategy can be \emph{arbitrarily bad}, as one can see in the following small example:
\[
\begin{tikzpicture}[scale=0.27,font=\tiny]
\useasboundingbox (0,0) rectangle (20,5);
\node at (22,2.5) {\normalsize \textcolor{red}{\Lightning}};

% background
\fill[gray!15] (0,0) rectangle (20,5);

% styles
\tikzset{
  site/.style   ={fill=black, draw, minimum size=0.1cm, inner sep=0pt},
  center/.style ={fill=blue!70!black,draw=blue!30!black,  circle, minimum size=0.1cm, inner sep=0pt}
}

% helper macro for a site square
\newcommand{\siteat}[2]{\node[site] at (#1,#2) {};}

% left cluster of sites
\siteat{2.3}{3.4}
\siteat{3.0}{3.6}
\siteat{3.7}{3.4}
\siteat{2.4}{2.9}
\siteat{3.1}{3.1}
\siteat{3.8}{2.9}
\siteat{2.6}{2.4}
\siteat{3.3}{2.6}
\siteat{4.0}{2.4}
\siteat{2.5}{1.9}
\siteat{3.2}{2.1}

% right cluster of sites
\siteat{11.8}{3.1}
\siteat{12.5}{3.3}
\siteat{13.2}{3.1}
\siteat{12.0}{2.6}
\siteat{12.7}{2.8}
\siteat{13.4}{2.6}
\siteat{12.2}{2.1}
\siteat{12.9}{2.3}
\siteat{13.6}{2.1}
\siteat{12.8}{1.6}
\siteat{13.5}{1.8}

% greedy center
\node[center] (greedy_center) at (8,2.7) {};
\node[below=0.05 of greedy_center] {greedy center};

% % k text
% \node at (8,1.0) {$k = 2$ centers};


% legend

\draw[rectangle] (15.5,0.2) rectangle (19.8,2.0);
\node[center] (c) at (16.2,1.4) {};
\node[right=0.05cm of c] {center};
\node[site] (s) at (16.2,0.65) {};
\node[right=0.05cm of s] {site};

\end{tikzpicture}
\]
So, greedy placement by optimizing current covering radius does \emph{not} yield a good approximation!
\end{caution}

so to make the problem more managable, let's restrict ourselves to choosing centers from the given sites (i.e. to the discrete \nameref{problem:k_center} problem) and thus avoid a situation as in \autoref{caution:greedy_k_center_arbitrarily_bad}.

\medskip


\textcolor[HTML]{660066}{Idea: use the given sites as centers, i.e., design a greedy algorithm for the \nameref{problem:k_center} problem}
\begin{itemize}
  \item start with an arbitrary site as first center
  \item repeatedly add as next center the site that is \emph{farthest} from all existing centers
  \item interpretation: always satisfy the currently most ``unhappy'' site
\end{itemize}

\begin{algorithm}[h]
\caption{Greedy Approximation to \nameref{problem:k_center} / \nameref{problem:k_selection}}\label{alg:greedy_k_center}
\begin{algorithmic}[1]
\Require sites \(P = \{s_1,\ldots,s_n\}\), integer \(k\)
\State \(C \gets \emptyset\)
\ForAll{$s \in P$}
  \State \(\attrib{s}{dist} \gets \infty\) \Comment{distance to nearest center (initially none)}
\EndFor
\For{$i = 1 \TO k$}
  \State choose \(s^* \in P\) with maximum \(\attrib{s^*}{dist}\) \Comment{farthest site} \label{line:ci_farthest_site}
  \State \(C \gets C \cup \{s^*\}\)
  \ForAll{$s \in P$}
    \State \(\attrib{s}{dist} \gets \min\{\attrib{s}{dist}, \delta(s, s^*)\}\)
  \EndFor
\EndFor
\State \Return \(C\)
\end{algorithmic}
\end{algorithm}

% Proof of Lemma 1: As more centers are added, the distance of any point to its closest center cannot increase (it can only stay the same or decrease). Therefore, since G i ✓

% G i+1 , we have % i $ % i+1 .

% Proof of Lemma 2: When g j is added, it is at distance % i$1 from its closest center in G j$1 .

% Since i < j, g i 2 G j$1 , and therefore " (g i , g j ) $ % i$1 , and by Lemma 1, this is at least

% as large as % k = %(G).

% Proof of Lemma 3: Consider any set C of k centers whose bottleneck distance is smaller than % min . By the pigeonhole principal, there exists at least two centers g i , g j 2 G k+1 , where 1  i < j  k + 1 that are in the same neighborhood of some center c 2 C. Since %(c)  % min , by combining Lemma 2 with the triangle inequality we have

% %(G)  " (g i , g j )  " (g i , c) + " (c, g j ) < %(C) + %(C) = 2%(C).

% Therefore, %(C) > %(G)/2 = % min . Since this holds for any set of centers, it holds for the optimum set O.

\begin{lemma}\label{lem:greedy_k_center_monotonic_radius}
  During the execution of \autoref{alg:greedy_k_center}, the covering radius \(r(C)\) is monotonically decreasing, that is 
  \(
  r(C^{(i)} = \{c_1, \ldots, c_{i-1}, c_i\}) \leq r(C^{(i-1)} = \{c_1, \ldots, c_{i-1}\})
  \).
  (radii of disks decreases with each added center.)
\end{lemma}
\begin{proof}
As more centers are added, the distance of any point to its closest center cannot increase (it can only stay the same or decrease).
Therefore, since \(C^{(i-1)} \subseteq C^{(i)}\), we have 
\(
r(C^{(i)}) \le r(C^{(i-1)})
\).
\end{proof}

\begin{lemma}\label{lem:greedy_k_center_pairwise_separation}
  Upon termination of \autoref{alg:greedy_k_center}, all centers \(C = \{c_1, \ldots, c_k\}\) are pairwise at least \(r(C)\) apart.
  (no disk contains the center of any other disk.)
\end{lemma}
\begin{proof}
% When a new center \(
% c_i = s^* 
% \) is added, it is the site farthest from all existing centers (i.e. at a distance \(r(C^{(i-1)})\)).
% Thus, its distance to any existing center is at least \(r(C)\), since by \autoref{lem:greedy_k_center_monotonic_radius} we have \(r(C^{(i-1)}) \ge r(C)\).
Let \(r_i = r(C^{(i)})\) be the covering radius after the \(i\)-th center \(c_i\) is added in Line~\ref{line:ci_farthest_site} of \autoref{alg:greedy_k_center}.
By \autoref{lem:greedy_k_center_monotonic_radius}, the sequence \(r_1 \ge \cdots \ge r_k = r(C)\) is non-increasing.

When center \(c_i\) is chosen in Line~\ref{line:ci_farthest_site}, it is the site farthest from \(C_{i-1}\), thus
\[
  \delta(c_j, c_i) \ge r_i \quad \forall j < i
\]

Since \(r_i \ge r_k = r(C)\), we get \(\delta(c_{j_1}, c_{j_2}) \ge r(C)\) for all \(j_1, j_2 \in \{1, \ldots, k\}\) with \(j_1 \neq j_2\).
\end{proof}

Running time: \(O(k \, n)\) (in general \(k \le n\))


\begin{lemma}\label{lem:greedy_min}
Let \(
r_{\min} = r(C) / 2
\), where \(C\) is the set of centers returned by \autoref{alg:greedy_k_center}.
Then for any set \(C'\) of \(k\) centers, we have
\begin{equation}\label{eq:greedy_min}
  r(C') \ge r_{\min} = \frac{r(C)}{2}
\end{equation}
(it is not possible to cover all points of \(P\) using \(k\) disks whose radii are smaller than \(r_{\min}\))
\end{lemma}
\begin{proof}
Let \(C\) be the set of centers returned by \autoref{alg:greedy_k_center}, and let \(r(C)\) be its covering radius.
% If \(r(C) = 0\), then \eqref{eq:greedy_min} is trivial, so assume \(r(C) > 0\).
Choose a site \(s^\star \in P\) with
\(
  \delta(s^\star, C) = r(C)
\).
By definition of \(\delta(s^\star, C)\), we have
\(
  \delta(s^\star, c) \ge r(C)
\) for all \(c \in C\).
Moreover, by \autoref{lem:greedy_k_center_pairwise_separation}, any two centers
\(c_{j_1}, c_{j_2} \in C\) satisfy \(\delta(c_{j_1}, c_{j_2}) \ge r(C)\).
Thus, every pair of distinct points in the set
\(
  A := C \cup \{s^\star\}
\)
is at distance at least \(r(C)\).
Note that \(|A| = k+1\).

Now let \(C'\) be any set of \(k\) centers, with covering radius \(r(C')\).
Assign each point in \(A\) to its nearest center in \(C'\).
Since \(|A| = k+1\) but \(|C'| = k\),
by the pigeonhole principle there are two distinct points
\(x, y \in A\) assigned to the same center \(c' \in C'\).
Then
\(
  \delta(x, c') \le r(C')
\)
and
\(
  \delta(y, c') \le r(C')
\)
and by the triangle inequality,
\[
  \delta(x, y)
  \le
  \delta(x, c') + \delta(c', y)
  \le
  2 r(C')
\]
On the other hand, since \(x, y \in A\) and \(x \neq y\),
we have \(\delta(x, y) \ge r(C)\).
Therefore
\[
  r(C) \le 2 r(C')
\]
which proves \eqref{eq:greedy_min}.
\end{proof}

\begin{theorem}\label{thm:greedy_k_center_2_approx}
Let \(C\) be the set of centers returned by \autoref{alg:greedy_k_center} and let \(C^*\) be an optimal set of \(k\) centers.
Then
\[
{\color{Red}\boxed{\color{black}
  r(C) \le 2 r(C^*)
}}
\]
i.e., \autoref{alg:greedy_k_center} is a \(2\)-approximation for both the \nameref{problem:k_center} and \nameref{problem:k_selection} problems.
\end{theorem}
\begin{proof}
By \autoref{lem:greedy_min}, we have
\(r(C^*) \ge r_{\min} = r(C) / 2
\), 
which can be rearranged to give the desired result.
\end{proof}



\begin{remark}
The greedy algorithm \emph{always} chooses centers at sites, but it is still within a factor \(2\) of the optimal
continuous center-selection solution that may place centers anywhere in the metric space.
\end{remark}

\begin{theorem}\label{thm:k_center_hardness}
Unless P = NP, there is no polynomial-time approximation algorithm for the \nameref{problem:k_center} or \nameref{problem:k_selection} problem with approximation factor \(\rho < 2\).
\end{theorem}

% \begin{caution}
% Greedy is essentially best possible in the worst case: improving the approximation factor below \(2\) would solve an NP-hard problem.
% \end{caution}






\clearpage
% !TEX root = ../algo-summary.tex
\section{Dynamic Programming}\label{sec:dynamic_programming}


\textcolor{AccentBlue}{Greedy}. 
Build up a solution incrementally, myopically optimizing some local criterion.

\medskip

\textcolor{AccentBlue}{Divide-and-conquer}. 
Break up a problem into two sub-problems of roughly equal size; 
solve each sub-problem independently; 
combine the two sub-problem solutions to form solution to original problem.

\medskip

\textcolor{AccentRed}{Dynamic programming}. 
Break up a problem into many overlapping sub-problems; 
combine solutions to small subproblems to build up solutions to larger and larger sub-problems.
\begin{itemize}
\item overlapping sub-problem = a sub-problem whose results can be reused many times
\item Hint: express your optimal solution by a recursive formula
\end{itemize}

\medskip

\textcolor{AccentBlue}{Some famous dynamic programming algorithms}:
\begin{itemize}
  \item Viterbi for Hidden Markov Models
  \item Unix diff for comparing two files
  \item Smith-Waterman for sequence alignment
  \item \nameref{alg:bellmanford} for shortest path routing in networks
  \item Cocke-Kasami-Younger for parsing context-free grammars
\end{itemize}


\bigskip


Powerful technique for solving optimization problems, which have certain well-defined clean structural properties.

\bigskip

\begin{definition}[Principle of optimality]\label{def:principle_of_optimality}
\hl[5]{For the global problem to be solved optimally, each subproblem must be solved optimally.}
\end{definition}
\indent \(\rightarrow\) this principle must hold to apply DP

\bigskip
\smallskip

The global optimal solution consists of optimal solutions to subproblems.

\medskip
Polynomial time DP:
\begin{itemize}
\item Polynomially-many distinct overlapping subproblems (polynomial in input size).
\item DP does not necessarily lead to a polynomial time algorithm
\end{itemize}
\medskip

DP selection principle: \hl{When given a set of feasible options to choose from, try them all and take the best}












\subsection{Weighted Interval Scheduling}








We denote by \(p(j)\) the largest index \(i\) with \(i < j\) and such that job \(i\) is compatible with \(j\), i.e.
\begin{equation}\label{eq:interval_scheduling_pj}
  p(j)
  :=
  \begin{cases}
    \max\{ i \mid f_i \leq s_j\} & \exists i \text{ with } f_i \leq s_j\\
    0 & \text{otherwise}
  \end{cases}
\end{equation}



\begin{algorithm}[h]
\caption{Compute \(p(j)\) values}\label{alg:compute_pj}
\begin{algorithmic}[1]
\Require Jobs \(1, \ldots, n\) sorted according to their finish time \(f_1 \leq \cdots \leq f_n\)
\State Sort all starting and finish times in a single array \(A\)
\State \(\texttt{index} \gets 0\)
\ForAll{\(e \in A\)} \Comment{go through all times (`events') in ascending order}
  \If{\(e = f_i\)}
    \State \(\texttt{index} \gets i\)
  \ElsIf{\(e = s_j\)}
    \State \(p[j] \gets \texttt{index}\)
  \EndIf
\EndFor
\end{algorithmic}
\end{algorithm}

The algorithm goes through every time slot, and it remembers the index of the last finish time.
When we encounter a starting time $s_j$, its $p$ value is the largest index $i$ such that $f_i \leq s_j$, which is exactly the value of \texttt{index} in the algorithm. 
Hence, \autoref{alg:compute_pj} is correct.

Sorting takes $O(n \log n)$ time, and it is easy to see that the for loop takes $O(n)$ time.



% \textcolor{AccentBlue}{Dynamic programming formulation.}
Assume jobs are labeled so that \(f_1 \le \cdots \le f_n\).
Let \(v_j\) be the value/weight of job \(j\), and let
\(
\operatorname{OPT}(j)
\)
denote the maximum total weight of a subset of mutually compatible jobs among \(1,\ldots,j\).
We use the convention \(\operatorname{OPT}(0) = 0\).

\textcolor{AccentBlue}{Recurrence.}
For \(j \ge 1\) we have
\[
\operatorname{OPT}(j)
=
\max\bigl\{
  v_j + \operatorname{OPT}(p(j)),\;
  \operatorname{OPT}(j-1)
\bigr\}
\]
\begin{itemize}
  \item Either the optimal solution does \emph{not} take job \(j\): then its value is \(\operatorname{OPT}(j-1)\).
  \item Or the optimal solution \emph{does} take job \(j\): then we gain \(v_j\) plus the best compatible subset among jobs \(\{1,\ldots,p(j)\}\), which has value \(\operatorname{OPT}(p(j))\).
\end{itemize}

% To reconstruct an optimal set of jobs, start from \(j = n\) and trace backwards:
% \begin{itemize}
%   \item if \(v_j + \operatorname{OPT}[p[j]] \ge \operatorname{OPT}[j-1]\) then take job \(j\) and continue with \(p(j)\);
%   \item otherwise skip job \(j\) and continue with \(j-1\).
% \end{itemize}


\begin{algorithm}[h]
\caption{Weighted Interval Scheduling}\label{alg:weighted_interval_scheduling_dp}
\begin{algorithmic}[1]
\Require jobs \(1,\ldots,n\) sorted by finish time, values \(v_j\), and precomputed \(p(j)\) as in \eqref{eq:interval_scheduling_pj}
\State \(\operatorname{OPT}[0] \gets 0\)
\For{$j = 1 \TO n$}
  \State \(\operatorname{OPT}[j] \gets \max\{\,v_j + \operatorname{OPT}[p[j]],\; \operatorname{OPT}[j-1]\,\}\)
\EndFor
\State \Return \(\operatorname{OPT}[n]\)
\end{algorithmic}
\end{algorithm}

\textcolor{AccentBlue}{Running time.}
Computing the \(\operatorname{OPT}[j]\) table takes \(O(n)\) time once the jobs are sorted and the \(p(j)\) values are known.
Together with sorting and the \(p(j)\)-computation, the total running time is \(O(n \log n)\) and the space usage is \(O(n)\).


\subsection{Segmented Least Squares}
Given \(n\) points \((x_1,y_1),\ldots,(x_n,y_n)\) in the plane (assume \(x_1 < \cdots < x_n\)),
we want to approximate them by a small number of straight-line segments.

\begin{itemize}
  \item For any \(1 \le i \le j \le n\), consider fitting a single line to the points
        \((x_i,y_i),\ldots,(x_j,y_j)\) by least squares.
  \item Let \(e(i,j)\) be the sum of squared vertical errors of this best-fit line:
        \[
          e(i,j)
          = \sum_{k=i}^{j}
            \bigl(y_k - (\hat a_{ij} x_k + \hat b_{ij})\bigr)^2,
        \]
        where \(\hat a_{ij}, \hat b_{ij}\) are the least-squares coefficients.
  \item Each segment we use incurs a fixed penalty \(C > 0\) (to discourage using too many segments).
\end{itemize}

\textcolor{AccentBlue}{Goal.}
Partition the sequence of points into contiguous blocks
\(
[1,\ell_1], [\ell_1+1,\ell_2], \ldots, [\ell_{m-1}+1,n]
\)
and fit each block with its own least-squares line so as to minimize
\[
  \sum_{r=1}^{m} e(\ell_{r-1}+1,\ell_r) + m C
  \quad
  (\ell_0 := 0)
\] 

\textcolor{AccentBlue}{Precomputation.}
Using prefix sums of \(x_k, y_k, x_k^2, x_k y_k\), all \(e(i,j)\) values can be computed in total \(O(n^2)\) time. 

Define
\[
\operatorname{OPT}(j)
:= \text{minimum total cost for approximating points } (x_1,y_1),\ldots,(x_j,y_j)
\]
using any number of segments, and set \(\operatorname{OPT}(0) := 0\).

\textcolor{AccentBlue}{Recurrence.}
For \(j \ge 1\),
\[
\operatorname{OPT}(j)
=
\min_{1 \le i \le j}
\bigl\{
  \operatorname{OPT}(i-1) + C + e(i,j)
\bigr\}.
\]
\begin{itemize}
  \item The last segment in an optimal solution for the first \(j\) points must cover
        some suffix \(i,\ldots,j\).
  \item Its cost is \(e(i,j)\) plus the penalty \(C\), plus the optimal cost for the prefix \(1,\ldots,i-1\).
\end{itemize}

\begin{algorithm}[h]
\caption{Segmented Least Squares via DP}\label{alg:segmented_least_squares}
\begin{algorithmic}[1]
\Require points \((x_1,y_1),\ldots,(x_n,y_n)\) with \(x_1 < \cdots < x_n\), segment penalty \(C\)
\State precompute all \(e(i,j)\) for \(1 \le i \le j \le n\)
\State \(\operatorname{OPT}[0] \gets 0\)
\For{$j = 1 \TO n$}
  \State \(\operatorname{OPT}[j] \gets \infty\)
  \For{$i = 1 \TO j$}
    \State \(\operatorname{OPT}[j] \gets
           \min\bigl\{
             \operatorname{OPT}[j],\;
             \operatorname{OPT}[i-1] + C + e(i,j)
           \bigr\}\)
  \EndFor
\EndFor
\State \Return \(\operatorname{OPT}[n]\)
\end{algorithmic}
\end{algorithm}

\textcolor{AccentBlue}{Running time.}
Precomputing all errors \(e(i,j)\) takes \(O(n^2)\) time.
The DP itself also takes \(O(n^2)\) time and \(O(n)\) space for the \(\operatorname{OPT}\) array.
Backtracking through the choices of \(i\) at each \(j\) yields the optimal segmentation.


\subsection{Knapsack Problem}

\begin{problem}[0/1 Knapsack Problem]\label{prob:knapsack}
Given \(n\) items, each with a weight \(w_i \in \N\) and a value \(v_i \in \N\), and a ``knapsack'' with capacity \(W \in \N\),
fill the knapsack (respecting its capacity) to maximize the total value.
\end{problem}

0/1 refers to the fact that the items are not divisible, i.e. each item can either be included (1) or excluded (0) from the knapsack, but not fractionally included (otherwise it would be the fractional knapsack problem, which can be solved greedily by taking items in descending order of value-to-weight ratio).

\[
\begin{tikzpicture}[line join=round, line cap=round, font=\sffamily\footnotesize, scale=0.4]

% ---------- helpers ----------
\definecolor{bagyellow}{RGB}{248,188,46}
\definecolor{baggold}{RGB}{230,160,40}
\definecolor{bagred}{RGB}{194,62,40}

% book(pic): (x,y), angle, fill, price, weight, scale
\newcommand{\book}[7]{%
    \colorlet{bookbase}{#4}
    \colorlet{bookback}{bookbase!60!black}
    \colorlet{bookspine}{bookbase!50!white}
  \begin{scope}[shift={(#1,#2)},rotate=#3,scale=#7,]
    % cover
    \fill[rounded corners=0pt,fill=bookbase] (-1.9,-2.6) rectangle (1.9,2.6);
    \fill[rounded corners=0pt,fill=bookback] (-1.9,-2.6) rectangle (-1.6,2.6);
    % pages strip
    \fill[rounded corners=0pt,white] (-1.7,-2.5) rectangle (1.9,-2.2);
    % \draw[black!12, very thin] (-1.65,-2.45) -- (-1.65,-2.25);
    \foreach \y in {-2.45, -2.4, -2.35, -2.3, -2.25}{%
      \draw[black!12, very thin] (-1.65,\y) -- (1.9,\y);
    }
    % text
    \node[align=center,scale=1.2, rotate=#3] at (0,0.5) {\$#5};
    \node[align=center,scale=1.1, rotate=#3] at (0,-0.4) {#6 kg};
  \end{scope}%
}

% ---------- backpack ----------
% side pockets
\fill[bagred,rounded corners=6pt] (-3.6,-3) rectangle (-2.6,0.6);
\fill[bagred,rounded corners=6pt] ( 2.6,-3) rectangle ( 3.6,0.6);

% top handle
\draw[line width=6pt,bagred] (0.55,4.1) arc (0:180:0.55 and 0.48);
% \draw[line width=8pt,bagred] (0,4.1) arc (90:270:0.55 and 0.48);


% main body
\fill[baggold!85!black,rounded corners=6pt] (-3.0,-3.4) rectangle (3.0,4.1);
\fill[bagyellow!95!white,rounded corners=6pt] (-3.0,-2.4) rectangle (3.0,4.1);


% front pocket
\fill[bagyellow!95!white] (-2.0,-0.4) -- (2.0, -0.4) [rounded corners=5pt] -- (2.0,-3) [rounded corners=5pt] -- (-2, -3) [rounded corners=0pt] -- cycle;
\fill[baggold!85!black] (-2.0,-0.4) -- (2.0, -0.4)  -- (2.0,-1.5)  .. controls (1,-2) and (-1,-2) .. (-2, -1.5) -- cycle;
\fill[black] (0,-1.25) circle (0.2);

% flap
\fill[baggold!85!black] (-2.6,4.1) -- (2.6, 4.1) [rounded corners=5pt] -- (2.6, 1) .. controls (1.0, 0) and (-1, 0) .. (-2.6, 1) [rounded corners=0pt] -- cycle;


% shoulder straps + buckles
\fill[rounded corners=2pt,bagred] (-2.3,4.2) rectangle (-1.7,0.2);
\fill[rounded corners=2pt,bagred] ( 2.3,4.2) rectangle ( 1.7,0.2);
\draw[black,rounded corners=2pt, line width=2pt] (-2.4,1.4) rectangle (-1.6,2.2);
\draw[black,rounded corners=2pt, line width=2pt] ( 1.6,1.4) rectangle (2.4,2.2);

% weight label
\node[font=\sffamily] at (0,2.5) {15 kg};

% ---------- books ----------
\book{-6}{2.0}{20}{green!40}{4}{12}{0.8}
\book{-6.2}{-2.0}{-15}{gray!30}{2}{1}{0.4}
\book{-9}{0}{15}{orange!40}{1}{1}{0.5}
\book{5.7}{2.6}{-25}{cyan!35}{2}{2}{0.6}
\book{5.7}{-1.0}{25}{yellow!60}{10}{4}{0.7}

\end{tikzpicture}
\]


% \[
% \operatorname{opt}(i, w) 
% =
% \begin{cases}
% 0 & \text{if } i = 0\\
% \operatorname{opt}(i-1, w) & \text{if } w_i > w\\
% \max\{ v_i + \operatorname{opt}(i-1, w - w_i), \operatorname{opt}(i-1, w)\} & \text{otherwise}
% \end{cases}
% \]
% % where \(i\) is the index of the current item, \(w\) is the remaining weight capacity, \(w_i\) is the weight of item \(i\) and \(v_i\) is the value of item \(i\).     



\definecolor{knapblue}{RGB}{0,130,130}
\definecolor{knapred}{RGB}{180,0,0}
recursive formulation:
\[
\operatorname{opt}(i, w)
=
\left\{
\begin{array}{@{}l@{\quad}l@{\quad}l@{}}
{0}
  & {i = 0}
  & \textcolor{knapred}{\triangleright\ \text{no items left}}\\[0.3em]
{\operatorname{opt}(i-1, w)}
  & {w_i > w}
  & \textcolor{knapred}{\triangleright\ \text{item too heavy}}\\[0.3em]
{\max\{v_i + \operatorname{opt}(i-1, w - w_i),\ \operatorname{opt}(i-1, w)\}}
  & {\text{otherwise}}
  & \textcolor{knapred}{\triangleright\ \text{include or exclude item $i$?}}
\end{array}
\right.
\]



% \begin{algorithmic}[1]
% \Function{opt}{$i, w$}
%   \If{$i = 0$} \Comment{no items left}
%     \State \Return \(0\)
%   \ElsIf{$w_i > w$} \Comment{item too heavy}
%     \State \Return \Call{opt}{$i-1, w$}
%   \Else \Comment{include or exlude item \(i\)?}
%     \State \Return \(\max\{ v_i + \Call{opt}{i-1, w - w_i}, \Call{opt}{i-1, w} \}\)
%   \EndIf
% \EndFunction
% \end{algorithmic}



% \[
% \operatorname{opt}(i,w) 
% =
% \begin{cases}
% 0 & \text{if } i = 0 \text{ (no items left)}\\[0.3em]
% \operatorname{opt}(i-1, w) & \text{if } w_i > w \text{ (too heavy)}\\[0.3em]
% \max\{ v_i + \operatorname{opt}(i-1, w - w_i), \operatorname{opt}(i-1, w)\}
%   & \text{otherwise (include or exclude item $i$?)}
% \end{cases}
% \]



top down memoized:
\begin{algorithm}[h]
\caption{Knapsack (top-down, memoized)}\label{alg:knapsack_topdown}
\begin{algorithmic}[1]
\Function{Opt}{$i, w$}
  \If{\(M[i,w] = \nil\)} 
    \If{\(i = 0\)} 
    \State \(M[i,w] \gets 0\) \Comment{no items left}
    \ElsIf{\(w_i > w\)} 
    \State \(M[i,w] \gets \Call{Opt}{i-1, w}\) \Comment{item too heavy}      
    \Else 
    \State \(M[i,w] \gets \max\{ v_i + \Call{Opt}{i-1, w - w_i}, \Call{Opt}{i-1, w} \}\) \Comment{include or exclude?}
    \EndIf
  \EndIf
  \State \Return \(M[i,w]\)
\EndFunction
\end{algorithmic}
\end{algorithm}


\textcolor{AccentBlue}{Bottom-up (tabulation) version.}
Instead of recursion + memoization, we can fill a DP table iteratively.
Let \(W\) be the knapsack capacity.
We build a table \(\operatorname{OPT}[i,w]\) for \(i = 0,\ldots,n\) and \(w = 0,\ldots,W\),
where
\(
\operatorname{OPT}[i,w]
\)
denotes the maximum value achievable using items \(1,\ldots,i\) with capacity \(w\).


\begin{algorithm}[h]
\caption{Knapsack (bottom-up DP)}\label{alg:knapsack_bottomup}
\begin{algorithmic}[1]
\Require items \(1,\ldots,n\) with weights \(w_i\), values \(v_i\); capacity \(W\)
\For{$w = 0 \TO W$}
  \State \(\operatorname{OPT}[0,w] \gets 0\) \Comment{no items}
\EndFor
% \For{$i = 0 \TO n$}
%   \State \(\operatorname{OPT}[i,0] \gets 0\) \Comment{zero capacity}
% \EndFor
\For{$i = 1 \TO n$}
  \For{$w = 0 \TO W$}
    \If{$w_i > w$}
      \State \(\operatorname{OPT}[i,w] \gets \operatorname{OPT}[i-1,w]\) \Comment{item \(i\) too heavy}
    \Else
      \State \(\operatorname{OPT}[i,w] \gets
        \max\bigl\{
          v_i + \operatorname{OPT}[i-1, w - w_i],\;
          \operatorname{OPT}[i-1, w]
        \bigr\}\)
    \EndIf
  \EndFor
\EndFor
\State \Return \(\operatorname{OPT}[n,W]\)
\end{algorithmic}
\end{algorithm}

\textcolor{AccentBlue}{Running time.}
The table has \((n+1)(W+1) = O(nW)\) entries, 
and each is filled in \(O(1)\) time, so the running time is \(O(nW)\) and the space is \(O(nW)\).
By keeping only two rows at a time, the space can be reduced to \(O(W)\).

\begin{remark}
0/1 Knapsack is NP-hard in general, so we do not expect a polynomial-time algorithm in the input size.
The DP runs in time polynomial in \(n\) and \(W\), which is called \emph{pseudo-polynomial} time,
because \(W\) is exponential in the number of bits needed to represent it (i.e. the input size).
\end{remark}


\textcolor{AccentBlue}{Approximation algorithm}.
There exists a polynomial algorithm that produces a feasible solution that has value within 0.01\% of the optimum.






\subsection{Sequence Alignment}\label{sec:sequence_alignment}

\textcolor{AccentBlue}{Goal}.
Given two strings \(X = x_1x_2\cdots x_m\) and \(Y = y_1y_2\cdots y_n\),
find an {alignment of minimum total cost}.


\begin{definition}[Alignment]\label{def:alignment}
An \textcolor{AccentRed}{alignment} \(M\) between two strings \(X\) and \(Y\) is a set of ordered pairs \((x_i, y_j)\) such that each item occurs in at most one pair and no two pairs cross.
\end{definition}

\begin{definition}\label{def:cross}
Two pairs \((x_i, y_j)\) and \((x_{i'}, y_{j'})\) \textcolor{AccentRed}{cross} if \(i < i'\) but \(j > j'\).
\end{definition}

\begin{definition}[Cost Model]\label{def:cost_model}
% \textcolor{AccentBlue}{Cost model}.
Aligning \(x_i\) with \(y_j\) incurs a \emph{mismatch} penalty \(\alpha_{x_i y_j}\).
Leaving a character unmatched (aligning it with a gap ``\(-\)'') incurs a \emph{gap} penalty \(\delta > 0\).
\end{definition}
  % The \hl{edit distance} between \(X\) and \(Y\) is the minimum cost over all alignments.

\begin{definition}[Cost]\label{def:alignment_cost}
The \textcolor{AccentRed}{cost} of an alignment \(M\) is
\[
  \operatorname{cost}(M)
  =
  \underbrace{\sum_{(x_i, y_j) \in M} \alpha_{x_i y_j}}_{\text{mismatch}} 
  + 
  \underbrace{\sum_{i \colon \text{\(x_i\) unmatched}} \delta+\sum_{j \colon \text{\(y_j\) unmatched}} \delta}_{\text {gap }}
\]
i.e. it is the sum of all gap and mismatch penalties.
\end{definition}

\begin{definition}[Edit distance]\label{def:edit_distance}
The \textcolor{AccentRed}{edit distance} between strings \(X\) and \(Y\) is the minimum \nameref{def:alignment_cost} over all \nameref{def:alignment}s  
\end{definition}

\textcolor{AccentBlue}{DP formulation}.
Let
\(
\operatorname{OPT}(i,j)
\) 
denote the minimum cost of aligning the prefixes \(x_1\cdots x_i\) and \(y_1\cdots y_j\).
If either of the prefixes is empty and the other has length \(k\), the only option is to align all characters of the other prefix with gaps, incurring a cost of \(\delta\) per gap, i.e.
\(\operatorname{OPT}(k,0) = \operatorname{OPT}(0,k) = k \cdot \delta\).
Otherwise we have three choices:
\begin{itemize}
  \item Case 1: match \(x_i\) with \(y_j\): \(\alpha_{x_i y_j} + \operatorname{OPT}(i-1,j-1)\)
  \begin{itemize}
    \item[-] pay mismatch cost \(\alpha_{x_i y_j}\) plus whatever the minimum cost is for aligning the prefixes \(x_1\cdots x_{i-1}\) and \(y_1\cdots y_{j-1}\)
  \end{itemize}
  \item Case 2a: leave \(x_i\) unmatched (align with gap): \(\delta + \operatorname{OPT}(i-1,j)\)
  \begin{itemize}
    \item[-] pay gap cost \(\delta\) for \(x_i\) plus min cost for aligning \(x_1\cdots x_{i-1}\) and \(y_1\cdots y_{j}\)
  \end{itemize}
  \item Case 2b: leave \(y_j\) unmatched (align with gap): \(\delta + \operatorname{OPT}(i,j-1)\)
  \begin{itemize}
    \item[-] pay gap cost \(\delta\) for \(y_j\) plus min cost for aligning \(x_1\cdots x_{i}\) and \(y_1\cdots y_{j-1}\)
  \end{itemize}
\end{itemize}


We get the recurrence:
\begin{equation}\label{eq:sequence_alignment_recurrence}
\operatorname{OPT}(i,j)
=
\begin{cases}
j \cdot \delta & \text{if \(i = 0\)}\\
i \cdot \delta & \text{if \(j = 0\)}\\
\min
\begin{cases}
\alpha_{x_i y_j} + \operatorname{OPT}(i-1,j-1)\\
\delta + \operatorname{OPT}(i-1,j)\\
\delta + \operatorname{OPT}(i,j-1)
\end{cases}
& \text{otherwise}
\end{cases}
\end{equation}

\medskip
To compute \(\operatorname{OPT}(m,n)\):
\begin{itemize}
\item build up the values of \(\operatorname{OPT}(i,j)\) using the recurrence -- fill an \(m \times n\) table
\item \(O(mn)\) subproblems
\item to compute \(\operatorname{OPT}(i,j)\) we need: \(\operatorname{OPT}(i-1,j)\), \(\operatorname{OPT}(i-1,j-1)\), \(\operatorname{OPT}(i,j-1)\)
\item can fill table \emph{row by row} or \emph{column by column}
\item cost we are looking for will be in bottom-right cell \(\operatorname{OPT}(m,n)\)
\end{itemize}


\begin{algorithm}[h]
\caption{Sequence Alignment}\label{alg:sequence_alignment}
\begin{algorithmic}[1]
\Require strings \(X=x_1\cdots x_m\), \(Y=y_1\cdots y_n\); gap penalty \(\delta\); mismatch costs \(\alpha_{pq}\)
\For{$i = 0 \TO m$}
  \State \(M[i,0] \gets i\delta\)
\EndFor
\For{$j = 0 \TO n$}
  \State \(M[0,j] \gets j\delta\)
\EndFor
\For{$i = 1 \TO m$}
  \For{$j = 1 \TO n$}
    \State \(M[i,j] \gets \min\bigl\{\alpha_{x_i y_j}+M[i-1,j-1], \delta+M[i-1,j], \delta+M[i,j-1]\bigr\}\) \label{line:seqalign_dp_decision}
  \EndFor
\EndFor
\State \Return \(M[m,n]\)
\end{algorithmic}
\end{algorithm}

\textcolor{AccentBlue}{Running time}.
The table has \((m+1)(n+1) = O(mn)\) entries and each entry is computed in \(O(1)\),
so the running time is \(O(mn)\) and the space usage is \(O(mn)\).


\begin{remark}
We can reduce the space to \(O(\min\{m,n\})\) if only the optimal cost is needed by keeping only two rows/columns in memory at a time while ``filling the table''.
Because to compute \(M[i,j]\) we only need \(M[i-1,j-1]\), \(M[i-1,j]\), and \(M[i,j-1]\).
\end{remark}


\textcolor{AccentBlue}{Recovering an optimal alignment}.
Two options:
\begin{itemize}
\item Do a postprocessing step after filling the table:
Starting at \((m,n)\), backtrack to \((0,0)\) by comparing \(M[i,j]\) with the three candidate predecessors:
\((i-1,j-1)\), \((i-1,j)\), \((i,j-1)\).
Record the corresponding aligned characters (\(x_i\) vs. \(y_j\), \(x_i\) vs. ``\(-\)'', or ``\(-\)'' vs. \(y_j\)).
Ties correspond to multiple optimal alignments.
\item Store another array (hint table):
Every time we make a decision in \autoref{line:seqalign_dp_decision} of \autoref{alg:sequence_alignment}, 
keep a hint of where we came from.
\end{itemize}


% \clearpage










\subsection{Longest Common Subsequence} \label{sec:lcs}
Let us think of character strings as sequences of characters. 
Given $X=\left\langle x_1, \ldots, x_l\right\rangle$, we say that $Z=\left\langle z_1, \ldots, z_k\right\rangle$ is a subsequence of $X$ if there is a strictly increasing sequence of $k$ indices $\left\langle i_1, \ldots, i_k\right\rangle$ such that $Z=\left\langle x_{i_1}, \ldots, x_{i_k}\right\rangle$. 

Given two strings \(X=\left\langle x_1, \ldots, x_m\right\rangle\) and \(Y = \left\langle y_1, \ldots, y_n \right\rangle\), the longest common subsequence of $X$ and $Y$ is a longest sequence $Z$ that is a subsequence of both $X$ and $Y$. 

\subsubsection{Brute force}
A brute force approach would be to enumerate all subsequences of $X$ and check for each whether it is also a subsequence of $Y$. 
There are $2^m$ subsequences of \(X\) and checking wether it is a subsequence of \(Y\) takes \(O(n)\) time:
Scan \(Y\) for first occurence, from there scan for second, and so on.
In total this takes \(\Theta(n 2^m)\) time.


\subsubsection{Dynamic programming formulation}
A prefix \(X_i\) of a sequence \(X\) is just an initial string of values, \(X_i = \langle x_1, \ldots, x_i \rangle\).
\(X_0\) is the empty sequence.
The idea is to compute the longest common subsequence for every possible pair of prefixes (there are \(O(mn)\) such pairs).

Let $\operatorname{lcs}(i, j)$ denote the length of the longest common subsequence of $X_i$ and $Y_j$.
We have the following recursive relation:
\[
\operatorname{lcs}(i, j)
= 
\begin{cases}
  0 & \text { if } i=0 \text { or } j=0 \\ 
  \operatorname{lcs}(i-1, j-1)+1 & \text { if } i, j>0 \text { and } x_i=y_j \\ 
  \max (\operatorname{lcs}(i-1, j), \operatorname{lcs}(i, j-1)) & \text { if } i, j>0 \text { and } x_i \neq y_j
\end{cases}
\]


Last characters match:
\[
\begin{tikzpicture}[scale=0.5, font=\footnotesize]

%--- left strings Xi, Yj ------------------------------------------
% Xi row
\node at (0,1.2) {$X_i$};
\draw[fill=gray!30] (0.6,0.8) rectangle (3.6,1.6);   % prefix of Xi
\draw              (3.6,0.8) rectangle (4.4,1.6);     % last char
\node at (4.0,1.2) {$A$};
\node at (4.0,2.0) {$x_i$};

% Yj row
\node at (0,-1.2) {$Y_j$};
\draw[fill=gray!30] (0.6,-1.6) rectangle (2.8,-0.8);  % prefix of Yj
\draw              (2.8,-1.6) rectangle (3.6,-0.8);   % last char
\node at (3.2,-1.2) {$A$};
\node at (3.2,-0.4) {$y_j$};

%--- branching + arrow --------------------------------------------
\coordinate (c) at (5.2,0);
\draw (4.8,1.0) -- (c) -- (4.8,-1.0);
\draw[->] (c) -- (7.4,0);

%--- right strings Xi-1, Yj-1 -------------------------------------
% Xi-1 row
\node at (7.4,1.2) {$X_{i-1}$};
\draw[fill=gray!30] (8.2,0.8) rectangle (11.2,1.6);

% Yj-1 row
\node at (7.4,-1.2) {$Y_{j-1}$};
\draw[fill=gray!30] (8.2,-1.6) rectangle (10.4,-0.8);
% vertical lcs arrow and label
\draw[<->] (8.5,0.8) -- (8.5,-0.8);
\node[right] at (8.5,0) {$\operatorname{lcs}(i-1,j-1)$};

%--- "+1" and two A boxes -----------------------------------------
\node[left] at (14,0) {$+1$};

\draw (13.6,0.8)  rectangle (14.4,1.6);
\draw (13.6,-1.6) rectangle (14.4,-0.8);
\node at (14.0,1.2) {$A$};
\node at (14.0,-1.2) {$A$};
\draw[<->] (14.0,0.8) -- (14.0,-0.8);

\end{tikzpicture}
\]


Last characters do not match:
\[
\begin{tikzpicture}[scale=0.5, font=\footnotesize]

%--- left strings Xi, Yj ------------------------------------------
% Xi row
\node[left] at (0.6,1.2) {$X_i$};
\draw[fill=gray!30] (0.6,0.8) rectangle (3.6,1.6);   % prefix of Xi
\draw              (3.6,0.8) rectangle (4.4,1.6);     % last char
\node at (4.0,1.2) {$A$};
\node at (4.0,2.0) {$x_i$};

% Yj row
\node[left] at (0.6,-1.2) {$Y_j$};
\draw[fill=gray!30] (0.6,-1.6) rectangle (2.8,-0.8);  % prefix of Yj
\draw              (2.8,-1.6) rectangle (3.6,-0.8);   % last char
\node at (3.2,-1.2) {$B$};
\node at (3.2,-0.4) {$y_j$};

%--- branching + arrow --------------------------------------------
\coordinate (c) at (5.2,0);
\draw (4.8,1.0) -- (c) -- (4.8,-1.0);
\draw[->] (c) -- (6.6,0);
\node[above] at (5.9,0) {$\max$};
\draw (6.6,3.0) -- (6.6,-3.0);
\draw[->] (6.6,3) -- (8.4,3);
\draw[->] (6.6,-3) -- (8.4,-3);


\begin {scope}[shift={(1, 3)}]
%--- right strings Xi-1, Yj-1 -------------------------------------
% Xi-1 row
\node[left] at (8.2,1.2) {$X_{i-1}$};
\draw[fill=gray!30] (8.2,0.8) rectangle (11.2,1.6);

\draw (12,0.8) rectangle (12.8,1.6);
\node at (12.4,1.2) {$A$};

% cross
\draw[line width=0.6pt] (11.9,0.7) -- (12.9,1.7);
\draw[line width=0.6pt] (11.9,1.7) -- (12.9,0.7);
\node[right] at (13,1.2) {skip $x_i$};

% Yj-1 row
\node[left] at (8.2,-1.2) {$Y_{j}$};
\draw[fill=gray!30] (8.2,-1.6) rectangle (10.4,-0.8);
\draw[fill=gray!30] (10.4,-1.6) rectangle (11.2,-0.8);
\node at (10.8,-1.2) {$B$};
% vertical lcs arrow and label
\draw[<->] (8.5,0.8) -- (8.5,-0.8);
\node[right] at (8.5,0) {$\operatorname{lcs}(i-1,j)$};
\end{scope}

\begin {scope}[shift={(1, -3)}]
%--- right strings Xi-1, Yj-1 -------------------------------------
% Xi-1 row
\node[left] at (8.2,1.2) {$X_{i}$};
\draw[fill=gray!30] (8.2,0.8) rectangle (11.2,1.6);
\draw[fill=gray!30] (11.2,0.8) rectangle (12.0,1.6);
\node at (11.6,1.2) {$A$};

% Yj-1 row
\node[left] at (8.2,-1.2) {$Y_{j-1}$};
\draw[fill=gray!30] (8.2,-1.6) rectangle (10.6,-0.8);

\draw (12.0,-1.6) rectangle (12.8,-0.8);
\node at (12.4,-1.2) {$B$};

% cross
\draw[line width=0.6pt] (11.9,-1.7) -- (12.9,-0.7);
\draw[line width=0.6pt] (11.9,-0.7) -- (12.9,-1.7);
\node[right] at (13, -1.2) {skip $y_j$};


% vertical lcs arrow and label
\draw[<->] (8.5,0.8) -- (8.5,-0.8);
\node[right] at (8.5,0) {$\operatorname{lcs}(i,j-1)$};
\end{scope}


\end{tikzpicture}
\]





\textcolor{AccentBlue}{DP table.}
We store values in a table \(L[0\ldots m, 0\ldots n]\) where
\(L[i,j] = \operatorname{lcs}(i,j)\).
We can go recursive/top-down with memoization (\autoref{alg:lcs_topdown}) or iterative/bottom-up (\autoref{alg:lcs}).


\begin{algorithm}[h]
\caption{Longest Common Subsequence (top-down DP)} \label{alg:lcs_topdown}
\begin{algorithmic}[1]
\Function{LCS}{$i, j$}
  \If{\(L[i,j] = \nil\)}
    \If{\(i = 0 \OR j = 0\)} \Comment{base case}
      \State \(L[i,j] \gets 0\)
    \ElsIf{\(x_i = y_j\)} \Comment{last characters match}
      \State \(L[i,j] \gets \Call{LCS}{i-1, j-1} + 1\)
    \Else \Comment{last characters do not match}
      \State \(L[i,j] \gets \max\{\Call{LCS}{i-1, j}, \Call{LCS}{i, j-1}\}\)
    \EndIf
  \EndIf
  \State \Return \(L[i,j]\) \Comment{return stored value}
\EndFunction
\end{algorithmic}
\end{algorithm}


\begin{algorithm}[h]
\caption{Longest Common Subsequence (bottom-up DP)}\label{alg:lcs}
\begin{algorithmic}[1]
\Require strings \(X = x_1\cdots x_m\), \(Y = y_1\cdots y_n\)
\State \(L \gets \text{new array \([0 \ldots m, 0 \ldots n]\)}\)
\For{$i = 0 \TO m$}
  \State \(L[i,0] \gets 0\)
\EndFor
\For{$j = 0 \TO n$}
  \State \(L[0,j] \gets 0\)
\EndFor
\For{$i = 1 \TO m$} \Comment{iterate over rows}
  \For{$j = 1 \TO n$} \Comment{iterate over columns}
    \If{$x_i = y_j$} 
      \State \(L[i,j] \gets L[i-1,j-1] + 1\) \Comment{take \(x_i = y_j\) for LCS}
    \Else
      \State \(L[i,j] \gets \max\{L[i-1,j], L[i,j-1]\}\)
    \EndIf
  \EndFor
\EndFor
\State \Return \(L[m,n]\)
\end{algorithmic}
\end{algorithm}




\textcolor{AccentBlue}{Running time} of \autoref{alg:lcs}.
The table has \((m+1)(n+1) = O(mn)\) entries, each filled in \(O(1)\) time,
so the running time is \(O(mn)\) and the space is \(O(mn)\).




\subsubsection{Reconstructing the actual subsequence}

To retrieve the actual subsequence, we can use another table, a ``hint'' table \(h[0\ldots m, 0\ldots n]\) that records which of the three cases was used to compute each \(L[i,j]\) (\autoref{alg:lcs_with_hints}).

\begin{algorithm}[h]
\caption{LCS (bottom-up) with hints}\label{alg:lcs_with_hints}
\begin{algorithmic}[1]
\Require strings \(X = x_1\cdots x_m\), \(Y = y_1\cdots y_n\)
\State \(L \gets \text{new array \([0 \ldots m, 0 \ldots n]\)}\) \Comment{stores lcs lengths}
\State \(h \gets \text{new array \([0 \ldots m, 0 \ldots n]\)}\) \Comment{hint table}
\For{$i = 0 \TO m$} \Comment{initialize column 0}
  \State \(L[i,0] \gets 0\)
  \State \(h[i,0] \gets \text{skipX}\)
\EndFor
\For{$j = 0 \TO n$} \Comment{initialize row 0}
  \State \(L[0,j] \gets 0\)
  \State \(h[0,j] \gets \text{skipY}\)
\EndFor
\For{$i = 1 \TO m$}
  \For{$j = 1 \TO n$}
    \If{$x_i = y_j$}
      \State \(L[i,j] \gets L[i-1,j-1] + 1\)
      \State \(h[i,j] \gets \text{addXY}\)
    \Else
      \If{$L[i-1,j] \ge L[i,j-1]$}
        \State \(L[i,j] \gets L[i-1,j]\)
        \State \(h[i,j] \gets \text{skipX}\)
      \Else
        \State \(L[i,j] \gets L[i,j-1]\)
        \State \(h[i,j] \gets \text{skipY}\)
      \EndIf
    \EndIf
  \EndFor
\EndFor
\State \Return \(L[m,n]\), \(h\)
\end{algorithmic}
\end{algorithm}


Now we can use the hints to reconstruct the LCS:
\begin{itemize}
  \item start at the last entry of the table, \([m,n]\)
  \item if \(h[i,j] = \text{addXY}\) (\(\nwarrow\)), then \(x_i = y_j\) is appended to LCS, continue with \([i-1,j-1]\)
  \item if \(h[i,j] = \text{skipX}\) (\(\uparrow\)), then \(x_i\) is not in LCS, continue with \([i-1,j]\)
  \item if \(h[i,j] = \text{skipY}\) (\(\leftarrow\)), then \(y_j\) is not in LCS, continue with \([i,j-1]\)
  \item stop when reaching \([0,0]\)
\end{itemize}

Because the characters are generated in revers order, prepend them to a sequence, so that when we are done, the sequence is in correct order.
Or use recursion to print them in correct order directly (\autoref{alg:print_lcs}).

\begin{algorithm}[h]
\caption{recursive LCS printing using hints}\label{alg:print_lcs}
\begin{algorithmic}[1]
\Function{Print-LCS}{$h, X, i, j$}
  \If{\(i = 0 \OR j = 0\)}
    \State \Return
  \EndIf
  \If{\(h[i,j] = \text{addXY}\)}
    \State \Call{Print-LCS}{$h, X, i-1, j-1$}
    \State {print} \(x_i\)
  \ElsIf{\(h[i,j] = \text{skipX}\)}
    \State \Call{Print-LCS}{$h, X, i-1, j$}
  \Else
    \State \Call{Print-LCS}{$h, X, i, j-1$}
  \EndIf
\EndFunction
\end{algorithmic}
\end{algorithm}





Alternatively, the hints can be discovered based on the filled \(L\) table (\autoref{alg:print_lcs_discovered}, \autoref{alg:print_lcs_discovered_alt}).
If \(x_i = y_j\) and \(L[i,j] = L[i-1,j-1]+1\), then \(x_i\) is part of some LCS and we move to \([i-1,j-1]\); 
otherwise, if \(L[i-1,j] \ge L[i,j-1]\) move to \([i-1,j]\), else move to \([i,j-1]\).


\begin{algorithm}[h]
\caption{recursive LCS printing discovering hints}\label{alg:print_lcs_discovered}
\begin{algorithmic}[1]
\Function{Print-LCS}{$L, X, Y, i, j$}
  \If{\(i = 0 \OR j = 0\)}
    \State \Return
  \EndIf
  \If{\(x_i = y_j \AND L[i,j] = L[i-1,j-1]+1\)}
    \State \Call{Print-LCS}{$L, X, Y, i-1, j-1$} \Comment{take \(x_i = y_j\)}
    \State {print} \(x_i\)
  \ElsIf{\(L[i-1,j] \ge L[i,j-1]\)}
    \State \Call{Print-LCS}{$L, X, Y, i-1, j$} \Comment{skip \(x_i\)}
  \Else
    \State \Call{Print-LCS}{$L, X, Y, i, j-1$} \Comment{skip \(y_j\)}
  \EndIf
\EndFunction
\end{algorithmic}
\end{algorithm}


\begin{algorithm}[h]
\caption{recursive LCS printing discovering hints (alternative condition)}\label{alg:print_lcs_discovered_alt}
\begin{algorithmic}[1]
\Function{Print-LCS}{$L, X, i, j$}
  \If{\(i = 0 \OR j = 0\)}
    \State \Return
  \EndIf
  \If{\(L[i,j] = L[i-1,j]\)}
    \State \Call{Print-LCS}{$L, X, i-1, j$} \Comment{move up i.e. skip \(x_i\)}
  \ElsIf{\(L[i,j] = L[i,j-1]\)}
    \State \Call{Print-LCS}{$L, X, i, j-1$} \Comment{move left i.e. skip \(y_j\)}
  \Else \Comment{\(L[i,j]\) must be different from all 3 neighbors, so \(x_i = y_j\)}
    \State \Call{Print-LCS}{$L, X, i-1, j-1$} \Comment{move diagonally, take \(x_i = y_j\)}
    \State {print} \(x_i\)
  \EndIf
\EndFunction
\end{algorithmic}
\end{algorithm}

 \autoref{alg:print_lcs}, \autoref{alg:print_lcs_discovered} and \autoref{alg:print_lcs_discovered_alt} are called with initial parameters \(i = m\) and \(j = n\).




\begin{remark}
Again, we can reduce the space to \(O(\min\{m,n\})\) if only the length of the LCS is needed by keeping only two rows/columns in memory at a time while ``filling the table''.
Because to compute \(L[i,j]\) we only need \(L[i-1,j-1]\), \(L[i-1,j]\), and \(L[i,j-1]\).
\end{remark}



\subsection{Sequence Alignment in linear space}\label{sec:seqalign_lcs_linear_space}
There is a cool trick that allows linear space without compromising the time complexity asymptotically.
It combines Dynamic Programming with Divide and Conquer.
The same trick can be used for \nameref{sec:lcs}.

Can we avoid using quadratic space?

\textcolor{AccentBlue}{easy:} \textcolor{AccentRed}{optimal value}: \(O(m + n)\) space and \(O(mn)\) time.
\begin{itemize}
  \item use a \(2 \times n\) matrix. compute \(\operatorname{OPT}(i, \cdot)\) from \(\operatorname{OPT}(i-1, \cdot)\)
  \item but no longer a simple way to recover the alignment itself
\end{itemize}

\begin{theorem}[Hirschberg 1975]\label{thm:hirschberg}
Optimal alignment in \(O(m + n)\) space and \(O(mn)\) time. Clever combination of divide-and-conquer and dynamic programming.
\end{theorem}

A DP matrix can be seen as a directed acyclic graph (DAG).
The optimal alignment corresponds to the shortest path in the DAG from node \((0,0)\) to \((m,n)\).

\textcolor{AccentBlue}{Edit distance graph}:
Add a node for every entry \(M[i,j]\) and an edge from \(M[i-1,j-1]\), \(M[i-1,j]\), \(M[i,j-1]\) to \(M[i,j]\) with weights according the \nameref{def:cost_model}:
\[
\begin{tikzpicture}[
  scale=1.5,
  font=\footnotesize,
  dpnode/.style={circle, minimum size=3mm, inner sep=0pt, draw=gray!60, fill=gray!20},
  startend/.style={dpnode, minimum size=6mm, draw=black, fill=none},
  focus/.style={dpnode, minimum size=6mm, draw=black, fill=red!20},
  ed/.style={-Latex, draw=gray!55, line width=0.6pt},
  edD/.style={ed, densely dotted}, % dashed "collapsed" connections
  edF/.style={ed, color=black}
]

% --- nodes (create ALL nodes once, with correct style/label)
\foreach \r in {0,1,2,3,4,5}{
  \foreach \c in {0,1,2,3,4,5,6,7}{
    \def\nodestyle{dpnode}
    \def\nodelabel{}%

    % special nodes (same names!)
    \ifnum\r=0 \ifnum\c=0 \def\nodestyle{startend}\def\nodelabel{$0$-$0$}\fi\fi
    \ifnum\r=3 \ifnum\c=4 \def\nodestyle{focus}\def\nodelabel{$i$-$j$}\fi\fi
    \ifnum\r=5 \ifnum\c=7 \def\nodestyle{startend}\def\nodelabel{$m$-$n$}\fi\fi

    \node[\nodestyle] (n\r\c) at (\c,-\r) {\nodelabel};
  }
}

% ---- horizontal edges (skip n33 -> n34)
\foreach \r in {0,1,2,3,4,5}{
  \foreach \c in {0,1,2,3,4,5,6}{
    \pgfmathtruncatemacro{\cp}{\c+1}

    % choose style
    \def\estyle{ed}
    \ifnum\c=2 \def\estyle{edD}\fi
    \ifnum\c=4 \def\estyle{edD}\fi

    % skip only if (r,c)=(3,3)
    \ifnum\r=3
      \ifnum\c=3
        % skip
      \else
        \draw[\estyle] (n\r\c) -- (n\r\cp);
      \fi
    \else
      \draw[\estyle] (n\r\c) -- (n\r\cp);
    \fi
  }
}

% ---- vertical edges (skip n24 -> n34)
\foreach \r in {0,1,2,3,4}{
  \pgfmathtruncatemacro{\rp}{\r+1}
  \foreach \c in {0,1,2,3,4,5,6,7}{

    % choose style
    \def\estyle{ed}
    \ifnum\r=1 \def\estyle{edD}\fi
    \ifnum\r=3 \def\estyle{edD}\fi

    % skip only if (r,c)=(2,4)
    \ifnum\r=2
      \ifnum\c=4
        % skip
      \else
        \draw[\estyle] (n\r\c) -- (n\rp\c);
      \fi
    \else
      \draw[\estyle] (n\r\c) -- (n\rp\c);
    \fi
  }
}

% ---- diagonal edges (skip n23 -> n34)
\foreach \r in {0,1,2,3,4}{
  \pgfmathtruncatemacro{\rp}{\r+1}
  \foreach \c in {0,1,2,3,4,5,6}{
    \pgfmathtruncatemacro{\cp}{\c+1}

    % choose style
    \def\estyle{ed}
    \ifnum\r=1 \def\estyle{edD}\fi
    \ifnum\r=3 \def\estyle{edD}\fi
    \ifnum\c=2 \def\estyle{edD}\fi
    \ifnum\c=4 \def\estyle{edD}\fi

    % skip only if (r,c)=(2,3)
    \ifnum\r=2
      \ifnum\c=3
        % skip
      \else
        \draw[\estyle] (n\r\c) -- (n\rp\cp);
      \fi
    \else
      \draw[\estyle] (n\r\c) -- (n\rp\cp);
    \fi
  }
}

% ---- highlighted incoming edges into (i,j) = n34
\draw[edF] (n24) -- node[midway, right] {$\delta$} (n34);                 % (i-1,j)
\draw[edF] (n33) -- node[midway, above] {$\delta$} (n34);                 % (i,j-1)
\draw[edF] (n23) -- node[pos=0.42, above, inner sep=0pt, outer sep=0pt, anchor=south west] {$\alpha_{x_i y_j}$} (n34); % (i-1,j-1)


\node[above=2.5mm] at (n00) {$\varepsilon$};
\node[left=2.5mm]  at (n00)  {$\varepsilon$};

\node[above=2.5mm] at (n01) {$ y_1$};
\node[above=2.5mm] at (n02) {$ y_2$};
\node[above=2.5mm] at (n03) {$ y_{j-1}$};
\node[above=2.5mm] at (n04) {$ y_j$};
\node[above=2.5mm] at (n05) {$ y_{n-2}$};
\node[above=2.5mm] at (n06) {$ y_{n-1}$};
\node[above=2.5mm] at (n07) {$ y_n$};

\node[left=2.5mm] at (n10) {$ x_1$};
\node[left=2.5mm] at (n20) {$ x_{i-1}$};
\node[left=2.5mm] at (n30) {$ x_i$};
\node[left=2.5mm] at (n40) {$ x_{m-1}$};
\node[left=2.5mm] at (n50) {$ x_m$};

% ---- pünktchen on axes at the collapsed gaps
\node[above=2.5mm] at ($(n02)!0.5!(n03)$) {$\cdots$};
\node[above=2.5mm] at ($(n04)!0.5!(n05)$) {$\cdots$};

\node[left=2.5mm]  at ($(n10)!0.5!(n20)$) {$\vdots$};
\node[left=2.5mm]  at ($(n30)!0.5!(n40)$) {$\vdots$};

\end{tikzpicture}
\]


Let \(f(i,j)\) be the (length of) shortest path from \((0,0)\) to \((i,j)\) and \(g(i,j)\) be the (length of) shortest path from \((i,j)\) to \((m,n)\).
\begin{observation}\label{obs:length_shortest_path_equals_opt}
 \(f(i,j) = M[i,j] = \operatorname{OPT}(i,j)\). 
\end{observation}
\begin{observation}\label{obs:opt_decomposition_node_on_shortest_path}
Let \(P\) be the shortest path from \((0,0)\) to \((m,n)\).
Then 
\begin{equation}\label{eq:opt_decomposition_node_on_shortest_path}
\operatorname{OPT}(m,n) = f(i,j) + g(i,j)
\end{equation}
for any node \((i,j)\) that lies on the path \(P\).
\end{observation}


\begin{observation}[Computing \(f(\cdot, j)\)]\label{obs:compute_f}
Given a column \(j\), we can compute the value of \(f(\cdot, j)\) = \(\operatorname{OPT}[\cdot, j]\) in \(O(mn)\) time and \(O(m + n)\) space.
\begin{itemize}
  \item[-] run space efficient sequence alignment to compute \(\operatorname{OPT}(m,n)\)
  \item[-] keep the entries of column \(j\) in a separate array
  \qedhere
\end{itemize}
\end{observation}

We can compute \(g(i,j)\) by reversing the edge orientations (formula indices) and inverting the roles of \((0,0)\) and \((m,n)\).
\begin{itemize}
  \item[-] we start with \(g(m,n) = 0\) and reverse indices in the recursive formula \eqref{eq:sequence_alignment_recurrence} for \(\operatorname{OPT}(i,j)\)
\end{itemize}

\medskip

The backward function \(g(i,j)\):
\begin{itemize}
  \item \(g(m,n) = 0\)
  \item \(g(m,j) = (n-j) \cdot \delta\); \(g(i,n) = (m-i) \cdot \delta\)
  \item substitute \(i-1\) by \(i+1\) and \(j-1\) by \(j+1\) in \eqref{eq:sequence_alignment_recurrence} for \(\operatorname{OPT}(i,j)\)
\end{itemize}

\begin{observation}[Computing \(g(\cdot, j)\)]\label{obs:compute_g}
Given a column \(j\), we can compute the values of \(g(\cdot, j)\) in \(O(mn)\) time and \(O(m + n)\) space.
\begin{itemize}
  \item[-] run space efficient sequence alignment to compute \(\operatorname{OPT}(0,0)\), (reversing the role of nodes \((m,n)\) and \((0,0)\) and reversing indices) 
  \item[-] keep the entries of column \(j\) in a separate array
  \qedhere
\end{itemize}
\end{observation}


Given a column \(j\), compute \(f(\cdot, j) + g(\cdot, j)\) in a separate array for all \(i\) (i.e. the entire column).
Let \(q\) be the index of the minimum
\begin{equation}\label{eq:q_minimizing_f_plus_g}
{\color{Red}\boxed{\color{black}
q \coloneq \argmin_{i \in [1 \ldots m]} \bigl(f(i,j) + g(i,j)\bigr)
}}
\end{equation}
then 
\begin{equation}\label{eq:f_plus_g_equals_opt}
{\color{Red}\boxed{\color{black}
f(q,j) + g(q,j) = \operatorname{OPT}(m,n)
}}
\end{equation}
i.e. we have found one pair $x_q$-$y_j$ (alignment \(x_q\)-\(y_j\) / \(x_q\)-gap / \(y_j\)-gap) in \(O(mn)\) time and \(O(m + n)\) space! % :-)

\medskip

Now we just need to put everything together using divide and conquer principle:
\begin{itemize}
  \item Divide \& Conquer: choose column \(j = n/2\)
  \item Compute \(f(\cdot, n/2)\). Compute \(g(\cdot, n/2)\). Compute \(f(\cdot, n/2) + g(\cdot, n/2)\)
  \item Let \(q\) be an index that minimizes \(f(i,n/2) + g(i,n/2)\)
  \item Store the pair \(x_q\)-\(y_{n/2}\) (possible alignment \(x_q\)-\(y_{n/2}\) / \(x_q\)-gap / \(y_{n/2}\)-gap)
  \item Recurse for \(X[1\ldots q]\), \(Y[1\ldots n/2]\) (top left subproblem) and \(X[q \ldots m]\), \(Y[n/2 \ldots n]\) (bottom right subproblem)
\end{itemize}

\medskip


\textcolor{AccentBlue}{Divide}:
Compute \(f(i,n/2)\) and \(g(i,n/2)\), for all \(i\), using two space-efficient DP algorithms (\autoref{obs:compute_f} and \autoref{obs:compute_g}) in \(O(mn)\) time and \(O(m+n)\) space.
\begin{itemize}
  \item find the index \(q\) that minimizes \(f(i, n/2) + g(i, n/2)\)
  \item store pair \(x_q\)-\(y_{n/2}\) (align \(x_q\)-\(y_{n/2}\) or \(x_q\)-gap or \(y_{n/2}\)-gap)
\end{itemize}

\smallskip
\textcolor{AccentBlue}{Conquer}:
recursively compute optimal alignment in top left and bottom right pieces.



\begin{algorithm}[h]
\caption{Divide-and-Conquer-Alignment (linear space)}
\label{alg:divide_and_conquer_alignment}
\begin{algorithmic}[1]
\Function{Divide-and-Conquer-Alignment}{$X,Y$}
  \State \(m \gets |X|\) \Comment{number of symbols in \(X\)}
  \State \(n \gets |Y|\) \Comment{number of symbols in \(Y\)}
  \If{\(m \le 2 \OR n \le 2\)}
    \State compute optimal alignment using \textsc{Alignment}(\(X,Y\))
  \EndIf
  \State \Call{Forward-Space-Efficient-Alignment}{\(X,Y[1:n/2]\)} \Comment{array \(f(\cdot,n/2)\)} \label{line:forward_alignment}
  \State \Call{Backward-Space-Efficient-Alignment}{\(X,Y[n/2 : n]\)} \Comment{array \(g(\cdot,n/2)\)} \label{line:backward_alignment}
  \State \(q \gets \argmin_{i \in [1 \ldots m]} \bigl(f(i,n/2) + g(i,n/2)\bigr)\) \Comment{applying \eqref{eq:q_minimizing_f_plus_g}}
  \State add pair \((q,n/2)\) to global list \(P\)
  \State \Call{Divide-and-Conquer-Alignment}{\(X[1:q],Y[1:n/2]\)} \label{line:recurse_top_left}
  \State \Call{Divide-and-Conquer-Alignment}{\(X[q:m],Y[n/2:n]\)} \label{line:recurse_bottom_right}
  \State \Return \(P\)
\EndFunction
\end{algorithmic}
\end{algorithm}

What is the running time of \autoref{alg:divide_and_conquer_alignment}?
For sure it takes at least as long as \autoref{alg:sequence_alignment} because we need to compute \(f(\cdot, n/2)\) and \(g(\cdot, n/2)\) at each level of recursion.


Let \(T(m,n)\) be the maximum running time of \autoref{alg:divide_and_conquer_alignment} on strings \(X\) and \(Y\) of length \(m\) and \(n\).
Then
\[
T(m,n) = \underbrace{T(q,n/2)}_{\text{Line~\ref{line:recurse_top_left}}} + \underbrace{T(m-q,n/2)}_{\text{Line~\ref{line:recurse_bottom_right}}} + \underbrace{O(mn)}_{\text{Lines~\ref{line:forward_alignment},\ref{line:backward_alignment}}}
\]
for some \(q \in [1 \ldots m]\) (see \autoref{eq:q_minimizing_f_plus_g}). 
We can also upper bound \(T(m,n)\) as
\[
T(m,n) \le 2 T(m,n/2) + O(mn) \overset{\text{\eqref{eq:master_theorem}}}\implies T(m,n) = O(mn \log n) 
\]
but this analysis is not tight, actually \(T(m,n) = O(mn)\).
We increase time by only a constant factor.
But we reuse space during recursive calls.



\begin{theorem}
\(T(m,n) = O(mn)\).
\end{theorem}
\begin{proof}[Induction on \(n\)]
Choose a constant \(c > 0\) such that
\begin{align}
T(m, 2) &\le cm \label{eq:DaC_base_case_m}\\
T(2, n) &\le cn \label{eq:DaC_base_case_n}\\
T(m, n) &\le cmn + T(q, n/2) + T(m-q, n/2) \label{eq:DaC_recurrence}
\end{align}
for all \(m,n \ge 1\) and all \(q \in [1 \ldots m]\).

\medskip

We prove by induction on \(n\) the statement
\[
P(n): \quad T(m,n) \le 2cmn \quad \forall m \ge 1
\]

\begin{enumerate}[label=(\roman*)]\label{proof:time_complexity_divide_and_conquer_alignment}
\item \textbf{Base cases:}
If \(m = 2\) or \(n = 2\), then by \eqref{eq:DaC_base_case_m} and \eqref{eq:DaC_base_case_n} we have
\[
T(m,n) \le \max\{cm,cn\} \le 2cmn
\quad
\text{\textcolor{Green}{\ding{52}}}
\]

\item \textbf{Induction hypothesis:}
Assume \(P(\tilde n)\) holds for all \(\tilde n < n\), i.e.
\[
T(\tilde m, \tilde n) \le 2 c \tilde m \tilde n
\]
for all \(\tilde m \ge 1\) and all \(\tilde n < n\).
\label{proof:time_complexity_divide_and_conquer_alignment:induction_hypothesis}

\item \textbf{Induction step:}
Fix any \(m \ge 1\). Then
\[
\begin{aligned}
T(m,n)
&\overset{\text{\eqref{eq:DaC_recurrence}}}\le cmn + T(q,n/2) + T(m-q,n/2)\\
&\overset{\text{\ref{proof:time_complexity_divide_and_conquer_alignment:induction_hypothesis}}}{%
\le}
cmn +
2c\cdot q\cdot \frac n2 + 2c\cdot (m-q)\cdot \frac n2\\
&= 
cmn +
cqn + c(m-q)n\\
&=
cmn +
cmn\\
&= 2cmn \quad \text{\textcolor{Green}{\ding{52}}}
\end{aligned}
\]
Since \(m\) was arbitrary, this proves \(P(n)\).
\end{enumerate}

Hence \(T(m,n) = O(mn)\).
\end{proof}

























\pagebreak[3]
\subsection{Shortest Paths with Negative Weights}



For graphs with possibly negative edge weights (but no negative-weight cycles reachable from the source),
\nameref{alg:dijkstra}'s greedy algorithm no longer works.
% The \nameref{alg:bellmanford} algorithm is a dynamic-programming, label-correcting method that works for arbitrary real edge weights and detects negative-weight cycles.






\begin{caution}
It may be tempting to use \nameref{alg:dijkstra} with negative weights by adding a constant to all edge weights to make them non-negative.
However, this approach does not preserve shortest paths (the new shortest paths may differ from the original ones).
\end{caution}


\begin{observation}\label{obs:negative_cost_cycle}
If some path from \(s\) to \(t\) contains a negative cost cycle,
then there does not exist a well-defined shortest \(s\)-\(t\) path (cost can be made arbitrarily small by traversing the negative cycle multiple times).
If there is no negative cost cycle, the shortest path is simple (no cycle) and has at most \(n-1\) edges
(otherwise some vertex would be repeated, forming a cycle that could be removed to yield a shorter path).
\end{observation}


% We want to find a recursive DP formulation for the cost of shortest path from a node \(v\) to the target \(t\) (equiv. from the source \(s\) to a node \(v\)).

Recursively formulate \(\operatorname{OPT}(v, ?)\): length of shortest \(v\)-\(t\) path.
\begin{itemize}
  \item Think of a parameter to facilitate the recursive formulation of a shortest path.
  \item Idea: use number of edges \(i\).
\end{itemize}
\begin{notation}
\(\operatorname{OPT}(i, v)
\coloneq
\text{length of shortest \(v\)-\(t\) path \(P\) using \emph{at most} \(i\) edges}\).
\end{notation}


compute \(\operatorname{OPT}(n-1,v)\) for every node \(v\).

\textcolor{AccentBlue}{Goal}: \(\operatorname{OPT}(n-1,s)\)

\medskip

We need to distinguish the following cases:
\begin{itemize}
  \item Case 1: \(P\) uses \(< i\) (\(\le i-1\)) edges: \(\operatorname{OPT}(i, v) = \operatorname{OPT}(i-1, v)\)
  \item Case 2: \(P\) uses exactly \(i\) edges.
  \begin{itemize}
  \item[-] 
  Let \((v, w)\) be the first edge out of \(v\).
  \(\operatorname{OPT}\) uses \((v, w)\) and selects best \(w\)-\(t\) path using at most \(i-1\) edges: \(\operatorname{OPT}(i-1, w)\).
  \item[-] 
  Choose minimum among all possibilities of edges incident to \(v\).
  \end{itemize}
  \item Base case (\(i=0\)): \(\operatorname{OPT}(0,v) = \{\infty \text{ if } v \neq t; 0 \text{ if } v = t\}\)
\end{itemize}
\medskip

We thus have the following recursive formula:
\begin{equation}\label{eq:bellmanford}
\operatorname{OPT}(i, v)
=
\begin{cases}
0  \text{ if } v = t; \;
\infty \text{ if } v \neq t
& \text{if } i = 0\\
\min\left\{
\operatorname{OPT}(i-1, v),\;
{\displaystyle \min_{(v,w) \in E} \left\{\operatorname{OPT}(i-1, w) + c_{vw}\right\}}
\right\}
& \text{otherwise}
\end{cases}
\end{equation}

Assuming there is no negative cycle from \(v\) to \(t\),
then \(\operatorname{OPT}(n-1, v)\) is the length of the shortest \(v\)-\(t\) path
(by \autoref{obs:negative_cost_cycle})

Thus, \(\operatorname{OPT}(n-1, s)\) is the length of the shortest path from \(s\) to \(t\)
(assuming there is no negative cycle from \(s\) to \(t\)).


\begin{algorithm}[h]
\caption{Shortest Path (1st attempt)}\label{alg:shortest_path_first_attempt}
\begin{algorithmic}[1]
\Function{Shortest-Path}{$G,t$}
  \ForAll{$v \in V$}
    \State \(M[0,v] \gets \infty\)
  \EndFor
  \State \(M[0,t] \gets 0\)
  \For{$i = 1 \TO n-1$}
    \ForAll{$v \in V$}
      \State \(M[i,v] \gets M[i-1,v]\)
      \ForAll{$(v,w) \in E$} \Comment{use adjacency list of node \(v\)}
        \State \(M[i,v] \gets \min\{M[i,v], M[i-1,w] + c_{vw}\}\)
      \EndFor
    \EndFor
  \EndFor
\EndFunction
\end{algorithmic}
\end{algorithm}

\textcolor{AccentBlue}{Analysis} of \autoref{alg:shortest_path_first_attempt}:
\begin{itemize}
\item Time: \(\Theta(mn)\), since \(\sum_{v \in V} \deg^+(v) = m\)
\item Space: \(\Theta(n^2)\) for the \(n \times n\) table \(M\)
\end{itemize}



% \subsubsection{Space improvement}


% slide 12

We can improve the space usage by keeping only two rows of of the table \(M\) and different from \nameref{sec:sequence_alignment},
we do not need to come up with a different algorithm to reconstruct the shortest paths.

We only need to maintain the latest ``successor'' for each node.
% We only need to know the best known distance to each node so far and who its successor is on the path achieving that distance.
No need for a full matrix of hints.
Just additional \(\Theta(n)\) space for a successor array.


So we  use a \(2 \times n\) table plus a \(n\)-array for the successors.
After \(i\) iterations, \(M[2,v]\) is length of shortest \(v\)-\(t\) path using \(\le i\) edges 
and we can use the successor array to retrieve the path.



\subsubsection{Practical improvement: one array only}

We can do even better.
Maintain only one \emph{single} \(n\)-array \(M[1\ldots n]\) and update it \textcolor{AccentRed}{in place}.

In practice, this not only reduces space but potentially also time,
because we do not insist to at most use \(i\) edges, but allow the best we can so far using any number of edges.



\begin{observation}\label{obs:bf_one_array_invariant}
Throughout the algorithm, each value \(M[v]\) is the length of \emph{some} \(v\)-\(t\) path (or \(\infty\)).
After \(i\) rounds of updates, we have
\(
M[v] \le \operatorname{OPT}(i,v)
\)
for all \(v\).
But it could be much smaller because a sequence of updates may discover a very good path that uses \emph{more than} \(i\) edges
(this is why the one-array version is a ``practical improvement'' rather than a faithful DP table).
\end{observation}


So with this approach we only need two \(n\)-arrays, one for the values and one for the successors.
In the worst case we still have the same time complexity, but usually it converges faster doing this trick.

% slide 13

\subsubsection{Practical improvement: early termination}

\begin{observation}\label{obs:no_need_to_check_edges_to_nodes_that_did_not_change}
No need to check edges of the form \((v, w)\) unless \(M[w]\) changed in the previous iteration.
\end{observation}


% slide 14
\begin{observation}\label{obs:if_no_change_stop}
Once no change occurs at an iteration \(i\), the algorithm can stop:
\begin{itemize}
  \item no change at iteration \(i\) means: \(\operatorname{OPT}(i,v) = \operatorname{OPT}(i-1,v)\) for all \(v\).
  \item the values of \(\operatorname{OPT}(i+1,v)\) are computed using \(\operatorname{OPT}(i,v)\).
  \item then \(\operatorname{OPT}(i+1,v) = \operatorname{OPT}(i,v) = \operatorname{OPT}(i-1,v)\), for all \(v\).
  \item thus, no change at iteration \(i+1\).
  \item by induction, no change for any \(j > i\).
  \qedhere
\end{itemize}
\end{observation}



% slide 17 (slide 16 does not exist)
\begin{algorithm}[h]
\caption{Bellman-Ford}\label{alg:bellmanford_push}
\begin{algorithmic}[1]
\Function{Push-Based-Shortest-Path}{$G,t$}
  \ForAll{$v \in V$}
    \State \(M[v] \gets \infty\) \Comment{distance array}
    \State \(\texttt{succ}[v] \gets \nil\) \Comment{successor array}
  \EndFor
  \State \(M[t] \gets 0\)
  \For{$i = 1 \TO n-1$}
    \ForAll{$w \in V$}
    \If{\(M[w]\) has been updated in previous iteration} \Comment{\autoref{obs:no_need_to_check_edges_to_nodes_that_did_not_change}}
      \ForAll{$(v,w) \in E$} \Comment{incoming edges into \(w\)}
        \If{\(M[w] + c_{vw} < M[v]\)}
          \State \(M[v] \gets M[w] + c_{vw}\)
          \State \(\texttt{succ}[v] \gets w\)
        \EndIf
      \EndFor
    \EndIf
    \EndFor
    \If{no \(M[v]\) changed in this iteration} \Comment{\autoref{obs:if_no_change_stop}}
      \State \Return
    \EndIf
  \EndFor
\EndFunction
\end{algorithmic}
\end{algorithm}
% end slide 17


% slide 15
\autoref{alg:bellmanford_push} implements the Bellman-Ford algorithm with the practical improvements discussed above.
\begin{itemize}
\item Memory: \(O(m + n)\).
\item Running time: \(O(mn)\) worst case, but substantially faster in practice: once no change on \(M\), the iteration stops.
\item The path whose length is \(M[v]\) after \(i\) iterations, can have \(\gg i\) edges.
\end{itemize}
% end slide 15


\subsubsection{Standard implementation}

\autoref{alg:bellmanford_alt} shows the standard variant of the Bellman-Ford algorithm.
It starts from the source \(s\), which is why it uses \(pred[u]\) (predecessor) pointers instead of successor pointers.





% slide 18: alternative efficient implementation (from the book)
\begin{algorithm}[h]
\caption{Bellman-Ford: Alternative Implementation}\label{alg:bellmanford_alt}
\begin{algorithmic}[1]
\Function{Bellman-Ford}{$G,s$}
  \ForAll{$u \in V$}
    \State \(d[u] \gets \infty\) \Comment{shortest path (so far) from \(s\) to \(u\)}
    \State \(\texttt{pred}[u] \gets \nil\) \Comment{predecessor of \(u\) on this path}
  \EndFor
  \State \(d[s] \gets 0\)
  \Repeat
    \State \(\texttt{converged} \gets \tru\)
    \ForAll{$(u,v) \in E$} \Comment{relax along each edge}
      \If{\(d[u] + w(u,v) < d[v]\)}
        \State \(d[v] \gets d[u] + w(u,v)\)
        \State \(\texttt{pred}[v] \gets u\)
        \State \(\texttt{converged} \gets \fals\)
      \EndIf
    \EndFor
  \Until{\texttt{converged}}
  \LComment{The \(\texttt{pred}\) pointers define an inverted shortest path tree}
\EndFunction
\end{algorithmic}
\end{algorithm}


After \(\le n-1\) iterations of \autoref{alg:bellmanford_alt}, the \(d\)-values of all vertices will have their final values.
Time: \(O(mn)\).
% end slide 18




% slide 20: advantages (19 does not exist)
\subsubsection{Comparison Bellman-Ford and Dijkstra}
Even though \nameref{alg:dijkstra}'s algorithm theoretically has better time complexity,
it is centralized and you need to make choices based on the entire graph.
So if a graph is huge, it is difficult to implement \nameref{alg:dijkstra}.

\nameref{alg:bellmanford_push} has some advantages:
\nameref{alg:dijkstra} requires global information of network, 
whereas \nameref{alg:bellmanford_push} uses only \emph{local} knowledge of neighboring nodes.
More flexible and \emph{decentralized} than \nameref{alg:dijkstra}'s.

So \nameref{alg:bellmanford_push} is used much more often in practice, e.g.
for the design of distributed routing algorithms to determine the most efficient path in a communication network.
If the graph is reasonable and there are no negative weights, \nameref{alg:dijkstra} should be used.







\subsubsection{Negative cycles}\label{sec:negative_cycles}
% slide 26

\begin{problem}[Negative Cycle]\label{prob:negative_cycle}
Given a directed graph \(G=(V,E)\) with edge weights \(c: E \to \mathbb{R}\),
determine if \(G\) contains a negative cycle.
\end{problem}
% end slide 26

% slide 27
\begin{lemma}\label{lem:bf:no_negative_cycle_on_path_to_t}
If \(\operatorname{OPT}(n,v) = \operatorname{OPT}(n-1,v)\) for all \(v\),
then there is no negative cycle on any path to \(t\).
\end{lemma}
\begin{proof}[Contrapositive]
If there were a negative cycle from \(v\) to \(t\),
then \(\operatorname{OPT}(i,v)\) would keep on decreasing in every iteration of \(i > n-1\).
\end{proof}
% end slide 27


% slide 30 (slide 28 and 29 do not exist)
\begin{lemma}\label{lem:bf:negative_cycle_on_path_to_t}
If \(\operatorname{OPT}(n,v) < \operatorname{OPT}(n-1,v)\) for some node \(v\),
then some path from \(v\) to \(t\) contains a cycle \(C\) with negative cost.
\end{lemma}


\[
\begin{tikzpicture}[>=Stealth, scale=0.6, font=\footnotesize, line cap=round, line join=round]
\useasboundingbox (-4,-2) rectangle (4,0);

% nodes
\tikzset{term/.style={circle,draw=black,fill=green!50!black,minimum size=9pt,inner sep=0pt,text=white}}
\node[term] (v) at (-4,-0.5) {\(v\)};
\node[term] (r) at (0,0) {\(r\)};
\node[term] (t) at (4,-0.5) {\(t\)};

% key point (attachment)
\coordinate (A) at (-1,-1.2);
\coordinate (B) at (-0.2,-2);
\coordinate (C) at (0.7,-1.6);
\coordinate (D) at (1,-0.9);

% wavy path v -> t
\draw[->, rounded corners=5pt] (v) -- (-2.7,-0.7) -- (-1.3,0.1) -- (r) -- (1.2,-0.3) --  (2.7,0) -- (t);

% blob boundary with arrow at the top junction
\draw[->, rounded corners=6pt] (r) --(A) --(B) --(C) --(D) --(r);

\node[] at ($0.4*(A)+0.1*(B)+0.1*(C)+0.4*(D)$) {\(C\)};
\end{tikzpicture}
\]


\begin{proof}
Any path with \(n\) edges contains a cycle (it must have a vertex \(r\) repeating by the pigeonhole principle).
Since \(\operatorname{OPT}(n,v) < \operatorname{OPT}(n-1,v)\),
this cycle must have negative cost.
\end{proof}


\autoref{lem:bf:negative_cycle_on_path_to_t} gives a condition to check if there is a negative cycle on a path to a given \(t\).





% end slide 30





% slide 31
\begin{claim}\label{claim:detect_negative_cycle}
Can detect the existence of a negative cost cycle in \(G\) in \(O(mn)\) time.
\end{claim}


Add a ``super-sink'' \(t\) and connect all nodes to \(t\) with a \(0\)-cost edge.
Check if \(\operatorname{OPT}(n,v) = \operatorname{OPT}(n-1,v)\) for all nodes \(v\) on new graph \(G'\):
\begin{itemize}
  \item if yes, then no negative cycle in \(G\) (\autoref{lem:bf:no_negative_cycle_on_path_to_t})
  \item if no, then extract cycle from shortest path from \(v\) to \(t\) (from the entry whose value reduced) 
\end{itemize}


\begin{theorem}\label{thm:negative_cycle_reduction}
\(G\) has a negative cycle iff there is a non-simple path from some node \(v\) to \(t\) in \(G'\) that contains a negative cycle.
\end{theorem}
\begin{proof}
Let \(G'=(V',E')\) with \(V' = V \cup \{t\}\) and 
\(E' = E \cup \{(v,t) : v \in V\}\), where each new edge has cost \(0\).

\(\Rightarrow\):
Suppose \(G\) contains a negative cycle \(C\), and let \(v\) be a vertex on \(C\).
In \(G'\) there is a path from \(v\) to \(t\) that traverses \(C\) once and then takes the edge \((v,t)\).
This path contains the negative cycle \(C\), hence there exists a path to \(t\) containing a negative cycle.

\(\Leftarrow\):
Suppose there exists a path from some \(v\) to \(t\) in \(G'\) that contains a negative cycle \(C\).
No directed cycle in \(G'\) can use \(t\), since all added edges enter \(t\) and \(t\) has no outgoing edges.
Therefore \(C\) uses only edges in \(E\), so \(C\) is a negative cycle in \(G\).
\end{proof}






% \clearpage

% \textcolor{AccentBlue}{DP viewpoint.}
% Let \(G = (V,E)\) be a directed graph with edge weights \(w:E\to\R\), and let \(s\) be the source.
% Define
% \(
% \operatorname{OPT}(i,v)
% \)
% to be the \hl{length of the shortest \(s\)-\(v\) path that uses \emph{at most} \(i\) edges}.
% \begin{itemize}
%   \item Base cases (\(i=0\)):
%   \(
%   \operatorname{OPT}(0,s) = 0
%   \), 
%   \(
%   \operatorname{OPT}(0,v) = \infty
%   \) for \(v \neq s\)
%   \item For \(i \ge 1\) and any \(v \in V\),
%   \[
%   \operatorname{OPT}(i,v)
%   =
%   \min\Bigl\{
%     \operatorname{OPT}(i-1,v),\;
%     \min_{(u,v)\in E}\bigl(\operatorname{OPT}(i-1,u) + w(u,v)\bigr)
%   \Bigr\}
%   \]
% \end{itemize}

% \hl[2]{Any simple path in a graph on \(n\) vertices has at most \(n - 1\) edges}.
% If there is no negative-weight cycle reachable from \(s\), the shortest paths are simple,
% so the true shortest-path distances are given by \(\operatorname{OPT}(n-1,v)\).


% Instead of explicitly storing the 2D table \(\operatorname{OPT}(i,v)\),
% Bellman-Ford iteratively relaxes all edges in the graph.

% \begin{algorithm}[h]
% \caption{Bellman-Ford}\label{alg:bellmanford}
% \begin{algorithmic}[1]
% \Function{BellmanFord}{$G,w:E\to\R,s$}
%   \ForAll{$v \in V(G)$}
%     \State \(\attrib{v}{\delta} \gets \infty\)
%     \State \(\attrib{v}{\pi} \gets \nil\) \Comment{predecessor on shortest path tree}
%   \EndFor
%   \State \(\attrib{s}{\delta} \gets 0\)
%   \For{$i = 1 \TO |V(G)| - 1$} \Comment{at most \(|V|-1\) edges on a simple path}
%     \ForAll{$(u,v) \in E(G)$}
%       \If{\(\attrib{u}{\delta} + w(u,v) < \attrib{v}{\delta}\)}
%         \State \(\attrib{v}{\delta} \gets \attrib{u}{\delta} + w(u,v)\)
%         \State \(\attrib{v}{\pi} \gets u\)
%       \EndIf
%     \EndFor
%   \EndFor
%   \ForAll{$(u,v) \in E(G)$}
%     \If{\(\attrib{u}{\delta} + w(u,v) < \attrib{v}{\delta}\)}
%       \State \Return \fals \Comment{negative-weight cycle detected}
%     \EndIf
%   \EndFor
%   \State \Return \tru \Comment{distances \(\attrib{v}{\delta}\) are correct}
% \EndFunction
% \end{algorithmic}
% \end{algorithm}

% \textcolor{AccentBlue}{Running time.}
% The algorithm performs \(O(|V|)\) passes, each relaxing all \(|E|\) edges once.
% Thus the running time is \(O(|V|\cdot|E|)\) and the space usage is \(O(|V|)\).
% The \(\attribute{\pi}\) pointers define a shortest-path tree rooted at \(s\).






\subsection{Matrix Chain Multiplication}

% \textcolor{AccentBlue}{Motivation.}
% Matrix multiplication is associative (so we may parenthesize freely) but not commutative (order fixed).
% Different parenthesizations can have wildly different running times.


\begin{problem}[Matrix Chain Multiplication]
Given matrices \(\matr{A}_1,\ldots,\matr{A}_n\) with dimensions
\(\matr{A}_i \in \mathbb{R}^{p_{i-1}\times p_i}\) (so the dimension vector is \(\vect{p} = (p_0,\ldots,p_n)^\top\)),
find a parenthesization of 
\(
% \prod_{i=1}^n \matr{A}_i 
% =
\matr{A}_1 \cdots \matr{A}_n
\) 
that minimizes the number of scalar multiplications.
\end{problem}

\textcolor{AccentBlue}{Cost of one multiplication}.
Multiplying an \(a\times b\) matrix by a \(b\times c\) matrix yields an \(a\times c\) matrix.
Each of the \(ac\) entries takes \(b\) scalar multiplications and \(b-1\) additions, so the total cost is
\(abc\) scalar multiplications and \(ac(b-1)\) additions, i.e. \(\Theta(abc)\) arithmetic operations.

\medskip

\textcolor{AccentBlue}{DP formulation}.
For \(1 \le i \le j \le n\), let \(m(i,j)\) be the minimum cost to compute the product
\(
\matr{A}_{i..j} 
:=
\prod_{l=i}^j \matr{A}_l
=
 \matr{A}_i
 \cdots 
 \matr{A}_j
\)
(where \(\matr{A}_{i..j}\) has dimension \(p_{i-1}\times p_j\)).
We seek \(m(1,n)\).

\begin{observation}
\(\matr{A}_{i..j} = \matr{A}_{i..k} \cdot \matr{A}_{k+1..j}\) for any \(i \le k < j\).
\end{observation}

\textcolor{AccentBlue}{Recurrence}.
If \(i=j\), no multiplication is needed, so \(m(i,i)=0\).
Now consider \(i<j\).
We must split the product somewhere \(i \le k < j\) (top-level parenthesization).
The miniumum cost of those subproblems is \(m(i,k)\) and \(m(k+1,j)\) by definition.
Finally, the cost to multiply the two resulting matrices is \(p_{i-1} p_k p_j\).
Thus,
\[
m(i,j)
=
\begin{cases}
0 & i = j\\
\displaystyle
\min_{i\le k<j}\Bigl\{ m(i,k)+m(k+1,j)+p_{i-1}\,p_k\,p_j \Bigr\} & i < j
\end{cases}
\]
Interpretation: choose the {last}/top-level split \(k\) optimally, compute both subchains optimally, then multiply the results
(\(p_{i-1}\times p_k\) times \(p_k\times p_j\)).

\medskip

\textcolor{AccentBlue}{Bottom-up computation} (\autoref{alg:matrix_chain}).
Compute by increasing chain length \(L=j-i+1\) (diagonal-by-diagonal), since \(m(i,j)\) depends on shorter subchains.
\[
\begin{tikzpicture}[scale=0.85, font=\footnotesize, line cap=round, line join=round]
% grid with for loop

% define number of rows and columns
\def\n{5}

\foreach \i in {1,...,\n} {
  \foreach \j in {1,...,\n} {
    \ifnum\i=\j
      \draw[] ({\j-1},-\i+1) -- (\j-1,-\i) -- (\j,-\i) -- (\j,-\i+1) -- cycle;
      \ifnum\i=\numexpr\n-1\relax
        \node at (\j-0.5, -\i+0.5) {$\ddots$};
      \else
      \node at (\j-0.5, -\i+0.5) {0};
      \fi
    \else
      \ifnum\i>\j
        \draw[fill=gray!20!white] (\j-1,-\i+1) -- (\j-1,-\i) -- (\j,-\i) -- (\j,-\i+1) -- cycle;
      \else
        \draw[] (\j-1,-\i+1) -- (\j-1,-\i) -- (\j,-\i) -- (\j,-\i+1) -- cycle;
      \fi
    \fi
  }
}
\draw[fill=green!20!white] (4,0) -- (4, -1) -- (5, -1) -- (5,0) -- cycle;
% \node at (4.5, -0.5) {\(m(1,n)\)};
\draw[<-] (4.5, -0.5) to[bend right] (5.5, -0.5) node[above right, outer sep=0pt, inner sep=0pt] {\(m(1,n)\)};



% --- row labels on the left
\foreach \i in {1,...,\n} {
  \ifnum\i=\numexpr\n-1\relax
    \node[left] at (-0.1,{-\i+0.5}) {$\vdots$};
  \else
    \ifnum\i=\n
      \node[left] at (-0.1,{-\i+0.5}) {$i=n$};
    \else
      \node[left] at (-0.1,{-\i+0.5}) {$i=\i$};
    \fi
  \fi
}

% --- column labels at the bottom
\foreach \j in {1,...,\n} {
  \ifnum\j=\numexpr\n-1\relax
    \node[below] at ({\j-0.5},{-\n-0.1}) {$\cdots$};
  \else
    \ifnum\j=\n
      \node[below] at ({\j-0.5},{-\n-0.1}) {$j=n$};
    \else
      \node[below] at ({\j-0.5},{-\n-0.1}) {$j=\j$};
    \fi
  \fi
}

\draw[<-] (-0.25, 0.25) -- (-0.75, 0.75) node[above] {\(L = 1\)};
\draw[<-] (0.75, 0.25) -- (0.25, 0.75) node[above] {\(L = 2\)};
\draw[<-] (1.75, 0.25) -- (1.25, 0.75) node[above] (l3) {\(L = 3\)};
\draw[<-] (\n-1.25, 0.25) -- (\n-1.75, 0.75) node[above] (ln) {\(L = n\)};
\node at ($(l3)!0.5!(ln)$) {\(\dots\)};
\end{tikzpicture}
\]


\begin{algorithm}[h]
\caption{Matrix Chain Multiplication}\label{alg:matrix_chain}
\begin{algorithmic}[1]
\Require dimensions \(p_0,\ldots,p_n\) where \(\matr{A}_i\) is \(p_{i-1}\times p_i\)
\State allocate \(m[1\ldots n,1\ldots n]\) \Comment{costs}
\State allocate \(s[1\ldots n-1, 2\ldots n]\) \Comment{split points}
\For{$i=1\TO n$}
  \State $m[i,i]\gets 0$
\EndFor
\For{$L=2\TO n$} \Comment{length of subchain}
  \For{$i=1\TO n-L+1$}
    \State $j\gets i+L-1$
    \State $m[i,j]\gets \infty$
    \For{$k=i\TO j-1$} \Comment{try all splits}
      \State $\texttt{cost} \gets m[i,k] + m[k+1,j] + p_{i-1} p_k p_j$
      \If{$\texttt{cost} < m[i,j]$} \Comment{found new minimum?}
        \State $m[i,j]\gets \texttt{cost}$ \Comment{save its cost}
        \State $s[i,j]\gets k$ \Comment{save split point}
      \EndIf
    \EndFor
  \EndFor
\EndFor
\State \Return $m[1,n]$ and split table $s$
\end{algorithmic}
\end{algorithm}




\textcolor{AccentBlue}{Time and space complexity} of \autoref{alg:matrix_chain}:
There are three nested loops, and each may iterate at most \(n\) times.
Thus the running time is \(O(n^3)\).
The tables use \(\Theta(n^2)\) space.


\textcolor{AccentBlue}{Reconstructing the optimal parenthesization}.
Given the split table \(s\) from \autoref{alg:matrix_chain}, we can reconstruct the optimal parenthesization recursively (\autoref{alg:multiply_optimal}).

\begin{algorithm}[h]
\caption{Matrix Chain Multiplication}\label{alg:multiply_optimal}
\begin{algorithmic}[1]
\Require marices \(\matr{A}_1,\ldots,\matr{A}_n\), split table \(s\)
\Function{Multiply}{$i, j$}
  \If{$i = j$} \Comment{base case}
    \State \Return $\matr{A}_i$
  \Else
    \State $k \gets s[i,j]$ \Comment{optimal split point}
    \State $\matr{X} \gets$ \Call{Multiply}{$i, k$} \Comment{\(\matr{A}_{i..k}\)}
    \State $\matr{Y} \gets$ \Call{Multiply}{$k+1, j$} \Comment{\(\matr{A}_{k+1..j}\)}
    \State \Return $\matr{X} \cdot \matr{Y}$ \Comment{\(\matr{A}_{i..j} = \matr{A}_{i..k} \cdot \matr{A}_{k+1..j}\)}
  \EndIf
\EndFunction
\end{algorithmic}
\end{algorithm}





\clearpage
% !TEX root = ../algo-summary.tex

\section{Network Flow}\label{sec:network_flow}

restricted version of a more general optimization problem, called linear programming.

\begin{definition}[Flow Network]\label{def:flow_network}
Abstraction for stuff \emph{flowing} through the edges of a network.
A flow network is a directed graph \(G = (V, E)\) with a \emph{source} and a \emph{sink} node \(s, t \in V\) (where flow originates and is consumed, respectively),
Each edge \((u,v) \in E\) has a capacity \(c(u,v) \in \N_0\), where \(c(u,v) = 0\) if \((u,v) \notin E\).
\end{definition}

\begin{remark}
Multiple sources/sinks can be modeled by adding a \emph{super-source} and \emph{super-sink} with infinite capacity.
\end{remark}

\begin{caution}
If we have further restrictions, such as flow from certain sources can only go to certain sinks, we have the \emph{multi-commodity flow problem}, which is NP-hard.
\end{caution}

\subsection{Flows and Cuts}\label{sec:flows_cuts}

Two rich algorithmic problems with a beautiful duality. 
Cornerstones in combinatorial optimization.


\begin{definition}[Flow]\label{def:flow}
A \emph{\(s\)-\(t\) flow} is a function \(f : E \to \R_{\ge 0}\) that satisfies
\begin{itemize}
  \item \emph{Capacity:} For each \(e \in E\)
  \begin{equation}\label{eq:capacity_constraints}
    0 \le f(e) \le c(e)
  \end{equation}
  \item \emph{Conservation:} For each \(v \in V \setminus \{s,t\}\)
  \begin{equation}\label{eq:flow_conservation}
    \sum_{u \in V} f(u,v) = \sum_{w \in V} f(v,w)
  \end{equation}
\end{itemize}
where \(s, t \in V\) are the source and sink nodes.
\end{definition}
\begin{definition}\label{def:flow_value}
The \emph{value} \(v(f)\) (also denoted by \(|f|\)) of a flow \(f\) is
\begin{equation}\label{eq:flow_value}
  {\color{Red}\boxed{\color{black}
  v(f) \coloneqq \sum_{w \in V} f(s,w) = \sum_{u \in V} f(u,t)
  }}
\end{equation}
where equality follows from flow conservation.
\end{definition}

\begin{problem}[Max Flow]\label{prob:max_flow}
Find an \(s\)-\(t\) flow \(f\) of maximum value \eqref{eq:flow_value}.
\end{problem}

% \begin{remark}
% If all capacities \(c(u,v)\) are integers, there always exists a maximum flow \(f\) such that all \(f(u,v)\) are integers.
% \end{remark}




\begin{definition}\label{def:cut}
An \emph{\(s\)-\(t\) cut} is a partition \((A,B)\) of \(V\) with \(s \in A\) and \(t \in B = V \setminus A\).
\end{definition}
\begin{definition}
[Cut Capacity]
\label{def:cut_capacity}
The \emph{capacity} of the cut is
  \begin{equation}\label{eq:cut_capacity}
  {\color{Red}\boxed{\color{black}
    c(A,B) \coloneqq \sum_{a \in A} \sum_{b \in B} c(a,b)
  }}  
  \end{equation}
i.e. the summation of all edges going from \(A\) to \(B\).
\end{definition}

\begin{problem}[Min Cut]\label{prob:min_cut}
Find an \(s\)-\(t\) cut \((A,B)\) of minimum capacity \eqref{eq:cut_capacity}.
\end{problem}





\begin{definition}\label{def:cut_flow}
The \emph{net flow} across the cut is
  \begin{equation}\label{eq:cut_flow}
  {\color{Red}\boxed{\color{black}
  f(A,B)
  \coloneqq 
  % \sum_{u \in A, v \in B} f(u,v) - \sum_{u \in B, v \in A} f(u,v) 
  \sum_{u \in A} \sum_{v \in B} f(u,v) - \sum_{u \in B} \sum_{v \in A} f(u,v)
  }}
  \qedhere
  \end{equation}
\end{definition}



\begin{lemma}[Flow Value]\label{lem:flow_value}
Let \(f\) be any flow and \((A,B)\) be any cut. Then
\begin{equation}\label{eq:flow_value_cut}
  f(A,B)
  = 
  v(f)
\end{equation}
i.e. the \hl[2]{net flow across any cut equals the amount leaving \(s\)}. %value of the flow.
\end{lemma}
\begin{proof}
\[
\begin{verticalhack}
\begin{aligned}
  v(f) &\overset{\text{\eqref{eq:flow_value}}}{=} \sum_{v \in V} f(s,v) \\
       &=\sum_{v \in V} f(s,v) + 0 \\
       &\overset{\text{\eqref{eq:flow_conservation}}}{=} \sum_{v \in V} f(s,v) + \sum_{u \in A \setminus \{s\}} \left( \sum_{v \in V} f(u,v) - \sum_{v \in V} f(v,u) \right) \\
        &= \sum_{u \in A} \left( \sum_{v \in V} f(u,v) - \sum_{v \in V} f(v,u) \right) \\
       &= \sum_{u \in A} \sum_{v \in V} f(u,v) - \sum_{u \in A} \sum_{v \in V} f(v,u) \\
        &= \sum_{u \in A} \left( \sum_{v \in A} f(u,v) + \sum_{v \in B} f(u,v) \right) - \sum_{u \in A} \left( \sum_{v \in A} f(v,u) + \sum_{v \in B} f(v,u) \right) \\
        &= \cancel{\sum_{u \in A} \sum_{v \in A} f(u,v)} + \sum_{u \in A} \sum_{v \in B} f(u,v) - \cancel{\sum_{u \in A} \sum_{v \in A} f(v,u)} - \sum_{u \in A} \sum_{v \in B} f(v,u) \\
        &= \sum_{u \in A} \sum_{v \in B} f(u,v) - \sum_{u \in B} \sum_{v \in A} f(u,v) \\
        &= f(A,B) 
\end{aligned}
\end{verticalhack}
\qedhere
\]
\end{proof}




\begin{theorem}
[Weak Duality]
\label{lem:weak_duality_flow_capacity}
Let \(f\) be any flow and \((A,B)\) be any cut. Then
\begin{equation}\label{eq:weak_duality}
{\color{Red}\boxed{\color{black}
  v(f) \le c(A,B)
}}
\end{equation}
i.e. the value of the flow is at most the capacity of the cut.
\end{theorem}

\begin{proof}\label{proof:weak_duality}
\[
\begin{verticalhack}
\begin{aligned}
  v(f) 
  \overset{\text{\eqref{eq:flow_value_cut}}}{=} f(A,B) 
  \overset{\text{\eqref{eq:cut_flow}}}{=}& \sum_{u \in A} \sum_{v \in B} f(u,v) - \sum_{u \in B} \sum_{v \in A} f(u,v) \\
  \le& \sum_{u \in A} \sum_{v \in B} f(u,v) \\
  \overset{\text{\eqref{eq:capacity_constraints}}}{\le}& \sum_{u \in A} \sum_{v \in B} c(u,v) \\
  \overset{\text{\eqref{eq:cut_capacity}}}{=}& c(A,B)
\end{aligned}
\end{verticalhack}
\qedhere
\]
\end{proof}
\begin{corollary}[Certificate of Optimality]\label{cor:certificate_optimality}
\hl{Let \(f\) be any flow and \((A,B)\) be any cut.
If \(v(f) = c(A,B)\), then \(f\) is a \nameref{prob:max_flow} and \((A,B)\) is a \nameref{prob:min_cut}.}
\end{corollary}
\begin{observation}
\label{obs:flow_equals_cut_flow}
In addition, if \(v(f) = c(A,B)\), then
\begin{equation}\label{eq:max_flow_in}
f_{\text{in}}(A) \coloneqq  \sum_{u \in B} \sum_{v \in A} f(u,v) = 0
\end{equation}
and
\begin{equation}\label{eq:max_flow_out}
f_{\text{out}}(A) \coloneqq  \sum_{u \in A} \sum_{v \in B} f(u,v) = v(f) = c(A,B)
\end{equation}
i.e. all edges from \(A\) to \(B\) are saturated and all edges from \(B\) to \(A\) carry zero flow.
\end{observation}
\begin{proof}
From the \autoref{proof:weak_duality} of \autoref{lem:weak_duality_flow_capacity}, we have
\[
v(f) = f_{\text{out}}(A) - f_{\text{in}}(A) \le f_{\text{out}}(A) \le c(A,B)
\]
but if \(v(f) = c(A,B)\), all inequalities must be equalities, yielding \eqref{eq:max_flow_in} and \eqref{eq:max_flow_out}.
\end{proof}


\begin{caution}
  \nameref{sec:dynamic_programming} (\autoref{sec:dynamic_programming}) cannot be applied because the problem does not exhibit optimal substructure (\autoref{def:principle_of_optimality}).
\end{caution}

A na\"ive greedy strategy (\autoref{sec:greedy_algorithms}) will not work for \nameref{prob:max_flow} either, because local optimality does not imply global optimality.
This happens, because once we send flow along an edge, we might later want to ``take it back'' to reroute it more efficiently, so in addition to increase flow along edges, we need the ability to \emph{reduce} it as well.

We need the notion of a




\subsection{Residual Network}\label{sec:residual_network}

Recall the transpose graph \(G^\top = (V, E^\top)\), where \(E^\top = \{(v,u) \mid (u,v) \in E\}\).

In order to ``undo'' flow sent along edges, we consider original edges \(E\) (`forward arcs') as well as transpose edges \(E^\top\) (`backward arcs'), with capacities defined as follows.
Given flow \(f\), the \emph{residual capacity} of any arc is
\begin{equation}\label{eq:residual_capacity}
c_f(e) =
\begin{cases}
  c(e) - f(e) & e \in E \\
  f(e^R) & e \in E^\top
\end{cases}
\end{equation}
% where \(e^{R} = (u,v)\) denotes the original (forward) arc for a given backward arc \(e = (v,u) \in E^\top\).
where for any arc $a=(x,y)$ we define $a^R\coloneqq (y,x)$.
Note that if $e\in E^\top$, then $e^R\in E$.

\begin{definition}[Residual Network]\label{def:residual_network}
The \emph{residual network} \(G_f = (V, E_f)\) with respect to flow \(f\) contains the same vertices as \(G\) with the arc set
\begin{equation}\label{eq:residual_edges}
  E_f \coloneqq  \{ e \in E : f(e) < c(e) \} \cup \{ e \in E^\top: f(e^R) > 0 \}
\end{equation}
i.e., it contains all (forward and backward) arcs with positive residual capacity.
\end{definition}

Each edge in \(G\) may result in the generation of at most two arcs in \(G_f\).
Thus, \(G_f\) is of the same asymptotic size as \(G\).
% Also note that when \(v(f) = 0\), we have \(G_f = G\). % only true if there is no cycle
Note that when \(f \equiv 0\), we have \(G_f = G\).



\begin{definition}[Augmenting Path]\label{def:augmenting_path}
An \emph{augmenting path} is a simple \(s\)-\(t\) path \(P\) in the residual network \(G_f\).
\end{definition}

The \emph{bottleneck capacity} (also called {residual capacity}) of an \nameref{def:augmenting_path} \(P\) is
\begin{equation}\label{eq:augmenting_path_capacity}
  {\color{Red}\boxed{\color{black}
  c_f(P) \coloneqq \min_{e \in P} c_f(e)
  }}
\end{equation}
where the minimum is taken over all arcs on the path \(P\).
Since all arcs in \(P\) have positive residual capacity, \(c_f(P) > 0\).


% ```
% Augment(f, c, P) {
%     b \leftarrow bottleneck-capacity(P)
%     foreach e \in P {
%         if (e \in E) f(e) \leftarrow f(e) + b
%         else f(eR) \leftarrow f(e) - b
%     }
%     return f
% }
% ```

% forward edge reverse edge
% ```
% Ford-Fulkerson(G, s, t, c) {
%     foreach e \in E f(e) \leftarrow 0
%     Gf \leftarrow residual graph
%     while (there exists an augmenting path P in }\mp@subsup{G}{f}{}\mathrm{ )
% {
%         f \leftarrow Augment (f, c, P) //Augment f by cap(P)
%         update }\mp@subsup{G}{f}{
%     }
%     return f
% }
% ```

\begin{algorithm}[h]
\caption{Augment}\label{alg:augment}
\begin{algorithmic}[1]
\Function{Augment}{\(f, c, P\)}
\State \(b \gets c_f(P)\) \Comment{bottleneck capacity \eqref{eq:augmenting_path_capacity}}
\ForAll{\(e \in P\)}
  \If{\(e \in E\)}
     \State \(f(e) \gets f(e) + b\) \Comment{forward arc: increase flow}
  \Else  \Comment{\(e \in E^\top\)}
     \State \(f(e^R) \gets f(e^R) - b\) \Comment{backward arc: decrease flow along original edge}
  \EndIf
\EndFor
\State \Return \(f\)
\EndFunction
\end{algorithmic}
\end{algorithm}

\begin{algorithm}[h]
\caption{Ford-Fulkerson Method}\label{alg:ford_fulkerson}
\begin{algorithmic}[1]
\Require Flow network \(G = (V,E)\), source \(s\), sink \(t\), capacities \(c\)
\ForAll{\(e \in E\)} 
 \State \(f(e) \gets 0\) \Comment{initial flow}
\EndFor
\State \(G_f \gets G\) \Comment{since the initial flow $f\equiv 0$, the initial residual network equals $G$}
% \While{\(\exists P \subset G_f\)} \Comment{there is an \nameref{def:augmenting_path} in \nameref{def:residual_network}}
 \While{there exists an \nameref{def:augmenting_path} \(P\) in \nameref{def:residual_network} \(G_f\)}
  \State \(f \gets \Call{Augment}{f, c, P}\) \Comment{augment flow along \(P\)}
  \State Update \(G_f\) based on new flow \(f\)
\EndWhile
\State \Return \(f\)
\end{algorithmic}
\end{algorithm}






At the end of the \autoref{alg:ford_fulkerson}, the \nameref{def:residual_network} \(G_f\) contains no \nameref{def:augmenting_path}s.








\subsection{Max-Flow Min-Cut Theorem}\label{sec:max_flow_min_cut_theorem}


\begin{theorem}[Augmenting Path]\label{thm:augmenting_path}
Flow \(f\) is a \nameref{prob:max_flow} iff there are no \nameref{def:augmenting_path}s in the \nameref{def:residual_network} \(G_f\).
\end{theorem}

\begin{theorem}[\nameref{prob:max_flow} / \nameref{prob:min_cut}, Ford-Fulkerson 1956]
The value of the \nameref{prob:max_flow} is equal to the value of the \nameref{prob:min_cut}.
\end{theorem}




\begin{theorem}[\nameref{prob:max_flow} / \nameref{prob:min_cut}]\label{thm:max-flow-min-cut}
Let $f$ be a flow in \(G\).
The following 3 are equivalent:
\begin{enumerate}[label=\textbf{(\roman*)}, itemindent=0.25cm, labelsep=0.25cm]
  \item There is cut $(A,B)$ such that $v(f) = c(A,B)$.
  \label{thm:mcmc-tight-cut}
  \item $f$ is a \nameref{prob:max_flow} and $(A,B)$ is a \nameref{prob:min_cut}.
  \label{thm:mcmc-maxflow-mincut}
  \item The \nameref{def:residual_network} $G_f$ contains no \nameref{def:augmenting_path}s.
  \label{thm:mcmc-no-augmenting-path}
\qedhere
\end{enumerate}
\end{theorem}

\begin{proof}
We prove the implications in the following triangle:
\(
\newcommand{\scalefraction}{0.68}
\begin{tikzcd}[column sep={\scalefraction*1cm,between origins},
               row sep={\scalefraction*1.732050808cm,between origins},
               arrows=Rightarrow]
  & \ref{thm:mcmc-tight-cut}
    \arrow[dr,""{pos=.5,sloped,overlay, inner sep=0pt,
      label={[black]center:\hyperlink{proof:mcmc-1-2}{\phantom{\rule{10pt}{15pt}}}}}]
  &
  \\
  \ref{thm:mcmc-no-augmenting-path}
    \arrow[ur,""{pos=.5,sloped,overlay, inner sep=0pt,
      label={[black]center:\hyperlink{proof:mcmc-3-1}{\phantom{\rule{10pt}{15pt}}}}}]
  &
  &
  \ref{thm:mcmc-maxflow-mincut}
    \arrow[ll,""{pos=.5,sloped,overlay, inner sep=0pt,
      label={[black]center:\hyperlink{proof:mcmc-2-3}{\phantom{\rule{15pt}{5pt}}}}}]
\end{tikzcd}
\)

\vspace{-2\baselineskip}
\hypertarget{proof:mcmc-1-2}{
\underline{\ref{thm:mcmc-tight-cut} $\Rightarrow$ \ref{thm:mcmc-maxflow-mincut}:}}

\autoref{cor:certificate_optimality}.

\hypertarget{proof:mcmc-2-3}{
\underline{\ref{thm:mcmc-maxflow-mincut} $\Rightarrow$ \ref{thm:mcmc-no-augmenting-path} (contrapositive):}}

If there were an augmenting path in \(G_f\), we could improve \(f\) by sending flow along this path.


\hypertarget{proof:mcmc-3-1}{
\underline{\ref{thm:mcmc-no-augmenting-path} $\Rightarrow$ \ref{thm:mcmc-tight-cut}:}}

Let \(A\) be the set of vertices reachable from the source \(s\) in \(G_f\).
By construction, \(s \in A\).
Since \(G_f\) contains no augmenting path, \(t \notin A\).
If we set \(B = V \setminus A\), then \((A,B)\) is a cut.
%
We now want to show again that the inqualities
% \[
% v(f) = f_{\text{out}}(A) - f_{\text{in}}(A) \le f_{\text{out}}(A) \le c(A,B)
% \]
in the \autoref{proof:weak_duality} of \autoref{lem:weak_duality_flow_capacity} are actually equalities.
Let \(u \in A\) and \(v \in B\).
If \((u,v) \in E\), then \(f(u,v) = c(u,v)\) because otherwise \((u,v)\) would be in \(G_f\) and \(v\) would be in \(A\).
If \((v,u) \in E\), then \(f(v,u) = 0\) because otherwise \((u,v)\) would be in \(G_f\) and \(v\) would be in \(A\).

This closes the triangle of implications and proves \autoref{thm:max-flow-min-cut}.
\end{proof}


\begin{assumption}\label{ass:integer_capacities}
All capacities are integers between \(1\) and \(C\).
\end{assumption}

\begin{invariant}\label{inv:integer_flows}
Every flow value \(f(e)\) and every residual capacity \(c_f(e)\) remains an integer throughout the execution of \autoref{alg:ford_fulkerson}.
\end{invariant}

\begin{theorem}\label{thm:ford_fulkerson_iteration}
\autoref{alg:ford_fulkerson} terminates in at most \(v(f^*)\) iterations, where \(v(f^*) \leq nC\) is the value of a maximum flow.
\end{theorem}
\begin{proof}
Each \nameref{alg:augment} increases \(v(f)\) by at least 1 and \(v(f^*) \leq \deg^+(s) \cdot C \leq nC\).
\end{proof}

\begin{corollary}
\autoref{alg:ford_fulkerson} runs in time \(O(v(f^*) \cdot m)\) (not polynomial in \(n\)\textcolor{red}{!}), since an \nameref{def:augmenting_path} can be found in \(O(m)\) time.
\end{corollary}
\begin{corollary}
If \(C = 1\), \autoref{alg:ford_fulkerson} runs in \(O(n m)\) time.
\end{corollary}

\begin{theorem}[Integrality]\label{thm:integrality}
If all capacities are integers (\autoref{ass:integer_capacities}), there exists a \nameref{prob:max_flow}, such that \(f(e)\) is an integer for all \(e \in E\).
\end{theorem}
\begin{proof}
\autoref{alg:ford_fulkerson} terminates (\autoref{thm:ford_fulkerson_iteration}) and maintains \autoref{inv:integer_flows}.
\end{proof}

\begin{corollary}\label{cor:integral-flow-given-value}
If all capacities are integers (\autoref{ass:integer_capacities}) and there exists a  flow of value $k$, there exists an \emph{integral} flow of value $k$.
\end{corollary}
\begin{proof}
% Let $f^*$ be a \nameref{prob:max_flow}. By \autoref{thm:integrality}, $f^*$ is integral and $v(f^*) \ge k$.
Deleting $v(f^*)-k$ unit augmenting paths from an integral \nameref{prob:max_flow} $f^*$.
\end{proof}



\subsection{Choosing Augmenting Paths}\label{sec:choosing_augmenting_paths}

Running time of generic \nameref{alg:ford_fulkerson} is \emph{pseudo-polynomial}, not polynomial in size of the input \(n, m\).

\textcolor{AccentBlue}{Use care when selecting \nameref{def:augmenting_path}s}:
\begin{itemize}
  \item Some choices lead to exponential algorithms.
  \item Clever choices lead to polynomial algorithms (Edmonds-Karp).
  \item If capacities are irrational, may not terminate!
\end{itemize}




% \[
% \begin{tikzpicture}[
%   font=\footnotesize,
%   scale=0.6,
%   node/.style={circle, draw, minimum size=3mm, inner sep=0pt},
%   edge/.style={-latex},
%   path/.style={-latex, thick, draw=orange!90!red},
%   lbl/.style={midway, fill=white, inner sep=1pt}
% ]

% % nodes
% \node[node] (s) at (0,2) {$s$};
% \node[node] (a) at (-3,0) {};
% \node[node] (u) at (-1,0) {};
% \node[node] (v) at (1,0) {};
% \node[node] (b) at (3,0) {};
% \node[node] (t) at (0,-2) {$t$};

% % outer black diamond edges
% \draw[edge] (s) -- node[lbl] {$x$} (a);
% \draw[edge] (s) -- node[lbl] {$x$} (b);
% \draw[edge] (a) -- node[lbl] {$x$} (t);
% \draw[edge] (b) -- node[lbl] {$x$} (t);

% % inner black edges
% \draw[edge] (u) -- node[lbl] {$1$} (a);
% \draw[edge] (b) -- node[lbl] {$\phi$} (v);

% % highlighted orange path
% \draw[path] (s) -- node[lbl] {$x$} (u);
% \draw[path] (u) -- node[lbl] {$1$} (v);
% \draw[path] (v) -- node[lbl] {$x$} (t);

% \end{tikzpicture}
% \]

\begin{example}[Uri Zwick, 1993]\label{ex:zwick_network}
Consider the following network
\[
\begin{tikzpicture}[
  font=\footnotesize,
  scale=0.5,
  node/.style={circle, draw, minimum size=3mm, inner sep=0pt},
  smallnode/.style={circle, draw, minimum size=2mm, inner sep=0pt},
  edge/.style={-latex},
  path/.style={-latex, thick, draw=orange!90!red},
  lbl/.style={midway, fill=white, inner sep=1pt}
]

%======================
% TOP NETWORK + PATH
%======================

% nodes
\node[node] (s) at (0,2) {$s$};
\node[node] (a) at (-3,0) {};
\node[node] (u) at (-1,0) {};
\node[node] (v) at (1,0) {};
\node[node] (b) at (3,0) {};
\node[node] (t) at (0,-2) {$t$};

% % start / sink arrows
% \draw[edge] (0,3) -- (s);
% \draw[edge] (t) -- (0,-3);

% outer black diamond edges
\draw[edge] (s) -- node[lbl] {$x$} (a);
\draw[edge] (s) -- node[lbl] {$x$} (b);
\draw[edge] (a) -- node[lbl] {$x$} (t);
\draw[edge] (b) -- node[lbl] {$x$} (t);

% inner black edges
\draw[edge] (u) -- node[lbl] {$1$} (a);
\draw[edge] (b) -- node[lbl] {$\phi$} (v);

% highlighted orange path (s -> u -> v -> t)
\draw[path] (s) -- node[lbl] {$x$} (u);
\draw[path] (u) -- node[lbl] {$1$} (v);
\draw[path] (v) -- node[lbl] {$x$} (t);

%======================
% HELPER: base graph (no labels)
%======================

\newcommand{\basegraph}[6]{%
  % #1..#6 are node names: S A U V B T

  % nodes
  \node[smallnode] (#1) at (0,2) {};
  \node[smallnode] (#2) at (-3,0) {};
  \node[smallnode] (#3) at (-1,0) {};
  \node[smallnode] (#4) at (1,0) {};
  \node[smallnode] (#5) at (3,0) {};
  \node[smallnode] (#6) at (0,-2) {};

  % % edge
  % \draw[edge] (#1) -- (#2);
  % \draw[edge] (#1) -- (#5);
  % \draw[edge] (#2) -- (#6);
  % \draw[edge] (#5) -- (#6);

  % \draw[edge] (#3) -- (#2);
  % \draw[edge] (#5) -- (#4);
  % \draw[edge] (#1) -- (#3);
  % \draw[edge] (#3) -- (#4);
  % \draw[edge] (#4) -- (#6);
}

%======================
% BOTTOM: PATH A
%======================

\begin{scope}[scale= 0.7, yshift=-7.25cm, xshift=-9cm]
  \basegraph{sA}{aA}{uA}{vA}{bA}{tA}

  % orange path A: top -> left side -> u -> v -> bottom
  \draw[path] (sA) -- (aA);
  \draw[path] (aA) -- (uA);
  \draw[path] (uA) -- (vA);
  \draw[path] (vA) -- (tA);

  % other edges
  \draw[edge] (sA) -- (bA);
  \draw[edge] (sA) -- (uA);
  \draw[edge] (bA) -- (vA);
  \draw[edge] (aA) -- (tA);
  \draw[edge] (bA) -- (tA);

  \node at (2.75,-1.75) {$P_A$};
\end{scope}

%======================
% BOTTOM: PATH B
%======================

\begin{scope}[scale= 0.7, yshift=-7.25cm]
  \basegraph{sB}{aB}{uB}{vB}{bB}{tB}

  % orange path B: top -> right side -> v -> u -> left side -> bottom
  \draw[path] (sB) -- (bB);
  \draw[path] (bB) -- (vB);
  \draw[path] (vB) -- (uB);
  \draw[path] (uB) -- (aB);
  \draw[path] (aB) -- (tB);

  % other edges
  \draw[edge] (sB) -- (aB);
  \draw[edge] (sB) -- (uB);
  \draw[edge] (vB) -- (tB);
  \draw[edge] (bB) -- (tB);

  \node at (2.75,-1.75) {$P_B$};
\end{scope}

%======================
% BOTTOM: PATH C
%======================

\begin{scope}[scale= 0.7, yshift=-7.25cm, xshift=9cm]
  \basegraph{sC}{aC}{uC}{vC}{bC}{tC}

  % orange path C: top -> u -> v -> right side -> bottom
  \draw[path] (sC) -- (uC);
  \draw[path] (uC) -- (vC);
  \draw[path] (vC) -- (bC);
  \draw[path] (bC) -- (tC);

  % other edges
  \draw[edge] (sC) -- (aC);
  \draw[edge] (sC) -- (bC);
  \draw[edge] (uC) -- (aC);
  \draw[edge] (aC) -- (tC);
  \draw[edge] (vC) -- (tC);

  \node at (2.75,-1.75) {$P_C$};
\end{scope}

\end{tikzpicture}
\]
where \(\phi = \frac{\sqrt{5} - 1}{2} \approx 0.618\) denotes the golden ratio, satisfying \(\phi^2 + \phi = 1\).
\end{example}


Goal: Choose augmenting paths so that:
\begin{itemize}
  \item can find augmenting paths efficiently
  \item result in few iterations
\end{itemize}

\medskip

Strategies:
\begin{itemize}
  \item max bottleneck capacity (dependency on \(C\))
  \item sufficiently large bottleneck capacity \textcolor{AccentGray}{[Dinitz 1970]}. reduce dependency on \(C\) to \(\log C\). Time \(O(m^2 \log C)\).
  \item simple idea: \\
  don't do anything fancy with capacities, just choose the path with fewest number of edges \textcolor{AccentGray}{[Edmonds-Karp 1972]}. Polynomial time algorithm. Time \(O(n m^2)\).
\end{itemize}


\subsubsection{Edmonds-Karp}

\emph{Idea}: Always choose an augmenting path with the \hl{minimum number of edges} in the residual network.

\begin{itemize}
  \item Implemented by running \nameref{alg:bfs} in \(G_f\)
  \item Each augmentation increases the shortest path distance
  \item Total number of iterations: \(O(nm)\), so total running time is \hl[2]{\(O(n m^2)\)}
\end{itemize}









\subsection{Applications}
\subsubsection{Bipartite Matching}\label{sec:bipartite_matching}

\begin{definition}[Matching]\label{def:graph-matching}
Given an undirected graph \(G = (V,E)\), \(M \subseteq E\) is a \emph{matching} if each node appears in at most one edge in \(M\).
\end{definition}

\begin{definition}[Bipartite Matching]\label{def:graph-bipartite-matching}
A \emph{bipartite matching} is a \nameref{def:graph-matching} in an undirected, bipartite graph \(G = (L \cup R, E)\).
\end{definition}

\begin{problem}[Max Bipartite Matching]\label{prob:max_bipartite_matching}
Find a \nameref{def:graph-bipartite-matching} of maximum cardinality.
\end{problem}

\textcolor{AccentBlue}{\nameref{prob:max_flow} formulation of \autoref{prob:max_bipartite_matching}}:
\begin{itemize}
  \item create digraph \(G' = (L \cup R \cup \{s,t\}, E')\)
  \item direct all edges from \(L\) to \(R\) and assign infinite (or unit) capacitiy
  \item add \textcolor{AccentRed}{unit capacity} edges from \(s\) to all nodes in \(L\)
  \item add \textcolor{AccentRed}{unit capacity} edges from all nodes in \(R\) to \(t\)
\end{itemize}

\begin{theorem}\label{thm:matching_flow_equivalence}
a bipartite graph \(G\) has a \nameref{def:graph-matching} of size \(k\) iff the network formulation \(G'\) has a flow of value \(k\).
\end{theorem}
or equivalently:
\begin{theorem}\label{thm:max_matching_via_max_flow}
max cardinality matching in \(G\) = value of \nameref{prob:max_flow} in \(G'\).
\end{theorem}

% \begin{proof}
% `\(\leq\)' (\(|M|=k\), then \(G'\) has a flow of value \(k\)):
% \begin{itemize}
%   \item 
%   given a max matching \(M\) of \(G\) with \(|M| = k\); 
%   consider a flow \(f\) that sends 1 unit along each edge in \(M\): 
%   \begin{itemize}
%     \item for each \((u,v) \in M\), set \(f(s,u) = 1\), \(f(u,v) = 1\), \(f(v,t) = 1\)
%     \item all other edges have flow \(f(e) = 0\)
%   \end{itemize}
%   \item \(f\) is a valid flow: consider a \(x \in L \cup R\):
%   \begin{itemize}
%     \item if \(x\) is incident to \(M\): \(f_{\text{in}}(x) = 1 = f_{\text{out}}(x)\) since \(M\) is a \nameref{def:graph-matching}
%     \item if \(x\) is not incident to \(M\): \(f_{\text{in}}(x) = 0 = f_{\text{out}}(x)\)
%   \end{itemize}
%   \item value of flow \(f\): \(v(f) = |M| = k \leq v(f^*)\), where \(f^*\) is a \nameref{prob:max_flow}
%   \qedhere
% \end{itemize}
% \end{proof}

% \begin{proof}
% `\(\geq\)' (\(v(f^*) \leq k = |M|\), where \(M\) is a max matching):
% \begin{itemize}
%   \item 
%   let \(f\) be a \nameref{prob:max_flow} in \(G'\) with value \(r = v(f) = v(f^*)\)
%   \item 
%   apply \nameref{thm:integrality} theorem:
%   \(r\) is integral, thus can assume \(f\) is \(\{0,1\}\)-valued
%   \item 
%   let \(M\) be the set of edges from \(L\) to \(R\) with \(f(e) = 1\).
%   \item 
%   \(v(f^*) = r\), thus \(r\) edges out of \(s\) carry flow 1; thus, \(r\) nodes in \(L\) are picked for \(M\); by conservation constraint, \(r\) incident \(L\)-\(R\) edges are picked. 
%   Thus, \(r \leq |M|\).
%   \item 
%   By the conservation constraint, no two edges out of one node \(x \in L\) can be picked in \(M\); thus, \(M\) is a \nameref{def:graph-matching}.
%   \qedhere
% \end{itemize}
% \end{proof}

\begin{proof}[``$\le$'']
Let $M$ be a max matching in $G$ with $|M|=k$. 
Define a flow $f$ in $G'$ by
\(f(s,u)=1\), \(f(u,v)=1\), \(f(v,t)=1\) 
for each \((u,v)\in M\),
and $f(e)=0$ for all other edges. 
Capacity constraints \eqref{eq:capacity_constraints} hold since all used edges have capacity at least \(1\).
For any vertex $x\in L\cup R$, 
either $x$ is incident to no edge of $M$ (then $f_{\mathrm{in}}(x) = f_{\mathrm{out}}(x)=0$), 
or to exactly one edge of $M$ (then $f_{\mathrm{in}}(x) = f_{\mathrm{out}}(x)=1$);
hence flow conservation \eqref{eq:flow_conservation} holds. 
Thus $f$ is a feasible flow in $G'$ and
\(
v(f)=\sum_{u\in L} f(s,u) = |M| = k
\).
In particular, if $f^*$ is a max flow in $G'$, then $v(f^*) \ge v(f) = k$.
\end{proof}

\begin{proof}[``$\ge$'']
Let $f$ be a max flow in $G'$ with $v(f) = r$. 
By integrality (\autoref{thm:integrality}), we may assume $f$ is integral, 
hence $f(e) \in \{0,1\}$ by capacity constraints \eqref{eq:capacity_constraints}.
Define
\(
M \coloneqq \{(u,v)\in E \mid u\in L, v\in R, f(u,v) = 1\}
\).
For each $u \in L$ we have $\sum_{v \in R} f(u,v) = f(s,u) \le 1$ by flow conservation \eqref{eq:flow_conservation} at $u$, 
so at most one edge of $M$ is incident to $u$.
Similarly, 
for each $v \in R$ we have $\sum_{u \in L} f(u,v) = f(v,t) \le 1$, so at most one edge of $M$ is incident to $v$.
Thus $M$ is a matching in $G$ and
\(
|M| = \sum_{u \in L} \sum_{v \in R} f(u,v) = \sum_{u \in L} f(s,u) = v(f) = r
\).
In particular, if $M^*$ is a max matching in $G$, then $|M^*| \ge |M| = r$.
\end{proof}

The edges of the matching are the ones that carry flow from \(L\) to \(R\).


\medskip


We reduced the problem `\nameref{prob:max_bipartite_matching}' to the known problem `\nameref{prob:max_flow}'.




\begin{definition}[Perfect Matching]\label{def:perfect_matching}
A matching \(M \subseteq E\) is perfect if each node appears in exactly one edge in \(M\).  
\end{definition}

A necessary condition for a \nameref{def:perfect_matching} in a bipartite graph is \(|L| = |R|\).

\medskip

\textcolor{AccentBlue}{To compute a perfect matching (if one exists), compute a max cardinality matching}:
\begin{itemize}
  \item if cardinality \( = n = |L| = |R|\), then the matching is perfect
  \item if cardinality \(< n\), then no perfect matching exists
\end{itemize}

\medskip

\hl{A cut with capacity \(< n\) provides a \emph{certificate} that no perfect matching exists!}

\begin{notation}[Neighborhood]\label{not:neighborhood}
For a subset \(S \subseteq G\) of nodes in a graph, we write 
\begin{equation}
  N(S) \coloneqq \bigcup_{u \in S} \Gamma(u) %= \{ v \in V : \exists u \in S \text{ with } (u,v) \in E \}
\end{equation}
for the set of nodes adjacent to at least one node in \(S\).
\end{notation}


\begin{observation}\label{obs:hall-necessity}
If a bipartite graph \(G = (L \cup R, E)\) has a \nameref{def:perfect_matching}, then \(|N(S)| \ge |S|\) for all subsets \(S \subseteq L\).
\end{observation}
\begin{proof}
Suppose \(G\) has a perfect matching.
Then each node in any \(S \subseteq L\) is matched to a different node in \(N(S)\).
Thus, \(|S| \le |N(S)|\).
\end{proof}

\begin{theorem}[Marriage \textcolor{AccentGray}{[Frobenius 1917, Hall 1935]}]\label{thm:hall-marriage}
Let \(G = (L \cup R, E)\) be a bipartite graph with \(|L| = |R|\).
Then, \(G\) has a perfect matching iff \(|N(S)| \ge |S|\) for all subsets \(S \subseteq L\).
\end{theorem}

\begin{proof}[`$\Rightarrow$']
\autoref{obs:hall-necessity}.
\end{proof}

% \begin{proof}[`$\Leftarrow$']
% Suppose G does not have a perfect matching 
% \begin{itemize}
% \item Formulate $\mathrm{G}^{\prime}$ as max flow problem and let $({A}, {B})$ be min cut in ${G}^{\prime}$.
% \item Since no perfect matching, $v\left(f^*\right)<n$ and $\operatorname{cap}({A}, {B})<{n}(|L|=n)$
% \item Define $L_A=L \cap A, L_B=L \cap B, R_A=R \cap A$.
% \item Since min cut can't use $\infty$ edges, no edge from $L$ to $R$ can be in cut; only edges from $s$ to $B\left(L_B\right)+$ edges from $A$ to $t\left(R_A\right)$
% \item Thus, $\operatorname{cap}(A, B)=\left|L_B\right|+\left|R_A\right|$. Similarly, $N\left(L_A\right) \subseteq R_A$.
% \item $\left|N\left(L_A\right)\right| \leq\left|R_A\right|=\operatorname{cap}(A, B)-\left|L_B\right|<|L|-\left|L_B\right|=\left|L_A\right|$.
% \item $\left|N\left(L_A\right)\right|<\left|L_A\right|$. Chose $S=L_A$.
% \qedhere
% \end{itemize}
% \end{proof}
\begin{proof}[`$\Leftarrow$']
We prove the contrapositive, i.e.
\[
\text{\(G\) has no perfect matching} \; \implies \; \exists S \subseteq L: |N(S)| < |S|
\]

Suppose \(G\) does not have a perfect matching 
\begin{itemize}
\item 
Formulate ${G}^{\prime}$ as \nameref{prob:max_flow} problem and let $({A}, {B})$ be \nameref{prob:min_cut} in ${G}^{\prime}$.
\item 
Since no perfect matching, 
value of \nameref{prob:max_flow} $v\left(f^*\right)<n$ (\autoref{thm:max_matching_via_max_flow})
and therefore (by \autoref{thm:max-flow-min-cut}),
capacity of the \nameref{prob:min_cut} $\operatorname{cap}({A}, {B})<{n}=|L|$.
\item 
Define $L_A=L \cap A$,  $L_B=L \cap B$, $R_A=R \cap A$, $R_B=R \cap B$.
\item 
Since \nameref{prob:min_cut} can't use $\infty$ edges (otherwise the cut capacity would be \(\infty\)), no edge from $L$ to $R$ can be in cut; 
only edges from $s$ to $B$ ($L_B$) and edges from $A$ to $t$ ($R_A$).
\item 
Since all these edges have unit capacity, $\operatorname{cap}(A, B)=\left|L_B\right|+\left|R_A\right|$. 
Furthermore, no edge from $L_A$ to $R_B$ implies $N\left(L_A\right) \subseteq R_A$.
\item 
$\left|N\left(L_A\right)\right| \leq\left|R_A\right|=\operatorname{cap}(A, B)-\left|L_B\right|<|L|-\left|L_B\right|=\left|L_A\right|$.
\item 
$\left|N\left(L_A\right)\right|<\left|L_A\right|$. 
Choose $S=L_A$.
\qedhere
\end{itemize}
\end{proof}

\bigskip

\textcolor{AccentBlue}{Which max flow algorithm to use for \nameref{def:graph-bipartite-matching}?}
\begin{itemize}
  \item Generic augmenting path: \(O(m \cdot \text{val}(f^*)) = O(mn)\), since \(\text{val}(f^*) \leq n\)
  \item \textcolor{gray}{Capacity scaling: \(O(m^2 \log C) = O(m^2)\)}
  \item Shortest augmenting path: \(O(m \sqrt{n})\)
\end{itemize}

\medskip

\textcolor{AccentBlue}{Non-bipartite \nameref{def:graph-matching}}:
\begin{itemize}
  \item Structure of non-bipartite graphs is more complicated, but well-understood \textcolor{AccentGray}{[Tutte-Berge, Edmonds-Galai]}
  \item Blossom algorithm: \(O(n^4)\) \textcolor{AccentGray}{[Edmonds 1965]}
  \item Best known: \(O(m \sqrt{n})\) \textcolor{AccentGray}{[Micali-Vazirani 1980]}
\end{itemize}



\subsubsection{Edge Disjoint Paths}\label{sec:edge_disjoint_paths}

\begin{definition}[Edge Disjoint Paths]\label{def:edge_disjoint_paths}
Two paths are \textcolor{AccentRed}{edge disjoint} if they have no edge in common.
\end{definition}

\begin{problem}[Max Edge Disjoint Paths]\label{prob:max_edge_disjoint_paths}
Given a directed graph \(G = (V,E)\) and two nodes \(s,t \in V\), find the maximum number of edge disjoint \(s\)-\(t\) paths.
\end{problem}

\textcolor{AccentBlue}{\nameref{prob:max_flow} formulation of \autoref{prob:max_edge_disjoint_paths}}:
assign unit capacity to each edge in \(G\).


\begin{theorem}\label{thm:max_edge_disjoint_paths_max_flow_equivalence}
\(G\) has \(k\) edge-disjoint \(s\)-\(t\) paths iff \(G'\) has a flow of value \(k\).
\end{theorem}

\begin{theorem}\label{thm:k_edge_disjoint_paths_k_flow_equivalence}
Max number of edge-disjoint \(s\)-\(t\) paths equals value of \nameref{prob:max_flow}.
\end{theorem}


\begin{proof}[``$\le$'']
Let \(P_1, \ldots, P_k\) be \(k\) edge disjoint \(s\)-\(t\) paths.
Create a flow \(f\) as follows:
Make each path \(P_i\) carry 1 unit of flow from \(s\) to \(t\),
i.e. set \(f(e) = 1\) if \(e\) participates in some path \(P_i\), and \(f(e) = 0\) otherwise.
Since the paths are edge disjoint, no edge carries more than 1 unit of flow,
so capacity constraints \eqref{eq:capacity_constraints} are satisfied.
Flow conservation \eqref{eq:flow_conservation} is satisfied since at an internal node, each path has exactly one incoming and one outgoing edge.
The value of the flow is \(v(f) = k\) (\(k\) edges out of \(s\) carry flow one unit of flow).
\end{proof}

\begin{proof}[``$\ge$'']
Let \(f\) be a flow of value \(k\).
By \nameref{thm:integrality}, there exists a 0-1 flow \(f\) of value \(k\).
Consider an edge \((s,u)\) with \(f(s,u) = 1\).
By flow conservation \eqref{eq:flow_conservation}, there exists an edge \((u,v)\) with \(f(u,v) = 1\).
Continue until reaching \(t\), always choosing a new edge with flow 1.
Once reaching \(t\) we found one path.
Remove all edges of this path from the graph and repeat \(k\) times.
We obtain \(k\) (not necessarily sim\tikzmark{simple}ple) edge disjoint \(s\)-\(t\) paths (one for each edge our of \(s\) with flow 1).
\end{proof}
\newcommand{\annotdropcycle}{0.5cm}  % how far below the text the note sits
\newcommand{\arrowpadcycle}{0.1cm}   % small offset between mark and arrow target
\begin{tikzpicture}[remember picture,overlay, font=\scriptsize]

  \node[anchor = north west] (note1) at ($(pic cs:simple)+(0.4,-\annotdropcycle)$) {can eliminate cycles to get simple paths if desired}; % depth arrow
  \draw[->, outer sep=0cm, inner sep=0cm] (note1.west) -- ($(pic cs:simple)+(0,-\arrowpadcycle)$);  % offset label

\end{tikzpicture}
% \vspace{\dimexpr\annotdrop-0.1cm\relax}


We reduced the problem `\nameref{prob:max_edge_disjoint_paths}' to the known problem `\nameref{prob:max_flow}'.





\subsubsection{Network Connectivity}\label{sec:network_connectivity}

\begin{definition}
A set of edges \(F \subseteq E\) \textcolor{AccentRed}{disconnects \(t\) from \(s\)} if all \(s\)-\(t\) paths use at least one edge in \(F\).
\end{definition}

\begin{problem}[Network Connectivity]\label{prob:network_connectivity}
Given a directed graph \(G = (V,E)\) and two nodes \(s,t \in V\), 
find the minimum number of edges whose removal disconnects \(t\) from \(s\).
\end{problem}

We can solve \autoref{prob:network_connectivity} by finding a \nameref{prob:min_cut} in a network with unit capacities on all edges.

\begin{theorem}[\textcolor{AccentGray}{[Menger, 1927]}]\label{thm:network_connectivity_max_edge_disjoint_paths_equivalence}
The max number of edge disjoint \(s\)-\(t\) paths is equal to the min number of edges whose removal disconnects \(t\) from \(s\).
\end{theorem}

Special case of \nameref{thm:max-flow-min-cut} (\autoref{thm:max-flow-min-cut}):
\begin{itemize}
  \item derive a \nameref{def:flow_network} with unit capacities
  \item min number of edges whose removal disconnects \(t\) from \(s\) = \nameref{prob:max_flow}
  \item max number of edge disjoint \(s\)-\(t\) paths = \nameref{prob:max_flow}
\end{itemize}



\clearpage
% !TEX root = ../algo-summary.tex

\section{Complexity Theory}\label{sec:complexity_theory}

`efficiently solvable' = solvable in polynomial time.

many problems arre `hard', i.e. no known polynomial time solution is known

a large class of such problems has been characterized and shown to be equivalent:
NP-complete problems.
a polynomial algo for one would mean a polynomial time algo for \emph{all} NP-complete problems.

NP-complete in practice: prolems that are computationally hard for all partical purposes. 
look for an approximation algorithm.

\smallskip
{
\renewcommand{\arraystretch}{1.4}
\centering
\begin{tabular}{>{\raggedright\arraybackslash}p{0.4\linewidth} >{\raggedright\arraybackslash}p{0.45\linewidth}}
{\textcolor{AccentBlue}{Algorithm design patterns.}} & {\textcolor{AccentBlue}{Example}} \\
% \hline
Greed. & $O(n\log n)$ interval scheduling. \\
Divide-and-conquer. & $O(n\log n)$ FFT, merge sort. \\
Dynamic programming. & $O(n^{2})$ edit distance. \\
Duality. & $O(n^{3})$ bipartite matching. \\
\textcolor{AccentRed}{Reductions.} & \\ 
\textcolor{gray}{Local search.} & \\
\textcolor{gray}{Randomization.} & \\[2mm]
% \hline
{\textcolor{AccentBlue}{Algorithm design anti-patterns.}} & \\
% \hline
\textcolor{AccentRed}{NP-completeness.} & $O(n^k)$ algorithm unlikely. \\
PSPACE-completeness. & $O(n^k)$ certification algorithm unlikely. \\
Undecidability. & No algorithm possible. \\
\end{tabular}
}

\medskip

\textcolor{AccentBlue}{Working Definition} [Cobham 1964, Edmonds 1965, Rabin 1966].
In practice, we are able to solve problems with polynomial time algorithms.


Which problems will we be able to solve in practice?

\begin{tabular}{>{\raggedright\arraybackslash}p{0.34\linewidth} |>{\raggedright\arraybackslash}p{0.45\linewidth}}
{\textcolor{AccentBlue}{Yes}} & {\textcolor{AccentBlue}{Probably No}} \\
\hline
Shortest path & Longest path \\
Matching & 3D-matching \\
\nameref{prob:min_cut} & Max Cut \\
2-SAT & 3-SAT \\
Planar 4-color & Planar 3-color \\
Bipartite vertex cover & Vertex cover \\
Primality testing & Factoring \\
\end{tabular}


\medskip


\textcolor{AccentBlue}{Desired}.
Classify problems according to those that can be solved in polynomial-time and those cannot.

...those that cannot: provably require exponential-time.


\textcolor{AccentBlue}{Frustrating news}.
Huge number of fundamental problems have defied classification for decades.

\textcolor{AccentBlue}{This chapter}.
Show that some fundamental problems are ``computationally equivalent'' and appear to be different manifestations of \textcolor{AccentRed}{one really hard} problem.



\textcolor{AccentBlue}{Decision problem}.
A problem whose output is Yes/No

\begin{example}[decision version of \nameref{def:mst}]\label{ex:mst_decision}
Given a weighted graph \(G\) and an integer \(k\), 
does \(G\) have a spanning tree of weight (at most) \(k\)?
\end{example}

A decision problem can be seen as a language recognition problem.

\begin{example}[continues=ex:mst_decision]
Define the language 
\[
L_{\text{\nameref{def:mst}}} = \{  (G, k) \mid \text{\(G\) has a spanning tree of weight at most \(k\)} \}.
\]
where \((G, k)\) is a reasonable encoding of the pair \(G, k\) as a string.
Given a string \(x=(G, k)\), does \(x\) belong to the language \(L_{\text{\nameref{def:mst}}}\)?
\end{example}

So solving the decision problem = solving a language membership problem.

The language of a decision problem \(\Pi\):
\[
L(\Pi) = \{ x \in \{0,1\}* \mid \text{\(x\) is a representation of a Yes-instance of \(\Pi\)} \}.
\]

Solving \(\Pi\):
determine the anser (Y/N) to an arbitrary instance of \(\Pi\)

Solving \(L(\Pi)\): 
determine if a string \(x\) (representing an arbitrary instance of \(\Pi\)) belongs in \(L(\Pi)\).

Given a string \(x\) we can determine if \(x \in L(\Pi)\):
\begin{itemize}
  \item decode \(x\) as an instance of \(\Pi\)
  \item feed \(x\) to an algorithm that solves \(\Pi\)
  \item if answer is Y, accept \(x\); otherwise reject \(x\)
\end{itemize}

time required: time to decode \(x\) + time to solve \(\Pi\)



\begin{definition}[P]\label{def:P}
set of all languages (i.e., decision problems) for which membership can be determined in (worst case) polynomial time.
\end{definition}











\subsection{Reductions}\label{sec:reductions}



\begin{definition}[polynomial time reduction]\label{def:polynomial_time_reduction}
Given two decision problems \(X\) and \(Y\), we write
\begin{equation}\label{eq:polynomial_reduction}
{\color{Red}\boxed{\color{black}
X \leq_{P} Y
}}
\end{equation}
if there is a polynomial-time computable function \(f\) such that %\(x\) is a Yes-instance of problem \(X\) if and only if \(f(x)\) is a Yes-instance of problem \(Y\).
\begin{equation}\label{eq:poly_time_reduction_equivalence}
x \in X \iff f(x) \in Y
\end{equation}
where membership `\(\in\)' means `is a Yes-instance of the problem'.
\end{definition}

% Conceptually, this can be visualized as follows:
% Conceptually, $X \leq_{P} Y$ can be visualized as follows:
\autoref{def:polynomial_time_reduction} can be viewed as follows:
\[
\begin{tikzpicture}[>=Stealth, scale=0.5, font=\footnotesize]

% \node[font=\normalsize] at (-6, 2) {$X \leq_{P} Y$};

% ---------------- outer box ----------------
\draw (-2.5,-2) rectangle (12,2);
\node[anchor=west] at (-1.75,2.3) {subroutine for $X$};

% ---------------- translate block ----------------
\draw (-1.0,0.55) .. controls (-0.5,1.2) and (0.5,1.2) .. (1.0, 0.55);
\draw (-1.0,-0.55) .. controls (-0.5,-1.2) and (0.5,-1.2) .. (1.0, -0.55);
\draw (-1.6,-0.55) rectangle (1.6,0.55);
\node at (0,0) {translate};

\node at (-0.9,1.05) {${f}$};

% ---------------- U block ----------------
\draw (6,-1.) rectangle (11,1);
\node at (8.5,0) {subroutine for $Y$};

% ---------------- arrows ----------------
\draw[->] (-5,0) -- node[above, midway] {$x$} (-1.6,0);
\draw[->] (1.6,0) -- node[above, midway] {$x' = f(x)$} (6.0,0);
\draw[->] (11,0.55) -- (14,0.55) node[right] {yes};
\draw[->] (11,-0.55) -- (14,-0.55) node[right] {no};

\end{tikzpicture}
\]



\begin{remark}
  We say ``\(X\) reduces polynomially to \(Y\)''.
  \eqref{eq:poly_time_reduction_equivalence} means:
  \begin{itemize}
    \item \(x \in X \implies f(x) \in Y\)
    \item \(f(x) \in Y \implies x \in X\)
  \end{itemize}
  or equivalently, by taking contrapositives,
  \begin{itemize}
    \item \(f(x) \notin Y \implies x \notin X\)
    \item \(x \notin X \implies f(x) \notin Y\)
  \end{itemize}
  where `\(\notin\)' means `is a No-instance of the problem'.
\end{remark}

\textcolor{AccentBlue}{Purpose}.
Classify problems according to their \textcolor{AccentRed}{relative} difficulty.



\begin{property}[transitivity]
\(L_1 \leq_{P} L_2\) and  \(L_2 \leq_{P} L_3\) implies \(L_1 \leq_{P} L_3\) 
\end{property}

\begin{proof}
  Idea: compose the two algorithms.

  Solve an instance \(x\) of \(X\) by an algorithm to solve \(Z\):
  \begin{itemize}
    \item transform \(x\) to \(y = f_{X \to Y}(x)\) (an instance of \(Y\)).
    \item transform \(y\) to \(z = f_{Y \to Z}(y)\) (an instance of \(Z\))
    \item feed \(z\) to the algorithm for \(Z\). If the answer is yes, then it is yes for \(y\). Output yes for \(x\).
    \qedhere
  \end{itemize}
\end{proof}

\begin{example}
  \(\text{\nameref{prob:3-sat}} \leq_{P} \text{\nameref{prob:independent-set}} \leq_{P} \text{\nameref{prob:vertex-cover}} \leq_{P} \text{\nameref{prob:set-cover}}\)
\end{example}



\begin{fact}
Assume \(X \leq_{P} Y\).

\textcolor{AccentBlue}{Design algorithms}.
If we can solve \(Y\) in poly time, we \textcolor{AccentRed}{can also} solve \(X\) in poly time as well (the algorithm that solves \(Y\) in poly time can be used to solve \(X\) in poly time).

\textcolor{AccentBlue}{Establish intractability}.
If we cannot solve \(X\) in poly time, we \textcolor{AccentRed}{cannot} solve \(Y\) in poly time either.
\end{fact}

\textcolor{AccentBlue}{Establish equivalence}.
If \(X \leq_{P} Y\) and \(Y \leq_{P} X\), we write \(X \equiv_{P} Y\).


\medskip


Next, we introduce \emph{three basic reduction strategies};
\nameref{sec:reduction_by_simple_equivalence},
\nameref{sec:reduction_from_special_case_to_general_case},
\nameref{sec:reduction_by_encoding_with_gadgets}.

\subsubsection{Reduction by simple equivalence}\label{sec:reduction_by_simple_equivalence}

\begin{problem}[Independent Set]\label{prob:independent-set}
Given a graph \(G=(V,E)\) and an integer \(k\),
is there a subset of vertices \(S \subseteq V\) such that \(|S| \geq k\) and 
no two vertices in \(S\) are joined by an edge?
(for each edge at most one of its endpoints is in \(S\))
\end{problem}

\begin{problem}[Vertex Cover]\label{prob:vertex-cover}
Given a graph \(G=(V,E)\) and an integer \(k\),
is there a subset of vertices \(T \subseteq V\) such that \(|T| \leq k\) and, for each edge,
at least one of its endpoints is in \(T\)?
(for each edge at least one of its endpoints is in \(T\))
\end{problem}


\begin{claim}\label{claim:vc_equiv_is}
\(\text{\nameref{prob:vertex-cover}} \equiv_{P} \text{\nameref{prob:independent-set}}\). (\(\text{IS} \leq_{P} \text{VC}\) and \(\text{VC} \leq_{P} \text{IS}\))
\end{claim}
\begin{proof}
The idea is to show that \(S\) is an independent set in \(G\) iff \(V \setminus S\) is a vertex cover in \(G\).
And then
\begin{itemize}
  \item \(\text{IS} \leq_{P} \text{VC}\):
    given an instance \((G, k)\) of \hyperref[prob:independent-set]{IS}, 
    construct the instance \((G, n - k)\) of \hyperref[prob:vertex-cover]{VC}.
  \item \(\text{VC} \leq_{P} \text{IS}\):
    given an instance \((G, k)\) of \hyperref[prob:vertex-cover]{VC}, 
    construct the instance \((G, n - k)\) of \hyperref[prob:independent-set]{IS}.
\end{itemize}

To prove the idea, we show the two directions of the equivalence:

`\(\Rightarrow\)':  
Let \(S\) be an independent set in \(G\).
Consider an arbitrary edge \((u,v) \in E\).
\(S\) independent set \(\Rightarrow\) \(u \notin S\) or \(v \notin S\) \(\Rightarrow\) \(u \in V \setminus S\) or \(v \in V \setminus S\).
Thus, \(V \setminus S\) is a vertex cover in \(G\).

`\(\Leftarrow\)':
Let \(T\) be a vertex cover in \(G\).
Set \(S = V \setminus T\).
Consider two nodes \(u, v \in S\).
An edge \((u,v)\) would not be covered by \(T\), thus no such edge can exit.
So, no two nodes in \(S\) are joined by an edge \(\Rightarrow\) \(S\) is an independent set in \(G\).
\end{proof}




\subsubsection{Reduction from special case to general case}\label{sec:reduction_from_special_case_to_general_case}

\begin{problem}[Set Cover]\label{prob:set-cover}
Given a set \(U\) (\emph{universe}) of elements, a collection \(F =\{S_1, \ldots, S_m\}\) of subsets of \(U\), and an integer \(k\),
does there exist a collection \(C\) of at most \(k\) of these subsets whose union is \(U\)?
\end{problem}

\begin{claim}\label{claim:vc_reduces_to_sc}
\(\text{\nameref{prob:vertex-cover}} \leq_{P} \text{\nameref{prob:set-cover}}\).
\end{claim}
\begin{proof}
Given an instance \((G=(V,E), k)\) of \nameref{prob:vertex-cover}, we construct an instance \((U, \{S_i\}, k)\) of \nameref{prob:set-cover} such that 
\(G\) has a \hyperref[prob:vertex-cover]{VC} of size \(k\) iff \(U\) has a \hyperref[prob:set-cover]{SC} of size \(k\).

Create the \nameref{prob:set-cover} instance as follows:
\begin{itemize}
  \item  \(U = E\) (universe = edges of \(G\))
  \item \(k\) remains the same
  \item \(S_v = \{ e \in E \mid e \text{ is incident to } v \}\) for each \(v \in V\)
\end{itemize}

\(G\) has a \hyperref[prob:vertex-cover]{VC} of size \(\leq k\) iff \(U\) has a \hyperref[prob:set-cover]{SC} of size \(\leq k\):
\begin{itemize}
\item 
Vertex cover \(S\) has nodes incident to all edges in \(E\).
Thus, the sets \(S_v\), where \(v \in S\), must ``cover'' \(E\) (their union is \(U\)).
\item
Given a set cover for \(U\), we take the vertex of each set of the set cover.
The result will be a vertex cover.
\qedhere
\end{itemize}
\end{proof}

\begin{caution}
Note that for every \(e \in E\), there are exactly two sets \(S_u\) and \(S_v\) such that \(e \in S_u\) and \(e \in S_v\) (where \(e=(u,v)\)).
A reduction from \nameref{prob:set-cover} to \nameref{prob:vertex-cover} can therefore \textcolor{AccentRed}{not} work, since in \nameref{prob:set-cover} an element of the universe can belong to an arbitrary number of sets.
So \nameref{prob:vertex-cover} is a \emph{special case} of \nameref{prob:set-cover} where each element of the universe belongs to exactly two sets.
\end{caution}







\subsubsection{Reduction by encoding with gadgets}\label{sec:reduction_by_encoding_with_gadgets}

% \textcolor{AccentPurple}{Satisfiability}

\textcolor{AccentBlue}{Literal}: boolean variable or its negation

\textcolor{AccentBlue}{Clause}: disjunction of literals

% \textcolor{AccentBlue}{Conjunctive Normal Form}: a propositional formula \(\Phi\) that is the conjunction of clauses

\begin{definition}[Disjunctive Normal Form]\label{def:DNF}
A formula \(\Phi\) is in \emph{disjunctive normal form} (DNF) if there exist literals \(L_{i,j}\) such that
\[
\begin{verticalhack}
  \Phi=  \bigvee_{i=1}^n\left(\bigwedge_{j=1}^{m_i} L_{i, j}\right) =  \left(L_{1,1} \wedge \ldots \wedge L_{1, m_1}\right) \vee \ldots  \vee\left(L_{n, 1} \wedge \ldots \wedge L_{n, m_n}\right) 
\end{verticalhack}
\qedhere
\]
\end{definition}
\begin{definition}[Conjunctive Normal Form]\label{def:CNF}
A formula \(\Phi\) is in \emph{conjunctive normal form} (CNF) if there exist literals \(L_{i,j}\) such that
\[
\begin{verticalhack}
  \Phi=  \bigwedge_{i=1}^n\left(\bigvee_{j=1}^{m_i} L_{i, j}\right) =  \left(L_{1,1} \vee \ldots \vee L_{1, m_1}\right) \wedge \ldots  \wedge\left(L_{n, 1} \vee \ldots \vee L_{n, m_n}\right)
\end{verticalhack}
\qedhere
\]
\end{definition}


\begin{problem}[SAT]\label{prob:sat}
Given a boolean formula \(\Phi\), does it have a satisfying truth assignment?
\end{problem}

\begin{remark}
\nameref{prob:sat} is trivial if the formulas are restricted to those in \hyperref[def:DNF]{DNF}, that is, they are a disjunction of conjunctions of literals. 
Such a formula is indeed satisfiable if and only if at least one of its conjunctions is satisfiable, and a conjunction is satisfiable if and only if it does not contain both $x$ and $\neg x$ for some variable $x$. 
This can be checked in linear time.

But it can take \hl[3]{exponential time and space} to convert a general \nameref{prob:sat} problem into an equisatisfiable \hyperref[def:DNF]{DNF}. 
% To obtain an example, exchange $\wedge$ and $\vee$ in the definitions above exponential blow-up example for conjunctive normal forms.

Furthermore, checking if a \hyperref[def:CNF]{CNF} is a tautology is also trivial, since a formula in \hyperref[def:CNF]{CNF} is a tautology iff all its clauses are tautologies, and a clause of a \hyperref[def:CNF]{CNF} is a tautology iff it contains both $x$ and $\neg x$ for some $x$.
On the other hand, for a formula in \hyperref[def:DNF]{DNF}, the tautology problem does not ``localize'' in the same way; the relation between the clauses is important here.
\end{remark}

\begin{caution}
While for a \hyperref[def:DNF]{DNF} formula both checking satisfiability as well as unsatisfiability are easy (both in P), 
the problem of checking unsatisfiability for a \hyperref[def:CNF]{CNF} formula is coNP-complete.
Checking tautology for a \hyperref[def:DNF]{DNF} formula is coNP-complete as well.
See Section~\ref{sec:co-np}.
\end{caution}

\begin{problem}[CNF-SAT]\label{prob:cnf-sat}
Given \hyperref[def:CNF]{CNF} formula \(\Phi\), does it have a satisfying truth assignment?
\end{problem}

\begin{remark}
Converting a general \nameref{prob:sat} problem into an equisatisfiable \hyperref[def:CNF]{CNF} formula can be done in polynomial time and space (using auxiliary variables if necessary), using the \emph{Tseitin transformation}.
\end{remark}

We have \(\text{\nameref{prob:sat}} \equiv_{P} \text{\nameref{prob:cnf-sat}}\).

\begin{problem}[3-SAT]\label{prob:3-sat}
\nameref{prob:sat} where each clause has exactly 3 literals (each corresponding to a different variable).
\end{problem}

\begin{claim}\label{claim:3sat_reduces_to_is}
\(\text{\nameref{prob:3-sat}} \leq_{P} \text{\nameref{prob:independent-set}}\).
\end{claim}

\begin{proof}
  Given an instance \(\Phi\) of \nameref{prob:3-sat} with \(k\) clauses, 
  we construct an instance \((G, k)\) of \nameref{prob:independent-set}, which has an independent set of size \(k\) iff \(\Phi\) is satisfiable.

  \textcolor{AccentBlue}{Construction}.
  \(G\) contains one vertex for each literal (3 vertices for each clause).
  Connect the 3 literals of one clause in a triangle (one clause - one triangle).
  Connect each literal to each of its negations (\emph{conflict links}).
  If \(\Phi\) has \(k\) clauses, \(G\) has \(k\) triangles.

  % \begin{remark}
  \textcolor{AccentBlue}{Remarks}.
    \begin{itemize}
    \item
    The reason for conflict links is that that \(x\) and \(\neg x\) cannot both be in the same independent set.
    Enforces the restriction that only one of \(x\) and \(\neg x\) can be set to 1.
    \item
    Reduction \hl{does not attempt to solve the \nameref{prob:3-sat} problem}.
    % \qedhere
    \end{itemize}
  % \end{remark}

\textcolor{AccentBlue}{Claim}:
\(G\) has an independent set of size \(k = |\Phi|\) iff \(\Phi\) is satisfiable.
  
`\(\Rightarrow\)':  
Let \(S\) be an independent set of size \(k\).
\(S\) can contain \(\leq 1\) vertex from each triangle.
Since \(|S| = k\), and there are \(k\) triangles, \(S\) must contain exactly one vertex from each triangle.
Set these literals to true, and any other variables in a consistent way.
(Then at least 1 literal is set to true in each clause.)
Truth assignment is consistent (not both \(x\) and \(\neg x\) can be in \(S\) because they are connected by a conflict link in \(G\)).
All clauses are satisfied.
Thus, \(\Phi\) is satisfiable.

`\(\Leftarrow\)':
Suppose \(\Phi\) has a satisfying assignment.
Each clause of \(\Phi\) is true in this assigment.
Select one true literal from each clause (one vertex from each triangle) and put it in a set \(S\).
\(S\) is an independent set of size \(k\) by construction, 
because no two nodes in \(S\) belong to the same triangle,
and no two conflicting literals can be in \(S\) (not both can be true).
Thus, there can be no edge connecting any two vertices in \(S\), 
because non-triangle edges connect only conflicting literals \(x_i, \neg x_i\).
\end{proof}



\textcolor{AccentBlue}{Decision problem}. Does there \textcolor{AccentRed}{exist} a vertex cover of size (\(\leq\)) \(k\)?

\textcolor{AccentBlue}{Search problem}. \textcolor{AccentRed}{Find} a vertex cover of minimum cardinality.

Decision problem: Yes/No answer. It is easier than search problem but not necessarily fundamentally easier.

\textcolor{AccentBlue}{Why deal with decision problems?}
\begin{itemize}
\item
In answering complexity questions, our goal is to show that a problem cannot be solved efficiently.
If we can do this for the simpler decision problem, then the same is true for the more general optimization problem.
(Decision problems are simpler to establish proofs)
\item
Typically, if we can solve the decision problem efficiently, we can construct an efficient solution for the optimization problem too,
i.e.
\(\text{optimization problem} \equiv_{P} \text{decision problem}\) for many problems of interest.
\end{itemize}






\subsection{Clique, Vertex Cover, Dominating Set}\label{sec:clique_vertex_cover_dominating_set}

\subsubsection{Clique}\label{sec:clique}

\begin{problem}[Clique]\label{prob:clique}
Given a graph \(G=(V,E)\) and an integer \(k\),
does \(G\) have a subset \(V' \subseteq V\) of size \(|V'| \geq k\) that is complete (for each \(u,v \in V'\), \((u,v) \in E\))?
\end{problem}

\begin{claim}\label{claim:clique_equiv_is_equiv_vc}
\(\text{\nameref{prob:clique}} \equiv_{P} \text{\hyperref[prob:independent-set]{Independent-Set}} \equiv_{P} \text{\hyperref[prob:vertex-cover]{Vertex-Cover}}\).
\end{claim}

\begin{lemma}\label{lem:clique_independent_vc}
Let $G=(V,E)$ be a graph with $n \coloneqq |V|$ vertices, and a subset $V' \subseteq V$ of size $|V'| = k$.
The following are equivalent:
\begin{enumerate}[label=\textbf{(\roman*)}, itemindent=0.25cm, labelsep=0.25cm]
  \item $V'$ is a clique of size $k$ in the complement graph $\overline{G}$.
  \label{lem:civc-clique}
  \item $V'$ is an independent set of size $k$ in $G$.
  \label{lem:civc-independent}
  \item $V \setminus V'$ is a vertex cover of size $n-k$ in $G$.
  \label{lem:civc-vertex-cover}
\qedhere
\end{enumerate}
\end{lemma}

\[
\begin{tikzpicture}[
  scale=0.68,
  font=\footnotesize,
  edge/.style={line width=0.5pt},
  ver/.style={circle,draw=black,line width=0.9pt,minimum size=2.4mm,inner sep=0pt,fill=white},
  sel/.style={ver,fill=gray!60},
]

% ---- parameters ----
\def\h{4}                 % total n = 2h
\pgfmathtruncatemacro{\n}{2*\h}
% \def\k{5}                 % |V'|
\pgfmathtruncatemacro{\k}{\h+1} 
\def\R{1.55}

\pgfmathsetmacro{\step}{360/\n}
% \pgfmathsetmacro{\offset}{270 - \step/2}  % flat bottom edge
\pgfmathsetmacro{\offset}{270}  % peaky bottom node
\pgfmathtruncatemacro{\nminus}{\n-1}
\pgfmathtruncatemacro{\Vstart}{\n-\k+1}   % V' = {Vstart,...,n} (on the left, if k<=h)

% --- place vertices on the circle ---
% mode=0 : highlight V'
% mode=1 : highlight V\V'
\newcommand{\PlaceCycle}[2]{%
  \foreach \i in {1,...,\n}{%
    \pgfmathsetmacro{\ang}{\i*\step+\offset}%
    \edef\mystyle{ver}%
    \ifnum#2=0 % highlight V'
      \ifnum\i<\Vstart\relax\else\edef\mystyle{sel}\fi
    \else      % highlight V\V'
      \ifnum\i<\Vstart\relax\edef\mystyle{sel}\fi
    \fi
    \node[\mystyle] (#1\i) at (\ang:\R) {};%
  }%
}


% ========================= Left: \bar G =========================
\begin{scope}[shift={(0,0)}]
  \PlaceCycle{l}{0}

  %  % debugging: add labels to vertices
  % \foreach \i in {1,...,\n}{%
  %   \node at (\i*360/\n+\offset:\R+0.25) {\i};%
  % }%
  % \node[above, outer sep = 5pt] at (l\Vstart) {$v_{\text{start}}$};

  % edges clique
  \foreach \i in {\Vstart,...,\nminus}{%
    \pgfmathtruncatemacro{\ip}{\i+1}%
    \foreach \j in {\ip,...,\n}{%
      \draw[edge] (l\i)--(l\j);%
    }%
  }%
  % remaining outer circle edges
  \foreach \i in {1,...,\Vstart}{%
    \pgfmathtruncatemacro{\iminus}{\i-1}%
    \ifnum\iminus=0\relax\pgfmathtruncatemacro{\iminus}{\n}\fi
    \draw[edge] (l\iminus)--(l\i);%
  }%

  \node[anchor=west] at (-1.95,1.55) {$\overline{G}$};
  \node[align=center] at (0,-2.35)
    {$V'$ is a clique\\ of size $k$ in $\overline{G}$};
\end{scope}

\node at (3,-2.35) {$\Leftrightarrow$};

% ========================= Middle: G =========================
\begin{scope}[shift={(6,0)}]
  \PlaceCycle{m}{0}
  \pgfmathtruncatemacro{\Vstartm}{\Vstart-1}%
  \foreach \i in {1,...,\Vstartm}{
    \pgfmathtruncatemacro{\ipp}{\i+2}%
    \foreach \j in {\ipp,...,\n}{
      \ifnum\i=1\relax
      \ifnum\j=\n\relax
        % skip edge (1,n)
      \else
        \draw[edge] (m\i)--(m\j);%
      \fi
    \else
      \draw[edge] (m\i)--(m\j);%
    \fi
    }
  }

  \node[anchor=west] at (-1.95,1.55) {$G$};
  \node[align=center] at (0,-2.35)
    {$V'$ is an independent\\set of size $k$ in $G$};
\end{scope}

\node at (9,-2.35) {$\Leftrightarrow$};

% ========================= Right: G (vertex cover highlighted) =========================
\begin{scope}[shift={(12,0)}]
  \PlaceCycle{r}{1}
    \pgfmathtruncatemacro{\Vstartm}{\Vstart-1}%
  \foreach \i in {1,...,\Vstartm}{
    \pgfmathtruncatemacro{\ipp}{\i+2}%
    \foreach \j in {\ipp,...,\n}{
      \ifnum\i=1\relax
      \ifnum\j=\n\relax
        % skip edge (1,n)
      \else
        \draw[edge] (r\i)--(r\j);%
      \fi
    \else
      \draw[edge] (r\i)--(r\j);%
    \fi
    }
  }

  \node[anchor=west] at (-1.95,1.55) {$G$};
  \node[align=center] at (0,-2.35)
    {$V\setminus V'$ is a vertex cover\\ of size $n-k$ in $G$};
\end{scope}

\end{tikzpicture}
\]

\begin{proof}
We prove the implications in the following triangle:
\(
\newcommand{\scalefraction}{0.68}
\begin{tikzcd}[column sep={\scalefraction*1cm,between origins},
               row sep={\scalefraction*1.732050808cm,between origins},
               arrows=Rightarrow]
  & \ref{lem:civc-clique}
    \arrow[dr,""{pos=.5,sloped,overlay, inner sep=0pt,
      label={[black]center:\hyperlink{proof:civc-1-2}{\phantom{\rule{10pt}{15pt}}}}}]
  &
  \\
  \ref{lem:civc-vertex-cover}
    \arrow[ur,""{pos=.5,sloped,overlay, inner sep=0pt,
      label={[black]center:\hyperlink{proof:civc-3-1}{\phantom{\rule{10pt}{15pt}}}}}]
  &
  &
  \ref{lem:civc-independent}
    \arrow[ll,""{pos=.5,sloped,overlay, inner sep=0pt,
      label={[black]center:\hyperlink{proof:civc-2-3}{\phantom{\rule{15pt}{5pt}}}}}]
\end{tikzcd}
\)

\vspace{-2\baselineskip}
\hypertarget{proof:civc-1-2}{
\underline{\ref{lem:civc-clique} $\Rightarrow$ \ref{lem:civc-independent}:}}

If \(V'\) is a clique in \(\overline{G}\),
then for each \(u,v \in V'\), \((u,v)\) is an edge of \(\overline{G}\)
implying that \((u, v)\) is not an edge of \(G\),
implying that \(V'\) is an independent set in \(G\).
\tikzmark{edge_mark}
\definecolor{vertRed}{RGB}{204, 1, 0}
\begin{tikzpicture}[remember picture,overlay, font=\footnotesize,
  myedge/.style={line width=0.5pt},
  myvert/.style={circle,draw=black,line width=0.9pt,minimum size=2.4mm,inner sep=0pt,fill=vertRed},
]
\coordinate (edge) at (pic cs:edge_mark);
\coordinate (left) at ($(edge)+(1.4,-0.22)$);
\node[myvert] (u) at (left) {};
\node[myvert] (v) at ($(left)+(1.2,0)$) {};
% \draw[myedge] (u) -- (v);
\node[anchor=east] at (u.west) {$u$};
\node[anchor=west] at (v.east) {$v$};
\end{tikzpicture}

\hypertarget{proof:civc-2-3}{
\underline{\ref{lem:civc-independent} $\Rightarrow$ \ref{lem:civc-vertex-cover}:}}

\autoref{claim:vc_equiv_is}.

\hypertarget{proof:civc-3-1}{
\underline{\ref{lem:civc-vertex-cover} $\Rightarrow$ \ref{lem:civc-clique}:}}

% If V\V′ is a vertex cover for G, then for any u,v ∈ V′ there is no edge (u,v) in G (because not covered by V\V’), implying that there is an edge (u,v) in G, implying that V ′ is a clique in G.
If \(V \setminus V'\) is a vertex cover for \(G\),
then for any \(u,v \in V'\) there is no edge \((u,v)\) in \(G\) (because not covered by \(V \setminus V'\)),
implying that there is an edge \((u,v)\) in \(\overline{G}\),
implying that \(V'\) is a clique in \(\overline{G}\).

This closes the triangle of implications and proves \autoref{lem:clique_independent_vc}.
\end{proof}




\subsubsection{Dominating Set}\label{sec:dominating_set}

\begin{definition}[Dominating Set]\label{def:dominating-set}
  A dominating set in a graph \(G=(V,E)\) is a subset of vertices \(V' \subseteq V\)
  such that every vertex in the graph is either in \(V'\) or is adjacent to some vertex in \(V'\).
\end{definition}
\begin{problem}[Dominating Set]\label{prob:dominating-set}
Given a graph \(G=(V,E)\) and an integer \(k\),
does \(G\) have a dominating set of size \(k\)?
\end{problem}

\textcolor{AccentBlue}{Dominating Set is NP-complete}.
Reduction from \nameref{prob:vertex-cover}. 
\(\text{\hyperref[prob:vertex-cover]{VC}} \leq_{P} \text{\hyperref[prob:dominating-set]{DS}}\).

\begin{caution}
Note:
\begin{itemize}
\item
If \(G\) has no isolated vertices, a vertex cover for \(G\) is a dominating set for \(G\).
\item
But a dominating set for \(G\) need not be a vertex cover
\qedhere
\end{itemize}
\end{caution}

\textcolor{AccentBlue}{Transformation from \hyperref[prob:vertex-cover]{VC} to \hyperref[prob:dominating-set]{DS}}.
\begin{itemize}
\item
Given \((G,k)\) an instance of \hyperref[prob:vertex-cover]{VC},
produce \((G',k')\) an instance of \hyperref[prob:dominating-set]{DS} such that
\(G\) has a vertex cover of size \(k\) iff \(G'\) has a dominating set of size \(k'\).
\item
Let \(V'\) be the VC of \(G\).
Let \(V''\) be the DS of \(G'\).
\item
\textcolor{AccentBlue}{Observation}.
If \(G\) has isolated vertices we must include them in \(V''\).
\item
Map the notion of ``incident edge'' to notion of ``adjacent vertex''.
Map edges to vertices.
\end{itemize}

So, given \((G,k)\) for \hyperref[prob:vertex-cover]{VC} create graph \(G'\) as follows:
\begin{itemize}
\item
Initially, \(G' = G\).
\item
For each edge \((u,v)\) in \(G\) we create a new vertex \(w_{uv}\) in \(G'\).
\item 
Add edges \((u, w_{uv})\) and \((v, w_{uv})\) in \(G'\).
\item
Let \(I\) denote the set of isolated vertices in \(G\).
\item
Set \(k' = k + |I|\).
\item
Output \((G', k')\).
\end{itemize}




\begin{lemma}\label{lem:vc_reduces_to_ds}
\(G\) has a vertex cover of size \(k\) iff \(G'\) has a dominating set of size \(k'\).
\end{lemma}
\begin{proof}
`\(\Rightarrow\)' (if \(V'\) is a vertex cover for \(G\), then \(V'' = V' \cup I\) is a dominating set for \(G'\)):
Indeed all vertices of \(G'\) are either in \(V''\) or adjacent to \(V''\).
\(|V''| = |V'| + |I| \leq k + |I| = k'\).

`\(\Leftarrow\)' (if \(G'\) has a dominating set \(V''\) of size \(k'\) then \(G\) has a vertex cover of size \(k = k' - n_I\)):
All isolated vertices of \(G'\) must be in \(V''\).
Let \(V'''\) = \(V'' \setminus V_I\) be the remaining \(k = k' - n_I\) vertices.
Modify \(V'''\) so that it contains no middle vertex 
(substitute the ``middle vertex'' by any adjacent regular vertex. Still dominates the same vertices).
Let \(V'\) be the resulting set.
\(V'\) must be a \hyperref[prob:vertex-cover]{VC} of \(G\)
(if an edge not covered then middle vertex not adjacent to DS).
\(V'\) must be a vertex cover for \(G\).
If there is a edge \((u,v)\) in \(G\) not covered by \(V'\) 
(neither \(u\) nor \(v\) is in \(V'\)),
then the middle vertex \(w_{uv}\) would not be adjacent to any vertex of \(V''\) in \(G'\).
This contradicts that \(V''\) was a dominating set for \(G'\).
\end{proof}
























\subsection{NP-Completeness}\label{sec:np_completeness}

\subsubsection{Complexity Classes}
\textcolor{AccentBlue}{Complexity class}:
collection of languages (decision problems), which are similar in terms of how hard it is to determine membership (solve).

Recall \autoref{def:P}:
P is the set of all languages (i.e., decision problems) for which membership can be \emph{determined} in (worst case) polynomial time.

\nameref{def:P} is a complexity class

\renewcommand{\arraystretch}{1.35}
\setlength{\tabcolsep}{0.5pt}
\arrayrulecolor{black}
\begin{tabularx}{\linewidth}{|>{\arraybackslash}p{0.16\linewidth}|>{\arraybackslash}X|>{\arraybackslash}p{0.28\linewidth}|}
\hline
{ Problem}      & { Description}                                          & { Algorithm} \\
\hhline{|=|=|=|}
Multiple        & Is $x$ a multiple of $y$?                               & Grade school division \\
\hline
RelPrime        & Are $x$ and $y$ relatively prime?                       & Euclid (300 BCE) \\
\hline
Primes          & Is $x$ prime?                                           & AKS (2002) \\
\hline
Edit-Distance   & Is the edit distance between $x$ and $y$ less than 5?   & Dynamic programming \\
\hline
LSolve          & Is there a vector $x$ that satisfies $Ax=b$?            & Gauss-Edmonds elim. \\
\hline
\end{tabularx}

Not all languages are in \nameref{def:P}.

\begin{definition}[Simple Cycle]\label{def:simple-cycle}
Given a graph \(G=(V,E)\),
a simple cycle is a closed path that visits each vertex at most once.
\end{definition}

\begin{problem}[Hamiltonian Cycle]\label{prob:ham-cycle}
Given a graph \(G=(V,E)\),
does \(G\) contain a \nameref{def:simple-cycle} that visits each vertex exactly once?
\end{problem}
or in terms of languages:
\[
\text{HC} = \{  G  \mid \text{\(G\) has a simple cycle that visits all vertices}\}
\]

There is no known polynomial time algorithm for \autoref{prob:ham-cycle}.

NP: non-deterministic polynomial time.

alternative (equivalent) definition (to avoid non-determinism):
set of all languages (i.e., decision problems) for which membership can be \emph{verified} in (worst case) polynomial time.


A decision problem (language recognition problem) may be hard to solve, but, given a string \(y\) (a potential solution) it may be easy to verify that \(y\) corresponds to a solution, a yes answer of the decision problem.
\begin{example}[Hamiltonian Cycle Verification]\label{ex:hamiltonian_cycle_verification}
Hard to find a Hamiltonian cycle in a graph.
But suppose that someone gives a permutation of vertices.
then it is easy to check if this is a legal cycle that visits all vertices in the graph exactly once.
So it is easy to \emph{verify} a \nameref{prob:ham-cycle}.

A sequence of vertices that forms a \hyperref[prob:prob:ham-cycle]{HC} is called a \textcolor{AccentRed}{certificate}.
\end{example}


Note that not all decision problems are easy to verify:
\begin{itemize}
\item 
Given a graph \(G\), does \(G\) have a \emph{unique} Hamiltonian cycle?
\[ \text{UHC} = \{ G \mid \text{\(G\) has a unique Hamiltonian cycle} \} \]
\item
Given a graph \(G\), does \(G\) have \emph{no} Hamiltonian cycle?
\[ \overline{\text{HC}} = \{ G \mid \text{\(G\) has no Hamiltonian cycle} \} \]
\end{itemize}
No known polynomial time verification algorithm for either.



\begin{definition}[Certificate]\label{def:certificate}
% Piece of information which allows us to verify that a decision problem has a solution (a Yes answer) in polynomial time.
% \end{definition}
% \begin{definition}[Certificate]\label{def:certificate}
Piece of information (a string \(t\)) which helps to verify that an instance \(x\) of problem \(X\) has a solution, i.e. that \(x \in X\).
\end{definition}

\begin{definition}[Certifier]\label{def:certifier}
An algorithm \(C(s, t)\) is a \emph{certifier} (also called \emph{verification algorithm}), if for every string \(s\) with \(s \in X\) (\(s\) is a yes instance of \(X\)), 
iff there exists a string \(t\) such that \(C(s, t) = \text{Yes}\).
\begin{itemize}
\item
\(s\): input string (a possible instance of problem \(X\))
\item
\(t\): certificate
\item
for a poly-time certifier, \(|t| \leq p(|s|)\) for some polynomial \(p(\cdot)\)
\end{itemize}
Note, \(C(s, t) = \text{No}\) when \(t\) is not a certificate.
\(C\) is a Yes/No algorithm.
\end{definition}

\begin{definition}[NP]\label{def:NP}
decision problems for which there exists a \emph{poly-time} certifier, i.e. 
\(C(s,t)\) is a poly-time algorithm and \(|t| \leq p(|s|)\) for some polynomial \(p(\cdot)\).
\end{definition}


\begin{problem}[Composites]\label{prob:composites}
Given an integer \(s\), is \(s\) composite (i.e., not prime)?
\end{problem}

\textcolor{AccentBlue}{Certificate}. 
A non-trivial factor \(t\) of \(s\) (i.e., \(1 < t < s\) and \(t \mid s\)).
Note that such a certificate exists iff \(s\) is composite.
Moreover \(|t| \leq |s|\).

\begin{algorithm}[h]
\caption{Certifier for \nameref{prob:composites}}
\label{alg:certifier-composites}
\begin{algorithmic}[1]
\Function{BooleanC}{$s, t$}
  \If{$t \leq 1 \OR t \geq s$} \Return false
  \ElsIf{$s$ is a multiple of $t$} \Return true
  \Else \ \Return false
  \EndIf
\EndFunction
\end{algorithmic}
\end{algorithm}

\textcolor{AccentBlue}{Conclusion}.
\nameref{prob:composites} is in NP.


\begin{fact}\label{fact:sat_in_np}
For \nameref{prob:sat} a \textcolor{AccentBlue}{\nameref{def:certificate}} is a satisfying assignment of variables
and the \textcolor{AccentBlue}{\nameref{def:certifier}} checks that each clause in \(\Phi\) has at least one true ltieral.
\end{fact}

\textcolor{AccentBlue}{Conclusion}.
\nameref{prob:sat} is in NP.

\begin{fact}\label{fact:ham_cycle_in_np}
For \hyperref[prob:ham-cycle]{HC} a \textcolor{AccentBlue}{\nameref{def:certificate}} is a permutation of the \(n\) nodes (which is a Ham. cycle if \(G\) has one)
and the \textcolor{AccentBlue}{\nameref{def:certifier}} checks that the certificate contains each node in \(V\) exactly once and that there is an edge between each pair of adjacent nodes in the permutation.
\end{fact}

\textcolor{AccentBlue}{Conclusion}.
\nameref{prob:ham-cycle} is in NP.

\medskip

\textcolor{AccentBlue}{P}: Decision problems for which there is a \textcolor{AccentRed}{poly-time algorithm}.

\textcolor{AccentBlue}{NP}: Decision problems for which there is a \textcolor{AccentRed}{poly-time certifier}.

\textcolor{AccentBlue}{EXP}: Decision problems for which there is a \textcolor{AccentRed}{exponential-time algorithm}.


\begin{claim}
\(\text{P} \subseteq \text{NP}\).
\end{claim}
\begin{proof}
Consider any problem \(X\) in P.
There exists a poly-time algorithm \(A(s)\) that solves \(X\): 
returns yes, if \(s \in X\), and returns no, otherwise.
Then there exists a poly-time certifier.
Certificate: \(t = \epsilon\) (empty string),
certifier \(C(s, t) = A(s)\).
Thus, \(X\) in NP.
\end{proof}



\begin{claim}
\(\text{NP} \subseteq \text{EXP}\).
\end{claim}
\begin{proof}
Consider any problem \(X\) in NP.
Since \(X\) is in NP, there exists a polynomial time certifier \(C(s, t)\) for \(X\).
Let \(p(|s|)\) be the polynomial time complexity of \(C(s,t)\); \(|t| \leq p(|s|)\).

We construct an algorithm that solves \(X\) on instance \(s\), based on \(C(s, t)\) (use brute force):
\begin{itemize}
\item
Generate all strings \(t\) with \(|t| \leq p(|s|)\).
Run \(C(s, t)\) on each string \(t\).
\item
If \(C(s, t)\) returns \texttt{yes} for any such string \(t\), stop and return \texttt{yes}.
\item
There is an exponential number of possible such strings \(t\), given \(p(|s|)\).
Thus, this is an exponential-time algorithm.
\item
After \(\leq p(|s|)\) rounds the algorithm stops with a \texttt{yes} or \texttt{no} answer.
\qedhere
\end{itemize}
\end{proof}


\begin{openquestion}[P vs NP]
Is $\mathrm{P}=\mathrm{NP}$, i.e. is the decision problem as easy as the verification problem?
\end{openquestion}
The consesus is probably no. There is \$1M prize for proof.




% \begin{tikzpicture}[x=1cm,y=1cm,>=Stealth]

% % --- parameters (keep consistent) ---
% \def\xL{3.0}   \def\yC{4.35} \def\rNP{2.10}
% \def\xR{9.0}
% \def\yRef{2.20}

% % --- horizontal dotted reference line ---
% \draw[gray!65, dotted, line width=0.8pt] (0,\yRef) -- (12,\yRef);

% % --- center complexity axis ---
% \draw[line width=1pt,->] (6,0) -- (6,8);
% \node[rotate=90] at (5.72,3.95) {Complexity};

% % ===================== LEFT: P != NP =====================

% % NP (circle)
% \draw[line width=0.1pt] (\xL,\yC) circle[radius=\rNP];

% % P : shares a SHORT ARC with NP (coincide, then diverge)

% \begin{scope}
%   \path (\xL,\yC) coordinate (cL);

%   \def\angL{250}
%   \def\angR{290}

%   % points on the NP boundary circle
%   \coordinate (pL) at ($(cL)+(\angL:\rNP)$);
%   \coordinate (pR) at ($(cL)+(\angR:\rNP)$);

%   % apex (topmost point of the dotted region)
%   \def\h{-0.5}
%   \coordinate (pA) at ($(cL)+(0,\h)$);

%   % tangent-handle lengths (tune these)
%   \def\t{0.6}  % handles at circle junctions
%   \def\a{1.2}  % handles at apex (horizontal)

%   % unit-ish tangent directions at circle points (CCW tangent)
%   \path let \p1 = ($(pL)-(cL)$) in coordinate (tL) at (-\y1,\x1);
%   \path let \p2 = ($(pR)-(cL)$) in coordinate (tR) at (-\y2,\x2);

%   % C1 at apex: last control of first spline and first control of second spline
%   % lie on the horizontal tangent through pA, mirrored.
%   \coordinate (cAin)  at ($(pA)+(\a,0)$);   % for segment ending at pA
%   \coordinate (cAout) at ($(pA)+(-\a,0)$);  % for segment starting at pA

%   \draw[densely dotted, thick]
%     (pL)
%       arc[start angle=\angL, end angle=\angR, radius=\rNP]
%     (pR)
%       .. controls ($(pR)+\t*(tR)$) and (cAin)  .. (pA)
%       .. controls (cAout) and ($(pL)-\t*(tL)$) .. (pL);
% \end{scope}


% % ---------- NP-hard (left) : Bezier -> circle arc -> Bezier ----------
% \begin{scope}
%   % endpoints at the top border of the left panel
%   \coordinate (S) at (0.5,8);
%   \coordinate (E) at (5.5,8);

%   % middle arc circle (TUNE THESE THREE)
%   \coordinate (C) at (3.00,6.6); % center of the NP-hard middle arc
%   \def\ain{250}                   % angle where we ENTER the arc
%   \def\aout{290}                  % angle where we LEAVE the arc

%   % join points on the circle
%   \coordinate (Jin)  at ($(C)+(\ain:\rNP)$);
%   \coordinate (Jout) at ($(C)+(\aout:\rNP)$);

%   % tangents at join points (CCW tangent = rotate radius by +90°)
%   \path let \p1 = ($(Jin)-(C)$)  in coordinate (Tin)  at (-\y1,\x1);
%   \path let \p2 = ($(Jout)-(C)$) in coordinate (Tout) at (-\y2,\x2);

%   % handle length along tangent (TUNE)
%   \def\k{0.8}

%   % also steer the ends downward a bit (TUNE these if needed)
%   \coordinate (cS) at ($0.5*($(Jin)-\k*(Tin)$)+0.5*(S)$);
%   \coordinate (cE) at ($0.5*($(Jout)+\k*(Tout)$)+0.5*(E)$);

%   \draw[gray, line width=0.9pt]
%     (S)
%       .. controls (cS) and ($(Jin)-\k*(Tin)$) .. (Jin)
%       arc[start angle=\ain, end angle=\aout, radius=\rNP]
%     (Jout)
%       .. controls ($(Jout)+\k*(Tout)$) and (cE) .. (E);
% \end{scope}

% % labels
% \node at (\xL,6.95) {NP-Hard};
% \node at (\xL,5.45) {NP-Complete};
% \node at (\xL-1.2,4.35) {NP};
% \node at (\xL,3.0) {P};
% \node[font=\Large] at (\xL,1.15) {$\mathrm{P}\ \ne\ \mathrm{NP}$};


% % ===================== RIGHT: P = NP =====================

% % big circle (P = NP = NP-Complete)
% \draw[line width=1pt] (\xR,\yC) circle[radius=\rNP];

% \node at (\xR,6.95) {NP-Hard};
% \node[align=center] at (\xR,5.05) {P $=$ NP $=$\\ NP-Complete};
% \node[font=\Large] at (\xR,1.15) {$\mathrm{P}\ =\ \mathrm{NP}$};



% \begin{scope}
%    % endpoints at the top border of the left panel
%   \coordinate (S) at (6.5,8);
%   \coordinate (E) at (11.5,8);

%   % middle arc circle (TUNE THESE THREE)
%   \coordinate (C) at (9.00,4.34); % center of the NP-hard middle arc
%   \def\ain{210}                   % angle where we ENTER the arc
%   \def\aout{330}                  % angle where we LEAVE the arc

%   % join points on the circle
%   \coordinate (Jin)  at ($(C)+(\ain:\rNP)$);
%   \coordinate (Jout) at ($(C)+(\aout:\rNP)$);

%   % tangents at join points (CCW tangent = rotate radius by +90°)
%   \path let \p1 = ($(Jin)-(C)$)  in coordinate (Tin)  at (-\y1,\x1);
%   \path let \p2 = ($(Jout)-(C)$) in coordinate (Tout) at (-\y2,\x2);

%   % handle length along tangent (TUNE)
%   \def\k{0.6}

%   % also steer the ends downward a bit (TUNE these if needed)
%   \coordinate (cS) at ($0.5*($(Jin)-\k*(Tin)$)+0.5*(S)$);
%   \coordinate (cE) at ($0.5*($(Jout)+\k*(Tout)$)+0.5*(E)$);

%   \draw[gray, line width=0.9pt]
%     (S)
%       .. controls (cS) and ($(Jin)-\k*(Tin)$) .. (Jin)
%       arc[start angle=\ain, end angle=\aout, radius=\rNP]
%     (Jout)
%       .. controls ($(Jout)+\k*(Tout)$) and (cE) .. (E);
% \end{scope}



% \end{tikzpicture}









If yes: efficient algorithms for \hyperref[prob:3-color]{3-COLOR}, \hyperref[prob:tsp]{TSP}, \hyperref[prob:factor]{FACTORING}\tikzmark{fact_mark}, \hyperref[prob:sat]{SAT}, ...
\definecolor{vertRed}{RGB}{204, 1, 0}
\begin{tikzpicture}[remember picture,overlay, font=\footnotesize]
\coordinate (fact) at (pic cs:fact_mark);
\coordinate (tiploc) at ($(fact)+(-0.1,0.3)$);
\draw[<-] (tiploc) -- ++(0.6,0.45) node[anchor=south west, outer sep = 0pt, inner sep = 0pt] {would break RSA};
\end{tikzpicture}

If no: no efficient algorithms for \hyperref[prob:3-color]{3-COLOR}, \hyperref[prob:tsp]{TSP}, \hyperref[prob:factor]{FACTORING}, \hyperref[prob:sat]{SAT}, ...

\[
% needs: \usetikzlibrary{calc}
\begin{tikzpicture}[scale=0.75,>=Stealth, font=\footnotesize]

% ---- params ----
\def\xL{3}\def\xR{9}\def\yC{4.35}\def\r{2.10}\def\yRef{2.20}

% ---- helpers (compact) ----
\tikzset{pDots/.style={densely dotted,thick},hard/.style={gray,line width=.9pt}}
\newcommand{\tangent}[3]{% #1=(out coord name) #2=center #3=point
  \path let \p1=($(#3)-(#2)$) in coordinate (#1) at (-\y1,\x1);
}
\newcommand{\Pregion}[5]{% center, r, angL, angR, apex-y-offset
  \begin{scope}
    \coordinate (c) at (#1);
    \def\angL{#3}\def\angR{#4}
    \coordinate (L) at ($(c)+(\angL:#2)$);
    \coordinate (R) at ($(c)+(\angR:#2)$);
    \coordinate (A) at ($(c)+(0,#5)$);
    \def\t{0.6}\def\a{1.2} % tweak if you want
    \tangent{tL}{c}{L}\tangent{tR}{c}{R}
    \draw[pDots]
      (L) arc[start angle=\angL,end angle=\angR,radius=#2]
      (R) .. controls ($(R)+\t*(tR)$) and ($(A)+(\a,0)$) .. (A)
          .. controls ($(A)+(-\a,0)$) and ($(L)-\t*(tL)$) .. (L);
  \end{scope}
}
\newcommand{\NPhard}[6]{% S, E, C, r, ain, aout  (k fixed inside; change if needed)
  \begin{scope}
    \coordinate (S) at (#1);\coordinate (E) at (#2);\coordinate (C) at (#3);
    \def\ain{#5}\def\aout{#6}\def\k{#4} % use #4 as "k" (handle length), radius is global \r
    \coordinate (I) at ($(C)+(\ain:\r)$);
    \coordinate (O) at ($(C)+(\aout:\r)$);
    \tangent{tI}{C}{I}\tangent{tO}{C}{O}
    \coordinate (cS) at ($.5*(S)+.5*($(I)-\k*(tI)$)$);
    \coordinate (cE) at ($.5*(E)+.5*($(O)+\k*(tO)$)$);
    \draw[hard]
      (S) .. controls (cS) and ($(I)-\k*(tI)$) .. (I)
      arc[start angle=\ain,end angle=\aout,radius=\r]
      (O) .. controls ($(O)+\k*(tO)$) and (cE) .. (E);
  \end{scope}
}

% ---- background stuff ----
\draw[gray!65,dotted,line width=.8pt] (0,\yRef) -- (12,\yRef);
\draw[line width=1pt,->] (6,1) -- (6,8);
\node[rotate=90] at (5.72,3.95) {Complexity};

% ---- left: P != NP ----
\draw[line width=1pt] (\xL,\yC) circle[radius=\r];
\Pregion{\xL,\yC}{\r}{250}{290}{-0.5}
\NPhard{0.5,8}{5.5,8}{3.0,6.6}{0.8}{250}{290}

\node at (\xL,7.25) {NP-Hard};
\node at (\xL,5.45) {NP-Complete};
\node at (\xL-1.25,4.2) {NP};
\node at (\xL,3.0) {P};
\node[] at (\xL,1.5) {$\text{P} \ne \text{NP}$};

% ---- right: P = NP ----
\draw[line width=1pt] (\xR,\yC) circle[radius=\r];
\NPhard{6.5,8}{11.5,8}{9.0,4.34}{0.6}{210}{330}

\node at (\xR,7.25) {NP-Hard};
\node[align=center] at (\xR,4.5) {P $=$ NP $=$\\ NP-Complete};
\node[] at (\xR,1.5) {$\text{P} = \text{NP}$};

\end{tikzpicture}
\]


\begin{definition}[NP-Hard]\label{def:np-hard}
A problem \(Y\) with the property that for every problem \(X\) in NP, \(X \leq_{P} Y\).
\end{definition}
\begin{definition}[NP-Complete]\label{def:np-complete}
A problem \(Y\) that is in NP and is NP-hard.
\end{definition}

\hl{If one NP-complete problem is found to have a poly time algorithm, then all problems in NP have}

\begin{theorem}
  Suppose \(Y\) is an NP-complete problem.
  \(Y\) is solvable in polynomial time iff P = NP.
\end{theorem}
\begin{proof}
`\(\Leftarrow\)' (if):
If P = NP, then \(Y\) can be solved in poly-time since \(Y\) is in P.

`\(\Rightarrow\)' (only if):
Suppose \(Y\) can be solved in poly-time.
Let \(X\) be any problem in NP.
Since \(X \leq_{P} Y\) and \(Y\) can be solved in poly-time,
we can solve \(X\) in poly-time.
Thus, \(X\) in P, \(\text{NP} \subseteq \text{P}\).
We already know \(\text{P} \subseteq \text{NP}\).
Thus, P = NP.
\end{proof}

\subsubsection{NP-complete Problems}
\begin{fundamentalquestion*}
Do there exist ``natural'' NP-complete problems?
\end{fundamentalquestion*}

\begin{problem}[CIRCUIT-SAT]\label{prob:circuit-sat}
Given a combinational logic circuit built out of AND, OR, and NOT gates,
is there a way to set the input so that the output is 1?
\end{problem}

\begin{theorem}[Cook 1971, Levin 1973]\label{thm:circuit-sat-np-complete}
\hyperref[prob:circuit-sat]{CIRCUIT-SAT} is NP-complete.
\end{theorem}
\begin{proof}[sketch]
Membership in NP is immediate, so we focus on NP-hardness:
 \begin{itemize}
  \item Any algorithm that takes a fixed number of bits $n$ as input and
  produces a yes/no answer can be represented by a circuit.
  Further, if algorithm takes poly in $n$ time, then circuit has poly-size.
    \begin{itemize}
    \item
    sketchy part of proof: fixing the number of bits is important,
    and reflects basic distinction between algorithms and circuits
    \end{itemize}
  \item Consider some problem $X$ in NP. It has a poly-time certifier $C(s,t)$.
  To determine whether $s$ is in $X$, need to know if there exists a
  certificate $t$ of length $\le p(|s|)$ such that $C(s,t)=\texttt{yes}$.

  \item $C(s,t)$ is an algorithm on $|s|+p(|s|)$ bits (input $s$, certificate $t$).
  For fixed input length $|s|$, $C(\cdot,\cdot)$ is an algorithm of fixed input length,
  so we convert it into a poly-size circuit family such that $K_s$ is satisfiable iff
  $C(s,t)=\texttt{yes}$ for some $t$.
  \begin{itemize}
    \item First $|s|$ bits are hard-coded with $s$; remaining $p(|s|)$ bits represent $t$.
  \end{itemize}

  \item $X \le_p \text{\nameref{prob:circuit-sat}}$:
  Given an arbitrary instance $s$ of $X$, transform $s$ (as sketched above) into a poly-size circuit $K_s$
  (in poly time), which is satisfiable iff $s$ is yes instance of $X$.
  \qedhere
 \end{itemize}
\end{proof}

\begin{remark}
  Once we establish first ``natural'' NP-complete problem (\nameref{prob:sat}), others fall like dominoes.
\end{remark}

% \textcolor{AccentBlue}{Recipe to establish NP-completeness of a problem \(Y\)}:
\hl[2]{Recipe to establish NP-completeness of a problem \(Y\)}:
\begin{enumerate}
\item Show that \(Y\) is in NP
\item Choose an NP-complete problem \(X\)
\item Make a reduction: prove that \(X \leq_{P} Y\)
\end{enumerate}

\begin{claim}
  If $X$ is an NP-complete problem and $X \le_p Y$, then $Y$ is NP-hard.
  Further, if $Y$ is in NP, then $Y$ is NP-complete.
\end{claim}
\begin{proof}
Let $W$ be any problem in NP.
Then $W \le_p X \le_p Y$.
By transitivity, $W \le_p Y$.
Thus, $Y$ is NP-hard.
\end{proof}

\begin{theorem}\label{thm:3sat-np-complete}
  \nameref{prob:3-sat} is NP-complete.
\end{theorem}
\begin{proof}
  It suffices to show that \nameref{prob:circuit-sat} $\le_p$ \nameref{prob:3-sat}, since \nameref{prob:3-sat} is in NP.
  Let \(K\) be any circuit.
  Create a 3-CNF formula \(K'\) which is satisfiable iff \(K\) is satisfiable.
  There is a mechanical way to do it (skipped).
\end{proof}




All problems below are NP-complete and polynomially reduce to one another:
\[
\begin{tikzpicture}[
  scale=1.5,
  font=\footnotesize,
  >=Latex,
  group/.style={green!70!black},
  problem/.style={
    draw, fill=gray!10, line width=0.4pt,
    minimum height=5mm, minimum width=22mm,
    inner sep=0pt, align=center,
    font=\footnotesize
  },
  arr/.style={-Latex, line width=0.5pt, outer sep=1pt, inner sep=0pt},
  bluefat/.style={draw=blue!70!black, line width=1.2pt, line join=round},
  bluar/.style={-Latex, draw=blue!70!black, line width=1.2pt, line join=round}
]

% clickable label helper
\newcommand{\plink}[2]{\hyperref[prob:#1]{#2}}
% clickable reduction-label helper (no nested link)
\newcommand{\rref}[1]{\hyperref[#1]{\tiny\autoref*{#1}}}

% --- nodes ---
\node[problem, minimum width=34mm, fill=gray!30] at (0,4.5) (csat) {\plink{circuit-sat}{CIRCUIT-SAT}};
\node[problem, minimum width=20mm] at (0,3.5) (sat3) {\plink{3-sat}{3-SAT}};
\node[group] at (1.5, 4) {Constraint Satisfaction};

\node[problem] at (-3, 3) (cliq) {\plink{clique}{CLIQUE}};
\node[problem] at (-3, 2)(is) {\plink{independent-set}{INDEP. SET}};
\node[problem] at (-3, 1) (vc) {\plink{vertex-cover}{VERTEX COV}};

\node[problem, font=\tiny, minimum height=2.0mm, minimum width=10mm, inner sep=1pt] at (-3.5, 0.35) (domset) {\plink{dominating-set}{DOM. SET}};

\node[problem] at (-3, 0) (sc) {\plink{set-cover}{SET COV}};

\node[group] at (-3, -0.65) {Packing and Covering};

\node[problem] at (-1, 2) (dhc) {\plink{dir-ham-cycle}{DIR-HAM-CYC}};
\node[problem] at (-1, 1) (hc) {\plink{ham-cycle}{HAM-CYC}};
\node[problem] at (-1, 0) (tsp) {\plink{tsp}{TSP}};
\node[group] at (-1, -0.65) {Sequencing};

\node[problem] at (1, 2) (g3c) {\plink{graph-3-color}{GRAPH 3-COL}};
\node[problem] at (1, 1) (ccov) {\plink{clique-cover}{CLIQUE COV}};
\node[group] at (1, -0.65) {Partitioning};

\node[problem] at (3, 2) (ss) {\plink{subset-sum}{SUBSET-SUM}};
\node[problem] at (3, 1) (sch) {\plink{scheduling}{SCHEDULING}};
\node[group] at (3, -0.65) {Numerical};

% % --- black reduction arrows ---

\draw[arr] (csat) -- node[midway, left] {\rref{thm:3sat-np-complete}} (sat3);

\draw[arr] (sat3.south) -- node[pos=0.7, right, anchor=north west, inner sep=1pt, outer sep=0pt] {\rref{claim:3-sat-reduces-to-dir-ham-cycle}} (dhc.north);
\draw[arr] (sat3.south) -- (g3c.north);
% \draw[arr] (sat3.south) -- node[midway, right] {\rref{thm:3sat-to-3color}} (g3c.north);
\draw[arr] (sat3.south) -- (ss.north);
% \draw[arr] (sat3.south) -- node[midway, right] {\rref{thm:3sat-to-subset-sum}} (ss.north);

\def\myshift{0.1}

\draw[arr] (sat3.south) -- node[pos=0.7, right, anchor=north west, inner sep=1pt, outer sep=0pt] {\rref{claim:3sat_reduces_to_is}} ($(is.north)!\myshift!(is.north east)$);

\draw[arr, <->] ($(cliq.south)!\myshift!(cliq.south west)$) -- node[midway, left] {\rref{claim:clique_equiv_is_equiv_vc}} ($(is.north)!\myshift!(is.north west)$);

\draw[arr, <->] ($(is.south)!\myshift!(is.south west)$) -- node[midway, left] {\rref{claim:clique_equiv_is_equiv_vc}} ($(vc.north)!\myshift!(vc.north west)$);
\draw[arr] ($(is.south)!\myshift!(is.south east)$) -- node[midway, right] {\rref{claim:vc_equiv_is}} ($(vc.north)!\myshift!(vc.north east)$);


% orange equivalence box (behind CLIQUE / IS / VC) 
\begin{scope}[on background layer]
  \node[
    draw=orange!80!black,
    fill=orange!12,
    rounded corners=2pt,
    line width=0.8pt,
    inner sep=4.5pt,
    fit=(cliq)(is)(vc)
  ] (eqbox) {};
\end{scope}
\node[
  anchor=south west,
  text=orange!80!black,
  font=\scriptsize
] at (eqbox.north west) {equivalent (\rref{lem:clique_independent_vc})};


\draw[arr] (vc) -- node[midway, right] {\rref{claim:vc_reduces_to_sc}} (sc);
\draw[arr] (vc) -- node[pos=0.5, left, inner sep=1pt] {\rref{lem:vc_reduces_to_ds}} (domset);


% \draw[arr] (dhc) -- (hc);
\draw[arr] (dhc) -- node[midway, right] {\rref{claim:dir-ham-cycle-reduces-to-ham-cycle}} (hc);
% \draw[arr] (hc) -- (tsp);
\draw[arr] (hc)  -- node[midway, right] {\rref{claim:ham-cycle-reduces-to-tsp}} (tsp);

\draw[arr] (g3c) -- node[midway, right] {\rref{claim:3col_reduces_to_clique_cover}} (ccov);
\draw[arr] (ss) -- (sch);
% \draw[arr] (ss)  -- node[midway, right] {\rref{thm:subset-sum-to-scheduling}} (sch);


% % labeled arrow to INDEPENDENT SET
% \draw[arr] (sat3) -- node[midway, above, sloped, align=left]
%   {3-SAT reduces to\\INDEPENDENT SET} (is);

% --- blue arrows ---
\coordinate (BL) at (-4.2, 0);
\coordinate (BR) at (4.2, 0);
\coordinate (TR) at ($(csat) + (4.2, 0)$);
\coordinate (TL) at ($(csat) + (-4.2, 0)$);
\draw[bluar] (tsp) -- (BR) -- (TR) -- node[pos=0.45, above] {\tiny by definition of NP-completeness} (csat);
\draw[bluefat] (ccov.south) -- ($(ccov.south |- BL)$);
\draw[bluefat] (sch.south) -- ($(sch.south |- BL)$);
\draw[bluar] (sc) -- (BL) -- (TL) -- node[pos=0.45, above] {\tiny \autoref{thm:circuit-sat-np-complete}} (csat);
\draw[bluefat] (domset.west) -- ($(domset.west -| BL)$);

\end{tikzpicture}
\]


Basic categories of NP-complete problems and paradigmatic examples:
\begin{itemize}
  \item Packing problems: \hyperref[prob:set-packing]{SET PACKING}, \hyperref[prob:independent-set]{INDEPENDENT SET}
  \item Covering problems: \hyperref[prob:set-cover]{SET COVER}, \hyperref[prob:vertex-cover]{VERTEX COVER}
  \item Constraint satisfaction problems: \hyperref[prob:sat]{SAT}, \hyperref[prob:3-sat]{3-SAT}
  \item Sequencing problems: \hyperref[prob:ham-cycle]{HAM-CYCLE}, \hyperref[prob:tsp]{TSP}
  \item Partitioning problems: \hyperref[prob:3d-matching]{3D-MATCHING}, \hyperref[prob:graph-3-color]{3-COLOR}
  \item Numerical problems: \hyperref[prob:subset-sum]{SUBSET-SUM}, \hyperref[prob:knapsack]{KNAPSACK}
\end{itemize}

In practice, most problems are either known to be in P or known to be NP-complete.
Notable exeptions: Factoring, graph isomorphism, Nash equilibrium.


\begin{problem}[3Col]\label{prob:graph-3-color}
Given a graph \(G\), can each of its vertices be labeled with one of three different ``colors'' such that no two adjacent vertices have the same label?
\begin{itemize}
\item arises in various partitioning problems (adjacent object not in same group)
\item planar graphs can be colored with 4 colors (well known)
\item determining if 3 colors are possible is hard (even for planar graphs)
\item \nameref{prob:graph-3-color} is known to be NP-complete
\qedhere
\end{itemize}
\end{problem}

\begin{problem}[Clique Cover]\label{prob:clique-cover}
Given a graph \(G=(V,E)\) and an integer \(k\), can we partition the vertex set into \(k\) cliques?
\begin{itemize}
\item every vertex needs to be in exactly one clique!
\item \autoref{prob:clique-cover} arises in clustering.
\item \(\text{\hyperref[prob:graph-3-color]{3Col}} \leq_{P} \text{\hyperref[prob:clique-cover]{CCov}}\) 
\qedhere
\end{itemize}
\end{problem}

\begin{claim}\label{claim:3col_reduces_to_clique_cover}
  A graph \(G = (V,E)\) is 3-colorable iff its complement \(\overline{G} = (V, \overline{E})\) has a clique cover with \(k=3\) cliques.
\end{claim}

\begin{proof}
`\(\Rightarrow\)': 
Suppose \(G\) is 3-colorable.
Let \(V_1, V_2, V_3\) be the 3 color classes.
Every pair of distinct vertices in \(V_i\) are not adjacent in \(G\).
Thus, they are adjacent in \(\overline{G}\).
So, \(V_i\) forms a complete subgraph in \(\overline{G}\).
Thus, this is a clique cover of size 3 for \(\overline{G}\).

`\(\Leftarrow\)':
Suppose \(\overline{G}\) has a clique cover of size 3, denoted \(V_1, V_2, V_3\).
Give the vertices of \(V_i\) color \(i\), \(i=1,2,3\).
No two vertices in \(V_i\) are adjacent in \(G\).
Thus, this is a legal coloring for \(G\).
\end{proof}


\subsubsection{co-NP and the Asymmetry of NP}\label{sec:co-np}
\textcolor{AccentBlue}{Asymmetry of NP}:
We only have short proofs of \texttt{yes} instances.
We do not necessarily have proofs for \texttt{no} inst  ances.

\begin{example}[SAT vs TAUTOLOGY]\label{ex:sat_vs_tautology}
Can prove a \hyperref[def:CNF]{CNF} formula is satisfiable by giving a truth assignment (\autoref{fact:sat_in_np}).
How could we prove that a formula is \emph{not} satisfiable, i.e., a contradiction?
How can we prove that a formula is always true, i.e., a tautology?
% TAUTOLOGY not known to be in NP.
\end{example}

\begin{example}[HAM-CYCLE vs NO-HAM-CYCLE]\label{ex:ham_cycle_vs_no_ham_cycle}
Can prove that a graph is Hamiltonian by giving a Hamiltonian cycle (\autoref{fact:ham_cycle_in_np}).
How can we prove that a graph is \emph{not} Hamiltonian?
\end{example}


The complement of SAT, i.e. \(\overline{\text{SAT}}\), is the set of all Boolean \hyperref[def:CNF]{CNF} formulas that are not satisfiable,
i.e.,
the CONTRADICTION (formulas that are always false).

\(\text{TAUTOLOGY} \equiv_{P} \text{CONTRADICTION}\): \(\Phi\) is a tautology iff \(\neg \Phi\) is a contradiction.

Thus, \(\overline{\text{SAT}} \equiv \text{CONTRADICTION} \equiv_{P} \text{TAUTOLOGY}\).


\begin{definition}[Complement]\label{def:complement}
Given a decision problem \(X\), its \textcolor{AccentRed}{complement} \(\overline{X}\) is the same problem with the \texttt{yes} and \texttt{no} answers reversed,
i.e.,
\(s\) in \(X\) iff \(s\) not in \(\overline{X}\).
\end{definition}

\begin{definition}[co-NP]\label{def:co-NP}
The \nameref{def:complement}s of decision problems in \nameref{def:NP},
i.e.,
a problem \(X\) belongs to co-NP iff \(\overline{X}\) belongs to NP.
\end{definition}

\setlength{\tabcolsep}{4.5pt}
\begin{tabular}{>{\raggedright\arraybackslash}p{0.25\linewidth} |>{\raggedright\arraybackslash}p{0.25\linewidth}}
{\textcolor{AccentBlue}{Examples \nameref{def:NP}}} & {\textcolor{AccentBlue}{Examples \nameref{def:co-NP}}} \\
\hline
\footnotesize SAT         & \footnotesize TAUTOLOGY \\
\footnotesize HAM-CYCLE   & \footnotesize NO-HAM-CYCLE \\
\footnotesize COMPOSITES  & \footnotesize PRIMES \\
\end{tabular}







\begin{openquestion}
Does $\mathrm{NP}=\mathrm{co\text{-}NP}$?
\begin{itemize}
  \item 
  Do \texttt{yes} instances have succinct certificates iff \texttt{no} instances do?
  \item
  Consensus opinion: no.
  \item
  Easy to get a simple short certificate for most problems in NP but to get a no certificate is often not easy.
  \qedhere
\end{itemize}
\end{openquestion}


\[
\begin{tikzpicture}[scale=0.35]
  \useasboundingbox (-5,-2.5) rectangle (5,3);

  \draw[rotate around={-40:(-2,0)}] (-1.2,0) ellipse (3.5 and 2.25);
  \draw[rotate around={40:( 2,0)}] ( 1.2,0) ellipse (3.5 and 2.25);
  \node[font=\footnotesize] at (-3.15,0.55) {NP};
  \node[font=\footnotesize] at ( 3.15,0.55) {co-NP};

  \draw (0,-2) circle [radius=1.35];
  \node[font=\footnotesize] at (0,-2) {P};
  
  \draw (0,0.5) ellipse (6 and 4);
  \node[font=\footnotesize] at (0,3.5) {EXP};
\end{tikzpicture}
\]

\begin{claim}\label{claim:p_eq_coP}
\(\mathrm{P} = \mathrm{co\text{-}P}\). 
\end{claim}
\begin{proof}
% If \(X\) is in P, then we can decide \(\overline{X}\) in poly-time by reversing the answer.
If we have an algorithm \(A\) that decides \(X\) in poly-time,
we can design an algorithm \(\overline{A}\) that decides \(\overline{X}\) in poly-time:
Run \(A\) on input \(x\) and flip its answer.
\end{proof}


\begin{theorem}
 If \(\mathrm{NP} \neq \mathrm{co\text{-}NP}\), then \(\mathrm{P} \neq \mathrm{NP}\). 
%  (Contrapositive: If \(\mathrm{P} = \mathrm{NP}\), then \(\mathrm{NP} = \mathrm{co\text{-}NP}\).)
\end{theorem}
\begin{proof}[idea]
% \begin{itemize}
  % \item 
  P is closed under complement (\autoref{claim:p_eq_coP}).
  % \item 
  If P = NP, then NP is also closed under complement, i.e., NP = co-NP.
  % \item 
  Contrapositive yields the result.
%   \qedhere
% \end{itemize}  
\end{proof}


\textcolor{AccentBlue}{Good characterization}. \textcolor{gray!70!black}{[Edmonds 1965]}
\(\text{NP} \cap \text{co-NP}\).
\begin{itemize}
  \item If problem \(X\) is in both NP and co-NP, then:
  \begin{itemize}
    \item for \texttt{yes} instance, there is a succinct certificate
    \item for \texttt{no} instance, there is a succinct disqualifier (certificate)
  \end{itemize}
  \item Provides conceptual leverage for reasoning about a problem.
\end{itemize}

\begin{example}
Given a bipartite graph, is there a perfect matching?
\begin{itemize}
  \item if yes, can exhibit a perfect matching.
  \item if no, can exhibit a set of nodes \(S\) such that \(|N(S)| < |S|\) (\autoref{thm:hall-marriage}).
  \item Bipartite perfect matching is in \(\text{NP} \cap \text{co-NP}\).
  \qedhere
\end{itemize}
\end{example}

\begin{observation}
  \(\text{P} \subseteq \text{NP} \cap \text{co-NP}\).
\end{observation}

\begin{openquestion}
  Does \(\text{P} = \text{NP} \cap \text{co-NP}\)?
  \begin{itemize}
    \item mixed opinions.
    \item many examples where problem found to have non-trivial good characterization, but only years later discovered to be in P.
    \begin{itemize}
      \item linear programming \textcolor{gray!70!black}{[Khachiyan, 1979]}
      \item primality testing \textcolor{gray!70!black}{[Agrawal-Kayal-Saxena, 2002]}
      \qedhere
    \end{itemize}
  \end{itemize}
\end{openquestion}

\begin{fact}
  Factoring is in \(\text{NP} \cap \text{co-NP}\), but not known to be in P.%
  \footnote{if poly-time algorithm for factoring, can break RSA cryptosystem.}
\end{fact}




\subsubsection{Sequencing Problems}\label{sec:sequencing_problems}

\begin{example}%[Hamiltonian Cycle]
Vertices and faces of a dodecahedron have a Hamiltonian cycle.
A bipartite graph with odd number of vertices cannot have a Hamiltonian cycle.
\end{example}

\begin{problem}[DIR-HAM-CYCLE]\label{prob:dir-ham-cycle}
Given a digraph \(G=(V,E)\),
does \(G\) contain a directed \nameref{def:simple-cycle} that visits each vertex exactly once?
\end{problem}

\begin{claim}\label{claim:dir-ham-cycle-reduces-to-ham-cycle}
\(\text{\nameref{prob:dir-ham-cycle}} \leq_{P} \text{\hyperref[prob:ham-cycle]{HAM-CYCLE}}\).
\end{claim}

\begin{proof}[gadget construction]
Given a directed graph \(G=(V,E)\) with \(n=|V|\) nodes, construct an undirected graph \(G'\) with \(3n\) nodes as follows:
Substitute each node \(v\) in \(G\) by three nodes \(v_{\text{in}}, v_{\text{mid}}, v_{\text{out}}\) in \(G'\).
Each directed edge \((u,v)\) in \(E\) becomes an undirected edge \((u_{\text{out}}, v_{\text{in}})\) in \(G'\).
\[
\begin{tikzpicture}[font=\footnotesize, scale=1.0]
\node[draw, fill=gray!20, circle, minimum size=0.45cm, inner sep=0cm] (v) at (0.25,0) {\(v\)};

\draw[|->] (1.5, 0) -- node[above, midway, inner sep=1pt] {\(f\)} (2.0, 0);

\begin{scope}[shift={(4,0)}]
\node[draw, fill=red!80!black, circle, minimum size=0.45cm, inner sep=0cm] (vin) at (-1,0) {\textcolor{white}{\(v_{\raisebox{-0.0ex}{\scalebox{0.5}{in}}}\)}};
\node[draw, fill=blue!80!black, circle, minimum size=0.45cm, inner sep=0cm] (vmid) at (0,0) {\textcolor{white}{\(v_{\raisebox{-0.0ex}{\scalebox{0.5}{mid}}}\)}};
\node[draw, fill=green!60!black, circle, minimum size=0.45cm, inner sep=0cm] (vout) at (1,0) {\textcolor{white}{\(v_{\raisebox{-0.0ex}{\scalebox{0.5}{out}}}\)}};
\draw[] (vin) -- (vmid) -- (vout);
\end{scope}
\end{tikzpicture}
\]

Now \(G\) has a directed Hamiltonian cycle iff \(G'\) has a Hamiltonian cycle.
\begin{itemize}
\item
`\(\Rightarrow\)': 
Suppose \(G\) has a directed Hamiltonian cycle.
Then \(G'\) has an undirected Hamiltonian cycle (same order).
\item
`\(\Leftarrow\)':
Suppose \(G'\) has an undirected Hamiltonian cycle.
must visit nodes either in order \(v_{\text{in}} \to v_{\text{mid}} \to v_{\text{out}}\) or \(v_{\text{out}} \to v_{\text{mid}} \to v_{\text{in}}\).
The \(v_{\text{mid}}\) nodes in the cycle make up a directed Hamiltonian cycle in \(G\), or reverse of one.
\qedhere
\end{itemize}
\end{proof}

\begin{problem}[Traveling Salesman]\label{prob:tsp}
Given a set of \(n\) cities and a pairwise distance function \(d(u,v)\),
is there a tour of length \(\leq D\) that visits each city exactly once and returns to the starting city?
\end{problem}

\begin{claim}\label{claim:ham-cycle-reduces-to-tsp}
\(\text{\hyperref[prob:ham-cycle]{HAM-CYCLE}} \leq_{P} \text{\hyperref[prob:tsp]{TSP}}\).  
\end{claim}
\begin{proof}
Given an instance \(G=(V,E)\) of \nameref{prob:ham-cycle}, create \(n=|V|\) cities with distance function
\[
d(u,v) 
= 
\begin{cases}
1 & \text{if } (u,v) \in E\\
2 & \text{otherwise}
\end{cases}
\]

\textcolor{AccentBlue}{Remark}. 
\hyperref[prob:tsp]{TSP} instance in reduction satisfies \(\triangle\)-inequality:
\(
d(u,v) \leq 2 \leq d(u,w) + d(w,v)
\)
% for all cities \(u,v,w\).

\hyperref[prob:tsp]{TSP} instance has tour of length \(\leq n\) iff \(G\) is Hamiltonian.

\begin{itemize}
\item
`\(\Leftarrow\)': 
If \(G\) has a Hamiltonian cycle, then it defines a tour of length \(n\) in the \hyperref[prob:tsp]{TSP} instance.
\item
`\(\Rightarrow\)':
Suppose there is a tour \(T\) of length \(\leq n\) in the \hyperref[prob:tsp]{TSP} instance.
The length of \(T\) is the sum of \(n\) terms, each of which has cost at least \(1\) (\(1\) or \(2\)).
Since the length of \(T\) is at most \(n\), each term must have cost \(1\).
Thus, each pair of consecutive cities in \(T\) must be connected by an edge in \(G\).
Thus, the order of \(T\) gives a Hamiltonian cycle in \(G\).
\qedhere
\end{itemize}
\end{proof}





\begin{claim}\label{claim:3-sat-reduces-to-dir-ham-cycle}
\(\text{\hyperref[prob:3-sat]{3-SAT}} \leq_{P} \text{\hyperref[prob:dir-ham-cycle]{DIR-HAM-CYCLE}}\).
\end{claim}

% \begin{proof}
% Given an instance \(\Phi\) of \nameref{prob:3-sat}, we construct an instance of \nameref{prob:dir-ham-cycle} that has a Hamiltonian cycle iff \(\Phi\) is satisfiable.

% \textcolor{AccentBlue}{Construction}.
% Create a graph that has \(2^n\) Hamiltonian cycles which correspond ina natural way to \(2^n\) truth assignments.

% Given \hyperref[prob:3-sat]{3-SAT} instance \(\Phi\) with \(n\) variables and \(k\) clauses,
% \begin{itemize}
%   \item 
%   construct \(G\): 
%   1 bi-directional path (\(3k+3\) nodes) for each variable (\(n\) paths).
%   For left and right endpoints of variable \(x_i\) path, connect them to the left and right endpoints of variable \(x_{i+1}\) path, for \(i=1, \ldots, n-1\) (left to left, left to right, right to left, right to right).
%   Add nodes \(s,t\). 
%   Add directed edges from \(s\) to left and right endpoints of first variable (\(x_1\)) path.
%   Add directed edges from left and right endpoints of last variable (\(x_n\)) path to \(t\).
%   Add a directed edge from \(t\) to \(s\).
%   (so far, \(G\) has \(2^n\) Hamiltonian cycles, one for each truth assignment).
%   \item
%   Traverse path \(i\) from left to right \(\Leftrightarrow\) set variable \(x_i = 1\).
%   \item
%   For each clause \(C_j\), add a node \(C_j\) and 6 directed edges
% \end{itemize}

% \medskip

% \textcolor{AccentBlue}{Claim}.
% \(\Phi\) is satisfiable iff \(G\) has a Hamiltonian cycle.

% `\(\Rightarrow\)':
% \begin{itemize}
% \item
% Suppose \nameref{prob:3-sat} instance has satisfying assignment \(x^*\).
% \item
% Then, define Hamiltonian cycle in \(G\) as follows:
% \begin{itemize}
%   \item if \(x^*_i = 1\), traverse row \(i\) from left to right
%   \item if \(x^*_i = 0\), traverse row \(i\) from right to left
%   \item for each clause \(C_j\), we splice \(C_j\) into the tour exactly once: for the literal that sets \(C_j\) to true.
%   There will be at least one row \(i\) in which we are going in ``correct'' direction to splice node \(C_j\) into tour
%   \item (Note we can splice node \(C_j\) into tour only if we have the right direction)
% \end{itemize}
% \end{itemize}

% `\(\Leftarrow\)':
% \begin{itemize}
% \item
% Suppose \(G\) has a Hamiltonian cycle \(H\).
% \item
% If \(H\) enters clause node \(C_j\), it must depart on mate edge.
% \begin{itemize}
%   \item thus, nodes immediately before and after \(C_j\) are connected by an edge \(e\) in \(G\)
%   \item removing \(C_j\) from cycle \(H\), and replacing it with edge \(e\) yields Hamiltonian cycle on \(G \setminus \{ C_j \}\).
% \end{itemize}
% \item
% Continuing in this way, we are left with Hamiltonian cycle \(H'\) in \(G \setminus \{ C_1, \ldots, C_k \}\).
% \item
% Set \(x^*_i = 1\) iff \(H'\) traverses row \(i\) left to right.
% \item
% Since \(H\) visits each clause node \(C_j\), at least one of the paths is traversed in ``correct'' direction, and each clause is satisfied.
% \qedhere
% \end{itemize}
% \end{proof}


\begin{proof}
Let $\Phi$ be a 3-\hyperref[def:CNF]{CNF} formula with \(n\) variables $x_1,\dots,x_n$ and \(k\) clauses $C_1,\dots,C_k$, i.e.,
\[
\Phi 
= \bigwedge_{j=1}^k C_j
= \bigwedge_{j=1}^k (L_{j,1} \vee L_{j,2} \vee L_{j,3}),
\]
where each literal \(L_{j,m}\) is either \(x_i\) or \(\neg x_i\) for some \(i \in \{1,\dots,n\}\).

We construct a directed graph $G$ as follows.

% \paragraph{Variable paths.}
Let $b := 3k+3$.
For each variable $x_i$ construct a directed path
\[
P_i = (v_{i,1}, v_{i,2}, \dots, v_{i,b-1}, v_{i,b})
\]
where for every $q \in \{1,\dots,b-1\}$ we add both directed edges
$(v_{i,q},v_{i,q+1})$ and $(v_{i,q+1},v_{i,q})$.
Let $\ell_i := v_{i,1}$ and $r_i := v_{i,b}$.

For $i=1,\dots,n-1$, add all four directed edges
$\ell_i \to \ell_{i+1}$, $\ell_i \to r_{i+1}$, $r_i \to \ell_{i+1}$,
$r_i \to r_{i+1}$.
Add vertices $s,t$ and directed edges
$s \to \ell_1$, $s \to r_1$, $\ell_n \to t$, $r_n \to t$, and $t \to s$.
Let $G_0$ denote the resulting graph:
\[
\begin{tikzpicture}[
  >={Latex[length=1.35mm, width=0.9mm]},
  v/.style={circle,draw,fill=gray!20,minimum size=3.8mm,inner sep=0pt},
  ed/.style={->,thin},
  lab/.style={font=\footnotesize, outer sep=0pt, inner sep=3pt},
  font=\footnotesize,
  scale=0.8
]

% ---- geometry ----
\def\dx{1.4}
\def\dy{2}

% x-positions (schematic): left block, dots, right block
\coordinate (X0) at (0,0);
\coordinate (X1) at (\dx,0);
\coordinate (X2) at (2*\dx,0);
\coordinate (X3) at (3*\dx,0);
\coordinate (Xd) at (4*\dx,0);   % where \cdots goes
\coordinate (Xr3) at (5*\dx,0);
\coordinate (Xr2) at (6*\dx,0);
\coordinate (Xr1) at (7*\dx,0);
\coordinate (Xr0) at (8*\dx,0);

% y-positions: top row, second row, dots, last row
\coordinate (Y1) at (0,0);
\coordinate (Y2) at (0,-\dy);
\coordinate (Yd) at (0,-2.1*\dy);  % where \vdots goes
\coordinate (Yn) at (0,-3.2*\dy);

% ---- helper macro to draw one schematic row i at y ----
% creates endpoints Li,Ri and a few inner nodes + \cdots
\newcommand{\Row}[2]{% #1 = row name (1,2,n), #2 = y
  % endpoints
  \node[v] (L#1) at ($(X0)+(0,#2)$) {$\ell_{#1}$};
  \node[v] (R#1) at ($(Xr0)+(0,#2)$) {$r_{#1}$};

  % left few nodes
  \node[v] (A#1) at ($(X1)+(0,#2)$) {};
  \node[v] (B#1) at ($(X2)+(0,#2)$) {};
  \node[v] (C#1) at ($(X3)+(0,#2)$) {};

  % right few nodes
  \node[v] (D#1) at ($(Xr3)+(0,#2)$) {};
  \node[v] (E#1) at ($(Xr2)+(0,#2)$) {};
  \node[v] (F#1) at ($(Xr1)+(0,#2)$) {};

  % horizontal dots
  \node at ($(Xd)+(0,#2)$) {$\cdots$};

  % bidirectional edges (left block)
  \draw[<->, thin] (L#1) -- (A#1);
  \draw[<->, thin] (A#1) -- (B#1);
  \draw[<->, thin] (B#1) -- (C#1);

  % connect to dots region and out again (schematic connectors)
  \draw[<-, thin] (C#1) -- ($(Xd)+(-0.55, #2)$);
  \draw[->, thin] ($(Xd)+(0.55, #2)$) -- (D#1);

  % bidirectional edges (right block)
  \draw[<->, thin] (D#1) -- (E#1);
  \draw[<->, thin] (E#1) -- (F#1);
  \draw[<->, thin] (F#1) -- (R#1);
}


% draw rows x1, x2, xn 
\Row{1}{0}
\Row{2}{-\dy}
\Row{n}{-3*\dy}

% row labels
\node[lab,left=15pt] at (L1.west) {$P_1$:};
\node[lab,left=15pt] at (L2.west) {$P_2$:};
\node[lab,left=15pt] at (Ln.west) {$P_n$:};

% G_0 label
\node[draw, rectangle, rounded corners, align=left, text width=3.3cm, inner sep=2pt] at (7.8*\dx, 1.1*\dy) {\(G_0\): Hamiltonian cycles correspond to the \(2^n\) possible truth assignments};

% s,t 
\node[v] (s) at ($(Xd)+(0,1.25*\dy)$) {$s$};
\node[v] (t) at ($(Xd)+(0,-4.25*\dy)$) {$t$};


% show all 4 edges between x1 and x2
\draw[->, thin] (L1) -- (L2);
\draw[->, thin] (L1.south east) -- (R2.north west);
\draw[->, thin] (R1.south west) -- (L2.north east);
\draw[->, thin] (R1) -- (R2);


\draw[-, thin] (L2.south) -- ($(L2.south)!0.34!($(L2.south) + (0,-\dy)$)$);
\draw[-, thin] (L2.south east) -- ($(L2.south east)!0.34!($(R2) + (0,-\dy)$)$);
\draw[-, thin] (R2.south west) -- ($(R2.south west)!0.34!($(L2) + (0,-\dy)$)$);
\draw[-, thin] (R2.south) -- ($(R2.south)!0.34!($(R2.south) + (0,-\dy)$)$);

% vertical dots between rows
\node at ($(Xr0)+(0,-1.9*\dy)$) {$\vdots$};
\node at ($(X0)+(0,-1.9*\dy)$) {$\vdots$};
\node at ($(Xd)+(0,-1.9*\dy)$) {$\vdots$};

\draw[->, thin] ($(Ln.north)!0.34!($(Ln.north) + (0,\dy)$)$) -- (Ln.north);
\draw[->, thin] ($(Rn.north west)!0.34!($(Ln) + (0,\dy)$)$) -- (Rn.north west);
\draw[->, thin] ($(Ln.north east)!0.34!($(Rn) + (0,\dy)$)$) -- (Ln.north east);
\draw[->, thin] ($(Rn.north)!0.34!($(Rn.north) + (0,\dy)$)$) -- (Rn.north);


% ---- edges from s to first row endpoints ----
\draw[->, thin] (s) -- (L1);
\draw[->, thin] (s) -- (R1);

% ---- edges from last row endpoints to t ----
\draw[->, thin] (Ln) -- (t);
\draw[->, thin] (Rn) -- (t);

% ---- edge t -> s ----
\coordinate (svirtual) at ($(t)+(-4,0.2)$);
\draw[->, thin, bend left=30, dashed] (t) to (svirtual) node[above] {to $s$ ...};

% ---- length marker (schematic) ----
\draw[|-|] ($(Ln)+(-0.1,-3.5)$) -- node[below] {$3k+3$} ($(Rn)+(0.1,-3.5)$);

\end{tikzpicture}
\]
$G_0$ has exactly $2^n$ different Hamiltonian cycles, corresponding to the $n$ independent choices of direction for traversing each path $P_i$.

This naturally models the $n$ independent choices of how to set each variable \(x_1, \dots, x_n\) to \texttt{true} or \texttt{false},
and hence the $2^n$ truth assignments.
Thus, we identify each Hamiltonian cycle $H_0$ in $G_0$ uniquely with a truth assignment $a$ as follows:
If $H_0$ traverses path $P_i$ from left to right, set $a(x_i) := 1$;
if $H_0$ traverses path $P_i$ from right to left, set $a(x_i) := 0$.


\begin{lemma*}\label{lem:choice_of_directions_induces_ham_cycle}
Every Hamiltonian cycle in $G_0$ traverses each path $P_i$ entirely
from $\ell_i$ to $r_i$ or entirely from $r_i$ to $\ell_i$.
Conversely, every choice of directions for $P_1,\dots,P_n$
induces a unique Hamiltonian cycle in $G_0$.
\end{lemma*}
\begin{proof}
Since the only edges incident to internal vertices of $P_i$ are the two path edges
to their predecessor/successor, a Hamiltonian cycle cannot enter or leave $P_i$
at an internal vertex.
Hence it must enter at one endpoint and leave at the other, traversing $P_i$
completely in one direction.
The endpoint connections guarantee that all $2^n$ direction choices are feasible and independent.
\end{proof}



Next we add nodes to model the constraints imposed by the clauses of \(\Phi\).

% \paragraph{Clause nodes.}
For each clause $C_j$, add a vertex $c_j$.
We reserve positions $3j$ and $3j+1$ on each path $P_i$ for clause $C_j$.
For each literal $L \in C_j$:
\begin{itemize}
\item if $L = x_i$, add edges $(v_{i,3j},c_j)$ and $(c_j,v_{i,3j+1})$
\item if $L = \neg x_i$, add edges $(v_{i,3j+1},c_j)$ and $(c_j,v_{i,3j})$
\end{itemize}

This completes the construction of $G$.

\begin{lemma*}[Splicing]\label{lem:splicing_clause_into_ham_cycle}
Let $H$ be a Hamiltonian cycle containing a directed edge $p \to q$.
If $G$ contains edges $p \to c_j$ and $c_j \to q$, then replacing
$p \to q$ in $H$ by $p \to c_j \to q$ yields another Hamiltonian cycle.
\end{lemma*}
\begin{proof}
The replacement preserves indegree and outdegree $1$ at all vertices
and introduces no repetitions.
\end{proof}

Now \(\Phi\) is satisfiable iff \(G\) has a Hamiltonian cycle.
\begin{itemize}
\item
`\(\Rightarrow\)':
Suppose $\Phi$ is satisfiable, and let $a$ be a satisfying assignment.
By the \hyperref[lem:choice_of_directions_induces_ham_cycle]{first Lemma}, $a$ induces a Hamiltonian cycle $H_0$ in $G_0$.

Fix a clause $C_j$.
Choose a literal $L \in C_j$ satisfied by $a$ 
% (it must exist, since $a$ satisfies $\Phi$ iff it satisfies every $C_j$ and $C_j$ is satisfied only if at least one literal in $C_j$ is true).
(at least one must exist by assumption that $\Phi$ is satisfiable).
\begin{itemize}
\item 
If $L=x_i$, 
then $a(x_i)=1$, so $H_0$ traverses $P_i$ left-to-right and in particular uses the edge $v_{i,3j} \to v_{i,3j+1}$.
By construction, $G$ contains edges $v_{i,3j} \to c_j$ and $c_j \to v_{i,3j+1}$, so by
the \hyperref[lem:splicing_clause_into_ham_cycle]{second Lemma} 
we can splice $c_j$ into the tour between $v_{i,3j}$ and $v_{i,3j+1}$.
\item 
If $L=\neg x_i$, 
then $a(x_i)=0$, so $H_0$ traverses $P_i$ right-to-left and in particular uses the edge $v_{i,3j+1} \to v_{i,3j}$.
By construction, $G$ contains edges $v_{i,3j+1} \to c_j$ and $c_j \to v_{i,3j}$, so by
the \hyperref[lem:splicing_clause_into_ham_cycle]{second Lemma} 
we can splice $c_j$ into the tour between $v_{i,3j+1}$ and $v_{i,3j}$.
\end{itemize}
Do this once for each clause $C_j$.
The resulting directed cycle visits every vertex exactly once, hence is a Hamiltonian cycle of $G$.

\item
`\(\Leftarrow\)':
Suppose $G$ has a Hamiltonian cycle $H$.
Fix $j$.
Since $c_j$ has indegree and outdegree $1$ in $H$, the cycle contains a subpath
$p \to c_j \to q$.

Moreover, if $H$ enters $c_j$ from $v_{i,3j}$ then it must leave to $v_{i,3j+1}$:
otherwise $v_{i,3j+1}$ would remain unvisited and would then have only one unvisited 
neighbor $v_{i,3j+2}$ available, so the tour will not be able to visit this node while remaining Hamiltonian.
Symmetrically, if $H$ enters from $v_{i,3j+1}$ then it must leave to $v_{i,3j}$.

By construction of $G$, necessarily $(p,q)=(v_{i,3j},v_{i,3j+1})$ for some $i$ (if $c_j$ encodes a
positive occurrence $x_i$) or $(p,q)=(v_{i,3j+1},v_{i,3j})$ (if it encodes a negative occurrence $\neg x_i$).
In either case, the shortcut edge $p \to q$ is a path edge of $P_i$ and hence belongs to $G_0$.
Contract $p \to c_j \to q$ to $p \to q$.
Repeating this for all $j=1,\dots,k$ yields a Hamiltonian cycle $H_0$ of $G_0$.

By the \hyperref[lem:choice_of_directions_induces_ham_cycle]{first Lemma}, $H_0$ induces an assignment $a$ by setting
$a(x_i)=1$ iff $H_0$ traverses $P_i$ left-to-right.
Fix a clause $C_j$.
Since $H$ visits $c_j$, it must splice through $c_j$ using some path $P_i$ in the direction that matches
the corresponding literal in $C_j$; hence that literal is true under $a$.
Therefore every clause is satisfied by $a$, so $\Phi$ is satisfiable.
\end{itemize}

Complexity:
The graph has $O(n \cdot k)$ vertices and edges and is constructible in polynomial time.
\end{proof}












% \clearpage



\subsection{Approximation Algorithms}\label{sec:approximation_algorithms}

We saw all these problems that are NP-complete and reduce one to another.
So they are the same hard.
They appear extremely often in practice.

So how do we cope with NP-completeness?
\begin{itemize}
  \item \textbf{brute-force search}: viable only for small input sizes (e.g. \(n \leq 20\)).
  \item \textbf{heuristics}: strategy for producing a valid solution, but no guarantee on how close is to optimal
  \item \textbf{general search algorithms}: powerful techniques for solving general combinatorial optimization problems:
  \begin{itemize}
  \item branch-and-bound: breaking problem down into smaller subproblem and using a bounding function % https://en.wikipedia.org/wiki/Branch_and_bound
  \item metropolis-hastings: Markov chain Monte Carlo method for obtaining a sequence of random samples from a probability distribution from which direct sampling is difficult % https://en.wikipedia.org/wiki/Metropolis–Hastings_algorithm
  \item simulated annealing: models the physical process of heating a material and then slowly lowering the temperature to decrease defects, thus minimizing the system energy % https://www.mathworks.com/help/gads/what-is-simulated-annealing.html
  \item genetic algorithms: metaheuristic used to generate high-quality solutions to optimization and search problems via biologically inspired operators such as selection, crossover, and mutation % https://en.wikipedia.org/wiki/Genetic_algorithm
  \end{itemize}
  performance varies considerably from problem to problem and instance to instance
  \item \textbf{approximation algorithms}: algorithm that runs in polynomial time and produces a solution that is guaranteed to be within some factor of the optimal solution
\end{itemize}

\medskip

\textcolor{AccentBlue}{Performance Bounds}:
\begin{itemize}
  \item most NP-complete problems states as decision problems (because of theoretical resons involving complexity) 
  \item they are natural optimization problems, e.g. 
  \begin{itemize}
  \item \nameref{prob:vertex-cover}: find vertex cover of minimum size
  \item \nameref{prob:clique}: find the clique of maximum size
  \end{itemize}
  \item an approximation algorithm returns a legitimate answer, but not necessarily one of optimal size
\end{itemize}

\medskip

\textcolor{AccentBlue}{Measuring how good an approximation algorithm is}:
\begin{itemize}
  \item define the \emph{performance ratio} of an approximation
  \item given an instance \(I\) of our problem,
  \begin{itemize}
    \item let \(C(I)\) be the cost of solution produced by the approximation algorithm
    \item let \(C^*(I)\) be the cost of optimal solution
  \end{itemize}
  \item for a \emph{minimization problem}: \(C(I) / C^*(I) \geq 1\)
  \item for a \emph{maximization problem}: \(C^*(I) / C(I) \geq 1\)
\end{itemize}


\begin{definition}[Performance Ratio Bound]\label{def:performance_ratio_bound}
We say that an approximation algorithm achieves performance ratio bound \(\rho(n)\) if
\begin{equation}\label{eq:performance_ratio}
\max_I \left(\frac{C(I)}{C^*(I)}, \frac{C^*(I)}{C(I)}\right) \leq \rho(n)
\end{equation}
for all instances \(I\) of size \(|I| = n\) of the problem.
Note that \(\rho(n) \geq 1\) with equality iff the approximate solution is the true optimal solution.
\end{definition}
The \nameref{def:performance_ratio_bound} is an upper bound for the worst-case performance.


\medskip 

\hl{NP-complete problems are equivalent with respect to complexity, but their approximability varies considerably}.
\begin{itemize}
  \item some problems are \emph{inapproximable}: no polynomial-time algorithm achieves a ratio bound \(< \infty\) unless P=NP
  \item some can be approximated, but the ratio bound is a function of \(n\) (e.g. \nameref{prob:set-cover} can be approximated within a factor \(O(\log n)\))
  \item some can be approximated and the ratio bound is constant (e.g. \nameref{prob:vertex-cover} can be approximated within a factor \(2\))
  \item some can be approximated arbitrarily well (e.g. \hyperref[prob:knapsack]{Knapsack})
  \begin{itemize}
    \item user provides a parameter \(\epsilon > 0\) and the algorithm achieves a ratio bound \(1 + \epsilon\)
    \item as \(\epsilon\) approaches \(0\), the running time gets worse
    \item if it runs in polynomial time for any fixed \(\epsilon\), it is called a \emph{polynomial-time approximation scheme} (PTAS)
  \end{itemize}
\end{itemize}









\subsubsection{Vertex Cover}\label{sec:vertex_cover_approximation}
\nameref{prob:vertex-cover} has an approximation algorithm with ratio bound \(\rho(n) = 2\).

\begin{algorithm}[h]
\caption{2-for-1 heuristic for \nameref{prob:vertex-cover}}
\label{alg:vc-2-approx}
\begin{algorithmic}[1]
\Function{VC-2-Approx}{$G=(V,E)$}
  \State $T \gets \emptyset$
  \While{$E \neq \emptyset$}
    \State Select an arbitrary edge $(u,v)$ from $E$ \label{line:vc-2-approx-select-edge}
    \State $T \gets T \cup \{u,v\}$ \Comment{add both endpoints to cover}
    \State Remove from $E$ all edges incident to either $u$ or $v$
  \EndWhile
  \State \Return $T$
\EndFunction
\end{algorithmic}
\end{algorithm}

\begin{claim}\label{claim:vc-2-approx-ratio}
\autoref{alg:vc-2-approx} achieves a performance ratio bound of 
\(
\rho(n) = 2
\).
\end{claim}
\begin{proof}
Consider the set \(T\) output by \autoref{alg:vc-2-approx}.
Let \(A\) be the set of edges selected in line \ref{line:vc-2-approx-select-edge}.
\(|T| = 2|A|\) because both endpoints of each edge of \(A\) are added to \(T\).
But also the optimal solution \(T^*\) must cover the edges in \(A\), which are non-adjacent.
Thus, \(|T^*| \geq |A|\).
Therefore,
\(
|T| 
% = 2|A| 
\leq 2|T^*| 
% \Rightarrow
% \frac{|T|}{|T^*|} \leq 2
\).
\end{proof}

\begin{algorithm}[h]
\caption{Greedy heuristic for \nameref{prob:vertex-cover}}
\label{alg:vc-greedy-approx}
\begin{algorithmic}[1]
\Function{VC-Greedy-Approx}{$G=(V,E)$}
  \State $T \gets \emptyset$
  \While{$E \neq \emptyset$}
    \State Select the vertex \(u\) of maximum degree in \(G\)
    \State $T \gets T \cup \{u\}$ \Comment{add vertex to cover}
    \State Remove from $E$ all edges incident to $u$
  \EndWhile
  \State \Return $T$
\EndFunction
\end{algorithmic}
\end{algorithm}

The greedy heuristic (\autoref{alg:vc-greedy-approx}) does not achieve a constant performance bound, but merely one of \(\Theta(\log n)\).
Nevertheless, experimental studies show that it often works quite well in practice, and for ``typical'' graphs, it will perform better than the 2-for-1 heuristic (\autoref{alg:vc-2-approx}).





\subsubsection{Independent Set}\label{sec:independent_set_approximation}
Unfortunately, approximation factors are not preserved by our transformations.

\begin{claim}\label{claim:is-2-for-1-heuristic-factor}
If we apply the \autoref{alg:vc-2-approx} to find an \nameref{prob:independent-set} of maximum size, the performance ratio is
\[
\rho(n, k) = \frac{n - k}{n - 2k}
\]
which can be arbitrarily large (e.g. for \(k = (n-1)/2\), \(\rho(n, k) = (n + 1)/2\)).
\end{claim}
\begin{proof}
We know from \autoref{lem:clique_independent_vc} that \(V'\) is a \nameref{prob:vertex-cover} for \(G\) iff \(V \setminus V'\) is a \nameref{prob:independent-set} for \(G\).
Let \(T^*\) be an minimum (i.e. optimal) \nameref{prob:vertex-cover} of size \(|T^*| = k\).
Then \(S^* = V \setminus T^*\) is a maximum (i.e. optimal) \nameref{prob:independent-set} of size \(|S^*| = n - k\).
\autoref{alg:vc-2-approx} returns a \nameref{prob:vertex-cover} \(T\) with size \(|T| \leq 2k\) (\autoref{claim:vc-2-approx-ratio}).
Thus \(S = V \setminus T\) is an \nameref{prob:independent-set} of size \(|S| \geq n - 2k\).
The performance ratio is therefore
\(\frac{|S^*|}{|S|} = \frac{n - k}{|S|} \leq \frac{n - k}{n - 2k}\).
% \[
% \begin{verticalhack}
%   \frac{|S^*|}{|S|}
%   =
%   \frac{n - k}{|S|}
%   \leq
%   \frac{n - k}{n - 2k}
% \end{verticalhack}
% \qedhere
% \]
\end{proof}

\subsubsection{Traveling Salesman Problem}\label{sec:tsp_approximation}

\autoref{prob:tsp} was formulated in terms of cities and distances, now we reformulate it in terms of graphs:
\begin{problem}[TSP]\label{prob:graph-tsp}
Given a \emph{complete undirected graph} \(G = (V,E)\) with non-negative edge weights \(w(u,v)\), find a cycle that visits all vertices and has minimum cost.
\end{problem}

Often edge weights satisfy the \(\triangle\)-inequality:
\(
w(u,v) \leq w(u,x) + w(x,v)
\)
\begin{itemize}
\item euclidian distance
\item shortest path in a graph
\end{itemize}

\medskip 

When cost function satisfies \(\triangle\)-inequality, there is an approximation algorithm for \nameref{prob:graph-tsp} with a ratio bound of \(\rho(n) = 2\).

\begin{observation}\label{obs:tsp-geq-mst}
A \nameref{prob:graph-tsp} with one edge removed is just a spanning tree (not necessarily minimum).
%
Thus, cost min \nameref{prob:graph-tsp} tour \(\geq\) cost \nameref{def:mst}.
\end{observation}



\textcolor{AccentBlue}{Idea}:
\begin{itemize}
  \item compute a \nameref{def:mst} \(T\) of \(G\) efficiently, e.g. using \nameref{alg:kruskal}
  \item find some way to convert the \nameref{def:mst} \(T\) into a \nameref{prob:graph-tsp} tour \(H\) while increasing its cost by a constant factor
\end{itemize}

\begin{remark}
  \begin{itemize}
    \item given a free tree, there is a tour of the tree called a \emph{twice around tour} that traverses the edges of the tree twice, once in each direction
    \item this path is not simple, because it revisits vertices, but we can make it simple by \emph{short-cutting}, i.e. skipping over already visited vertices
    \item the order in which vertices are visited using short-cuts is a \emph{preorder traversal} (see \href{https://fabianbosshard.github.io/usi-informatics-course-summaries/usi-algorithms-and-data-structures-summary.pdf}{Algorithms \& Data Structures}, chapter `Binary Search Trees') of the \nameref{def:mst} \(T\)
    \item the triangle inequality assures that the path length will not increase when we take short-cuts
    \item in fact, any subsequence of the twice-around tour which visits each vertex exactly once will suffice (not necessarily in preorder)
    \qedhere
  \end{itemize}
\end{remark}

\begin{algorithm}[h]
\caption{Approximation for \nameref{prob:graph-tsp}}
\label{alg:tsp-approx}
\begin{algorithmic}[1]
\Function{TSP-Approx}{$G=(V,E)$}
  \State \(T \gets \text{MST}(G)\) \label{line:mst-calc}\Comment{e.g. using \nameref{alg:kruskal}}
  \State \(r \gets\) arbitrary root of \(T\)
  \State \(H \gets\) list of vertices visited by a preorder walk of \(T\) starting at \(r\)
  \State \Return \(H\)
\EndFunction
\end{algorithmic}
\end{algorithm}

\begin{claim}\label{claim:tsp-approx-ratio}
\autoref{alg:tsp-approx} achieves a performance ratio bound of 
\(
\rho(n) = 2
\).
\end{claim}
\begin{proof}
Let \(H^*\) be an optimal \nameref{prob:graph-tsp} tour and \(H\) be the tour returned by \autoref{alg:tsp-approx}.
Let \(T\) be the \nameref{def:mst} computed in line \ref{line:mst-calc}.
By \autoref{obs:tsp-geq-mst}, \(W(T) \leq W(H^*)\).
The twice around tour of \(T\) has cost \(2 \cdot W(T)\).
By the triangle inequality, short-cuts do not increase the cost of a tour, i.e. \(W(H) \leq 2 \cdot W(T)\).
Thus, \(W(H) \leq 2 \cdot W(T) \leq 2 \cdot W(H^*) 
\Rightarrow
\frac{W(H)}{W(H^*)} \leq 2\).
\end{proof}



\subsubsection{Set Cover}\label{sec:set_cover_approximation}

\nameref{prob:set-cover} can be generalized where each set \(S_i\) has a positive cost \(w_i\)
and we want to compute the set cover of minimum total weight.
\autoref{alg:set-cover-greedy-approx} can be generalized to this too.

The \nameref{alg:vc-2-approx} relies on the fact that each element (each edge) appears in exactly two sets (one for each endpoint).
This is not true for \nameref{prob:set-cover}\textcolor{red}{!}

\hl[2]{widely believed, there is no constant factor approximation for \nameref{prob:set-cover}}.



\begin{algorithm}[h]
\caption{Greedy heuristic for \nameref{prob:set-cover}}
\label{alg:set-cover-greedy-approx}
\begin{algorithmic}[1]
\Function{SC-Greedy-Approx}{$U, F = \{S_1, \ldots, S_n\}$}
  \State \(X \gets U\) \Comment{uncovered elements}
  \State \(C \gets \emptyset\) \Comment{sets in the cover}
  \While{\(X \neq \emptyset\)}
    \State Select \(S_i \in F\) that covers the most elements of \(X\)
    \State \(C \gets C \cup \{S_i\}\)
    \State \(X \gets X \setminus S_i\)
  \EndWhile
  \State \Return \(C\) \Comment{\autoref{thm:set-cover-greedy-approx-ratio}: \(|C| \le (1 + \ln |U|) \cdot |C^*|\)}
\EndFunction
\end{algorithmic}
\end{algorithm}


\autoref{alg:set-cover-greedy-approx} is a greedy heuristic that at each stage selects the set that covers the greatest number of uncovered elements.


The \nameref{alg:set-cover-greedy-approx} can be ``fooled'' into picking the wrong set over and over again, as the following example shows.
\begin{example}\label{ex:greedy-can-be-fooled}
Consider the following instance of \nameref{prob:set-cover}:
\[
\begin{tikzpicture}[font=\footnotesize,scale=0.8]

% --- parameters ---
\def\dx{0.60}      % horizontal spacing of elements
\def\r{0.085}      % dot radius

\def\smallsep{0.25}
\def\largesep{0.38}

\pgfmathsetmacro{\rowsep}{2*\largesep+\dx-2*\smallsep}  % vertical spacing of rows

% coordinates
\coordinate (T0) at (0,0);                 % first element in top row
\coordinate (B0) at (0,-\rowsep);          % first element in bottom row
\coordinate (T15) at (15*\dx,0);
\coordinate (B15) at (15*\dx,-\rowsep);

% --- dots (16 per row) ---
\foreach \i in {0,...,15}{
  \fill ($(T0)+(\i*\dx,0)$) circle (\r);
  \fill ($(B0)+(\i*\dx,0)$) circle (\r);
}

% ---  optimal sets s7 and s8 ---
\draw[densely dotted, thick] ($(T0)+(-\largesep,\smallsep)$) rectangle ($(T15)+(\largesep,-\smallsep)$);
\node[anchor=west] at ($(T15)+(\largesep,0)$) {$S_7$};
\draw[densely dotted, thick] ($(B0)+(-\largesep,\smallsep)$) rectangle ($(B15)+(\largesep,-\smallsep)$);
\node[anchor=west] at ($(B15)+(\largesep,0)$) {$S_8$};

% ---  greedy sets s1..s6 ---
% convention: s1..s4 span BOTH rows; s5 only top-left; s6 only bottom-left
% element blocks: 1 | 1 | 2 | 4 | 8  (total 16)
\draw[gray, thick] ($(T0)+(-\smallsep,\largesep)$) rectangle ($(T0)+(\smallsep,-\largesep)$); % s5 (top only, element 0)
\node[above] at ($(T0)+(0,\largesep)$) {$S_5$};
\draw[gray, thick] ($(B0)+(-\smallsep,\largesep)$) rectangle ($(B0)+(\smallsep,-\largesep)$); % s6 (bottom only, element 0)
\node[below] at ($(B0)+(0,-\largesep)$) {$S_6$};
\draw[gray, thick] ($(B0)+(\dx-\smallsep,-\largesep)$) rectangle ($(T0)+(\dx+\smallsep,\largesep)$); % s4 (both rows, element 1)
\node[above] at ($(T0)+(\dx,\largesep)$) {$S_4$};
\draw[gray, thick] ($(B0)+(2*\dx-\smallsep,-\largesep)$) rectangle ($(T0)+(3*\dx+\smallsep,\largesep)$); % s3 (both rows, elements 2..3)
\node[above] at ($(T0)+(3*\dx-\smallsep,\largesep)$) {$S_3$};
\draw[gray, thick] ($(B0)+(4*\dx-\smallsep,-\largesep)$) rectangle ($(T0)+(7*\dx+\smallsep,\largesep)$); % s2 (both rows, elements 4..7)
\node[above] at ($(T0)+(5.5*\dx,\largesep)$) {$S_2$};
\draw[gray, thick] ($(B0)+(8*\dx-\smallsep,-\largesep)$) rectangle ($(T0)+(15*\dx+\smallsep,\largesep)$); % s1 (both rows, elements 8..15)
\node[above] at ($(T0)+(11.5*\dx,\largesep)$) {$S_1$};

\end{tikzpicture}
\]
The optimal set cover consists of sets $S_7$ and $S_8$, each of size $16$, but...
\begin{itemize}
  \item initially the three sets \(S_1, S_7, S_8\) each cover \(16\) elements
  \item if ties are broken in the worst possible way, \autoref{alg:set-cover-greedy-approx} picks~\(S_1\)
  \item now the sets \(S_2, S_7, S_8\) each cover \(8\) of the remaining elements 
  \item again, if we choose poorly, \(S_2\) is chosen
  \item and so on...
  \qedhere
\end{itemize}
\end{example}

We generalize \autoref{ex:greedy-can-be-fooled} to any number of elements that is a power of \(2\).
\begin{itemize}
  \item although there is an optimal solution \(2\) sets, \autoref{alg:set-cover-greedy-approx} will select roughly \(\log m\) sets, where \(m = |U|\) 
  \item thus, the \nameref{alg:set-cover-greedy-approx} achieves an approximation factor of roughly \(\frac{\log m}{2}\) on this example
  \item it is possible to slighlty adjust the example such that there are no ties and yet \autoref{alg:set-cover-greedy-approx} has essentially the same ratio bound
\end{itemize}

\begin{theorem}\label{thm:set-cover-greedy-approx-ratio}
\autoref{alg:set-cover-greedy-approx} achieves an approximation factor of \(1 + \ln m\), where \(m = |U|\) denotes the number of elements to be covered.
\end{theorem}
\begin{proof}%[sketch]
Let \(c = |C^*|\) be the size of an optimal set cover.
At any iteration \(i\) with \(m_{i-1}\) uncovered elements remaining, some optimal set must cover at least \(m_{i-1} / c\) of them (Pigeonhole principle).
Since \autoref{alg:set-cover-greedy-approx} chooses the set covering the maximum number of uncovered elements, it removes at least this many, leaving
\[
m_i 
\le 
m_{i-1} - \frac{m_{i-1}}{c}
=
m_{i-1} \left(1 - \frac{1}{c}\right)
\]
elements uncovered after iteration \(i\). 
Applying this recursively, we obtain
\[
m_i
\le
m \left(1 - \frac{1}{c}\right)^i
\]
where we used \(m_0 = m = |U|\).

Now denote by \(g := |C|\) the number of sets returned by \autoref{alg:set-cover-greedy-approx}
(equivalently, the number of iterations).
Right before the \(g^\text{th}\) iterations of \autoref{alg:set-cover-greedy-approx} we therefore have
\[
m_{g-1} 
\le 
m \left(1 - \frac{1}{c}\right)^{g-1} 
=
m \left(\left(1 - \frac{1}{c}\right)^c\right)^{\frac{g-1}{c}}
\le 
m \left(\frac{1}{\eu}\right)^{\frac{g-1}{c}}
\]
elements uncovered,
where we used \((1 - 1/x)^x \le 1/\eu\).

Since at least one element remains uncovered before the last iteration, i.e. \(m_{g-1} \ge 1\), we have 
\[
1 \leq m \left(\frac{1}{\eu}\right)^{\frac{g-1}{c}}
\quad\Rightarrow\quad
\eu^{\frac{g-1}{c}} \leq m
\quad\Rightarrow\quad
\frac{g-1}{c} \leq \ln m
% \quad\Rightarrow\quad
% g \leq c \ln m + 1
\]
which implies \(g \le c \ln m + 1\).

Therefore, \autoref{alg:set-cover-greedy-approx} returns a \nameref{prob:set-cover} of size at most \(1 + \ln(m) \cdot c\).
Since \(1 + \ln(m) \cdot c \leq (1 + \ln m) \cdot c\), we have \(g \leq (1 + \ln m) \cdot c\), 
establishing the claimed \((1 + \ln m)\)-approximation ratio.
\end{proof}




% Even though the greedy heuristic has this relatively high approximation factor, it tends to perform well in practice.

% – the example in which the approximation bound is Ω(log m) is not typical of set cover instances.

% – Is there is a more sophisticated approximation algorithm that has a better approximation factor?

% Answer: such algorithms exist for special cases of set cover, but for the general problem no significantly better approximation bound is believed to be computable in polynomial time.

% – (Formally, such an algorithm would imply that NP has quasipolynomial time algorithms, and the experts do not believe that this is the case.)



Even though the \nameref{alg:set-cover-greedy-approx} has this relatively high approximation factor, it tends to perform well in practice.
\begin{itemize}
  \item the example in which the approximation bound is \(\Omega(\log m)\) is not typical of \nameref{prob:set-cover} instances
  \item more sophisticated approximation algorithms exist for special cases of \nameref{prob:set-cover}, but for the general problem no significantly better approximation bound is believed to be computable in polynomial time
  \item such an algorithm would imply that NP has quasipolynomial time algorithms (Uriel Feige, 1998), and the experts do not believe that this is the case
\end{itemize}




























































\clearpage


\section{Appendix}

\subsection{Homework 2}


\begin{algorithm}[h]
\caption*{Find cross edges and odd cycles in an undirected graph using \nameref{alg:bfs}}\label{alg:bfs_detect_cross_edges_odd_cycles}
\begin{algorithmic}[1]
\Function{BFS}{$G,\,s$} \Comment{\(s\) is the source}
  \ForAll{$u\in V \setminus \{s\}$}
    \State $\attrib{u}{color},\attribnormal{u}{\delta},\attrib{u}{\pi}\gets \white,\infty,\nil$ \Comment{initialization}
  \EndFor
  % \State Initialize $\attrib{v}{color}\gets\white$, $\attribnormal{v}{\delta}\gets\infty$, $\attrib{v}{\pi}\gets\nil$ for all $v\in V$
  \State $\attrib{s}{color},\attribnormal{s}{\delta},\attrib{s}{\pi}\gets \gray,0,\nil$ \Comment{initialize source \(s\)}
  \State $Q\gets\emptyset$ \Comment{initialize empty queue for vertices to visit}
  \State \Call{Enqueue}{$Q,s$}
  \While{$Q\neq\emptyset$} 
    \State $u\gets$ \Call{Dequeue}{$Q$} \Comment{get next vertex from the frontier}
    \ForAll{$v\in\Gamma(u)$} 
      \If{$\attrib{v}{color}=\white$} \Comment{first time we have seen \(v\)?}
        \State $\attrib{v}{color}\gets\gray$ \Comment{mark it discovered}
        \State $\attribnormal{v}{\delta}\gets\attribnormal{u}{\delta}+1$ \Comment{set its distance from \(s\)}
        \State $\attrib{v}{\pi}\gets u$ \Comment{set its parent}
        \State \Call{Enqueue}{$Q,v$}
      \BeginBox[draw=red]
      \Else
        \If{$\attrib{v}{color} = \black \AND v \neq \attrib{u}{\pi}$} \Comment{\(2^\text{nd}\) time we see \((u, v)\)}
          \State add \((u,v)\) to the list of cross edges
        \EndIf
        \If{$\attribnormal{v}{\delta} = \attribnormal{u}{\delta}$} \Comment{\(u\) and \(v\) are in the same layer}
          \State odd cycle detected!
        \EndIf
      \EndIf
      \EndBox
    \EndFor
    \State $\attrib{u}{color}\gets\black$ \Comment{we are done with \(u\)}
  \EndWhile
\EndFunction
\end{algorithmic}
\end{algorithm}










\end{document}
