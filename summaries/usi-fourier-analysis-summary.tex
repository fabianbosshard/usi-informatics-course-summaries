% author: Fabian Bosshard % © CC BY 4.0
\documentclass[9pt,headings=standardclasses,parskip=half]{scrartcl}

% ===================== basics =====================
\usepackage{ifthen,iftex,csquotes}
\usepackage[T1]{fontenc}
\usepackage[english]{babel}

% ===================== page style =====================
\usepackage[automark]{scrlayer-scrpage}
\pagestyle{scrheadings}

% ===================== graphics / color =====================
\usepackage{graphicx}
\usepackage[dvipsnames]{xcolor}
\usepackage{caption,subcaption}
\usepackage[left=38mm,right=38mm,top=20mm,bottom=30mm]{geometry}

% ===================== math =====================
\usepackage{amsmath,amssymb,amsthm,mathtools,mathdots,accents}
\numberwithin{equation}{section}

% ===================== lists =====================
\usepackage{enumitem}
\renewcommand{\labelitemi}{\textbullet}
\renewcommand{\labelitemii}{\raisebox{0.1ex}{\scalebox{0.8}{\textbullet}}}
\renewcommand{\labelitemiii}{\raisebox{0.2ex}{\scalebox{0.6}{\textbullet}}}
\renewcommand{\labelitemiv}{\raisebox{0.3ex}{\scalebox{0.4}{\textbullet}}}

% ===================== bibliography =====================
\usepackage[backend=biber,style=numeric]{biblatex}
\addbibresource{\jobname.bib}

% ===================== symbols =====================
\usepackage{pifont}

% ===================== TikZ / plots =====================
\usepackage{tikz}
\usetikzlibrary{arrows,arrows.meta,shapes,positioning,calc,fit,patterns,intersections,math,3d,tikzmark,decorations.pathreplacing,decorations.markings}
\usepackage{tikz-dependency,tikz-qtree,tikz-qtree-compat,tikz-3dplot,tikzpagenodes}
\usepackage{pgfplots}

% ===================== highlighting (SOUL + math-safe \hl) =====================
\usepackage{soulutf8}
\usepackage{xparse}

\ExplSyntaxOn
\tl_new:N \__l_SOUL_argument_tl
\cs_set_eq:Nc \SOUL_start:n { SOUL@start }
\cs_generate_variant:Nn \SOUL_start:n { V }
\cs_set_protected:cpn {SOUL@start} #1 {
  \tl_set:Nn \__l_SOUL_argument_tl { #1 }
  \regex_replace_all:nnN { \c{\(} (.*?) \c{\)} } { \cM\$ \1 \cM\$ } \__l_SOUL_argument_tl
  \SOUL_start:V \__l_SOUL_argument_tl
}
\ExplSyntaxOff

\let\SOULhl\hl
\renewcommand{\hl}[1]{\SOULhl{#1}}

\definecolor{HLgreen}{HTML}{77DD77}
\definecolor{HLyellow}{HTML}{FFFF66}
\definecolor{HLorange}{HTML}{FFB347}
\definecolor{HLred}{HTML}{FF6961}
\definecolor{HLpink}{HTML}{FFB6C1}
\definecolor{HLturquoise}{HTML}{40E0D0}

\RenewDocumentCommand{\hl}{O{1} m}{%
  \begingroup
  \IfEqCase{#1}{%
    {1}{\sethlcolor{HLgreen}\SOULhl{#2}}%
    {2}{\sethlcolor{HLyellow}\SOULhl{#2}}%
    {3}{\sethlcolor{HLorange}\SOULhl{#2}}%
    {4}{\sethlcolor{HLred}\SOULhl{#2}}%
    {5}{\sethlcolor{red}\SOULhl{#2}}%
    {6}{\sethlcolor{HLturquoise}\SOULhl{#2}}%
  }[\PackageError{hl}{Undefined highlight level: #1}{}]%
  \endgroup
}

% ===================== colors / emphasis =====================
\definecolor{funblue}{rgb}{0.10,0.35,0.66}
\definecolor{alizarincrimsonred}{rgb}{0.85,0.17,0.11}
\definecolor{amethyst}{rgb}{0.6,0.4,0.8}
\definecolor{mypurple}{rgb}{0.5,0,0.5}
\definecolor{highlightpurple}{rgb}{0.6,0,0.6}
\renewcommand{\emph}[1]{\textcolor{highlightpurple}{\textsl{#1}}}

% ===================== theorem environments =====================
\usepackage{thmtools}

\newlength{\thmspace}\setlength{\thmspace}{3pt plus 1pt minus 1pt}

\declaretheoremstyle[headfont=\bfseries,bodyfont=\normalfont,spaceabove=\thmspace,spacebelow=\thmspace,qed=\ensuremath{\vartriangleleft},postheadspace=1em]{assertionstyle}
\declaretheorem[style=assertionstyle,name=Theorem,numberwithin=section]{theorem}
\declaretheorem[style=assertionstyle,name=Lemma,sibling=theorem]{lemma}
\declaretheorem[style=assertionstyle,name=Corollary,sibling=theorem]{corollary}
\declaretheorem[style=assertionstyle,name=Proposition,sibling=theorem]{proposition}
\declaretheorem[style=assertionstyle,name=Conjecture,sibling=theorem]{conjecture}
\declaretheorem[style=assertionstyle,name=Claim,sibling=theorem]{claim}
\declaretheorem[style=assertionstyle,name=Fact,sibling=theorem]{fact}

\declaretheoremstyle[headfont=\bfseries,bodyfont=\normalfont,spaceabove=\thmspace,spacebelow=\thmspace,qed=\ensuremath{\blacktriangleleft},postheadspace=1em]{definitionstyle}
\declaretheorem[style=definitionstyle,name=Definition,numberwithin=section]{definition}
\declaretheorem[style=definitionstyle,name=Problem,sibling=definition]{problem}

\declaretheoremstyle[headfont=\bfseries,bodyfont=\normalfont,spaceabove=\thmspace,spacebelow=\thmspace,qed=\ding{45},postheadspace=1em]{exercisestyle}
\declaretheorem[style=exercisestyle,name=Exercise,numberwithin=section]{exercise}
\declaretheoremstyle[headfont=\bfseries\color{red},bodyfont=\normalfont,spaceabove=\thmspace,spacebelow=\thmspace,qed=\ensuremath{\color{red}\blacktriangleleft},postheadspace=1em]{solutionstyle}
\declaretheorem[style=solutionstyle,name=Solution,numbered=no]{solution}

\declaretheoremstyle[headfont=\bfseries,bodyfont=\normalfont,spaceabove=6pt,spacebelow=6pt,qed=\ensuremath{\square},postheadspace=1em]{proofstyle}
\let\proof\relax \let\endproof\relax
\declaretheorem[style=proofstyle,name=Proof,numbered=no]{proof}

\declaretheoremstyle[headfont=\bfseries,bodyfont=\normalfont\normalsize,spaceabove=\thmspace,spacebelow=\thmspace,qed=\ensuremath{\blacktriangleleft},postheadspace=1em]{remarkstyle}
\declaretheorem[style=remarkstyle,name=Remark,numberwithin=section]{remark}

\declaretheoremstyle[headfont=\bfseries\color{funblue},bodyfont=\normalfont\normalsize,spaceabove=\thmspace,spacebelow=\thmspace,qed=\ensuremath{\color{funblue}\blacktriangleleft},postheadspace=1em]{examplestyle}
\declaretheorem[style=examplestyle,name=Example,sibling=remark]{example}

\declaretheoremstyle[headfont=\color{alizarincrimsonred}\bfseries,bodyfont=\normalfont\normalsize,spaceabove=\thmspace,spacebelow=\thmspace,qed=\ensuremath{\color{alizarincrimsonred}\blacktriangleleft},postheadspace=1em]{cautionstyle}
\declaretheorem[style=cautionstyle,name=Caution,sibling=remark]{caution}

\declaretheoremstyle[headfont=\bfseries,bodyfont=\normalfont\footnotesize,spaceabove=\thmspace,spacebelow=\thmspace,postheadspace=1em]{smallremarkstyle}
\declaretheorem[style=smallremarkstyle,name=Remark,sibling=remark]{smallremark}

\declaretheoremstyle[headfont=\bfseries\color{amethyst},bodyfont=\normalfont,spaceabove=6pt,spacebelow=6pt,qed=\ensuremath{\color{amethyst}\blacktriangleleft},postheadspace=1em]{digressionstyle}
\declaretheorem[style=digressionstyle,name=Digression,sibling=remark]{digression}

% ===================== misc helpers =====================
\newenvironment{verticalhack}{\begin{array}[b]{@{}c@{}}\displaystyle}{\\\noalign{\hrule height0pt}\end{array}}

% ===================== algorithms =====================
\usepackage{algorithm,algorithmicx}
\usepackage[italicComments=false]{algpseudocodex}
\newcommand*{\algorithmautorefname}{Algorithm}

\algnewcommand{\TO}{, \ldots ,}
\algnewcommand{\DOWNTO}{, \ldots ,}
\algnewcommand{\OR}{\vee}
\algnewcommand{\AND}{\wedge}
\algnewcommand{\NOT}{\neg}
\algnewcommand{\LEN}{\operatorname{len}}
\algnewcommand{\tru}{\ensuremath{\mathrm{\texttt{true}}}}
\algnewcommand{\fals}{\ensuremath{\mathrm{\texttt{false}}}}
\algnewcommand{\append}{\circ}
\algnewcommand{\nil}{\ensuremath{\mathrm{\textsc{nil}}}}
\algnewcommand{\red}{\ensuremath{\mathrm{\textsc{red}}}}
\algnewcommand{\black}{\ensuremath{\mathrm{\textsc{black}}}}
\algnewcommand{\gray}{\ensuremath{\mathrm{\textsc{gray}}}}
\algnewcommand{\white}{\ensuremath{\mathrm{\textsc{white}}}}

% ===================== operators / notation =====================
\DeclareMathOperator*{\argmax}{arg\,max}
\DeclareMathOperator*{\argmin}{arg\,min}
\DeclareMathOperator{\Span}{span}
\DeclareMathOperator{\Kern}{kern}
\DeclareMathOperator{\Trace}{trace}
\DeclareMathOperator{\Rank}{rank}
\newcommand{\im}{\operatorname{Im}}
\newcommand{\re}{\operatorname{Re}}

\newcommand*\matrspace{0.8mu}
\newcommand{\matr}[1]{\mspace{\matrspace}\underline{\mspace{-\matrspace}\smash[b]{\boldsymbol{#1}}\mspace{-\matrspace}}\mspace{\matrspace}}
\newcommand{\vect}[1]{\vec{\boldsymbol{#1}}}

\newcommand{\dif}{\mathrm{d}}
\newcommand{\eu}{\mathrm{e}}
\newcommand{\iu}{\mathrm{i}}
\newcommand{\sign}{\operatorname{sgn}}

\newcommand{\R}{\mathbb{R}}
\newcommand{\N}{\mathbb{N}}
\newcommand{\Z}{\mathbb{Z}}
\newcommand{\Q}{\mathbb{Q}}
\newcommand{\C}{\mathbb{C}}
\newcommand{\K}{\mathbb{K}}
\newcommand{\F}{\mathbb{F}}

\newcommand{\Var}{\operatorname{Var}}
\newcommand{\Cov}{\operatorname{Cov}}
\newcommand{\Exp}{\operatorname{E}}
\newcommand{\Prob}{\operatorname{P}}
\newcommand{\numof}{\ensuremath{\#\,}}
\newcommand{\blackheight}{\operatorname{bh}}

\newcommand{\attrib}[2]{\ensuremath{#1\mathtt{.}\mathtt{#2}}}
\newcommand{\attribnormal}[2]{\ensuremath{#1\mathtt{.}#2}}

% ===================== metadata / hyperlinks =====================
\title{Fourier Analysis}
\author{Fabian Bosshard}
\date{\today}

\usepackage[
  linktoc=none,
  pdfauthor={Fabian Bosshard},
  pdftitle={USI - Fourier Analysis - Course Notes},
  pdfkeywords={USI, Fourier analysis, course notes, informatics},
  colorlinks=false,
  pdfborder={0 0 0},
  linkbordercolor={0 0.6 1},
  urlbordercolor={0 0.6 1},
  citebordercolor={0 0.6 1}
]{hyperref}

% ToC entries: dotted leaders + full-width clickable line
\DeclareTOCStyleEntry[linefill=\dotfill]{tocline}{section}
\DeclareTOCStyleEntry[linefill=\dotfill]{tocline}{subsection}
\DeclareTOCStyleEntry[linefill=\dotfill]{tocline}{subsubsection}

\makeatletter
\newlength\FB@toclinkht \newlength\FB@toclinkdp
\setlength\FB@toclinkht{.80\ht\strutbox}
\setlength\FB@toclinkdp{.40\dp\strutbox}

\let\FB@orig@contentsline\contentsline
\renewcommand*\contentsline[4]{%
  \begingroup
    \Hy@safe@activestrue
    \edef\Hy@tocdestname{#4}%
    \FB@orig@contentsline{#1}{%
      \Hy@raisedlink{\hyper@anchorstart{toc:\Hy@tocdestname}\hyper@anchorend}%
      \leavevmode
      \rlap{%
        \hyper@linkstart{link}{\Hy@tocdestname}%
          \raisebox{0pt}[\FB@toclinkht][\FB@toclinkdp]{%
            \hbox to \dimexpr\hsize-\parindent\relax{\hfil}%
          }%
        \hyper@linkend
      }%
      #2%
    }{#3}{#4}%
  \endgroup
}

\let\FB@orig@sectionlinesformat\sectionlinesformat
\renewcommand{\sectionlinesformat}[4]{%
  \ifx\@currentHref\@empty
    \FB@orig@sectionlinesformat{#1}{#2}{#3}{#4}%
  \else
    \leavevmode
    \hyper@linkstart{link}{toc:\@currentHref}%
      \@hangfrom{\hskip #2\relax #3}{#4}%
    \hyper@linkend
    \par
  \fi
}
\makeatother

% hyperref anchor uniqueness for section-relative numbering
\makeatletter
\renewcommand{\theHequation}{\thesection.\arabic{equation}}
\renewcommand{\theHtheorem}{\thesection.\arabic{theorem}}
\renewcommand{\theHlemma}{\theHtheorem}
\renewcommand{\theHcorollary}{\theHtheorem}
\renewcommand{\theHproposition}{\theHtheorem}
\renewcommand{\theHconjecture}{\theHtheorem}
\renewcommand{\theHclaim}{\theHtheorem}
\renewcommand{\theHfact}{\theHtheorem}
\renewcommand{\theHdefinition}{\thesection.\arabic{definition}}
\renewcommand{\theHproblem}{\theHdefinition}
\renewcommand{\theHexercise}{\thesection.\arabic{exercise}}
\renewcommand{\theHremark}{\thesection.\arabic{remark}}
\renewcommand{\theHexample}{\theHremark}
\renewcommand{\theHcaution}{\theHremark}
\renewcommand{\theHsmallremark}{\theHremark}
\renewcommand{\theHdigression}{\theHremark}
\makeatother

% ===================== license / cleveref =====================
\usepackage[type={CC},modifier={by},version={4.0}]{doclicense}
\usepackage{cleveref}

% ===================== acronyms =====================
\usepackage[acronym,nomain,toc,nonumberlist]{glossaries-extra}
\setabbreviationstyle[acronym]{long-short}
\makeglossaries
\newacronym{rbf}{RBF}{radial basis function}
\newacronym{rkhs}{RKHS}{reproducing kernel Hilbert space}

\begin{document}


\maketitle

\tableofcontents

\clearpage
\section{Preliminaries}
% aligned label + content
\newlength{\propw}
\newlength{\propsep}
\newcommand{\propitem}[2]{%
  \item \makebox[\propsep][l]{#1}%
  \parbox[t]{\dimexpr\linewidth-\propsep\relax}{#2}%
}
\settowidth{\propw}{{Distributive:}}
\setlength{\propsep}{\propw + 1.5em}

\newlength{\formw}
\settowidth{\formw}{$(a\circ b)\circ c = a\circ(b\circ c)$} % longest formula (adjust if needed)
\newlength{\formsep}
\setlength{\formsep}{\formw + 1.5em}

\paragraph{Algebraic properties}
\begin{itemize}
\propitem{{Associative:}}{%
  \makebox[\formsep][l]{$(a\circ b)\circ c = a\circ(b\circ c)$}%
  \emph{parentheses (grouping) do not matter}}
\propitem{{Commutative:}}{%
  \makebox[\formsep][l]{$a\circ b = b\circ a$}%
  \emph{order does not matter}}
\propitem{{Distributive:}}{%
    $a\cdot(b+c)=a\cdot b+a\cdot c$\\
    $(a+b)\cdot c=ac+bc$}
\end{itemize}




\paragraph{Trigonometric identities}
\begin{itemize}
    \item \(\cos( - y) = \cos(y)\)
    \item \(\sin( - y) = - \sin(y)\)
    \item \(\cos(a + b) = \cos(a) \cos(b) - \sin(a) \sin(b)\)
    \item \(\sin(a + b) = \sin(a) \cos(b) + \cos(a) \sin(b)\)
\end{itemize}

\paragraph{Complex functions}
If \(f: \R \to \C\), we may write \(f(x) = \Re(f(x)) + \iu \Im(f(x))\).
We have \(|f(x)|^2 = f(x) \overline{f(x)}\), where \(\overline{f(x)} = \Re(f(x)) - \iu \Im(f(x))\).


\subsection{Lebesgue integration}

\(\mathcal{R}([a,b])\) denotes the class of Riemann integrable functions on \([a,b]\).
\begin{theorem}
Let \(f: [a,b] \to \R\) be bounded.
\begin{itemize}
\item 
If \(f \in \mathcal{R}([a,b])\), then the Lebesgue integral of \(|f|\) is finite.
Moreover, the Riemann integral of \(f\) equals the Lebesgue integral of \(f\).
\item 
\(f \in \mathcal{R}([a,b])\) if and only if \(f\) is continuous except on a set of measure zero.
\qedhere
\end{itemize}
\end{theorem}

The motivation for replacing Riemann with Lebesgue theory is that Lebesgue integrals are particularly well-suited for interchanginng limits and integration,
the order of integration and derivatives and integration.


\subsection{Complex analysis}


We define the curve integral of a function \(f: \C \to \C\) along a curve \(C\) parametrized by \(z(t)\), \(t \in [a,b]\) as
\begin{equation}\label{eq:curve-integral}
\int_C f(z) \dif z := \int_a^b f( z(t)) z'(t) \dif t
\end{equation}

For a positively oriented circle \(C\) centered at \(z_0\) with radius \(R\), we have
\begin{equation}\label{eq:central-integral}
\int_C (z - z_0)^n \dif z 
=
\begin{cases}
    0 & n \neq -1 \\
    2 \pi \iu & n = -1
\end{cases}
\end{equation}


\subsection{Function spaces}

Classically, functions are classified in terms of their regularity.
We write \(C(\R)\) for the space of continuous functions from \(\R\) to \(\C\).
For \(k \in \N\), we write \(C^k(\R)\) for the space of functions from \(\R\) to \(\C\) whose first \(k\) derivatives exist and are continuous.

% Recall that
% \[
% \begin{aligned}
% \text{$f$ continuous}             &\;\Leftarrow& \text{$f$ differentiable}        &\;\Leftarrow& \text{$f$ continuously differentiable} \\
% \phantom{\text{$f$ continuous}}   &\;\Leftarrow& \text{$f$ twice differentiable}  &\;\Leftarrow& \text{$f$ twice continuously differentiable} \\
% \phantom{\text{$f$ continuous}}   &\;\Leftarrow& \text{$f$ 3-times differentiable}&\;\Leftarrow& \dots \;\Leftarrow\; \text{$f$ smooth}
% \end{aligned}
% \]
Recall that
\[
\begin{aligned}
\text{$f$ continuous}           &\Leftarrow & \text{$f$ differentiable}         &\Leftarrow & \text{$f$ continuously differentiable} \\
\phantom{\text{$f$ continuous}} &\Leftarrow & \text{$f$ twice differentiable}   &\Leftarrow & \text{$f$ twice continuously differentiable} \\
\phantom{\text{$f$ continuous}} &\Leftarrow & \text{$f$ 3-times differentiable} &\Leftarrow & \dots \;\Leftarrow\; \text{$f$ smooth}
\end{aligned}
\]

\begin{example}
  The function \(f(x) = x^k \sign(x)\) is in \(C^{k-1}(\R)\) but not in \(C^k(\R)\).
\end{example}


The revolution in modern analysis is that functions are classified in terms of their integrability.
For \(1 \leq p < \infty\), we write \(f \in L^p(\R)\) if
\begin{equation}\label{eq:Lp-norm}
{\color{Red}\boxed{\color{black}
\| f \|_p  := \left( \int_{-\infty}^{\infty} |f(x)|^p \dif x \right)^{1/p} 
}}
\end{equation}
is finite.
To be precise, we also need the requierement that \(f: \R \to \C\) is measurable.

\begin{remark}
There is also a definition of \(L^{\infty}(\R)\), where \(f \in L^{\infty}(\R)\) if
\begin{equation}\label{eq:Linf-norm}
{\color{Red}\boxed{\color{black}
\| f \|_{\infty} := \operatorname{ess\,sup}_{x \in \R} |f(x)|
}}
\end{equation}
is finite.
The essential supremum is basically a supremum out of a set of measure zero.
\end{remark}
\begin{remark}
Strictly speaking, \eqref{eq:Lp-norm} is not a norm on functions defined pointwise, but on equivalence classes of functions that are equal almost everywhere.
\end{remark}

We write \(f \in C_b(\R)\) if \(f \in C(\R)\) and there exists \(M > 0\) such that \(|f(x)| \leq M\) for all \(x \in \R\) (i.e., \(f\) is bounded).
Observe that \hl[2]{\(C_b(\R)  = L^{\infty}(\R) \cap C(\R)\)}.

If \(f \in C_b(\R)\), then \(\|f\|_{\infty} = \max_{x \in \R} |f(x)|\).

% In general, $L^p(\R)$ is not closed under pointwise products: even if $f,g\in L^p(\R)$, it may happen that $fg\notin L^p(\R)$.
% For $1\le p\le \infty$, the \emph{conjugate exponent} $p'$ is defined by
% \[
% \frac{1}{p}+\frac{1}{p'}=1,
% \qquad \text{with } \frac{1}{\infty}:=0 \text{ and } \frac{1}{0}:=\infty.
% \]
% If $f\in L^p(\R)$ and $g\in L^{p'}(\R)$, then $fg\in L^1(\R)$ and H\"older's inequality holds:
% \[
% \|fg\|_1 \le \|f\|_p\,\|g\|_{p'}.
% \]

% A subspace $Y\subseteq X$ of a normed space $(X,\|\cdot\|_X)$ is \emph{dense} in $X$ if for every $x\in X$ and every $\varepsilon>0$ there exists $y\in Y$ such that
% \[
% \|x-y\|_X<\varepsilon.
% \]
% The space
% \[
% C_c^\infty(\R):=\{f\in C^\infty(\R): \exists R>0 \text{ such that } f(x)=0 \text{ for } |x|>R\}
% \]
% is dense in $L^p(\R)$ for every $1\le p<\infty$. Hence, many statements can be proved first for $C_c^\infty(\R)$ and then extended to $L^p(\R)$ by approximation.




\clearpage
\section{Fourier Series}

We write $f \in L^2([-\pi, \pi])$ if $f: \R \rightarrow \mathbb{C}$ is $2 \pi$-periodic and
\begin{equation}\label{eq:L2-norm}
{\color{Red}
% \setlength{\fboxrule}{1pt}
\boxed{ 
\color{black}
\|f\|_2=\left(\frac{1}{2 \pi} \int_{-\pi}^\pi|f(x)|^2 \dif x\right)^{\frac{1}{2}}
}}
\end{equation}
is finite, i.e. \(\|f\|_2 < \infty\).

Functions in $L^2([-\pi, \pi])$ are called finite-energy signals, the norm $\|f\|_2$ is the energy of $f$. 
Observe that we normalized the integral in \eqref{eq:L2-norm} by means of the factor $\frac{1}{2 \pi}$. 
% In this way, we normalized the length of $[-\pi, \pi]$:
% $$
% \operatorname{length}([-\pi, \pi])=\frac{1}{2 \pi} \int_{-\pi}^\pi 1 \dif x
% =
% 1
% $$
Loosely speaking, we change the unit of measurement so that the length of $[-\pi, \pi]$ becomes $1$.

% see Figure 2.1.
% Differently from the other $L^p$ spaces, the space $L^2([-\pi, \pi])$ has the additional structure provided by the (sesquilinear) inner product:

% $$
% \langle f, g\rangle_{L^2}=\frac{1}{2 \pi} \int_{-\pi}^\pi f(x) \overline{g(x)} d x, \quad f, g \in L^2([-\pi, \pi]) .
% $$


% For the benefit of the reader, we recall the definition of inner product, that reflects the usual definition of inner product on $\R^d$, with the necessary precaution needed when working in the complex framework.




% Here, we used the component-wise continuity of $\langle\cdot, \cdot\rangle_{L^2}$ to intertwine series and inner products. 
For every $f \in L^2([-\pi, \pi])$,
\begin{equation}\label{eq:fourier-series}
  f(x) \stackrel{L^2}{=} \sum_{n=-\infty}^{\infty} \hat{f}(n) \eu^{\iu n x}
\end{equation}
where, for \(n \in \Z\),
\begin{equation}\label{eq:fourier-coefficient}
{\color{Red}\boxed{\color{black}
\hat{f}(n)
=
\left\langle f, e_n\right\rangle_{L^2}
=
\frac{1}{2 \pi} \int_{-\pi}^\pi f(x) \eu^{-\iu n x} \dif x
}}
\end{equation}
is the $n$-th Fourier coefficient of $f$. 
\eqref{eq:fourier-series} is the Fourier series of $f$.


For every \(f \in L^2([-\pi,\pi])\), we have 
\begin{equation}\label{eq:parseval}
\|f\|_2^2=\sum_{n=-\infty}^{\infty}|\hat{f}(n)|^2
\end{equation}
which is known as \hl{Parseval's theorem}.




\clearpage

\section{Fourier Transform}



For $f \in L^1(\R)$ the Fourier transform
\begin{equation}\label{eq:fourier-transform}
{\color{Red}\boxed{\color{black}
\hat{f}(\xi)=\int_{-\infty}^{\infty} f(x) \eu^{-2 \pi \iu \xi x} \dif x
}}
\end{equation}
converges for every $\xi \in \R$. 
Moreover,
\[
|\hat{f}(\xi)| \leq \int_{-\infty}^{\infty}|f(x)| \dif x=\|f\|_1
\]
where \(\xi \in \R\),
implying that $\hat{f}$ is in $L^{\infty}(\R)$ and the operator 
\[
\mathcal{F}: f \in L^1(\R) \mapsto \hat{f} \in L^{\infty}(\R)
\]
is bounded with
\(
\|\hat{f}\|_{\infty} \leq\|f\|_1 
\).

\begin{definition}
The operator $\mathcal{F}$ defined above is called Fourier transform, and the function $\hat{f}$ is the Fourier transform of $f$.
\end{definition}

Recall that a function $f: \R \rightarrow \mathbb{C}$ is uniformly continuous on $\R$ if for every $\varepsilon>0$ there exists $\delta>0$ depending only on $\varepsilon$ such that
$$
|x-y|<\delta \quad \Longrightarrow \quad|f(x)-f(y)|<\varepsilon 
$$


\begin{fact}
The Fourier transform is a bounded operator from $L^1(\R)$ to $L^{\infty}(\R)$. Moreover, $\hat{f}$ is a uniformly continuous function on $\R$ for every $f \in L^1(\R)$.
\end{fact}








% So for \( \xi \neq 0\),
% \[
% \hat{f}(\xi) = \frac{1}{2 \pi \xi} 2 \sin( \pi \xi A) = \frac{\sin( \pi \xi A)}{\pi \xi}
% \]
% In conclusion
% \[
% \hat{f}(\xi) =
% \begin{cases}
%     \frac{\sin( \pi \xi A)}{\pi \xi}, & \xi \neq 0, \\
%     A, & \xi = 0.
% \end{cases}
% \]

% Oberserve that \(\hat{f}\) is indeed in \(\mathcal{C}_b(\R)\) 

% \(\frac{\sin( \pi \xi A)}{\pi \xi}\) is a continuous at every \(\xi \neq 0\) and
% \[
% \lim_{\xi \to 0} \frac{\sin( \pi \xi A)}{\pi \xi} = A = \hat{f}(0)
% \]
% by definition.
\begin{example}[Characteristic function]
For every \(\xi \ne 0\),
\[
\hat{f}(\xi) 
=
\frac{\sin( \pi A \xi)}{\pi \xi}
\]
and \(\hat{f}(0) = A\).
\[
\begin{tikzpicture}[scale=1.2, font=\footnotesize]
  \tikzset{>=latex}
  \colorlet{myblue}{green!80!black}
  \colorlet{mydarkblue}{myblue!80!black}
  \tikzstyle{xline}=[myblue,thick]
  \def\tick#1#2{\draw[thick] (#1) ++ (#2:0.1) --++ (#2-180:0.2)}

  % -------- parameters (feel free to tweak axes) ----------
  \def\xmax{4.5}
  \def\ymin{-0.4}
  \def\ymax{1.4}
  \def\Aparam{1.0} % this is the "A" in sin(pi*A*xi)/(pi*xi)
  % --------------------------------------------------------

  \draw[->,thick] (0,\ymin) -- (0,\ymax);% node[left] {$\hat f(\xi)$};
  \draw[->,thick] (-\xmax,0) -- (\xmax+0.1,0) node[below=1,right=0.05] {$\xi$};

  % plot: sin(pi*A*xi)/(pi*xi), trig in pgf is degrees -> use deg(...)
  \draw[xline,samples=200,smooth,variable=\t,domain=-0.94*\xmax:0.94*\xmax]  plot(\t,{ (abs(\t)<0.01) ? (\Aparam) : (sin(deg(pi*\Aparam*\t))/(pi*\t)) });

  % zeros at xi = k/A
  \tick{-4/\Aparam,0}{90} node[below=0,scale=1] {$-\frac{4}{A}$};
  \tick{-3/\Aparam,0}{90} node[below=0,scale=1] {$-\frac{3}{A}$};
  \tick{-2/\Aparam,0}{90} node[below=0,scale=1] {$-\frac{2}{A}$};
  \tick{-1/\Aparam,0}{90} node[below=0,scale=1] {$-\frac{1}{A}$};
  \tick{ 1/\Aparam,0}{90} node[below=0,scale=1] {$ \frac{1}{A}$};
  \tick{ 2/\Aparam,0}{90} node[below=0,scale=1] {$ \frac{2}{A}$};
  \tick{ 3/\Aparam,0}{90} node[below=0,scale=1] {$ \frac{3}{A}$};
  \tick{ 4/\Aparam,0}{90} node[below=0,scale=1] {$ \frac{4}{A}$};

  % peak value
  \tick{0,\Aparam}{0} node[left=0.05,scale=0.9] {$A$};

  \node[mydarkblue,right,scale=0.95] at (0.15*\xmax,0.85*\Aparam) {$\displaystyle \hat{f}(\xi)=\frac{\sin(\pi A \xi)}{\pi \xi}$};
\end{tikzpicture}
\]
Thus we conclude that \(\hat{f}(\xi) = \frac{\sin( \pi A \xi)}{\pi \xi}\), where we implicitly consider its continuous continuation.
In particular, for \(A = 1\), we have
\[
\hat{f}(\xi) = \frac{\sin( \pi \xi)}{\pi \xi}
=:
\operatorname{sinc}(\xi)
\]
which is called \emph{cardinal sine} or \emph{sinc function}.
\end{example}


\begin{example}[Gaussian]
The Fourier transform of \(f(x) = \eu^{-\pi x^2}\) is \(\hat{f}(\xi) = \eu^{-\pi \xi^2}\).
\end{example}











\end{document}
