% author: Fabian Bosshard % © CC BY 4.0
\documentclass[9pt,headings=standardclasses,parskip=half]{scrartcl}

% ===================== basics =====================
\usepackage{ifthen,iftex,csquotes}
\usepackage[T1]{fontenc}
\usepackage[english]{babel}

% ===================== page style =====================
\usepackage[automark]{scrlayer-scrpage}
\pagestyle{scrheadings}

% ===================== graphics / color =====================
\usepackage{graphicx}
\usepackage[dvipsnames]{xcolor}
\usepackage{caption,subcaption}
\usepackage[left=38mm,right=38mm,top=20mm,bottom=30mm]{geometry}

% ===================== math =====================
\usepackage{amsmath,amssymb,amsthm,mathtools,mathdots,accents}
\numberwithin{equation}{section}

% ===================== lists =====================
\usepackage{enumitem}
\renewcommand{\labelitemi}{\textbullet}
\renewcommand{\labelitemii}{\raisebox{0.1ex}{\scalebox{0.8}{\textbullet}}}
\renewcommand{\labelitemiii}{\raisebox{0.2ex}{\scalebox{0.6}{\textbullet}}}
\renewcommand{\labelitemiv}{\raisebox{0.3ex}{\scalebox{0.4}{\textbullet}}}

% ===================== bibliography =====================
\usepackage[backend=biber,style=numeric]{biblatex}
\addbibresource{\jobname.bib}

% ===================== symbols =====================
\usepackage{pifont}
\usepackage{marvosym}

% ===================== TikZ / plots =====================
\usepackage{tikz}
\usetikzlibrary{arrows,arrows.meta,shapes,positioning,calc,fit,patterns,intersections,math,3d,tikzmark,decorations.pathreplacing,decorations.markings}
\usepackage{tikz-dependency,tikz-qtree,tikz-qtree-compat,tikz-3dplot,tikzpagenodes}
\usepackage{pgfplots}
\usetikzlibrary{external}
\tikzexternalize[prefix=tikz-cache/]

% ===================== highlighting (SOUL + math-safe \hl) =====================
\usepackage{soulutf8}
\usepackage{xparse}

\ExplSyntaxOn
\tl_new:N \__l_SOUL_argument_tl
\cs_set_eq:Nc \SOUL_start:n { SOUL@start }
\cs_generate_variant:Nn \SOUL_start:n { V }
\cs_set_protected:cpn {SOUL@start} #1 {
  \tl_set:Nn \__l_SOUL_argument_tl { #1 }
  \regex_replace_all:nnN { \c{\(} (.*?) \c{\)} } { \cM\$ \1 \cM\$ } \__l_SOUL_argument_tl
  \SOUL_start:V \__l_SOUL_argument_tl
}
\ExplSyntaxOff

\let\SOULhl\hl
\renewcommand{\hl}[1]{\SOULhl{#1}}

\definecolor{HLgreen}{HTML}{77DD77}
\definecolor{HLyellow}{HTML}{FFFF66}
\definecolor{HLorange}{HTML}{FFB347}
\definecolor{HLred}{HTML}{FF6961}
\definecolor{HLpink}{HTML}{FFB6C1}
\definecolor{HLturquoise}{HTML}{40E0D0}

\RenewDocumentCommand{\hl}{O{1} m}{%
  \begingroup
  \IfEqCase{#1}{%
    {1}{\sethlcolor{HLgreen}\SOULhl{#2}}%
    {2}{\sethlcolor{HLyellow}\SOULhl{#2}}%
    {3}{\sethlcolor{HLorange}\SOULhl{#2}}%
    {4}{\sethlcolor{HLred}\SOULhl{#2}}%
    {5}{\sethlcolor{red}\SOULhl{#2}}%
    {6}{\sethlcolor{HLturquoise}\SOULhl{#2}}%
  }[\PackageError{hl}{Undefined highlight level: #1}{}]%
  \endgroup
}

% ===================== colors / emphasis =====================
\definecolor{funblue}{rgb}{0.10,0.35,0.66}
\definecolor{alizarincrimsonred}{rgb}{0.85,0.17,0.11}
\definecolor{amethyst}{rgb}{0.6,0.4,0.8}
\definecolor{mypurple}{rgb}{0.5,0,0.5}
\definecolor{highlightpurple}{rgb}{0.6,0,0.6}
\renewcommand{\emph}[1]{\textcolor{black}{\textsl{#1}}}

% ===================== theorem environments =====================
\usepackage{thmtools}

\renewcommand\thmcontinues[1]{%
  \ifcsname hyperref\endcsname
    \hyperref[#1]{continuing}%
  \else
    continuing%
  \fi
}

\newlength{\thmspace}\setlength{\thmspace}{3pt plus 1pt minus 1pt}

\declaretheoremstyle[headfont=\bfseries,bodyfont=\normalfont,spaceabove=\thmspace,spacebelow=\thmspace,qed=\ensuremath{\vartriangleleft},postheadspace=1em]{assertionstyle}
\declaretheorem[style=assertionstyle,name=Theorem,numberwithin=section]{theorem}
\declaretheorem[style=assertionstyle,name=Lemma,sibling=theorem]{lemma}
\declaretheorem[style=assertionstyle,name=Corollary,sibling=theorem]{corollary}
\declaretheorem[style=assertionstyle,name=Proposition,sibling=theorem]{proposition}
\declaretheorem[style=assertionstyle,name=Conjecture,sibling=theorem]{conjecture}
\declaretheorem[style=assertionstyle,name=Claim,sibling=theorem]{claim}
\declaretheorem[style=assertionstyle,name=Fact,sibling=theorem]{fact}

\declaretheoremstyle[headfont=\bfseries,bodyfont=\normalfont,spaceabove=\thmspace,spacebelow=\thmspace,qed=\ensuremath{\blacktriangleleft},postheadspace=1em]{definitionstyle}
\declaretheorem[style=definitionstyle,name=Definition,numberwithin=section]{definition}
\declaretheorem[style=definitionstyle,name=Problem,sibling=definition]{problem}

\declaretheoremstyle[headfont=\bfseries,bodyfont=\normalfont,spaceabove=\thmspace,spacebelow=\thmspace,qed=\ding{45},postheadspace=1em]{exercisestyle}
\declaretheorem[style=exercisestyle,name=Exercise,numberwithin=section]{exercise}
\declaretheoremstyle[headfont=\bfseries\color{red},bodyfont=\normalfont,spaceabove=\thmspace,spacebelow=\thmspace,qed=\ensuremath{\color{red}\blacktriangleleft},postheadspace=1em]{solutionstyle}
\declaretheorem[style=solutionstyle,name=Solution,numbered=no]{solution}

\declaretheoremstyle[headfont=\bfseries,bodyfont=\normalfont,spaceabove=6pt,spacebelow=6pt,qed=\ensuremath{\square},postheadspace=1em]{proofstyle}
\let\proof\relax \let\endproof\relax
\declaretheorem[style=proofstyle,name=Proof,numbered=no]{proof}

\declaretheoremstyle[headfont=\bfseries,bodyfont=\normalfont\normalsize,spaceabove=\thmspace,spacebelow=\thmspace,qed=\ensuremath{\blacktriangleleft},postheadspace=1em]{remarkstyle}
\declaretheorem[style=remarkstyle,name=Remark,numberwithin=section]{remark}

\declaretheoremstyle[headfont=\bfseries\color{funblue},bodyfont=\normalfont\normalsize,spaceabove=\thmspace,spacebelow=\thmspace,qed=\ensuremath{\color{funblue}\blacktriangleleft},postheadspace=1em]{examplestyle}
\declaretheorem[style=examplestyle,name=Example,sibling=remark]{example}

\declaretheoremstyle[headfont=\color{alizarincrimsonred}\bfseries,bodyfont=\normalfont\normalsize,spaceabove=\thmspace,spacebelow=\thmspace,qed=\ensuremath{\color{alizarincrimsonred}\blacktriangleleft},postheadspace=1em]{cautionstyle}
\declaretheorem[style=cautionstyle,name=Caution,sibling=remark]{caution}

\declaretheoremstyle[headfont=\bfseries,bodyfont=\normalfont\footnotesize,spaceabove=\thmspace,spacebelow=\thmspace,postheadspace=1em]{smallremarkstyle}
\declaretheorem[style=smallremarkstyle,name=Remark,sibling=remark]{smallremark}

\declaretheoremstyle[headfont=\bfseries\color{amethyst},bodyfont=\normalfont,spaceabove=6pt,spacebelow=6pt,qed=\ensuremath{\color{amethyst}\blacktriangleleft},postheadspace=1em]{digressionstyle}
\declaretheorem[style=digressionstyle,name=Digression,sibling=remark]{digression}

% ===================== misc helpers =====================
\newenvironment{verticalhack}{\begin{array}[b]{@{}c@{}}\displaystyle}{\\\noalign{\hrule height0pt}\end{array}}

% ===================== algorithms =====================
\usepackage{algorithm,algorithmicx}
\usepackage[italicComments=false]{algpseudocodex}
\newcommand*{\algorithmautorefname}{Algorithm}

\algnewcommand{\TO}{, \ldots ,}
\algnewcommand{\DOWNTO}{, \ldots ,}
\algnewcommand{\OR}{\vee}
\algnewcommand{\AND}{\wedge}
\algnewcommand{\NOT}{\neg}
\algnewcommand{\LEN}{\operatorname{len}}
\algnewcommand{\tru}{\ensuremath{\mathrm{\texttt{true}}}}
\algnewcommand{\fals}{\ensuremath{\mathrm{\texttt{false}}}}
\algnewcommand{\append}{\circ}
\algnewcommand{\nil}{\ensuremath{\mathrm{\textsc{nil}}}}
\algnewcommand{\red}{\ensuremath{\mathrm{\textsc{red}}}}
\algnewcommand{\black}{\ensuremath{\mathrm{\textsc{black}}}}
\algnewcommand{\gray}{\ensuremath{\mathrm{\textsc{gray}}}}
\algnewcommand{\white}{\ensuremath{\mathrm{\textsc{white}}}}

% ===================== operators / notation =====================
\DeclareMathOperator*{\argmax}{arg\,max}
\DeclareMathOperator*{\argmin}{arg\,min}
\DeclareMathOperator{\Span}{span}
\DeclareMathOperator{\Kern}{kern}
\DeclareMathOperator{\Trace}{trace}
\DeclareMathOperator{\Rank}{rank}
% \newcommand{\im}{\operatorname{Im}}
% \newcommand{\re}{\operatorname{Re}}
\newcommand{\im}{\Im}
\newcommand{\re}{\Re}

\newcommand*\matrspace{0.8mu}
\newcommand{\matr}[1]{\mspace{\matrspace}\underline{\mspace{-\matrspace}\smash[b]{\boldsymbol{#1}}\mspace{-\matrspace}}\mspace{\matrspace}}
\newcommand{\vect}[1]{{\boldsymbol{#1}}}

\newcommand{\dif}{\mathrm{d}}
% \newcommand{\eu}{\mathrm{e}}
\newcommand{\eu}{e}
% \newcommand{\iu}{\mathrm{i}}
\newcommand{\iu}{\imath}
\newcommand{\sign}{\operatorname{sgn}}

\newcommand{\oo}{\infty}

\newcommand{\R}{\mathbb{R}}
\newcommand{\N}{\mathbb{N}}
\newcommand{\Z}{\mathbb{Z}}
\newcommand{\Q}{\mathbb{Q}}
\newcommand{\C}{\mathbb{C}}
\newcommand{\K}{\mathbb{K}}
\newcommand{\F}{\mathbb{F}}

\newcommand{\Var}{\operatorname{Var}}
\newcommand{\Cov}{\operatorname{Cov}}
\newcommand{\Exp}{\operatorname{E}}
\newcommand{\Prob}{\operatorname{P}}
\newcommand{\numof}{\ensuremath{\#\,}}
\newcommand{\blackheight}{\operatorname{bh}}

\newcommand{\attrib}[2]{\ensuremath{#1\mathtt{.}\mathtt{#2}}}
\newcommand{\attribnormal}[2]{\ensuremath{#1\mathtt{.}#2}}

% ===================== metadata / hyperlinks =====================
\title{Fourier Analysis}
\author{Fabian Bosshard}
\date{\today}

\usepackage[
  linktoc=none,
  pdfauthor={Fabian Bosshard},
  pdftitle={USI - Fourier Analysis - Course Notes},
  pdfkeywords={USI, Fourier analysis, course notes, informatics},
  colorlinks=false,
  pdfborder={0 0 0},
  linkbordercolor={0 0.6 1},
  urlbordercolor={0 0.6 1},
  citebordercolor={0 0.6 1}
]{hyperref}

% ToC entries: dotted leaders + full-width clickable line
\DeclareTOCStyleEntry[linefill=\dotfill]{tocline}{section}
\DeclareTOCStyleEntry[linefill=\dotfill]{tocline}{subsection}
\DeclareTOCStyleEntry[linefill=\dotfill]{tocline}{subsubsection}

\makeatletter
\newlength\FB@toclinkht \newlength\FB@toclinkdp
\setlength\FB@toclinkht{.80\ht\strutbox}
\setlength\FB@toclinkdp{.40\dp\strutbox}

\let\FB@orig@contentsline\contentsline
\renewcommand*\contentsline[4]{%
  \begingroup
    \Hy@safe@activestrue
    \edef\Hy@tocdestname{#4}%
    \FB@orig@contentsline{#1}{%
      \Hy@raisedlink{\hyper@anchorstart{toc:\Hy@tocdestname}\hyper@anchorend}%
      \leavevmode
      \rlap{%
        \hyper@linkstart{link}{\Hy@tocdestname}%
          \raisebox{0pt}[\FB@toclinkht][\FB@toclinkdp]{%
            \hbox to \dimexpr\hsize-\parindent\relax{\hfil}%
          }%
        \hyper@linkend
      }%
      #2%
    }{#3}{#4}%
  \endgroup
}

\let\FB@orig@sectionlinesformat\sectionlinesformat
\renewcommand{\sectionlinesformat}[4]{%
  \ifx\@currentHref\@empty
    \FB@orig@sectionlinesformat{#1}{#2}{#3}{#4}%
  \else
    \leavevmode
    \hyper@linkstart{link}{toc:\@currentHref}%
      \@hangfrom{\hskip #2\relax #3}{#4}%
    \hyper@linkend
    \par
  \fi
}
\makeatother

% hyperref anchor uniqueness for section-relative numbering
\makeatletter
\renewcommand{\theHequation}{\thesection.\arabic{equation}}
\renewcommand{\theHtheorem}{\thesection.\arabic{theorem}}
\renewcommand{\theHlemma}{\theHtheorem}
\renewcommand{\theHcorollary}{\theHtheorem}
\renewcommand{\theHproposition}{\theHtheorem}
\renewcommand{\theHconjecture}{\theHtheorem}
\renewcommand{\theHclaim}{\theHtheorem}
\renewcommand{\theHfact}{\theHtheorem}
\renewcommand{\theHdefinition}{\thesection.\arabic{definition}}
\renewcommand{\theHproblem}{\theHdefinition}
\renewcommand{\theHexercise}{\thesection.\arabic{exercise}}
\renewcommand{\theHremark}{\thesection.\arabic{remark}}
\renewcommand{\theHexample}{\theHremark}
\renewcommand{\theHcaution}{\theHremark}
\renewcommand{\theHsmallremark}{\theHremark}
\renewcommand{\theHdigression}{\theHremark}
\makeatother

% ===================== license / cleveref =====================
\usepackage[type={CC},modifier={by},version={4.0}]{doclicense}
\usepackage{cleveref}

% ===================== acronyms =====================
\usepackage[acronym,nomain,toc,nonumberlist]{glossaries-extra}
\setabbreviationstyle[acronym]{long-short}
\makeglossaries
\newacronym{rbf}{RBF}{radial basis function}
\newacronym{rkhs}{RKHS}{reproducing kernel Hilbert space}

\begin{document}


\maketitle

\tableofcontents

\clearpage
\section{Preliminaries}



\paragraph{Greek letters}
\renewcommand{\arraystretch}{1.15}
\newcommand{\latinlike}[1]{\textcolor{gray}{\mathrm{#1}}}
\[
\begin{array}{@{}l@{\qquad}c@{\quad}c@{\quad}c@{}}
\textbf{Name} & \textbf{Low.} & \textbf{Var.} & \textbf{Upp.} \\
\hline
\text{alpha}   & \alpha   &              & \latinlike{A} \\
\text{beta}    & \beta    &              & \latinlike{B} \\
\text{gamma}   & \gamma   &              & \Gamma \\
\text{delta}   & \delta   &              & \Delta \\
\text{epsilon} & \epsilon & \varepsilon  & \latinlike{E} \\
\text{zeta}    & \zeta    &              & \latinlike{Z} \\
\text{eta}     & \eta     &              & \latinlike{H} \\
\text{theta}   & \theta   & \vartheta    & \Theta \\
\text{iota}    & \iota    &              & \latinlike{I} \\
\text{kappa}   & \kappa   & \varkappa    & \latinlike{K} \\
\text{lambda}  & \lambda  &              & \Lambda \\
\text{mu}      & \mu      &              & \latinlike{M} \\
\text{nu}      & \nu      &              & \latinlike{N} \\
\text{xi}      & \xi      &              & \Xi \\
\text{omicron} & {o}      &              & \latinlike{O} \\
\text{pi}      & \pi      & \varpi       & \Pi \\
\text{rho}     & \rho     & \varrho      & \latinlike{P} \\
\text{sigma}   & \sigma   & \varsigma    & \Sigma \\
\text{tau}     & \tau     &              & \latinlike{T} \\
\text{upsilon} & \upsilon &              & \Upsilon \\
\text{phi}     & \phi     & \varphi      & \Phi \\
\text{chi}     & \chi     &              & \latinlike{X} \\
\text{psi}     & \psi     &              & \Psi \\
\text{omega}   & \omega   &              & \Omega
\end{array}
\]


% aligned label + content
\newlength{\propw}
\newlength{\propsep}
\newcommand{\propitem}[2]{%
  \item \makebox[\propsep][l]{#1}%
  \parbox[t]{\dimexpr\linewidth-\propsep\relax}{#2}%
}
\settowidth{\propw}{{Distributive:}}
\setlength{\propsep}{\propw + 1.5em}

\newlength{\formw}
\settowidth{\formw}{$(a\circ b)\circ c = a\circ(b\circ c)$} % longest formula (adjust if needed)
\newlength{\formsep}
\setlength{\formsep}{\formw + 1.5em}

\paragraph{Algebraic properties}
\begin{itemize}
\propitem{{Associative:}}{%
  \makebox[\formsep][l]{$(a\circ b)\circ c = a\circ(b\circ c)$}%
  \emph{parentheses (grouping) do not matter}}
\propitem{{Commutative:}}{%
  \makebox[\formsep][l]{$a\circ b = b\circ a$}%
  \emph{order does not matter}}
\propitem{{Distributive:}}{%
    $a\cdot(b+c)=a\cdot b+a\cdot c$\\
    $(a+b)\cdot c=ac+bc$}
\end{itemize}




\paragraph{Trigonometric identities}
\begin{itemize}
    \item \(\cos( - y) = \cos(y)\)
    \item \(\sin( - y) = - \sin(y)\)
    \item \(\cos(a + b) = \cos(a) \cos(b) - \sin(a) \sin(b)\)
    \item \(\sin(a + b) = \sin(a) \cos(b) + \cos(a) \sin(b)\)
\end{itemize}


\subsection{Complex functions}
If \(f: \R \to \C\), we may write \(f(x) = \Re(f(x)) + \iu \Im(f(x))\).
We have \(|f(x)|^2 = f(x) \overline{f(x)}\), where \(\overline{f(x)} = \Re(f(x)) - \iu \Im(f(x))\).



The imaginary part of \(\log z := \log |z| + \iu \arg(z)\) is not unique (see \autoref{fig:helicoid-log}).

\begin{figure}[htbp]
\centering
\begin{tikzpicture}
\begin{axis}[
    trig format plots=rad,
    view={-50}{25},
    z buffer=sort,
    % unit vector ratio*=1 1 0.6,
    zmin=-3.0*pi,
    zmax=3.0*pi,
    xlabel={$\re(z)$},
    ylabel={$\im(z)$},
    zlabel={$\im(\log z)$},
    xtick={-3,-2,-1,0,1,2,3},
ytick={-3,-2,-1,0,1,2,3},
xticklabels={$-3$,$-2$,$-1$,$0$,$1$,$2$,$3$},
yticklabels={$-3$,$-2$,$-1$,$0$,$1$,$2$,$3$},
    % z ticks in multiples of pi/2
    ztick={-3*pi,-2*pi,-pi,0,pi,2*pi,3*pi},
    zticklabels={$-3\pi$,$-2\pi$,$-\pi$,$0$,$\pi$,$2\pi$,$3\pi$},
    ticklabel style={font=\footnotesize},
]

% \addplot3[
%     surf,
%     domain=0.001:4,
%     domain y=-3*pi:3*pi,
%     samples=25,
%     samples y=109,
%     draw=black,                 % black mesh lines
%     shader=flat,                % no gradient shading
%     colormap={grayfill}{        % constant gray fill
%       rgb255(0cm)=(220,220,220)
%       rgb255(1cm)=(220,220,220)
%     },
% ]
% ({x*cos(y)},{x*sin(y)},{y});

% lower sheets (gray): y in [-3pi,-pi]
\addplot3[
    surf,
    domain=0.001:4,
    domain y=-3*pi:-pi,
    samples=30,
    samples y=30,
    draw=black, line width=0.01pt,
    % draw=none,
    shader=flat,
    colormap={grayfill}{rgb255(0cm)=(220,220,220) rgb255(1cm)=(220,220,220)},
]
({x*cos(y)},{x*sin(y)},{y});

\addplot3[red,very thick,domain=0:4,samples=2] ({-x},{0},{-pi});

% principal branch (green): y in [-pi,pi]
\addplot3[
    surf,
    domain=0.001:4,
    domain y=-pi:pi,
    samples=30,
    samples y=30,
    draw=black, line width=0.01pt,
    % draw=none,
    shader=flat,
    colormap={greenfill}{rgb255(0cm)=(80,200,120) rgb255(1cm)=(80,200,120)},
]
({x*cos(y)},{x*sin(y)},{y});

\addplot3[red,very thick,domain=0:4,samples=2] ({-x},{0},{pi});

% upper sheets (gray): y in [pi,3pi]
\addplot3[
    surf,
    domain=0.001:4,
    domain y=pi:3*pi,
    samples=30,
    samples y=30,
    draw=black, line width=0.01pt,
    % draw=none,
    shader=flat,
    colormap={grayfill}{rgb255(0cm)=(220,220,220) rgb255(1cm)=(220,220,220)},
]
({x*cos(y)},{x*sin(y)},{y});

\end{axis}
\end{tikzpicture}
\caption{Imaginary part of \(\log(z)\) as a function of \(\re(z)\) and \(\im(z)\).
Green corresponds to the principal branch, red lines correspond to the branch cut along the negative real axis.}
\label{fig:helicoid-log}
\end{figure}


Consider \(\C^{-\ast} := \C \setminus \{z \in \R \mid z \leq 0\}\).
To obtain uniqueness, we define
\[
\begin{aligned}
\operatorname{Log} \colon &\C^{-\ast} \to \C \\
& z \mapsto \log |z| + \iu \operatorname{Arg}(z)
\end{aligned}
\]
where \(\operatorname{Arg}(z) := \arg(z) \cap (-\pi, \pi)\).
This is called the \emph{principal branch} of the logarithm.



\subsection{Lebesgue integration}

\(\mathcal{R}([a,b])\) denotes the class of Riemann integrable functions on \([a,b]\).
\begin{theorem}
Let \(f: [a,b] \to \R\) be bounded.
\begin{itemize}
\item 
If \(f \in \mathcal{R}([a,b])\), then the Lebesgue integral of \(|f|\) is finite.
Moreover, the Riemann integral of \(f\) equals the Lebesgue integral of \(f\).
\item 
\(f \in \mathcal{R}([a,b])\) if and only if \(f\) is continuous except on a set of measure zero.
\qedhere
\end{itemize}
\end{theorem}

The motivation for replacing Riemann with Lebesgue theory is that Lebesgue integrals are particularly well-suited for interchanging limits and integration,
the order of integration and derivatives and integration.

\begin{theorem}[Dominated convergence]\label{thm:dominated-convergence}
Let \((f_n)_{n \in \N}\) be a sequence of functions in \(L^1(\R)\) such that
\begin{enumerate}[label=(\roman*)]
\item there exists \(f \in L^1(\R)\) so that \(f(x) = \lim_{n \to \infty} f_n(x)\) for almost all \(x \in \R\)
\item there exists \(g \in L^1(\R)\) such that \(|f_n(x)| \leq |g(x)|\) for almost all \(x \in \R\) and all \(n \in \N\) 
\end{enumerate}
Then
\[
\begin{verticalhack}
\lim_{n \to \infty} \int_{-\infty}^{\infty} f_n(x) \dif x = \int_{-\infty}^{\infty} f(x) \dif x
\end{verticalhack}
\qedhere
\]
\end{theorem}

As mentioned before, we are also free to choose the order of integration.
According to Fubini-Tonelli, if \(f \in L^1(\R^2)\), we may calculate 
\(
\int_{-\infty}^{\infty} \int_{-\infty}^{\infty} f(x,y) \dif x \dif y
\) 
by first integrating with respect to \(x\) and then with respect to \(y\) or the other way around.

Finally, differentiation and integration can also be interchanged under certain conditions.
Indeed,
\begin{equation}\label{eq:diff-int}
\frac{\dif}{\dif t} \int_{-\infty}^{\infty} f(x,t) \dif x = \int_{-\infty}^{\infty} \frac{\partial}{\partial t} f(x,t) \dif x
\end{equation}
if certain conditions are met.
Generally speaking, if \(f(\cdot, t) \in L^1(\R)\) for all \(t\), \(\frac{\partial f}{\partial t}\) exists and
\[
\left|\frac{\partial}{\partial t} f(x,t)\right| \leq |g(x)|
\]
for some \(g \in L^1(\R)\) and every \((x, t) \in \R^2\) outside a set of measure zero, then \eqref{eq:diff-int} holds.

% \subsection{Function spaces: $C^k(\R)$, $L^p(\R)$, $C_b(\R)$, $C_c^\infty(\R)$}
\subsection{Function spaces: $C^k$, $L^p$, $C_b$, $C_c^\infty$}

Classically, functions are classified in terms of their regularity.
We write \(C(\R)\) for the space of continuous functions from \(\R\) to \(\C\).
For \(k \in \N\), we write \(C^k(\R)\) for the space of functions from \(\R\) to \(\C\) whose first \(k\) derivatives exist and are continuous.

% Recall that
% \[
% \begin{aligned}
% \text{$f$ continuous}             &\;\Leftarrow& \text{$f$ differentiable}        &\;\Leftarrow& \text{$f$ continuously differentiable} \\
% \phantom{\text{$f$ continuous}}   &\;\Leftarrow& \text{$f$ twice differentiable}  &\;\Leftarrow& \text{$f$ twice continuously differentiable} \\
% \phantom{\text{$f$ continuous}}   &\;\Leftarrow& \text{$f$ 3-times differentiable}&\;\Leftarrow& \dots \;\Leftarrow\; \text{$f$ smooth}
% \end{aligned}
% \]
% Recall that
% \[
% \begin{aligned}
% \text{$f$ continuous}           &\Leftarrow & \text{$f$ differentiable}         &\Leftarrow & \text{$f$ continuously differentiable} \\
% \phantom{\text{$f$ continuous}} &\Leftarrow & \text{$f$ twice differentiable}   &\Leftarrow & \text{$f$ twice continuously differentiable} \\
% \phantom{\text{$f$ continuous}} &\Leftarrow & \text{$f$ 3-times differentiable} &\Leftarrow & \dots \;\Leftarrow\; \text{$f$ smooth}
% \end{aligned}
% \]

\begin{example}
  The function \(f(x) = x^k \sign(x)\) is in \(C^{k-1}(\R)\) but not in \(C^k(\R)\).
\end{example}


The revolution in modern analysis is that functions are classified in terms of their integrability.
For \(1 \leq p < \infty\), we write \(f \in L^p(\R)\) if
\begin{equation}\label{eq:Lp-norm}
{\color{red}\boxed{\color{black}
\| f \|_p  := \left( \int_{-\infty}^{\infty} |f(x)|^p \dif x \right)^{1/p} 
}}
\end{equation}
is finite.
To be precise, we also need the requirement that \(f: \R \to \C\) is measurable.

\begin{remark}
There is also a definition of \(L^{\infty}(\R)\), where \(f \in L^{\infty}(\R)\) if
\begin{equation}\label{eq:Linf-norm}
{\color{red}\boxed{\color{black}
\| f \|_{\infty} := \operatorname{ess\,sup}_{x \in \R} |f(x)|
}}
\end{equation}
is finite.
The essential supremum is basically a supremum outside  a set of measure zero.
If \(f\) is in \(C(\R)\), then \(\|f\|_{\infty} = \sup_{x \in \R} |f(x)|\).
If, in addition, \(f(x) \to 0\) as \(|x| \to \infty\), i.e.\ \(f \in C_0(\R)\),
% then the supremum is attained and equals a maximum.
then \(\|f\|_{\infty} = \max_{x \in \R} |f(x)|\).
\end{remark}
\begin{remark}
Strictly speaking, \(\|\cdot\|_p\) for \(1 \le p \le \infty\), i.e. \cref{eq:Lp-norm,eq:Linf-norm}, is not a norm on functions defined pointwise, but on equivalence classes of functions that are equal almost everywhere.
\end{remark}

We write \(f \in C_b(\R)\) if \(f \in C(\R)\) and there exists \(M > 0\) such that \(|f(x)| \leq M\) for all \(x \in \R\) (i.e., \(f\) is \emph{bounded}).
Observe that \hl[2]{\(C_b(\R)  = L^{\infty}(\R) \cap C(\R)\)}.



Lebesgue \(L^p\) are not closed under product, i.e., \(f, g \in L^p(\R) \not\Rightarrow fg \in L^p(\R)\).
However, if \(f\) and \(g\) are in conjugate spaces, something can be said about the product \(fg\).

For \(1 \leq p \leq \infty\), the \emph{Lebesgue conjugate exponent} of \(p\) is the number \(p'\) defined by
\[
\frac{1}{p} + \frac{1}{p'} = 1
\]
where we set \(\frac{1}{\infty} := 0\) and \(\frac{1}{0} := \infty\).
\begin{theorem}[H\"older's inequality]\label{thm:holder}
If \(f \in L^p(\R)\) and \(g \in L^{p'}(\R)\), then \(fg \in L^1(\R)\) and
\begin{equation}\label{eq:holder}
\|fg\|_1 \leq \|f\|_p \|g\|_{p'}
\qedhere
\end{equation}
\end{theorem}
\begin{definition}\label{def:dense}
Let \((X, \|\cdot\|_X)\) be a normed space and \(Y \subseteq X\) a subspace of \(X\).
We say that \(Y\) is \emph{dense} in \(X\) if for every \(x \in X\) and every \(\varepsilon > 0\) there exists \(y \in Y\) such that \(\|x - y\|_X < \varepsilon\).
\end{definition}
\begin{example}
\(\Q\) is dense in \(\R\).
\end{example}
\begin{example}
Define
\hl[1]{\(
C_c^\infty(\R) := \{f \in C^\infty(\R) : \exists R > 0 \text{ such that } f(x) = 0 \text{ for } |x| > R\}
\)}
the space of \emph{compactly supported} smooth functions.
\(C_c^\infty(\R)\) is dense in \(L^p(\R)\) for every \(1 \leq p < \infty\).
\end{example}

Often easier to prove a statement in e.g. \(C_c^\infty(\R)\) and then extend it to e.g. \(L^1(\R)\) by invoking e.g. \autoref{thm:dominated-convergence}.



\subsection{Operator theory}
Let \(X, Y\) be vector spaces (over \(\C\)).
A linear operator \(T \colon X \to Y\) is a mapping that satisfies
\begin{enumerate}[label=(\roman*)]
\item \(T(x_1 + x_2) = T(x_1) + T(x_2)\) for all \(x_1, x_2 \in X\) \hfill (additivity)
\item \(T(\lambda x) = \lambda T(x)\) for all \(x \in X\) and \(\lambda \in \C\) \hfill (homogeneity)
\end{enumerate}
We often write \(T x\) instead of \(T(x)\), if it does not cause ambiguity.

\begin{example}
 If \(X = \C^n\) and \(Y = \C^m\), then \(T\) is linear iff there exists a matrix \(\matr{M} \in \C^{m \times n}\) such that \(T(\vect{x}) = \matr{M} \vect{x}\) for all \(\vect{x} \in \C^n\).
\end{example}
Linear operators between finite-dimensional spaces are always identified with their matrix.
They are fundamentally trivial.
For instance, they are always continuous, a property not shared by linear operators between infinite-dimensional spaces.

\begin{definition}
Let \((X, \|\cdot\|_X)\) and \((Y, \|\cdot\|_Y)\) be normed spaces.
A linear operator \(T: X \to Y\) is \emph{bounded} if there exists \(C > 0\) such that 
\begin{equation}\label{eq:bounded-operator}
\|T x\|_Y \leq C \|x\|_X
\end{equation}
for all \(x \in X\).
The infimum of the constants \(C\) for which \eqref{eq:bounded-operator} holds is called the \emph{operator norm} of \(T\).
\end{definition}

Roughly speaking, \(T\) is bounded when bounded sets in \(X\) cannot grow too much under the action of \(T\).
Observe  that if \((x_n)_{n \in \N}\) is a sequence in \(X\) converging to \(x\), then
\[
\|T x_n - T x\|_Y 
=
\|T(x_n - x)\|_Y
\leq 
C \|x_n - x\|_X \to 0
\]
as \(n \to \infty\), i.e. \(\lim_{n \to \infty} T x_n = T x\) in \(Y\), implying that \(T\) is continuous.
This is a characterization of bounded linear operators between normed spaces.


A normed space \((X, \|\cdot\|_X)\) is \emph{complete} if whenever \((x_n)_{n \in \N}\) satisfies
\[
\lim_{n,m \to \infty} \|x_n - x_m\|_X = 0
\]
there exists \(x \in X\) such that \(\lim_{n \to \infty} \|x - x_n\|_X = 0\), i.e. \(x = \lim_{n \to \infty} x_n\) in \(X\).
A complete normed vector space is called a \emph{Banach space}.
Loosely speaking, \(X\) is complete if it has no holes.
\begin{example}
\(\Q\) is not complete, \(\R\) and \(\C\) are complete.
\end{example}
\begin{example}
\((L^p(\R), \|\cdot\|_p)\) is complete for every \(1 \leq p \leq \infty\).
\end{example}

\begin{remark}
Finite-dimensional vector spaces (over a complete field) are always complete, with respect to any norm defined on them.
\end{remark}

\begin{theorem}
Let \((X, \|\cdot\|_X)\) and \((Y, \|\cdot\|_Y)\) be normed spaces, with \(Y\) complete.
Let \(Z\) be a dense subspace of \(X\) and \(T: Z \to Y\) be a linear bounded operator.
Then, there exists a unique linear bounded operator \(\tilde{T}: X \to Y\) such that \(\tilde{T} x = T x\) for every \(x \in Z\).
When there is no risk of ambiguity, we often write \(T\) instead of \(\tilde{T}\) for this \emph{extended operator}.
\end{theorem}




















\clearpage
% \section{Fourier Series on \(L^2([-\pi, \pi])\)}\label{sec:fourier-series-L2}
\section{Fourier Series}\label{sec:fourier-series-L2}

We write $f \in L^2([-\pi, \pi])$ if $f: \R \rightarrow \mathbb{C}$ is $2 \pi$-periodic and
\begin{equation}\label{eq:L2-norm}
{\color{red}
% \setlength{\fboxrule}{1pt}
\boxed{ 
\color{black}
\|f\|_2=\left(\frac{1}{2 \pi} \int_{-\pi}^\pi|f(x)|^2 \dif x\right)^{\frac{1}{2}}
}}
\end{equation}
is finite, i.e. \(\|f\|_2 < \infty\).

Functions in $L^2([-\pi, \pi])$ are called finite-energy signals, the norm $\|f\|_2$ is the energy of $f$. 
Observe that we normalized the integral in \eqref{eq:L2-norm} by means of the factor $\frac{1}{2 \pi}$. 
% In this way, we normalized the length of $[-\pi, \pi]$:
% $$
% \operatorname{length}([-\pi, \pi])=\frac{1}{2 \pi} \int_{-\pi}^\pi 1 \dif x
% =
% 1
% $$
% Loosely speaking, we change the unit of measurement so that the length of $[-\pi, \pi]$ becomes $1$.
Loosely speaking, we change the unit of measurement so that \(\operatorname{length}([-\pi, \pi]) = \frac{1}{2 \pi} \int_{-\pi}^\pi 1 \dif x = 1\).


% Differently from the other $L^p$ spaces, the space $L^2([-\pi, \pi])$ has the additional structure provided by the (sesquilinear) inner product:

% $$
% \langle f, g\rangle_{L^2}=\frac{1}{2 \pi} \int_{-\pi}^\pi f(x) \overline{g(x)} d x, \quad f, g \in L^2([-\pi, \pi]) .
% $$


Different from other \(L^p\) spaces, \(L^2([-\pi, \pi])\) has additional structure provided by the (sesquilinear) inner product
\begin{equation}\label{eq:L2-inner-product}
\langle f, g\rangle_{L^2}=\frac{1}{2 \pi} \int_{-\pi}^\pi f(x) \overline{g(x)} \dif x
\end{equation}
for \(f, g \in L^2([-\pi, \pi])\).
\eqref{eq:L2-inner-product} \emph{induces} \eqref{eq:L2-norm} since \(\left\|f\right\|_2 = \sqrt{\left\langle f, f\right\rangle_{L^2}}\).


\(L^2([-\pi, \pi])\) is not finite-dimensional, i.e. it does not admit a finite basis.
However, it does have a countable orthonormal (Schauder) basis, i.e. there exists an infinite sequence of functions
\(
(e_n)_{n \in \Z} \subseteq L^2([-\pi, \pi])
\) 
such that
\(
\langle e_n, e_m\rangle_{L^2} = \delta_{n,m}
\)
and for every \(f \in L^2([-\pi, \pi])\) there exist unique \(a_n \in \C\) such that
\begin{equation}\label{eq:fourier-series-convergence}
f \stackrel{L^2}{=} \sum_{n} a_n e_n
\end{equation}
where the symbol \(\stackrel{L^2}{=}\) means that 
\begin{equation}\label{eq:L2-convergence}
\lim_{N \to \infty} \left\|f - \sum_{n=-N}^N a_n e_n\right\|_2 = 0
\end{equation}
i.e.
the series converges in the \(L^2\) norm. % i.e. \(\lim_{N \to \infty} \|f - \sum_{n=-N}^N a_n e_n\|_2 = 0\).
We can compute the coefficients explicitly since
\[
\langle f, e_m\rangle_{L^2} 
=
\left\langle \sum_{n} a_n e_n, e_m\right\rangle_{L^2}
=
\sum_{n} a_n \langle e_n, e_m\rangle_{L^2}
=
a_m
\]
where we used the component-wise continuity of \(\langle\cdot, \cdot\rangle_{L^2}\) to intertwine series and inner products.


The prototype of such an orthonormal basis are the functions \(e_n(x) = \eu^{\iu n x}\) with \(n \in \Z\):
For every $f \in L^2([-\pi, \pi])$,
\begin{equation}\label{eq:fourier-series}
  f(x) \stackrel{L^2}{=} \sum_{n=-\infty}^{\infty} \hat{f}(n) \eu^{\iu n x}
\end{equation}
where, for \(n \in \Z\),
\begin{equation}\label{eq:fourier-coefficient}
{\color{red}\boxed{\color{black}
\hat{f}(n)
=
\left\langle f, e_n\right\rangle_{L^2}
=
\frac{1}{2 \pi} \int_{-\pi}^\pi f(x) \eu^{-\iu n x} \dif x
}}
\end{equation}
is the $n$-th Fourier coefficient of $f$. 
\eqref{eq:fourier-series} is the Fourier series of $f$.


\begin{theorem}\label{thm:parseval}
For every \(f \in L^2([-\pi,\pi])\), we have 
\begin{equation}\label{eq:parseval}
\|f\|_2^2=\sum_{n=-\infty}^{\infty}|\hat{f}(n)|^2
\end{equation}
which is known as \hl{Parseval's theorem}.
\end{theorem}
\begin{proof}
By the properties of the inner product,
\[
\begin{aligned}
\|f\|_2^2
&=
\langle f, f\rangle_{L^2}
=
\left\langle \sum_{n} \hat{f}(n) e_n, \sum_{m} \hat{f}(m) e_m\right\rangle_{L^2} \\
&=
\sum_{n} \sum_{m} \hat{f}(n) \overline{\hat{f}(m)} \underbrace{\langle e_n, e_m\rangle_{L^2}}_{\delta_{n,m}}
=
\sum_{n} |\hat{f}(n)|^2
\end{aligned}
\]
where we again used the component-wise continuity of \(\langle\cdot, \cdot\rangle_{L^2}\).
\end{proof}



\begin{corollary}\label{cor:fourier-zeros}
If \(f \in L^2([-\pi, \pi])\) has \(\hat{f}(n) = 0\) for all \(n \in \Z\), then \(f = 0\) in \(L^2([-\pi, \pi])\).
\end{corollary}

\begin{proof}
\(\hat{f}(n) = 0 \quad \forall n \in \Z 
\quad \overset{\text{\eqref{eq:parseval}}}{\Rightarrow}\quad 
 \|f\|_2^2 = 0
% \quad \Rightarrow\quad 
%  \|f\|_2 = 0
\quad \Rightarrow\quad 
 f = 0_{L^2([-\pi, \pi])}\)
\end{proof}

\begin{example}
The \(C^\infty\) function \(\eu^{-{1}/{x^2}}\) is a prototypical example of a non-zero function with all vanishing Taylor coefficients at \(0\).
\autoref{cor:fourier-zeros} shows that these pathologies do not occur for Fourier series.
\end{example}


\begin{corollary}[Riemann-Lebesgue lemma]\label{cor:riemann-lebesgue}
We have 
\begin{equation}\label{eq:riemann-lebesgue}
\lim_{|n| \to \infty} \hat{f}(n) = 0
\end{equation}
for every \(f \in L^2([-\pi, \pi])\).
\end{corollary}
\begin{proof}
\(f \in L^2([-\pi, \pi]) 
\quad \Rightarrow\quad 
%  \|f\|_2^2 < \infty 
% \quad \Rightarrow\quad 
\|f\|_2^2 \overset{\text{\eqref{eq:parseval}}}{=}
 \sum_{n} |\hat{f}(n)|^2 < \infty 
\quad \Rightarrow\quad 
 \lim_{|n| \to \infty} |\hat{f}(n)|^2 = 0\)
\end{proof}


Lebesgue spaces on finite measure spaces satisfy the inclusion relations
\begin{equation}\label{eq:Lp-inclusion}
L^q \subseteq L^p
\end{equation} 
for \(1 \leq p < q \leq \infty\).
Thus, \(L^1([-\pi, \pi])\) is the largest of the \(L^p\) spaces on \([-\pi, \pi]\).
Furthermore, the Fourier coefficients are well-defined for functions in \(L^1([-\pi, \pi])\), since
\[
|\hat{f}(n)| 
=
\left|\frac{1}{2 \pi} \int_{-\pi}^\pi f(x) \eu^{-\iu n x} \dif x\right|
\leq
\frac{1}{2 \pi} \int_{-\pi}^\pi |f(x)| \dif x
=
\|f\|_1
\]
exists and is finite for every \(f \in L^1([-\pi, \pi])\) and \(n \in \Z\).
Proving results such as Corollaries \ref{cor:fourier-zeros} and \ref{cor:riemann-lebesgue} is considerably simplified in the \(L^2\) setting thanks to the inner product.
Nevertheless, together with most of the results from this section, they also hold for \(L^1([-\pi, \pi])\).
In fact, except for \eqref{eq:fourier-series-convergence} and \eqref{eq:parseval}, all statements from this section are already true in \(L^1([-\pi, \pi])\), but need to be proved by different methods.


























\clearpage

% \section{Fourier Transform on \(L^1(\R)\)}
\section{Fourier Transform on $L^1$}



For $f \in L^1(\R)$ the Fourier transform
\begin{equation}\label{eq:fourier-transform}
{\color{red}\boxed{\color{black}
\hat{f}(\xi)=\int_{-\infty}^{\infty} f(x) \eu^{-2 \pi \iu \xi x} \dif x
}}
\end{equation}
converges for every $\xi \in \R$. 
Moreover,
\begin{equation}\label{eq:fourier-transform-bound}
|\hat{f}(\xi)| 
=
\left|\int_{-\infty}^{\infty} f(x) \eu^{-2\pi\iu\xi x} \dif x\right|
\leq 
\int_{-\infty}^{\infty}|f(x)| \underbrace{|\eu^{-2\pi\iu\xi x}|}_{=1}
\dif x
=
\|f\|_1
\end{equation}
for any \(\xi \in \R\),
\hl[2]{implying that $\hat{f}$ is in $L^{\infty}(\R)$} and the operator 
\[
\mathcal{F}: f \in L^1(\R) \mapsto \hat{f} \in L^{\infty}(\R)
\]
is bounded with
\begin{equation}\label{eq:fourier-transform-norm-bound}
\|\hat{f}\|_{\infty} \leq \|f\|_1
\end{equation}
% \(
% \|\hat{f}\|_{\infty} \leq\|f\|_1 
% \).

\begin{definition}
The operator $\mathcal{F}$ defined above is called Fourier transform, and the function $\hat{f}$ is the Fourier transform of $f$.
\end{definition}

\begin{definition}[Pointwise continuity]
Let \(f:\R\to\C\) and \(x_0\in\R\).  
We say \(f\) is continuous at \(x_0\) if for every \(\varepsilon>0\) there exists \(\delta>0\) such that
\[
|x-x_0|<\delta \Longrightarrow |f(x)-f(x_0)|<\varepsilon
\qedhere
\]
\end{definition}
\begin{definition}[Uniform continuity]
Let \(f:\R\to\C\).  
We say \(f\) is uniformly continuous on \(\R\) if for every \(\varepsilon>0\) there exists \(\delta>0\) such that for all \(x,y\in\R\)
\[
|x-y|<\delta \Longrightarrow |f(x)-f(y)|<\varepsilon
\qedhere
\]
\end{definition}

\begin{remark}
Uniform continuity is stronger than pointwise continuity and implies it.
\end{remark}


\begin{fact}\label{fact:fourier-transform-uniform-continuity} 
The Fourier transform is a bounded operator from $L^1(\R)$ to $L^{\infty}(\R)$. Moreover, $\hat{f}$ is a uniformly continuous function on $\R$ for every $f \in L^1(\R)$.
\end{fact}


% \begin{proof}
% Boundedness follows from \eqref{eq:fourier-transform-bound}.
% Fix \(\varepsilon > 0\).
% Choose \(R>0\) such that
% \[
% \int_{|x|>R} |f(x)|\,\dif x < \varepsilon
% \]
% i.e. the tail of \(f\) is smaller than \(\varepsilon\).
% For \(\xi,\eta\in\R\)
% \[
% |\hat f(\xi)-\hat f(\eta)|
% \le
% \int_{|x|\le R} |f(x)|\,\bigl|\eu^{-2\pi\iu(\xi-\eta)x}-1\bigr|\,\dif x
% +
% \int_{|x|>R} |f(x)|\,\bigl|\eu^{-2\pi\iu(\xi-\eta)x}-1\bigr|\,\dif x
% \]

% Using \(|\eu^{\iu t}-1|\le |t|\) and \(|\eu^{\iu t}-1|\le 2\) we estimate the two terms separately.

% For \(|x|\le R\) we have
% \[
% \bigl|\eu^{-2\pi\iu(\xi-\eta)x}-1\bigr|
% \le
% 2\pi|\xi-\eta||x|
% \le
% 2\pi R|\xi-\eta|
% \]
% hence
% \[
% \int_{|x|\le R} |f(x)|\,\bigl|\eu^{-2\pi\iu(\xi-\eta)x}-1\bigr|\,\dif x
% \le
% 2\pi R|\xi-\eta|\int_{|x|\le R}|f(x)|\,\dif x
% \le
% 2\pi R\|f\|_1|\xi-\eta|
% \]

% For \(|x|>R\) we use \(\bigl|\eu^{-2\pi\iu(\xi-\eta)x}-1\bigr|\le 2\) to get
% \[
% \int_{|x|>R} |f(x)|\,\bigl|\eu^{-2\pi\iu(\xi-\eta)x}-1\bigr|\,\dif x
% \le
% 2\int_{|x|>R}|f(x)|\,\dif x
% <
% 2\varepsilon
% \]

% Combining the two estimates yields
% \[
% |\hat f(\xi)-\hat f(\eta)|
% \le
% 2\pi R\|f\|_1|\xi-\eta| + 2\varepsilon
% \]

% Set
% \[
% \delta=\frac{\varepsilon}{3\,2\pi R\|f\|_1}
% \]

% Then \(|\xi-\eta|<\delta\) implies
% \[
% |\hat f(\xi)-\hat f(\eta)|<\varepsilon
% \]

% This proves uniform continuity of \(\hat f\)
% \end{proof}

\begin{proof}
Boundedness follows from \eqref{eq:fourier-transform-bound}.
For uniform continuity, consider 
\(
|\hat{f}(\xi + h) - \hat{f}(\xi)|
\),
apply \autoref{thm:dominated-convergence} and observe that the obtained expression goes to zero as \(h \to 0\) and does not depend on \(\xi\).
\end{proof}



% \begin{proof}
% Boundedness follows from \eqref{eq:fourier-transform-bound}.
% For uniform continuity, let \(h\in\R\) and \(\xi\in\R\). Then
% \[
% \hat f(\xi+h)-\hat f(\xi)
% =
% \int_{-\infty}^{\infty} f(x)\,\eu^{-2\pi\iu\xi x}\Bigl(\eu^{-2\pi\iu h x}-1\Bigr)\,\dif x
% \]
% hence
% \[
% |\hat f(\xi+h)-\hat f(\xi)|
% \le
% \int_{-\infty}^{\infty} |f(x)|\,\bigl|\eu^{-2\pi\iu h x}-1\bigr|\,\dif x
% \]
% For each \(x\), \(\bigl|\eu^{-2\pi\iu h x}-1\bigr|\to 0\) as \(h\to 0\), and
% \[
% |f(x)|\,\bigl|\eu^{-2\pi\iu h x}-1\bigr|
% \le
% 2|f(x)|
% \]
% By \autoref{thm:dominated-convergence},
% \[
% \int_{-\infty}^{\infty} |f(x)|\,\bigl|\eu^{-2\pi\iu h x}-1\bigr|\,\dif x \to 0
% \]
% as \(h\to 0\). The right-hand side depends only on \(h\), so \(\hat f\) is uniformly continuous
% \end{proof}




% So for \( \xi \neq 0\),
% \[
% \hat{f}(\xi) = \frac{1}{2 \pi \xi} 2 \sin( \pi \xi A) = \frac{\sin( \pi \xi A)}{\pi \xi}
% \]
% In conclusion
% \[
% \hat{f}(\xi) =
% \begin{cases}
%     \frac{\sin( \pi \xi A)}{\pi \xi}, & \xi \neq 0, \\
%     A, & \xi = 0.
% \end{cases}
% \]

% Oberserve that \(\hat{f}\) is indeed in \(\mathcal{C}_b(\R)\) 

% \(\frac{\sin( \pi \xi A)}{\pi \xi}\) is a continuous at every \(\xi \neq 0\) and
% \[
% \lim_{\xi \to 0} \frac{\sin( \pi \xi A)}{\pi \xi} = A = \hat{f}(0)
% \]
% by definition.
\begin{example}[Characteristic function]
Let 
\begin{equation}\label{eq:characteristic-function}
f(x) = \chi_{[-A/2, A/2]}(x):=
\begin{cases}
1, & x \in [-A/2, A/2] \\
0, & \text{otherwise}
\end{cases}
\end{equation}
be the characteristic function of the interval \([-A/2, A/2]\), with \(A > 0\).
Then for every \(\xi \ne 0\),
\[
\hat{f}(\xi) 
=
\int_{-A/2}^{A/2} \eu^{-2\pi\iu\xi x} \dif x
=
\frac{\sin( \pi A \xi)}{\pi \xi}
\]
and \(\hat{f}(0) 
=
% \int_{-\infty}^{\infty} f(x) \dif x
% =
\int_{-A/2}^{A/2} 1 \dif x
= A\).
\[
\begin{tikzpicture}[scale=1.2, font=\footnotesize]
  \tikzset{>=latex}
  \colorlet{myblue}{green!80!black}
  \colorlet{mydarkblue}{myblue!80!black}
  \tikzstyle{xline}=[myblue,thick]
  \def\tick#1#2{\draw[thick] (#1) ++ (#2:0.1) --++ (#2-180:0.2)}

  % -------- parameters (feel free to tweak axes) ----------
  \def\xmax{4.5}
  \def\ymin{-0.4}
  \def\ymax{1.4}
  \def\Aparam{1.0} % this is the "A" in sin(pi*A*xi)/(pi*xi)
  % --------------------------------------------------------

  \draw[->,thick] (0,\ymin) -- (0,\ymax);% node[left] {$\hat f(\xi)$};
  \draw[->,thick] (-\xmax,0) -- (\xmax+0.1,0) node[below=1,right=0.05] {$\xi$};

  % plot: sin(pi*A*xi)/(pi*xi), trig in pgf is degrees -> use deg(...)
  \draw[xline,samples=200,smooth,variable=\t,domain=-0.94*\xmax:0.94*\xmax]  plot(\t,{ (abs(\t)<0.01) ? (\Aparam) : (sin(deg(pi*\Aparam*\t))/(pi*\t)) });

  % zeros at xi = k/A
  \tick{-4/\Aparam,0}{90} node[below=0,scale=1] {$-\frac{4}{A}$};
  \tick{-3/\Aparam,0}{90} node[below=0,scale=1] {$-\frac{3}{A}$};
  \tick{-2/\Aparam,0}{90} node[below=0,scale=1] {$-\frac{2}{A}$};
  \tick{-1/\Aparam,0}{90} node[below=0,scale=1] {$-\frac{1}{A}$};
  \tick{ 1/\Aparam,0}{90} node[below=0,scale=1] {$ \frac{1}{A}$};
  \tick{ 2/\Aparam,0}{90} node[below=0,scale=1] {$ \frac{2}{A}$};
  \tick{ 3/\Aparam,0}{90} node[below=0,scale=1] {$ \frac{3}{A}$};
  \tick{ 4/\Aparam,0}{90} node[below=0,scale=1] {$ \frac{4}{A}$};

  % peak value
  \tick{0,\Aparam}{0} node[left=0.05,scale=0.9] {$A$};

  \node[mydarkblue,right,scale=0.95] at (0.15*\xmax,0.85*\Aparam) {$\displaystyle \hat{f}(\xi)=\frac{\sin(\pi A \xi)}{\pi \xi}$};
\end{tikzpicture}
\]
Thus we conclude that \(\hat{f}(\xi) = \frac{\sin( \pi A \xi)}{\pi \xi}\), where we implicitly consider its continuous continuation.
In particular, for \(A = 1\), we have
\begin{equation}\label{eq:sinc-function}
\hat{f}(\xi) = \frac{\sin( \pi \xi)}{\pi \xi}
=:
\operatorname{sinc}(\xi)
\end{equation}
which is called \emph{cardinal sine} or \emph{sinc function}.
\end{example}


\begin{example}[Gaussian]\label{ex:gaussian-fourier-transform}
The Fourier transform of \(f(x) = \eu^{-\pi x^2}\) is \(\hat{f}(\xi) = \eu^{-\pi \xi^2}\).
\end{example}
% \begin{example}[Gaussian]
% Let \(\varphi(x)=\eu^{-\pi x^2}\). For \(\xi\in\R\),
% \[
% \hat\varphi(\xi)=\int_{-\infty}^{\infty}\eu^{-\pi x^2}\,\eu^{-2\pi\iu\xi x}\,\dif x
% \]
% Since \(x\varphi(x)\in L^1(\R)\), we may differentiate under the integral sign and obtain
% \[
% \hat\varphi'(\xi)
% =
% -2\pi\iu\int_{-\infty}^{\infty} x\,\eu^{-\pi x^2}\,\eu^{-2\pi\iu\xi x}\,\dif x
% \]
% Using \(\varphi'(x)=-2\pi x\varphi(x)\), we get \(x\varphi(x)=-(2\pi)^{-1}\varphi'(x)\) and hence
% \[
% \hat\varphi'(\xi)
% =
% \iu\int_{-\infty}^{\infty} \varphi'(x)\,\eu^{-2\pi\iu\xi x}\,\dif x
% \]
% Integration by parts yields
% \[
% \hat\varphi'(\xi)
% =
% \iu\Bigl(\bigl[\varphi(x)\,\eu^{-2\pi\iu\xi x}\bigr]_{-\infty}^{\infty}
% -\int_{-\infty}^{\infty} \varphi(x)\,(-2\pi\iu\xi)\,\eu^{-2\pi\iu\xi x}\,\dif x\Bigr)
% =
% -2\pi\xi\,\hat\varphi(\xi)
% \]
% Thus \(\hat\varphi\) satisfies the ODE
% \[
% \hat\varphi'(\xi)=-2\pi\xi\,\hat\varphi(\xi)
% \]
% so
% \[
% \hat\varphi(\xi)=C\,\eu^{-\pi\xi^2}
% \]
% Evaluating at \(\xi=0\) gives
% \[
% C=\hat\varphi(0)=\int_{-\infty}^{\infty}\eu^{-\pi x^2}\,\dif x=1
% \]
% Therefore
% \[
% \hat\varphi(\xi)=\eu^{-\pi\xi^2}
% \qedhere
% \]
% \end{example}



\begin{example}\label{ex:decaying-exponential-fourier-transform}
The Fourier transform of \(f(x) = \eu^{-2 |x|}\) is \(\hat{f}(\xi) = \frac{1}{1 + \pi^2 \xi^2}\).
\end{example}


% --- operators (as in the notes) ---
\subsection{Important operators in harmonic analysis}
\begin{definition}\label{def:harmonic-operators}
\begin{align}
% {\color{red}\boxed{\color{black}
&\qquad& &\text{\textcolor{red}{translation}:}& T_{x_0} f(x) &= f(x-x_0),                 & x_0 &\in \R \quad &\quad&\\
&\qquad& &\text{\textcolor{red}{modulation}:}& M_{\xi_0} f(x) &= \eu^{2\pi \iu \xi_0 x} f(x), & \xi_0 &\in \R \quad &\quad&\\
&\qquad& &\text{\textcolor{red}{upper dilation}:}& f^{\lambda}(x) &= f(\lambda x),     & \lambda &> 0 \quad &\quad&\\
&\qquad& &\text{\textcolor{red}{lower dilation}:}& f_{\lambda}(x) &= \frac{1}{\lambda} f(x/\lambda), & \lambda &> 0 \quad &\quad&%%
\end{align}
% }}
\end{definition}


% \begin{theorem}[Duality]\label{thm:duality-dilations}
\subsubsection{Duality}
Let \(f \in L^1(\R)\) and \(\lambda > 0\).

\medskip

Translation/Modulation:
% {\color{red}\boxed{\color{black}
\begin{align}
\widehat{(T_{x_0} f)}(\xi)
&= M_{-x_0}\hat f(\xi)
\label{eq:duality-translation}\\
% = \eu^{-2\pi \iu x_0 \xi}\hat f(\xi)\\[0.4em]
\widehat{(M_{\xi_0} f)}(\xi)
&= T_{\xi_0}\hat f(\xi)
\label{eq:duality-modulation}
% = \hat f(\xi-\xi_0)
\end{align}
% }}


\medskip

Dilation:
% {\color{red}\boxed{\color{black}
\begin{align}
\widehat{(f^{\lambda})}(\xi)
&= (\hat f)_{\lambda}(\xi)
\label{eq:duality-dilation-upper}\\
% = \frac{1}{\lambda}\hat f(\xi/\lambda)\\[0.4em]
\widehat{(f_{\lambda})}(\xi)
&= (\hat f)^{\lambda}(\xi)
\label{eq:duality-dilation-lower}
% = \hat f(\lambda \xi)
\end{align}
% }}

% \end{itemize}
% \end{theorem}

\eqref{eq:duality-dilation-upper} and \eqref{eq:duality-dilation-lower} are a prelude to the \emph{uncertainty principle}, which states that a function and its Fourier transform cannot both be arbitrarily well localized.
Compressing a function in the time domain causes its Fourier transform to expand in the frequency domain and vice versa.



\subsection{Convolution}
For \(f, g \colon \R \to \C\), we define the convolution
\begin{equation}\label{eq:convolution}\
  {\color{red}\boxed{\color{black}
  (f * g)(x) := \int_{-\infty}^{\infty} f(y) g(x - y) \dif y
  }}
\end{equation}

If \(f, g \in L^1(\R)\), then \(\|f * g\|_1 \leq \|f\|_1 \|g\|_1\), so \(f * g \in L^1(\R)\).
Furthermore, the convolution is commutative, associative and distributive.


\begin{theorem}
If \(f \in L^1(\R)\), \(g \in L^p(\R)\) for \(1 \leq p \leq \infty\), then 
\begin{equation}\label{eq:young}
\|f * g\|_p \leq \|f\|_1 \|g\|_p
\end{equation}
which is known as \hl{Young's inequality}.
\end{theorem}




\begin{theorem}
The Fourier transform turns convolution into pointwise multiplication
\begin{equation}\label{eq:convolution-theorem}
% {\color{red}\boxed{\color{black}
\widehat{(f * g)}(\xi) = \hat{f}(\xi) \cdot \hat{g}(\xi)
% }}
\end{equation}
for every \(f, g \in L^1(\R)\) and \(\xi \in \R\).
\end{theorem}
\begin{proof}
Plug \eqref{eq:convolution} into \eqref{eq:fourier-transform}, use Fubini, substitute \(x - y = z\) and \(x = z + y\).
\end{proof}














\subsection{Differentiation}

Define 
\begin{equation}
C_0(\R) := \left\{f \in C(\R) \colon \lim_{|x| \to \infty} f(x) = 0\right\}
\end{equation}
i.e. \(f \in C_0(\R)\) if \(f\) is continuous and decays at infinity.


It is widely known that the normed space \((C_0(\R), \|\cdot\|_\infty)\) is complete, i.e. a Banach space.





\begin{theorem}\label{thm:differentiation-fourier}
Let \(f \in L^1(\R)\).
\begin{enumerate}[label=(\roman*)]
\item If \(xf \in L^1(\R)\), then \(\hat{f} \in C^1(\R)\) and 
\begin{equation}\label{eq:derivative-of-fourier}
\hat{f}'(\xi) = \widehat{(-2\pi \iu xf)}(\xi)
\end{equation}
\item If \(f \in C^1(\R) \cap C_0(\R)\) and \(f' \in L^1(\R)\), then 
\begin{equation}\label{eq:fourier-of-derivative}
% {\color{red}\boxed{\color{black}
\widehat{(f')}(\xi) = 2\pi \iu \xi \hat{f}(\xi)
% }}
\end{equation} 
\qedhere
\end{enumerate}
\end{theorem}



\begin{figure}[hb]
\center
\begin{tikzpicture}[scale=2,>=Latex, font=\footnotesize, every node/.style={font=\footnotesize}]

  % angle (degrees)
  \def\th{35}

  \coordinate (O) at (0,0);
  \coordinate (A) at (1,0);
  \coordinate (B) at ({cos(\th)},{sin(\th)});

  % --- styles (single source of truth, used also in legend) ---
  \tikzset{
    ax/.style   ={->, line width=0.35pt},
    circ/.style ={line width=0.5pt},
    rad/.style  ={line width=0.5pt, blue!60!black},
    chord/.style={line width=0.7pt,  green!60!black},
    arcl/.style ={line width=0.7pt,  orange},
  }

  % --- angle sector FIRST (so it sits behind lines) ---
  \fill[blue!25] (O) -- (0.38,0) arc (0:\th:0.38) -- cycle;
  \node at ({0.27*cos(\th/2)},{0.27*sin(\th/2)}) {$\vartheta$};

  % axes (a bit longer)
  \draw[ax] (-1.4,0) -- (1.5,0);
  \draw[ax] (0,-1.4) -- (0,1.4);

  % unit circle
  \draw[circ] (O) circle (1);

  % radius to e^{i\vartheta}
  \draw[rad] (O) -- (B);

  % green chord |e^{i\vartheta}-1|
  \draw[chord] (A) -- (B);

  % red arc length = \vartheta
  \draw[arcl] (A) arc (0:\th:1);

  % labels (move "1" off the circle a bit)
  \node[below left] at (O) {$0$};
  \node[below right] at (A) {$1$};
  \node[above right] at (B) {$\eu^{\iu \vartheta}$};

  % --- compact legend, short samples, matching line widths ---
  \node[
    draw,
    line width=0.35pt,
    inner sep=4pt,
    anchor=north west
  ] at (1.5,1.4) {%
    \begin{tabular}{@{}l@{\;}l@{}}
      \tikz{\draw[chord] (0,0)--(0.2,0);} & $|\eu^{\iu \vartheta}-1|$\\[2pt]
      \tikz{\draw[arcl]  (0,0)--(0.2,0);} & $\vartheta$
    \end{tabular}
  };

\end{tikzpicture}
\caption{Inequality \(|\eu^{\iu \vartheta}-1| \leq \vartheta\) for \(\vartheta > 0\).}
\label{fig:complex-inequality}
\end{figure}

% \begin{proof}
% \begin{enumerate}[label=(\roman*)]
% \item 
% % write the proof sketch in prose, referencing all rerqured ingredients 9aobve figure, dom conv., uniform continuousness of fourier transf
% \end{enumerate}
% \end{proof}

\begin{proof}
\begin{enumerate}[label=(\roman*)]
\item
Consider 
\smash{\(\frac{\hat f(\xi+h)-\hat f(\xi)}{h}\)},
factor out the common exponential, so that the dependence on \(h\) appears only through 
\smash{\(\frac{\eu^{-2\pi \iu h x}-1}{h}\)}.
Using \autoref{fig:complex-inequality} with \(\vartheta=-2\pi h x\), this quotient is dominated by a constant multiple of \(|x f(x)|\).
Since \(x f\in L^1(\R)\) by assumption, this provides an \(L^1\)-majorant, so we apply \autoref{thm:dominated-convergence}.
This proves that \(\hat f\) is differentiable and that the derivative is given by \eqref{eq:derivative-of-fourier}.
Moreover, \(\hat f'\) is a Fourier transform of an \(L^1\)-function, hence is (uniformly) continuous by \autoref{fact:fourier-transform-uniform-continuity}, so \(\hat f\in C^1(\R)\).

\item
Integrate by parts, using the fact that \(f\) vanishes at infinity.
\qedhere
\end{enumerate}
\end{proof}


By iterating \eqref{eq:derivative-of-fourier}, we obtain
\begin{theorem}\label{thm:higher-derivatives}
Let \(f \in L^1(\R) \cap C^k(\R)\), \(k \geq 1\). 
If
\begin{enumerate}[label=(\roman*)]
\item \(f^{(j)} \in L^1(\R)\) for every \(j \leq k\)
\item \(f^{(j)} \in C_0(\R)\) for every \(j \leq k-1\)
\end{enumerate}
then 
% \(
\begin{equation}\label{eq:higher-derivatives}
  {\color{red}\boxed{\color{black}
  \widehat{(f^{(j)})}(\xi) = (2\pi \iu \xi)^j \hat{f}(\xi)
  }}
\end{equation}
% \) 
for every \(j \leq k\) and \(\xi \in \R\).
\end{theorem}


As a corollary, if \(f \in C_c^\infty(\R)\), then \(\hat{f} \in C^\infty(\R)\) and \eqref{eq:higher-derivatives} holds for every \(j \in \N\).
In particular, if \(f \in C_c^\infty(\R)\), by choosing \(j = 1\), we get
\begin{equation}\label{eq:decay-of-fourier}
|\hat{f}(\xi)| 
\leq 
\frac{|\widehat{(f')}(\xi)|}{2\pi |\xi|}
\overset{\text{\eqref{eq:fourier-transform-norm-bound}}}{\leq}
\frac{\|f'\|_1}{|\xi|}
\to
0
\end{equation}
as \(|\xi| \to \infty\).
If \(f \in L^1(\R)\), by the density of \(C_c^\infty(\R)\) in \(L^1(\R)\), 
there exists a sequence of functions \((f_i)_{i \in \N}\), \(f_i \in C_c^\infty(\R)\) such that \(\lim_{j \to \infty} \|f - f_j\|_1 = 0\).
Then,
\[
\|\hat{f} - \hat{f_j}\|_\infty \overset{\text{\eqref{eq:fourier-transform-norm-bound}}}{\leq} \|f - f_j\|_1 \to 0
\]
as \(j \to \infty\).
Thus, \(\hat{f} \in C_0(\R)\), since every \(\hat{f_j} \in C_0(\R)\) and \(C_0(\R)\) is complete with respect to \(\|\cdot\|_\infty\).
In essence:
\begin{theorem}[Riemann-Lebesgue lemma]\label{thm:riemann-lebesgue-lemma-FT}
If \(f \in L^1(\R)\), then 
% \(
\begin{equation}\label{eq:riemann-lebesgue-FT}
{\color{red}\boxed{\color{black}
\hat{f}(\xi) \in C_0(\R)
}}
\end{equation}
% \), 
i.e., \(\hat{f}\) is continuous and decays at infinity.
\end{theorem}







\subsection{Inverse}

\begin{lemma}\label{lem:fourier-transform-switch}
If \(f,g \in L^1(\R)\), then \(\int_{-\infty}^{\infty} \hat{f}(y) g(y) \dif y = \int_{-\infty}^{\infty} f(x) \hat{g}(x) \dif x\).
\end{lemma}
\begin{proof}
Plug in \eqref{eq:fourier-transform} and use Fubini.
\end{proof}


\begin{lemma}\label{lem:L1-convergence-pointwise-convergence-implies-almost-everywhere-equality}
If \(u_n \to f\) in \(L^1(\R)\) and \(u_n(x) \to g(x)\) for almost every \(x \in \R\), then we have \(f = g\) almost everywhere.
\end{lemma}
\begin{remark}
Even if \(u_n(x) \to g(x)\) for all \(x \in \R\), we still only get \(f = g\) almost everywhere, not necessarily everywhere.
\end{remark}

\begin{theorem}\label{thm:approximate-identity}
Let \(f,K \in L^1(\R)\) and \(\int_{-\infty}^{\infty} K(y) \dif y = 1\).
Then
\begin{equation}\label{eq:approximate-identity}
\left\|f - K_\lambda * f\right\|_1 
% =
% \left\|f - \int_{-\infty}^{\infty} \frac{1}{\lambda} K({y}/{\lambda}) f(\cdot - y) \dif y\right\|_1
\to 0
\end{equation}
as \(\lambda \to 0\).
\end{theorem}
\autoref{thm:approximate-identity} states that \(K_\lambda \to \delta\) as \(\lambda \to 0\) in the sense of distributions, where
\(\delta \colon C(\R) \to \C\) 
is the operator \(\delta(g) = g(0)\), \(g \in C(\R)\).


% \begin{proof}
% Let $T_y f(x):=f(x-y)$. Since $\int_{\R}K=1$,
% \[
% (K_\lambda*f-f)(x)=\int_{\R}K_\lambda(y)\,\bigl(T_y f(x)-f(x)\bigr)\,\dif y .
% \]
% Hence, by Minkowski,
% \[
% \|K_\lambda*f-f\|_1
% \le \int_{\R}|K_\lambda(y)|\,\|T_y f-f\|_1\,\dif y. \tag{$\ast$}
% \]
% Fix $\varepsilon>0$. Using density, pick $g\in C_c^\infty(\R)$ with $\|f-g\|_1<\varepsilon$.
% Then for all $y$,
% \[
% \|T_y f-f\|_1 \le \|T_y(f-g)\|_1+\|T_y g-g\|_1+\|g-f\|_1
% \le 2\varepsilon+\|T_y g-g\|_1.
% \]
% Insert into $(\ast)$ and use $\|K_\lambda\|_1=\|K\|_1$:
% \[
% \|K_\lambda*f-f\|_1
% \le 2\varepsilon\|K\|_1 + \int_{\R}|K_\lambda(y)|\,\|T_y g-g\|_1\,\dif y.
% \]
% Now $g$ is uniformly continuous and compactly supported, hence $\|T_y g-g\|_1\to 0$ as $y\to 0$ and
% $\|T_y g-g\|_1\le 2\|g\|_1$. Moreover,
% \[
% \int_{\R}|K_\lambda(y)|\,\|T_y g-g\|_1\,\dif y
% =\int_{\R}|K(u)|\,\|T_{\lambda u}g-g\|_1\,\dif u \xrightarrow[\lambda\to 0]{} 0
% \]
% by dominated convergence. Therefore $\limsup_{\lambda\to 0}\|K_\lambda*f-f\|_1 \le 2\varepsilon\|K\|_1$.
% Letting $\varepsilon\to 0$ gives $\|K_\lambda*f-f\|_1\to 0$.
% \end{proof}







\begin{theorem}[Inverse]\label{thm:inversion}
Let \(f \in L^1(\R)\).
Assume that \(\hat{f} \in L^1(\R)\).
Then,
\begin{equation}\label{eq:inversion}
{\color{red}\boxed{\color{black}
f(x) = \int_{-\oo}^\oo \hat{f}(\xi) \eu^{2 \pi \iu \xi x} \dif \xi
}}
\end{equation}
for almost every \(x \in \R\). %, outside a set of measure zero.
If \(f \in C(\R)\), then \eqref{eq:inversion} holds for every \(x \in \R\).
\end{theorem}

\begin{proof}
Let \(\lambda > 0\) and \(x \in \R\). Consider 
\begin{equation}\label{eq:fourier-transform-phi-definition}
\phi(\xi) 
:= 
\eu^{2 \pi \iu \xi x} \eu^{-\pi \lambda^2 \xi^2}
=
M_x \varphi^\lambda(\xi)
\end{equation}
where \(\varphi(\xi) = \eu^{-\pi \xi^2}\) is the \nameref{ex:gaussian-fourier-transform}.
By \eqref{eq:duality-modulation} and \eqref{eq:duality-dilation-upper},
\begin{equation}\label{eq:fourier-transform-phi}
\hat{\phi}(y)
=
T_x \hat{\varphi}_\lambda(y)
=
T_x \varphi_\lambda(y)
=
\frac{1}{\lambda} \varphi\left(\frac{y - x}{\lambda}\right)
=
\frac{1}{\lambda} \eu^{-\pi |x-y|^2/\lambda^2}
=
\varphi_\lambda(x-y)
\end{equation}
where we used the fact that \(\varphi\) is even and that \(\hat{\varphi} = \varphi\) (\autoref{ex:gaussian-fourier-transform}).

Now on the one hand, by \autoref{lem:fourier-transform-switch},
\[
\begin{aligned}
\int_{-\oo}^\oo \hat{f}(\xi) \eu^{2 \pi \iu \xi x} \eu^{-\pi \lambda^2 \xi^2} \dif \xi
\overset{\text{\eqref{eq:fourier-transform-phi-definition}}}{=}&
\int_{-\oo}^\oo \hat{f}(\xi) \phi(\xi) \dif \xi
\overset{\text{\autoref{lem:fourier-transform-switch}}}{=}
\int_{-\oo}^\oo f(y) \hat{\phi}(y) \dif y \\
\overset{\text{\eqref{eq:fourier-transform-phi}}}{=}&
\int_{-\oo}^\oo f(y) \varphi_\lambda(x-y) \dif y
\overset{\text{\eqref{eq:convolution}}}{=}
(f * \varphi_\lambda)(x)
\end{aligned}
\]
and, since \(\int_{-\oo}^\oo \varphi(y) \dif y = 1\), by \autoref{thm:approximate-identity}
\(f * \varphi_\lambda \to f\) as \(\lambda \to 0\) in \(L^1(\R)\).

On the other hand, as \(\lambda \to 0\), for every \(x \in \R\),
\[
\int_{-\oo}^\oo \hat{f}(\xi) \eu^{2 \pi \iu \xi x} \eu^{-\pi \lambda^2 \xi^2} \dif \xi
\to
\int_{-\oo}^\oo \hat{f}(\xi) \eu^{2 \pi \iu \xi x} \dif \xi
\]
by \autoref{thm:dominated-convergence} (with \(g(\xi) = \hat{f}(\xi)\)), which is in \(L^1(\R)\) by assumption.

Setting e.g. \(u_n = f * \varphi_{1/n}\), by \autoref{lem:L1-convergence-pointwise-convergence-implies-almost-everywhere-equality}, we conclude that
\begin{equation}\label{eq:inversion-almost-everywhere}
f(x) = \int_{-\oo}^\oo \hat{f}(\xi) \eu^{2 \pi \iu \xi x} \dif \xi
\end{equation}
for almost all \(x \in \R\).
Note that 
\[
\int_{-\oo}^\oo \hat{f}(\xi) \eu^{2 \pi \iu \xi x} \dif \xi
=
\widehat{\hat{f}}(-x)
\]
and since \(\hat{f} \in L^1(\R)\),  by \autoref{fact:fourier-transform-uniform-continuity}, the right-hand side of \eqref{eq:inversion-almost-everywhere} defines a continuous function.
Therefore, if \(f\) is is also continuous, \eqref{eq:inversion-almost-everywhere} holds for all \(x \in \R\).
\end{proof}


\begin{corollary}\label{cor:fourier-transform-injective}
The operator \(\mathcal{F} \colon L^1(\R) \to L^\oo(\R)\) is injective.
\end{corollary}
\begin{proof}
A linear operator \(T \colon X \to Y\) is injective if \(\Kern(T) = \{0\}\), i.e., \(Tx = 0 \implies x = 0\).
Let \(f \in L^1(\R)\) and assume that \(\hat{f} = 0\).
Then,
\[
f(x) \overset{L^1}{=} \int_{-\oo}^\oo \hat{f}(\xi) \eu^{2 \pi \iu \xi x} \dif \xi = 0
\]
where we used the fact that \(\hat{f} = 0 \in L^1(\R)\) to apply \autoref{thm:inversion}.
\end{proof}

\begin{corollary}\label{cor:fourier-transform-periodicity}
Assume that \(f, \hat{f} \in L^1(\R)\). Then,
\begin{equation}\label{eq:fourier-transform-periodicity}
(\mathcal{F}^2 f)(x) = f(-x)
\end{equation}
in \(L^1(\R)\).
In particular, 
\begin{equation}\label{eq:fourier-transform-inverse-periodicity}
(\mathcal{F}^{-1} f)(x) = (\mathcal{F} f)(-x)
\end{equation}
and \((\mathcal{F}^4 f)(x) = f(x)\) in \(L^1(\R)\).
\end{corollary}

\begin{proof}
We have
\[
(\mathcal{F}^2 f)(x) 
=
(\mathcal{F}(\hat{f}))(x)
=
\int_{-\oo}^\oo \hat{f}(\xi) \eu^{-2 \pi \iu \xi x} \dif \xi
=
\int_{-\oo}^\oo \hat{f}(\xi) \eu^{2 \pi \iu \xi (-x)} \dif \xi
\overset{L^1}{=}
f(-x)
\]
where we used \autoref{thm:inversion} in the last step.

Therefore also,
\(
(\mathcal{F}^4 f)(x) = (\mathcal{F}^2 (\mathcal{F}^2 f))(x) = (\mathcal{F}^2 f)(-x) = f(x)
\) and
\(
\mathcal{F}^{-1} f(x) = \mathcal{F}^{-1} (\mathcal{F}^2 f)(-x) = \mathcal{F}(-x) = \mathcal{F} f(-x)
\) in the sense of \(L^1(\R)\).
\end{proof}




\clearpage

% \section{Fourier Transform on \(L^p(\R)\), \(1 < p \leq 2\)}
\section{Fourier Transform on \(L^p\)}

% \subsection{Fourier Transform on \(L^2(\R)\)}
\subsection{Fourier Transform on \(L^2\)}

Unlike the function on \([-\pi, \pi]\) considered in \autoref{sec:fourier-series-L2},
the \(L^p(\R)\) spaces do not satisfy \eqref{eq:Lp-inclusion}.
In particular, \(L^1(\R)\) is not the largest Lebesgue space on \(\R\).
Consequently, the results developed for \(L^1(\R)\) do not extend in an automatic way to \(L^2(\R)\).
Note that the functions \(\{\eu^{2\pi \iu \xi x}\}_{\xi \in \R}\) are not even in \(L^2(\R)\), whereas \(\{\eu^{\iu n x}\}_{n \in \Z}\) was an orthonormal basis of \(L^2([-\pi, \pi])\).
Instead of using orthonormal bases, the results for \(L^2(\R)\) are proven using functional analysis and operator theory, extending them by density from \(L^1(\R)\).

\subsubsection{Inner product and Hilbert space structure}
The one property that does also hold for \(L^2(\R)\) is that we have an inner product
\begin{equation}\label{eq:L2R-inner-product}
\langle f, g \rangle := \int_{-\infty}^\infty f(x) \overline{g(x)} \dif x
\end{equation}
for \(f, g \in L^2(\R)\). 
The definition \eqref{eq:L2R-inner-product} is well-defined because, by \nameref{thm:holder},
\[
|\langle f, g \rangle| 
=
\left|\int_{-\infty}^\infty f(x) \overline{g(x)} \dif x\right|
\leq
% \int_{-\infty}^\infty |f(x) \overline{g(x)}| \dif x
% =
\int_{-\infty}^\infty |f(x) g(x)| \dif x
=
\|f \cdot g\|_1
\leq
\|f\|_2 \|g\|_2
\]
which is finite since \(f, g \in L^2(\R)\).
The norm induced by \eqref{eq:L2R-inner-product} is precisely the \(L^2\)-norm, since
\(
\|f\|_2^2 = \langle f, f \rangle
\).
Since \(L^2(\R)\) is complete with respect to this norm, it is a \emph{Hilbert space}.

\begin{definition}
The \(L^2\)-norm of a (measurable) function \(f \to \C\) is called the \emph{energy} of \(f\).
A \emph{finite-energy signal} is a function in \(L^2(\R)\).
\end{definition}



\subsubsection{Definition and properties}

For \(f \in L^2(\R)\), the integral \eqref{eq:fourier-transform} may not converge.
\begin{theorem}
Let \(f \in L^2(\R)\).
Then,
\begin{equation}\label{eq:fourier-transform-L2}
\hat{f}(\xi) := \lim_{R \to \infty} \int_{-R}^R f(x) \eu^{-2\pi \iu \xi x} \dif x
\end{equation}
converges for almost every \(\xi \in \R\) and is in \(L^2(\R)\).
\end{theorem}

\begin{theorem}
The linear operator \(\mathcal{F} \colon f \in L^2(\R) \mapsto \hat{f} \in L^2(\R)\) satisfies:
\begin{enumerate}[label=(\roman*)]
\item \(\mathcal{F} \colon L^2(\R) \to L^2(\R)\) is bounded, injective and surjective. Its inverse is also bounded.
\item For every \(f, g \in L^2(\R)\), \(\langle f, \hat{g} \rangle = \langle f, g \rangle\), in particular \(\|f\|_2 = \|\hat{f}\|_2\), i.e., \(\mathcal{F}\) is a surjective isometry of \(L^2(\R)\).\footnote{in fact, a unitary operator}
\qedhere
\end{enumerate}
\end{theorem}

\begin{theorem}[Inverse]
Let \(f \in L^2(\R)\).
Then, 
\(
f(x) \overset{L^2}{=} 
\lim_{R \to \infty} \int_{-R}^R \hat{f}(\xi) \eu^{2\pi \iu \xi x} \dif \xi
\).
\end{theorem}

Different from \(L^1(\R)\), where where \(\hat{f}\) is continuous and defined for every \(\xi \in \R\), the Fourier transform of \(f \in L^2(\R)\) does not have the same nice behavior.
In general, the Fourier transform of \(f \in L^2(\R)\) is not a continuous function, and may even be infinite for some \(\xi \in \R\).

\begin{example}
For \(f(x) = \frac{\sin(\pi x)}{\pi x} \in L^2(\R)\), we have 
\[
\hat{f}(\xi) = 
\begin{cases}
0 & |\xi| > 1/2 \\
1/2 & |\xi| = 1/2 \\
1 & |\xi| < 1/2
\end{cases}
\]
which is not continuous.
\end{example}

\begin{example}
Take
\(
f(x) = \frac{1}{(1 + |x|)^\alpha}
\) for \(1/2 < \alpha < 1\).
Then, \(f \in L^2(\R) \setminus L^1(\R)\) and
\[
\hat{f}(0) = \lim_{R \to \infty} \int_{-R}^R f(x) \dif x = \int_{-\infty}^\infty f(x) \dif x = \|f\|_1 = \infty
\]
 by the monotone convergence theorem (since \(f \geq 0\)).
\end{example}





% \subsection{Fourier transform on \(L^p(\R)\), \(1 < p < 2\)}
\subsection{Fourier transform on \(L^p\), \(1 < p < 2\)}
For \(p = 2\), we have the complete characterization \(\mathcal{F}(L^2(\R)) = L^2(\R)\).
On the other hand, for \(p = 1\), \(\mathcal{F}(L^1(\R))\) is far from being \(L^\infty(\R)\).
By \autoref{fact:fourier-transform-uniform-continuity} and \autoref{thm:riemann-lebesgue-lemma-FT}, we know that
\[
\mathcal{F}(L^1(\R)) \subseteq C_0(\R) \cap C_\text{uni}(\R) \subsetneq L^\infty(\R)
\]
where \(C_\text{uni}(\R)\) denotes the space of uniformly continuous functions on \(\R\).
The first inclusion is also strict, i.e.,
\begin{equation}\label{eq:fourier-transform-image-L1}
{\color{red}\boxed{\color{black}
\mathcal{F}(L^1(\R)) 
\subsetneq 
C_0(\R) \cap C_\text{uni}(\R)
\subsetneq
L^\infty(\R)
}}
\end{equation}
since there exist functions in \(C_0(\R) \cap C_\text{uni}(\R)\) which are not the Fourier transform of any \(L^1\)-function.
Unlike \(p = 2\), the image of \(L^1(\R)\) under the Fourier transform does not admit a simple characterization.

In general, the behavior of \(\mathcal{F}\) on \(L^p(\R)\) is subtle.
It can be proven that if \(f \in L^p(\R)\), \(1 < p < 2\), there exist (non-unique) \(f_1 \in L^1(\R)\) and \(f_2 \in L^2(\R)\) such that \(f = f_1 + f_2\).
Hence, we may define the Fourier transform of \(f\) as \(\hat{f} = \hat{f_1} + \hat{f_2}\).
Moreover, this definition does not depend on the choice of \(f_1\) and \(f_2\).
We have the following celebrated result regarding boundedness:

\begin{theorem}[Hausdorff-Young]\label{thm:hausdorff-young}
The Fourier transform \(\mathcal{F}\) maps \(L^p(\R)\) to \(L^{p'}(\R)\) for every \(1 \leq p \leq 2\), where \(p' = \frac{p}{p-1}\) is the conjugate exponent of \(p\).
Moreover,
\begin{equation}\label{eq:hausdorff-young}
\|\hat{f}\|_{p'} 
\leq
\left(\frac{p^{1/p}}{(p')^{1/p'}}\right)^{1/2} \|f\|_p
\end{equation}
for every \(f \in L^p(\R)\).
Also known as the \hl{Babenko–Beckner inequality}.
\end{theorem}

The mapping \(\mathcal{F} \colon L^p(\R) \to L^{p'}(\R)\), \(1 \leq p \leq 2\), is surjective if and only if \(p = 2\).
In fact, \(\mathcal{F}(L^p(\R))\) admits an explicit characterization only in this case.
Moreover, for every \(p > 2\) there does always exist a function \(f \in L^p(\R)\) such that \(\hat{f} \notin L^q(\R)\) for any \(0 < q \leq \infty\).
Even worse, one can find \(f \in L^p(\R)\) such that \(\hat{f} \notin L^1_\text{loc}(\R)\), where \(L^1_\text{loc}(\R)\) denotes the space of measurable functions \(f \colon \R \to \C\) satisfying 
\[
\int_K |f(x)| \dif x < \infty
\]
for every compact \(K \subseteq \R\) (locally integrable functions).
In order to develop a Fourier theory for \(L^p\) functions beyond the case \(p = 2\), one is naturally led to the framework of \emph{distributions}.



\subsection{Multivariate Fourier transform}
For a multivariate function \(f \colon \R^d \to \C\), \(f \in L^1(\R^d)\), we define 
\begin{equation}\label{eq:multivariate-fourier-transform}
\hat{f}(\vect{\xi}) := \int_{\R^d} f(\vect{x}) \eu^{-2\pi \iu \vect{\xi} \cdot \vect{x}} \dif \vect{x}
\end{equation}
where \(\vect{\xi} \cdot \vect{x} = \sum_{j=1}^d \xi_j x_j\) is the standard inner product in \(\R^d\).

The lower dilation must be re-defined as
\[
f_\lambda(\vect{x}) = \frac{1}{\lambda^d} f\left(\frac{\vect{x}}{\lambda}\right)
\]
and Gaussians are replaced by multivariate Gaussians \(\varphi(\vect{x}) = \eu^{-\pi \|\vect{x}\|^2}\).

If \(f \in C^k(\R^d)\) and \(|\vect{\alpha}| = \alpha_1 + \cdots + \alpha_d \leq k\), we define
\[
D^{\vect{\alpha}} f(\vect{x}) 
=
\frac{\partial^{|\vect{\alpha}|}}{\partial x_1^{\alpha_1} \cdots \partial x_d^{\alpha_d}} f(\vect{x})
\]
where \(\vect{\alpha} = (\alpha_1, \ldots, \alpha_d) \in \N^d\) is a multi-index.
Under this notation, \autoref{eq:higher-derivatives} becomes
\[
\widehat{D^{\vect{\alpha}} f}(\vect{\xi}) 
=
(2\pi \iu \vect{\xi})^{\vect{\alpha}} \hat{f}(\vect{\xi})
\]
where \((2\pi \iu \vect{\xi})^{\vect{\alpha}} = (2\pi \iu)^{|\vect{\alpha}|} \xi_1^{\alpha_1} \cdots \xi_d^{\alpha_d}\).








\clearpage




\section{PDEs}

\subsection{Sobolev spaces}

% Recall \autoref{thm:higher-derivatives}, \autoref{eq:higher-derivatives}:
% \[
% \widehat{(f^{(j)})}(\xi) = (2\pi \iu \xi)^j \hat{f}(\xi)
% \]

Consider the ODE
\begin{equation}\label{eq:ODE}
\boxed{
-u_{xx} + u = f
}
\end{equation}
for a fixed forcing \(f \in L^1(\R)\).
Assuming that \(u \in L^1(\R)\), we take the Fourier transform to obtain
\[
(4\pi^2 \xi^2 + 1) \hat{u}(\xi) 
\overset{\text{\eqref{eq:higher-derivatives}}}{=} 
\hat{f}(\xi) \implies \hat{u}(\xi) = \frac{1}{1 + 4\pi^2 \xi^2} \cdot \hat{f}(\xi)
\]
whence,
\[
u = \mathcal{F}^{-1} \left( \frac{1}{1 + 4\pi^2 (\cdot)^2} \cdot \hat{f} \right) 
\overset{\text{\eqref{eq:convolution-theorem}}}{=}
\mathcal{F}^{-1} \left( \frac{1}{1 + 4\pi^2 (\cdot)^2} \right) \ast f
\]

By \autoref{ex:decaying-exponential-fourier-transform}
\[
\frac{1}{1 + \pi^2 (2\xi)^2} 
= 
\mathcal{F}(e^{-2 |\cdot|})(2\xi) 
=
\frac 1 2 \mathcal{F}(e^{- |\cdot|})(\xi)
\]
where we used \autoref{eq:duality-dilation-upper}.
Therefore, 
\(
\mathcal{F}^{-1} \left( \frac{1}{1 + 4\pi^2 (\cdot)^2} \right) = \frac{1}{2} e^{-|\cdot|}
\).

Thus, the solution of \autoref{eq:ODE} can be expressed as
\[
\begin{verticalhack}
u(x) = \frac{1}{2} \int_{-\infty}^\infty e^{-|x-y|} f(y) \dif y
\end{verticalhack}
\qedhere
\]


\begin{definition}[Lebesgue-Sobolev space]\label{def:lebesgue-sobolev-space}
For \(s \in \R\), we define 
\begin{equation}\label{eq:lebesgue-sobolev-space}
{\color{red}\boxed{\color{black}
H^s(\R) := \big\{f \in L^2(\R) : \Lambda_s f \in L^2(\R)\big\}
}}
\end{equation}
where 
\begin{equation}\label{eq:bessel-sobolev-potential}
\Lambda_s f = \mathcal{F}^{-1} \big( (1 + 4\pi^2 (\cdot)^2)^{s/2} \cdot \hat{f} \big)
\end{equation}
denotes the Bessel-Sobolev potential of order \(s\).
\end{definition}

\subsubsection{Norm}
A norm is naturally defined on \(H^s(\R)\) by \hl{\(\|f\|_{H^s} := \|\Lambda_s f\|_2\)}.
Recall that the Fourier and its inverse are isometries of \(L^2(\R)\), so
\[
\|f\|_{H^s}
=
\|\Lambda_s f\|_2
=
\big\|\mathcal{F}^{-1} \big( (1 + 4\pi^2 (\cdot)^2)^{s/2} \cdot \hat{f} \big)\big\|_2
=
\big\|(1 + 4\pi^2 (\cdot)^2)^{s/2} \cdot \hat{f}\big\|_2
\]
whence
\begin{equation}\label{eq:lebesgue-sobolev-norm}
{\color{red}
\boxed{\color{black}
\|f\|_{H^s} 
= 
\left( \int_{-\infty}^\infty (1 + 4\pi^2 \xi^2)^s |\hat{f}(\xi)|^2 \dif \xi \right)^{1/2}
}
}
\end{equation}



Trivially, \(\Lambda_0 f = f\) for every \(f \in L^2(\R)\), so \(H^0(\R) = L^2(\R)\).
Moreover, if \(s = k \in \N_{\geq 1}\),
\[
\begin{aligned}
\|f\|_{H^k}^2
&= 
\int_{-\infty}^\infty (1 + 4\pi^2 \xi^2)^k |\hat{f}(\xi)|^2 \dif \xi 
=
\int_{-\infty}^\infty \left( \sum_{j=0}^k \binom{k}{j} (4\pi^2 \xi^2)^j \right) |\hat{f}(\xi)|^2 \dif \xi \\
&=
\sum_{j=0}^k \binom{k}{j} \int_{-\infty}^\infty (4\pi^2 \xi^2)^j |\hat{f}(\xi)|^2 \dif \xi 
=
\|f\|_2^2 + \sum_{j=1}^k \binom{k}{j} \int_{-\infty}^\infty (4 \pi^2 \xi^2)^j |\hat{f}(\xi)|^2 \dif \xi \\
&=
\|f\|_2^2 + \sum_{j=1}^k \binom{k}{j} \int_{-\infty}^\infty |(2\pi \iu \xi)^j \hat{f}(\xi)|^2 \dif \xi  \\
&\overset{\text{\eqref{eq:higher-derivatives}}}{=}
\|f\|_2^2 + \sum_{j=1}^k \binom{k}{j} \int_{-\infty}^\infty |\widehat{f^{(j)}}(\xi)|^2 \dif \xi \\
&=
\|f\|_2^2 + \sum_{j=1}^k \binom{k}{j} \|\mathcal{F}(f^{(j)})\|_2^2 \\
&=
\|f\|_2^2 + \sum_{j=1}^k \binom{k}{j} \|f^{(j)}\|_2^2
\end{aligned}
\]

Therefore,
\begin{equation}\label{eq:lebesgue-sobolev-norm-derivatives}
{\color{red}\boxed{\color{black}
\|f\|_{H^k}
=
\sqrt{
  \|f\|_2^2 + \sum_{j=1}^k \binom{k}{j} \|f^{(j)}\|_2^2 
}
}}
\end{equation}



We have the estimate
\begin{equation}\label{eq:lebesgue-sobolev-norm-derivatives-estimate}
% \left( 
\sqrt{
  \|f\|_2^2 + \sum_{j=1}^k \|f^{(j)}\|_2^2 
}
% \right)^{\frac{1}{2}}
\;
\leq
\;
\|f\|_{H^k}
\;
\leq
\;
\binom{k}{\lfloor k/2 \rfloor}^{\frac{1}{2}}
% \left(
\sqrt{
   \|f\|_2^2 + \sum_{j=1}^k \|f^{(j)}\|_2^2 
}
% \right)^{1/2}
\end{equation}
since \(1 \leq \binom{k}{j} \leq \binom{k}{\lfloor k/2 \rfloor}\) for every \(0 \leq j \leq k\).



% \begin{lemma}\label{lem:sum-of-squares-inequality}
Furthermore, for \(x_1, \ldots, x_k \geq 0\) and \(k \in \N_{\geq 1}\),
\begin{equation}\label{eq:sum-of-squares-inequality}
\frac{1}{k} (x_1 + \cdots + x_k)^2 
\leq
x_1^2 + \cdots + x_k^2
\leq
(x_1 + \cdots + x_k)^2
\end{equation}
as can be proven by induction on \(k\).
% \end{lemma}


Combining \eqref{eq:sum-of-squares-inequality} and \eqref{eq:lebesgue-sobolev-norm-derivatives-estimate}, we obtain
\begin{equation}\label{eq:lebesgue-sobolev-norm-derivatives-estimate-2}
{\color{red}\boxed{\color{black}
\frac{1}{\sqrt{1 + k}} \sum_{j=0}^k \|f^{(j)}\|_2
\leq
\|f\|_{H^k}
\leq
\binom{k}{\lfloor k/2 \rfloor}^{\frac{1}{2}} 
\sum_{j=0}^k \|f^{(j)}\|_2
}}
\end{equation}
where we denoted \(f^{(0)} = f\) for compactness.




\subsubsection{Weak derivative}

\autoref{def:weak-derivative} clarifies the notion of derivative appearing in the next sections.
Recall that \(L^1_{\mathrm{loc}}(\R)\) is the space of measurable functions \(f : \R \to \C\) such that
\[
\int_K |f(x)| \dif x < \infty
\]
for every compact \(K\) in  \(\R\).



\begin{definition}[weak derivative]\label{def:weak-derivative}
Let \(f \in L^2(\R)\).
We say that \(f\) has weak derivative of order \(k\) if there exists a function \(F \in L^1_{\mathrm{loc}}(\R)\) such that for every \(g \in C_c^\infty(\R)\),
\begin{equation}\label{eq:weak-derivative}
\int_{-\infty}^\infty F(x) g(x) \dif x
=
(-1)^k \int_{-\infty}^\infty f(x) g^{(k)}(x) \dif x
\end{equation}
The function \(F\) is called the \(k\)-th weak derivative of \(f\) and will be denoted by \(f^{(k)}\).
\end{definition}

\begin{theorem}\label{thm:lebesgue-sobolev-characterization}
Let \(k \in \N\).
Then, \(f \in H^k(\R)\) if and only if \(f\) has weak derivatives up to order \(k\) and for every \(j \leq k\), \(f^{(j)} \in L^2(\R)\).
\end{theorem}























\subsection{Heat equation}


% Let \(u = u(t, x)\). % where \(t > 0\) and \(x \in \R\) are the time and space variables.
% Denote by \(u_t\) and \(u_{xx}\) the partial derivatives \(\partial_t u\) and \(\partial_{xx}^2 u\), respectively.
Consider the Cauchy problem
\begin{equation}\label{eq:heat-equation}
\begin{cases}
\partial_t u(t, x) = \partial_{xx}^2 u(t, x) & t > 0, x \in \R \\
u(0, x) = u_0(x) & x \in \R
\end{cases}
\end{equation}
where the initial datum \(u_0\) is assumed to be sufficiently regular.

% To solve \eqref{eq:heat-equation}, we use the Fourier transform wrt the space variable \(x\), i.e., we fix \(t > 0\) and take the Fourier transform of \(u(t, \cdot)\).

To solve \eqref{eq:heat-equation}, take FT wrt the space variable \(x\).
It commutes with \(\partial_t\) and to the RHS we apply \autoref{thm:higher-derivatives}:
\begin{equation}\label{eq:heat-equation-fourier}
\begin{cases}
\partial_t \hat{u}(t, \xi) = -4\pi^2 \xi^2 \hat{u}(t, \xi) & t > 0, \xi \in \R \\
\hat{u}(0, \xi) = \widehat{u_0}(\xi) & \xi \in \R
\end{cases}
\end{equation}

For every fixed \(\xi \in \R\), \eqref{eq:heat-equation-fourier} is a first-order ODE in \(t\) with general solution given by
\begin{equation}\label{eq:heat-equation-fourier-general-solution}
\hat{u}(t, \xi) = C e^{-4\pi^2 t \xi^2}
\end{equation}
where the constant \(C\) is chosen such that the initial condition is satisfied:
\[
\left[\hat{u}(t, \xi)\right]_{t=0} \overset{!}{=} \widehat{u_0}(\xi) \implies C = \widehat{u_0}(\xi)
\]
whence the solution of \eqref{eq:heat-equation-fourier} is
\begin{equation}\label{eq:heat-equation-specific-solution}
\hat{u}(t, \xi) = e^{-4\pi^2 t \xi^2} \widehat{u_0}(\xi)
\end{equation}


By taking the inverse Fourier transform, we obtain
\begin{equation}\label{eq:heat-equation-solution}
\begin{aligned}
u(t, x)
&= 
\mathcal{F}^{-1} \left( e^{-4\pi^2 t (\cdot)^2} \widehat{u_0} \right)(x) \\
&\overset{\text{\eqref{eq:convolution-theorem}}}{=}
\left(\mathcal{F}^{-1} \left( e^{-4\pi^2 t (\cdot)^2} \right) \ast \mathcal{F}^{-1} \left( \widehat{u_0} \right) \right)(x) \\
&\overset{\text{\eqref{eq:duality-dilation-lower}}}{=}
\left(\frac{1}{\sqrt{4\pi t}} e^{-\frac{(\cdot)^2}{4t}} \ast u_0 \right)(x)
\end{aligned}
\end{equation}
where we used \autoref{cor:fourier-transform-periodicity}, \eqref{eq:fourier-transform-inverse-periodicity}, and \autoref{eq:duality-dilation-upper} with \(\lambda = \sqrt{4 \pi t}\) 


In conclusion, 
\begin{equation}\label{eq:heat-equation-final-solution}
{\color{red}\boxed{\color{black}
u(t, x) = (K_t \ast u_0)(x)
}}
\end{equation}
where \(K_t(x) = \frac{1}{\sqrt{4\pi t}} e^{-\frac{x^2}{4t}}\), \(t > 0\), \(x \in \R\), is called \hl{heat kernel}.





\subsection{Wave equation}

% Let \(u = u(t, x)\).
% Denote by \(u_{tt}\) and \(u_{xx}\) the partial derivatives \(\partial_{tt}^2 u\) and \(\partial_{xx}^2 u\), respectively.
Consider the Cauchy problem
\begin{equation}\label{eq:wave-equation}
\begin{cases}
\partial_{tt}^2 u(t, x) = \partial_{xx}^2 u(t, x) & t > 0, x \in \R \\
u(0, x) = u_0(x) & x \in \R \\
\partial_t u(0, x) = v_0(x) & x \in \R
\end{cases}
\end{equation}
where the initial data \(u_0, v_0\) are assumed to be sufficiently regular.

To solve \eqref{eq:wave-equation}, take FT wrt the space variable \(x\).
It commutes with \(\partial_t\) and to the RHS we apply \autoref{thm:higher-derivatives}:
\begin{equation}\label{eq:wave-equation-fourier}
\begin{cases}
\partial_{tt}^2 \hat{u}(t, \xi) = -4\pi^2 \xi^2 \hat{u}(t, \xi) & t > 0, \xi \in \R \\
\hat{u}(0, \xi) = \widehat{u_0}(\xi) & \xi \in \R \\
\partial_t \hat{u}(0, \xi) = \widehat{v_0}(\xi) & \xi \in \R
\end{cases}
\end{equation}

For every fixed \(\xi \in \R\), \eqref{eq:wave-equation-fourier} is a second-order ODE in \(t\).
Searching for solutions of the form \(\hat u(t,\xi)=\eu^{\lambda t}\) yields
\[
\lambda^2 + 4\pi^2 \xi^2 = 0
\]
hence \(\lambda = \pm 2\pi \iu |\xi|\), and thus, for \(\xi \neq 0\) the general solution%
\footnote{For \(\xi = 0\) the solution is different (since the characteristic polynomial has a double root), but for our purposes we can ignore that, since the (inverse) FT does not care about a set of measure zero.}
is given by
\begin{equation}\label{eq:wave-equation-fourier-general-solution}
\hat{u}(t, \xi)
=
A(\xi)\,\eu^{-2\pi \iu |\xi| t}
+
B(\xi)\,\eu^{2\pi \iu |\xi| t}
\end{equation}
where \(A(\xi)\) and \(B(\xi)\) are obtained by imposing the initial conditions in \eqref{eq:wave-equation-fourier}:
% Imposing \(\hat u(0,\xi)=\widehat{u_0}(\xi)\) and \(\partial_t\hat u(0,\xi)=\widehat{v_0}(\xi)\) gives
% \[
% A(\xi)+B(\xi)=\widehat{u_0}(\xi),
% \qquad
% 2\pi\iu|\xi|\bigl(-A(\xi)+B(\xi)\bigr)=\widehat{v_0}(\xi),
% \]
% so that
\[
A(\xi)=\frac{1}{2}\widehat{u_0}(\xi)-\frac{1}{2}\frac{\widehat{v_0}(\xi)}{2\pi\iu|\xi|}
\qquad
B(\xi)=\frac{1}{2}\widehat{u_0}(\xi)+\frac{1}{2}\frac{\widehat{v_0}(\xi)}{2\pi\iu|\xi|}
\]
By associativity and the identities
\(
\frac{\eu^{\iu\theta}+\eu^{-\iu\theta}}{2}=\cos(\theta)
\)
and
\(
\frac{\eu^{\iu\theta}-\eu^{-\iu\theta}}{2\iu}=\sin(\theta)
\),
we obtain
\begin{equation}\label{eq:wave-equation-fourier-solution}
\hat{u}(t, \xi)
=
\cos(2\pi |\xi| t)\,\widehat{u_0}(\xi)
+
\frac{\sin(2\pi |\xi| t)}{2\pi |\xi|}\,\widehat{v_0}(\xi)
\end{equation}
with the inverse 
\begin{equation}\label{eq:wave-equation-solution}
u(t, x)
=
\mathcal{F}^{-1}\!\left(\cos(2\pi t|\cdot|)\,\widehat{u_0}\right)(x)
+
\mathcal{F}^{-1}\!\left(\frac{\sin(2\pi t|\cdot|)}{2\pi |\cdot|}\,\widehat{v_0}\right)(x)
\end{equation}

For the first term, using \(\cos(\theta)=\frac{1}{2}(\eu^{\iu\theta}+\eu^{-\iu\theta})\) and the inversion formula:
\begin{equation}\label{eq:wave-equation-first-term}
\mathcal{F}^{-1}\!\left(\cos(2\pi t|\cdot|)\,\widehat{u_0}\right)(x)
=
\frac{u_0(x+t) + u_0(x-t)}{2}
\end{equation}
% where we used that \(|\xi|\) is even and split the integral over \(\xi>0\) and \(\xi<0\).

% For the second term, we use that
% \[
% \frac{\sin(2\pi t\xi)}{\pi \xi}
% =
% \mathcal{F}\big(\chi_{[-t,t]}\big)(\xi)
% \]
% as in the characteristic function computation in the Fourier transform section (cf.\ \autoref{eq:characteristic-function} with \(A=2t\)).
For the second term, we use \eqref{eq:characteristic-function}, \eqref{eq:convolution-theorem} and \eqref{eq:fourier-transform-periodicity}:
\begin{equation}\label{eq:wave-equation-second-term}
\begin{aligned}
\mathcal{F}^{-1}\!\left(\frac{\sin(2\pi t|\cdot|)}{2\pi |\cdot|}\,\widehat{v_0}\right)(x)
&=
\frac{1}{2}\,(\chi_{[-t,t]} \ast v_0)(x) \\
&=
\frac{1}{2}\int_{-\infty}^\infty \chi_{[-t,t]}(x-y) v_0(y)\,\dif y \\
&=
\frac{1}{2}\int_{x-t}^{x+t} v_0(y)\,\dif y
\end{aligned}
\end{equation}

In conclusion, combining \eqref{eq:wave-equation-solution}, \eqref{eq:wave-equation-first-term}, and \eqref{eq:wave-equation-second-term}, we obtain \hl{d'Alembert's formula}
\begin{equation}\label{eq:wave-equation-final-solution}
{\color{red}\boxed{\color{black}
u(t, x)
=
\frac{u_0(x+t) + u_0(x-t)}{2}
+
\frac{1}{2}\int_{x-t}^{x+t} v_0(y)\,\dif y
}}
\end{equation}
for \(t>0\) and \(x\in\R\).



\end{document}
