% author: Fabian Bosshard % © CC BY 4.0
\documentclass[9pt,headings=standardclasses,parskip=half]{scrartcl}

% ===================== basics =====================
\usepackage{ifthen,iftex,csquotes}
\usepackage[T1]{fontenc}
\usepackage[english]{babel}

% ===================== page style =====================
\usepackage[automark]{scrlayer-scrpage}
\pagestyle{scrheadings}

% ===================== graphics / color =====================
\usepackage{graphicx}
\usepackage[dvipsnames]{xcolor}
\usepackage{caption,subcaption}
\usepackage[left=38mm,right=38mm,top=20mm,bottom=30mm]{geometry}

% ===================== math =====================
\usepackage{amsmath,amssymb,amsthm,mathtools,mathdots,accents}
\numberwithin{equation}{section}

% ===================== lists =====================
\usepackage{enumitem}
\renewcommand{\labelitemi}{\textbullet}
\renewcommand{\labelitemii}{\raisebox{0.1ex}{\scalebox{0.8}{\textbullet}}}
\renewcommand{\labelitemiii}{\raisebox{0.2ex}{\scalebox{0.6}{\textbullet}}}
\renewcommand{\labelitemiv}{\raisebox{0.3ex}{\scalebox{0.4}{\textbullet}}}

% ===================== bibliography =====================
\usepackage[backend=biber,style=numeric]{biblatex}
\addbibresource{\jobname.bib}

% ===================== symbols =====================
\usepackage{pifont}
\usepackage{marvosym}

% ===================== TikZ / plots =====================
\usepackage{tikz}
\usetikzlibrary{arrows,arrows.meta,shapes,positioning,calc,fit,patterns,intersections,math,3d,tikzmark,decorations.pathreplacing,decorations.markings}
\usepackage{tikz-dependency,tikz-qtree,tikz-qtree-compat,tikz-3dplot,tikzpagenodes}
\usepackage{pgfplots}

% ===================== highlighting (SOUL + math-safe \hl) =====================
\usepackage{soulutf8}
\usepackage{xparse}

\ExplSyntaxOn
\tl_new:N \__l_SOUL_argument_tl
\cs_set_eq:Nc \SOUL_start:n { SOUL@start }
\cs_generate_variant:Nn \SOUL_start:n { V }
\cs_set_protected:cpn {SOUL@start} #1 {
  \tl_set:Nn \__l_SOUL_argument_tl { #1 }
  \regex_replace_all:nnN { \c{\(} (.*?) \c{\)} } { \cM\$ \1 \cM\$ } \__l_SOUL_argument_tl
  \SOUL_start:V \__l_SOUL_argument_tl
}
\ExplSyntaxOff

\let\SOULhl\hl
\renewcommand{\hl}[1]{\SOULhl{#1}}

\definecolor{HLgreen}{HTML}{77DD77}
\definecolor{HLyellow}{HTML}{FFFF66}
\definecolor{HLorange}{HTML}{FFB347}
\definecolor{HLred}{HTML}{FF6961}
\definecolor{HLpink}{HTML}{FFB6C1}
\definecolor{HLturquoise}{HTML}{40E0D0}

\RenewDocumentCommand{\hl}{O{1} m}{%
  \begingroup
  \IfEqCase{#1}{%
    {1}{\sethlcolor{HLgreen}\SOULhl{#2}}%
    {2}{\sethlcolor{HLyellow}\SOULhl{#2}}%
    {3}{\sethlcolor{HLorange}\SOULhl{#2}}%
    {4}{\sethlcolor{HLred}\SOULhl{#2}}%
    {5}{\sethlcolor{red}\SOULhl{#2}}%
    {6}{\sethlcolor{HLturquoise}\SOULhl{#2}}%
  }[\PackageError{hl}{Undefined highlight level: #1}{}]%
  \endgroup
}

% ===================== colors / emphasis =====================
\definecolor{funblue}{rgb}{0.10,0.35,0.66}
\definecolor{alizarincrimsonred}{rgb}{0.85,0.17,0.11}
\definecolor{amethyst}{rgb}{0.6,0.4,0.8}
\definecolor{mypurple}{rgb}{0.5,0,0.5}
\definecolor{highlightpurple}{rgb}{0.6,0,0.6}
\renewcommand{\emph}[1]{\textcolor{highlightpurple}{\textsl{#1}}}

% ===================== theorem environments =====================
\usepackage{thmtools}

\newlength{\thmspace}\setlength{\thmspace}{3pt plus 1pt minus 1pt}

\declaretheoremstyle[headfont=\bfseries,bodyfont=\normalfont,spaceabove=\thmspace,spacebelow=\thmspace,qed=\ensuremath{\vartriangleleft},postheadspace=1em]{assertionstyle}
\declaretheorem[style=assertionstyle,name=Theorem,numberwithin=section]{theorem}
\declaretheorem[style=assertionstyle,name=Lemma,sibling=theorem]{lemma}
\declaretheorem[style=assertionstyle,name=Corollary,sibling=theorem]{corollary}
\declaretheorem[style=assertionstyle,name=Proposition,sibling=theorem]{proposition}
\declaretheorem[style=assertionstyle,name=Conjecture,sibling=theorem]{conjecture}
\declaretheorem[style=assertionstyle,name=Claim,sibling=theorem]{claim}
\declaretheorem[style=assertionstyle,name=Fact,sibling=theorem]{fact}

\declaretheoremstyle[headfont=\bfseries,bodyfont=\normalfont,spaceabove=\thmspace,spacebelow=\thmspace,qed=\ensuremath{\blacktriangleleft},postheadspace=1em]{definitionstyle}
\declaretheorem[style=definitionstyle,name=Definition,numberwithin=section]{definition}
\declaretheorem[style=definitionstyle,name=Problem,sibling=definition]{problem}

\declaretheoremstyle[headfont=\bfseries,bodyfont=\normalfont,spaceabove=\thmspace,spacebelow=\thmspace,qed=\ding{45},postheadspace=1em]{exercisestyle}
\declaretheorem[style=exercisestyle,name=Exercise,numberwithin=section]{exercise}
\declaretheoremstyle[headfont=\bfseries\color{red},bodyfont=\normalfont,spaceabove=\thmspace,spacebelow=\thmspace,qed=\ensuremath{\color{red}\blacktriangleleft},postheadspace=1em]{solutionstyle}
\declaretheorem[style=solutionstyle,name=Solution,numbered=no]{solution}

\declaretheoremstyle[headfont=\bfseries,bodyfont=\normalfont,spaceabove=6pt,spacebelow=6pt,qed=\ensuremath{\square},postheadspace=1em]{proofstyle}
\let\proof\relax \let\endproof\relax
\declaretheorem[style=proofstyle,name=Proof,numbered=no]{proof}

\declaretheoremstyle[headfont=\bfseries,bodyfont=\normalfont\normalsize,spaceabove=\thmspace,spacebelow=\thmspace,qed=\ensuremath{\blacktriangleleft},postheadspace=1em]{remarkstyle}
\declaretheorem[style=remarkstyle,name=Remark,numberwithin=section]{remark}

\declaretheoremstyle[headfont=\bfseries\color{funblue},bodyfont=\normalfont\normalsize,spaceabove=\thmspace,spacebelow=\thmspace,qed=\ensuremath{\color{funblue}\blacktriangleleft},postheadspace=1em]{examplestyle}
\declaretheorem[style=examplestyle,name=Example,sibling=remark]{example}

\declaretheoremstyle[headfont=\color{alizarincrimsonred}\bfseries,bodyfont=\normalfont\normalsize,spaceabove=\thmspace,spacebelow=\thmspace,qed=\ensuremath{\color{alizarincrimsonred}\blacktriangleleft},postheadspace=1em]{cautionstyle}
\declaretheorem[style=cautionstyle,name=Caution,sibling=remark]{caution}

\declaretheoremstyle[headfont=\bfseries,bodyfont=\normalfont\footnotesize,spaceabove=\thmspace,spacebelow=\thmspace,postheadspace=1em]{smallremarkstyle}
\declaretheorem[style=smallremarkstyle,name=Remark,sibling=remark]{smallremark}

\declaretheoremstyle[headfont=\bfseries\color{amethyst},bodyfont=\normalfont,spaceabove=6pt,spacebelow=6pt,qed=\ensuremath{\color{amethyst}\blacktriangleleft},postheadspace=1em]{digressionstyle}
\declaretheorem[style=digressionstyle,name=Digression,sibling=remark]{digression}

% ===================== misc helpers =====================
\newenvironment{verticalhack}{\begin{array}[b]{@{}c@{}}\displaystyle}{\\\noalign{\hrule height0pt}\end{array}}

% ===================== algorithms =====================
\usepackage{algorithm,algorithmicx}
\usepackage[italicComments=false]{algpseudocodex}
\newcommand*{\algorithmautorefname}{Algorithm}

\algnewcommand{\TO}{, \ldots ,}
\algnewcommand{\DOWNTO}{, \ldots ,}
\algnewcommand{\OR}{\vee}
\algnewcommand{\AND}{\wedge}
\algnewcommand{\NOT}{\neg}
\algnewcommand{\LEN}{\operatorname{len}}
\algnewcommand{\tru}{\ensuremath{\mathrm{\texttt{true}}}}
\algnewcommand{\fals}{\ensuremath{\mathrm{\texttt{false}}}}
\algnewcommand{\append}{\circ}
\algnewcommand{\nil}{\ensuremath{\mathrm{\textsc{nil}}}}
\algnewcommand{\red}{\ensuremath{\mathrm{\textsc{red}}}}
\algnewcommand{\black}{\ensuremath{\mathrm{\textsc{black}}}}
\algnewcommand{\gray}{\ensuremath{\mathrm{\textsc{gray}}}}
\algnewcommand{\white}{\ensuremath{\mathrm{\textsc{white}}}}

% ===================== operators / notation =====================
\DeclareMathOperator*{\argmax}{arg\,max}
\DeclareMathOperator*{\argmin}{arg\,min}
\DeclareMathOperator{\Span}{span}
\DeclareMathOperator{\Kern}{kern}
\DeclareMathOperator{\Trace}{trace}
\DeclareMathOperator{\Rank}{rank}
\newcommand{\im}{\operatorname{Im}}
\newcommand{\re}{\operatorname{Re}}

\newcommand*\matrspace{0.8mu}
\newcommand{\matr}[1]{\mspace{\matrspace}\underline{\mspace{-\matrspace}\smash[b]{\boldsymbol{#1}}\mspace{-\matrspace}}\mspace{\matrspace}}
\newcommand{\vect}[1]{{\boldsymbol{#1}}}

\newcommand{\dif}{\mathrm{d}}
\newcommand{\eu}{\mathrm{e}}
\newcommand{\iu}{\mathrm{i}}
\newcommand{\sign}{\operatorname{sgn}}

\newcommand{\R}{\mathbb{R}}
\newcommand{\N}{\mathbb{N}}
\newcommand{\Z}{\mathbb{Z}}
\newcommand{\Q}{\mathbb{Q}}
\newcommand{\C}{\mathbb{C}}
\newcommand{\K}{\mathbb{K}}
\newcommand{\F}{\mathbb{F}}

\newcommand{\Var}{\operatorname{Var}}
\newcommand{\Cov}{\operatorname{Cov}}
\newcommand{\Exp}{\operatorname{E}}
\newcommand{\Prob}{\operatorname{P}}
\newcommand{\numof}{\ensuremath{\#\,}}
\newcommand{\blackheight}{\operatorname{bh}}

\newcommand{\attrib}[2]{\ensuremath{#1\mathtt{.}\mathtt{#2}}}
\newcommand{\attribnormal}[2]{\ensuremath{#1\mathtt{.}#2}}

% ===================== metadata / hyperlinks =====================
\title{Fourier Analysis}
\author{Fabian Bosshard}
\date{\today}

\usepackage[
  linktoc=none,
  pdfauthor={Fabian Bosshard},
  pdftitle={USI - Fourier Analysis - Course Notes},
  pdfkeywords={USI, Fourier analysis, course notes, informatics},
  colorlinks=false,
  pdfborder={0 0 0},
  linkbordercolor={0 0.6 1},
  urlbordercolor={0 0.6 1},
  citebordercolor={0 0.6 1}
]{hyperref}

% ToC entries: dotted leaders + full-width clickable line
\DeclareTOCStyleEntry[linefill=\dotfill]{tocline}{section}
\DeclareTOCStyleEntry[linefill=\dotfill]{tocline}{subsection}
\DeclareTOCStyleEntry[linefill=\dotfill]{tocline}{subsubsection}

\makeatletter
\newlength\FB@toclinkht \newlength\FB@toclinkdp
\setlength\FB@toclinkht{.80\ht\strutbox}
\setlength\FB@toclinkdp{.40\dp\strutbox}

\let\FB@orig@contentsline\contentsline
\renewcommand*\contentsline[4]{%
  \begingroup
    \Hy@safe@activestrue
    \edef\Hy@tocdestname{#4}%
    \FB@orig@contentsline{#1}{%
      \Hy@raisedlink{\hyper@anchorstart{toc:\Hy@tocdestname}\hyper@anchorend}%
      \leavevmode
      \rlap{%
        \hyper@linkstart{link}{\Hy@tocdestname}%
          \raisebox{0pt}[\FB@toclinkht][\FB@toclinkdp]{%
            \hbox to \dimexpr\hsize-\parindent\relax{\hfil}%
          }%
        \hyper@linkend
      }%
      #2%
    }{#3}{#4}%
  \endgroup
}

\let\FB@orig@sectionlinesformat\sectionlinesformat
\renewcommand{\sectionlinesformat}[4]{%
  \ifx\@currentHref\@empty
    \FB@orig@sectionlinesformat{#1}{#2}{#3}{#4}%
  \else
    \leavevmode
    \hyper@linkstart{link}{toc:\@currentHref}%
      \@hangfrom{\hskip #2\relax #3}{#4}%
    \hyper@linkend
    \par
  \fi
}
\makeatother

% hyperref anchor uniqueness for section-relative numbering
\makeatletter
\renewcommand{\theHequation}{\thesection.\arabic{equation}}
\renewcommand{\theHtheorem}{\thesection.\arabic{theorem}}
\renewcommand{\theHlemma}{\theHtheorem}
\renewcommand{\theHcorollary}{\theHtheorem}
\renewcommand{\theHproposition}{\theHtheorem}
\renewcommand{\theHconjecture}{\theHtheorem}
\renewcommand{\theHclaim}{\theHtheorem}
\renewcommand{\theHfact}{\theHtheorem}
\renewcommand{\theHdefinition}{\thesection.\arabic{definition}}
\renewcommand{\theHproblem}{\theHdefinition}
\renewcommand{\theHexercise}{\thesection.\arabic{exercise}}
\renewcommand{\theHremark}{\thesection.\arabic{remark}}
\renewcommand{\theHexample}{\theHremark}
\renewcommand{\theHcaution}{\theHremark}
\renewcommand{\theHsmallremark}{\theHremark}
\renewcommand{\theHdigression}{\theHremark}
\makeatother

% ===================== license / cleveref =====================
\usepackage[type={CC},modifier={by},version={4.0}]{doclicense}
\usepackage{cleveref}

% ===================== acronyms =====================
\usepackage[acronym,nomain,toc,nonumberlist]{glossaries-extra}
\setabbreviationstyle[acronym]{long-short}
\makeglossaries
\newacronym{rbf}{RBF}{radial basis function}
\newacronym{rkhs}{RKHS}{reproducing kernel Hilbert space}

\begin{document}


\maketitle

\tableofcontents

\clearpage
\section{Preliminaries}



\paragraph{Greek letters}
\renewcommand{\arraystretch}{1.15}
\newcommand{\latinlike}[1]{\textcolor{gray}{\mathrm{#1}}}
\[
\begin{array}{@{}l@{\qquad}c@{\quad}c@{\quad}c@{}}
\textbf{Name} & \textbf{Low.} & \textbf{Var.} & \textbf{Upp.} \\
\hline
\text{alpha}   & \alpha   &              & \latinlike{A} \\
\text{beta}    & \beta    &              & \latinlike{B} \\
\text{gamma}   & \gamma   &              & \Gamma \\
\text{delta}   & \delta   &              & \Delta \\
\text{epsilon} & \epsilon & \varepsilon  & \latinlike{E} \\
\text{zeta}    & \zeta    &              & \latinlike{Z} \\
\text{eta}     & \eta     &              & \latinlike{H} \\
\text{theta}   & \theta   & \vartheta    & \Theta \\
\text{iota}    & \iota    &              & \latinlike{I} \\
\text{kappa}   & \kappa   & \varkappa    & \latinlike{K} \\
\text{lambda}  & \lambda  &              & \Lambda \\
\text{mu}      & \mu      &              & \latinlike{M} \\
\text{nu}      & \nu      &              & \latinlike{N} \\
\text{xi}      & \xi      &              & \Xi \\
\text{omicron} & {o}      &              & \latinlike{O} \\
\text{pi}      & \pi      & \varpi       & \Pi \\
\text{rho}     & \rho     & \varrho      & \latinlike{P} \\
\text{sigma}   & \sigma   & \varsigma    & \Sigma \\
\text{tau}     & \tau     &              & \latinlike{T} \\
\text{upsilon} & \upsilon &              & \Upsilon \\
\text{phi}     & \phi     & \varphi      & \Phi \\
\text{chi}     & \chi     &              & \latinlike{X} \\
\text{psi}     & \psi     &              & \Psi \\
\text{omega}   & \omega   &              & \Omega
\end{array}
\]


% aligned label + content
\newlength{\propw}
\newlength{\propsep}
\newcommand{\propitem}[2]{%
  \item \makebox[\propsep][l]{#1}%
  \parbox[t]{\dimexpr\linewidth-\propsep\relax}{#2}%
}
\settowidth{\propw}{{Distributive:}}
\setlength{\propsep}{\propw + 1.5em}

\newlength{\formw}
\settowidth{\formw}{$(a\circ b)\circ c = a\circ(b\circ c)$} % longest formula (adjust if needed)
\newlength{\formsep}
\setlength{\formsep}{\formw + 1.5em}

\paragraph{Algebraic properties}
\begin{itemize}
\propitem{{Associative:}}{%
  \makebox[\formsep][l]{$(a\circ b)\circ c = a\circ(b\circ c)$}%
  \emph{parentheses (grouping) do not matter}}
\propitem{{Commutative:}}{%
  \makebox[\formsep][l]{$a\circ b = b\circ a$}%
  \emph{order does not matter}}
\propitem{{Distributive:}}{%
    $a\cdot(b+c)=a\cdot b+a\cdot c$\\
    $(a+b)\cdot c=ac+bc$}
\end{itemize}




\paragraph{Trigonometric identities}
\begin{itemize}
    \item \(\cos( - y) = \cos(y)\)
    \item \(\sin( - y) = - \sin(y)\)
    \item \(\cos(a + b) = \cos(a) \cos(b) - \sin(a) \sin(b)\)
    \item \(\sin(a + b) = \sin(a) \cos(b) + \cos(a) \sin(b)\)
\end{itemize}

\paragraph{Complex functions}
If \(f: \R \to \C\), we may write \(f(x) = \Re(f(x)) + \iu \Im(f(x))\).
We have \(|f(x)|^2 = f(x) \overline{f(x)}\), where \(\overline{f(x)} = \Re(f(x)) - \iu \Im(f(x))\).


\subsection{Lebesgue integration}

\(\mathcal{R}([a,b])\) denotes the class of Riemann integrable functions on \([a,b]\).
\begin{theorem}
Let \(f: [a,b] \to \R\) be bounded.
\begin{itemize}
\item 
If \(f \in \mathcal{R}([a,b])\), then the Lebesgue integral of \(|f|\) is finite.
Moreover, the Riemann integral of \(f\) equals the Lebesgue integral of \(f\).
\item 
\(f \in \mathcal{R}([a,b])\) if and only if \(f\) is continuous except on a set of measure zero.
\qedhere
\end{itemize}
\end{theorem}

The motivation for replacing Riemann with Lebesgue theory is that Lebesgue integrals are particularly well-suited for interchanginng limits and integration,
the order of integration and derivatives and integration.






\subsection{Function spaces}

Classically, functions are classified in terms of their regularity.
We write \(C(\R)\) for the space of continuous functions from \(\R\) to \(\C\).
For \(k \in \N\), we write \(C^k(\R)\) for the space of functions from \(\R\) to \(\C\) whose first \(k\) derivatives exist and are continuous.

% Recall that
% \[
% \begin{aligned}
% \text{$f$ continuous}             &\;\Leftarrow& \text{$f$ differentiable}        &\;\Leftarrow& \text{$f$ continuously differentiable} \\
% \phantom{\text{$f$ continuous}}   &\;\Leftarrow& \text{$f$ twice differentiable}  &\;\Leftarrow& \text{$f$ twice continuously differentiable} \\
% \phantom{\text{$f$ continuous}}   &\;\Leftarrow& \text{$f$ 3-times differentiable}&\;\Leftarrow& \dots \;\Leftarrow\; \text{$f$ smooth}
% \end{aligned}
% \]
% Recall that
% \[
% \begin{aligned}
% \text{$f$ continuous}           &\Leftarrow & \text{$f$ differentiable}         &\Leftarrow & \text{$f$ continuously differentiable} \\
% \phantom{\text{$f$ continuous}} &\Leftarrow & \text{$f$ twice differentiable}   &\Leftarrow & \text{$f$ twice continuously differentiable} \\
% \phantom{\text{$f$ continuous}} &\Leftarrow & \text{$f$ 3-times differentiable} &\Leftarrow & \dots \;\Leftarrow\; \text{$f$ smooth}
% \end{aligned}
% \]

\begin{example}
  The function \(f(x) = x^k \sign(x)\) is in \(C^{k-1}(\R)\) but not in \(C^k(\R)\).
\end{example}


The revolution in modern analysis is that functions are classified in terms of their integrability.
For \(1 \leq p < \infty\), we write \(f \in L^p(\R)\) if
\begin{equation}\label{eq:Lp-norm}
{\color{red}\boxed{\color{black}
\| f \|_p  := \left( \int_{-\infty}^{\infty} |f(x)|^p \dif x \right)^{1/p} 
}}
\end{equation}
is finite.
To be precise, we also need the requierement that \(f: \R \to \C\) is measurable.

\begin{remark}
There is also a definition of \(L^{\infty}(\R)\), where \(f \in L^{\infty}(\R)\) if
\begin{equation}\label{eq:Linf-norm}
{\color{red}\boxed{\color{black}
\| f \|_{\infty} := \operatorname{ess\,sup}_{x \in \R} |f(x)|
}}
\end{equation}
is finite.
The essential supremum is basically a supremum out of a set of measure zero.
\end{remark}
\begin{remark}
Strictly speaking, \eqref{eq:Lp-norm} is not a norm on functions defined pointwise, but on equivalence classes of functions that are equal almost everywhere.
\end{remark}

We write \(f \in C_b(\R)\) if \(f \in C(\R)\) and there exists \(M > 0\) such that \(|f(x)| \leq M\) for all \(x \in \R\) (i.e., \(f\) is bounded).
Observe that \hl[2]{\(C_b(\R)  = L^{\infty}(\R) \cap C(\R)\)}.

If \(f \in C_b(\R)\), then \(\|f\|_{\infty} = \max_{x \in \R} |f(x)|\).

% In general, $L^p(\R)$ is not closed under pointwise products: even if $f,g\in L^p(\R)$, it may happen that $fg\notin L^p(\R)$.
% For $1\le p\le \infty$, the \emph{conjugate exponent} $p'$ is defined by
% \[
% \frac{1}{p}+\frac{1}{p'}=1,
% \qquad \text{with } \frac{1}{\infty}:=0 \text{ and } \frac{1}{0}:=\infty.
% \]
% If $f\in L^p(\R)$ and $g\in L^{p'}(\R)$, then $fg\in L^1(\R)$ and H\"older's inequality holds:
% \[
% \|fg\|_1 \le \|f\|_p\,\|g\|_{p'}.
% \]

% A subspace $Y\subseteq X$ of a normed space $(X,\|\cdot\|_X)$ is \emph{dense} in $X$ if for every $x\in X$ and every $\varepsilon>0$ there exists $y\in Y$ such that
% \[
% \|x-y\|_X<\varepsilon.
% \]
% The space
% \[
% C_c^\infty(\R):=\{f\in C^\infty(\R): \exists R>0 \text{ such that } f(x)=0 \text{ for } |x|>R\}
% \]
% is dense in $L^p(\R)$ for every $1\le p<\infty$. Hence, many statements can be proved first for $C_c^\infty(\R)$ and then extended to $L^p(\R)$ by approximation.











\clearpage

\section{Complex analysis}


A complex function \(f = u + \iu v\) is complex differentiable at \(z = x + \iu y \in D(f)\) iff it is real differentiable (as a function \(\vect{f}: \R^2 \to \R^2\)) at \((x,y)\) and additionally the \emph{Cauchy-Riemann equations}
\begin{equation}\label{eq:cauchy-riemann}
\frac{\partial u}{\partial x} = \frac{\partial v}{\partial y},
\qquad
\frac{\partial u}{\partial y} = - \frac{\partial v}{\partial x}
\end{equation}
hold at \((x,y)\), because the derivative needs to be a ``Drehstreckung'' (composition of a rotation and a scaling), i.e. its Jacobian needs to be of the form \(\vect{f}' =  \left[\begin{smallmatrix} a & -b \\ b & a \end{smallmatrix}\right]\).
If a function is complex differentiable at every point of an open set \(D \subseteq \C\),
then it is called \emph{holomorphic} on \(D\).


We define the curve integral of a function \(f: \C \to \C\) along a curve \(C\) parametrized by \(z(t)\), \(t \in [a,b]\) as
\begin{equation}\label{eq:curve-integral}
{\color{red}\boxed{\color{black}
\int_C f(z) \dif z := \int_a^b f( z(t)) z'(t) \dif t
}}
\end{equation}
% give me a \remark to add here, that explain how the complex curve integral can bee interpreded as a wegintegral 2. art... 
\begin{remark}
\eqref{eq:curve-integral} can be interpreted as a line integral of a vector field in \(\R^2\) (``Wegintegral 2. Art'').
To that end, we write \(f = u(x,y) + \iu v(x,y)\) with \(u,v : \R^2 \to \R\), and the parametrization of \(C\) as \(z(t) = x(t) + \iu y(t)\), \(t \in [a,b]\).
Then
\[
\begin{aligned}
f(z(t))\,z'(t)
&=
\big(u(x(t),y(t)) + \iu v(x(t),y(t))\big) \big(x'(t) + \iu y'(t)\big) \\
&=
% \big(u(x(t),y(t)) x'(t) - v(x(t),y(t)) y'(t)\big) + \iu \big(v(x(t),y(t)) x'(t) + u(x(t),y(t)) y'(t)\big) \\
\big(u(x,y) \, x' - v(x,y) \, y'\big) + \iu \big(v(x,y) \, x' + u(x,y) \, y'\big) \\
&=
% \begin{bmatrix}u(x(t),y(t))\\-v(x(t),y(t))\end{bmatrix}
% \cdot
% \begin{bmatrix}x'(t)\\y'(t)\end{bmatrix}
% \;+\;
% \iu\,\begin{bmatrix}v(x(t),y(t))\\u(x(t),y(t))\end{bmatrix}
% \cdot
% \begin{bmatrix}x'(t)\\y'(t)\end{bmatrix}
\begin{bmatrix}u(x,y)\\-v(x,y)\end{bmatrix}
\cdot
\begin{bmatrix}x'\\y'\end{bmatrix}
\;+\;
\iu\,\begin{bmatrix}v(x,y)\\u(x,y)\end{bmatrix}
\cdot
\begin{bmatrix}x'\\y'\end{bmatrix}
\end{aligned}
\]
where we omitted the dependence of \(x\) and \(y\) on \(t\) in the last 2 lines for better readability.
Introducing \(\vect{\gamma}(t) := [x(t), y(t)]^\top\) and the vector fields \(\vect{V}_1 := [u(x,y), -v(x,y)]^\top\) and \(\vect{V}_2 := [v(x,y), u(x,y)]^\top\), we can rewrite \eqref{eq:curve-integral} as
\[
\begin{aligned}
\int_C f(z) \dif z
&=
\int_a^b \vect{V}_1(\vect{\gamma}(t))
\cdot \vect{\gamma}'(t) \, \dif t
\;+\;
\iu \int_a^b \vect{V}_2(\vect{\gamma}(t))
\cdot \vect{\gamma}'(t) \, \dif t \\
&=
\int_\Gamma \vect{V}_1 \cdot \dif \vect{s} \;+\; \iu \int_\Gamma \vect{V}_2 \cdot \dif \vect{s}
\end{aligned}
\]
where \(\Gamma\) is the curve in \(\R^2\) associated with \(C\) via the parametrization \(\vect{\gamma}(t)\) and \(\dif \vect{s} = \vect{\gamma}'(t) \dif t\) is the line element along \(\Gamma\).
As we can see, \(\re \int_C f(z) \dif z\) and \(\im \int_C f(z) \dif z\) are (real) line integrals of the vector fields
\(
\vect V_1%=[u,-v]^\top
\)
and
\(
\vect V_2%=[v,u]^\top
\),
respectively.
\end{remark}



The formula \(|\int_a^b f(t) \dif t| \leq \int_a^b |f(t)| \dif t\) for complex valued functions of a real variable \(t\) implies the following useful estimate for curve integrals:
\[
\begin{aligned}
\left| \int_C f(z) \dif z \right| &= \left| \int_a^b f( z(t)) z'(t) \dif t \right| \\
&\leq \int_a^b |f( z(t)) z'(t)| \dif t \\
&= \int_a^b |f( z(t))| |z'(t)| \dif t \\
&\leq \int_a^b \left(\max_{\tilde{t} \in [a,b]} |f(z(\tilde{t}))|\right) |z'(t)| \dif t \\
&= \max_{\tilde{t} \in [a,b]} |f(z(\tilde{t}))| \int_a^b |z'(t)| \dif t \\
&= \left\|f\right\|_{C} \cdot \operatorname{length}(C)
\end{aligned}
\]
where \(\|f\|_C\) denotes the maximum of the modulus of \(f\) on the curve \(C\) and \(\operatorname{length}(C)\) denotes the length of the curve \(C\).


\begin{theorem}[central integral of complex analysis]
For a positively oriented circle \(C\) centered at \(z_0\) with radius \(R\), we have
\begin{equation}\label{eq:central-integral}
{\color{red}\boxed{\color{black}
\oint_C (z - z_0)^n \dif z 
=
\begin{cases}
    0 & n \neq -1 \\
    2 \pi \iu & n = -1
\end{cases}
}}
\end{equation}
as can easily be seen by parametrizing \(C: z(t) = z_0 + R \eu^{\iu t}\), \(t \in [0, 2 \pi)\) and evaluating \eqref{eq:curve-integral}.
\end{theorem}



\begin{theorem}\label{thm:cauchy-integral-theorem}
For a simply connected open set \(D\) and a holomorphic function \(f\) thereupon, we have, provided all considered curves are completely contained in \(D\):
\begin{itemize}
  \item for any closed curve \(C\): \(\int_C f(z) \dif z = 0\)
  \item for any two curves \(C_1\) and \(C_2\) with the same start and end point: \(\int_{C_1} f(z) \dif z = \int_{C_2} f(z) \dif z\)
  \item \(f\) has an antiderivative \(F\) with \(\frac{\dif F}{\dif z} = f\) and \(\int_{z_{\text{start}}}^{z_{\text{end}}} f(z) \dif z = F(z_{\text{end}}) - F(z_{\text{start}})\)
  \qedhere
\end{itemize}
\end{theorem}
\begin{remark}\label{rem:cauchy-integral-theorem-ass}
The conditions under which \autoref{thm:cauchy-integral-theorem} holds can be slightly weakened: 
We choose an arbitrary, but fixed point \(z^* \in D\) and require continuity on entire \(D\), but holomorphy only on \(D\setminus \{z^*\}\).

These conditions are technical (it turns out that a function that fulfills them is also holomorphic in \(z^*\) anyway), but they simplify the proof of \autoref{thm:cauchy-integral-formula}.
\end{remark}

In the case of holomorphic functions, \autoref{thm:cauchy-integral-theorem} simplifies the evaluation of curve integrals significantly, because we only need to find an antiderivative and evaluate it at the endpoints of the curve.





\subsection{Cauchy integral formula}

Choose any \(z_0 \in \C\) and consider the integral
\[
\oint_C \frac{1}{z - z_0} \dif z
\]
where \(C\) is a positively oriented circle that does not contain \(z_0\).
We need to distinguish 2 cases. If \(z_0\) is outside of \(C\), then by \autoref{thm:cauchy-integral-theorem} the integral is 0.
If \(z_0\) is inside of \(C\), we can use \eqref{eq:central-integral} to see that the integral equals \(n 2 \pi \iu\) if \(C\) travels \(n\) times around \(z_0\) in positive orientation and \(-n 2 \pi \iu\) if \(C\) travels \(n\) times around \(z_0\) in negative orientation.
So the above integral counts (up to the factor \(2 \pi \iu\)) how often \(C\) winds around \(z_0\).

This motivates the following definition.
\begin{definition}[Winding number]\label{def:winding-number}
Consider a closed curve \(C\) and a point \(z_0 \notin \operatorname{im}(C)\).
We call
\begin{equation}\label{eq:index}
{\color{red}\boxed{\color{black}
\operatorname{Ind}_C(z_0) := \frac{1}{2 \pi \iu} \oint_C \frac{1}{z - z_0} \dif z
}}
\end{equation}
the \emph{Index} or \emph{winding number} of the curve \(C\) with respect to the point \(z_0\).
\end{definition}
\(\operatorname{Ind}_C(z_0)\) counts how often \(C\) winds around \(z_0\) in positive orientation.
Clearly we have
\[
\operatorname{Ind}_{-C}(z_0) = - \operatorname{Ind}_C(z_0)
\]
where \(-C\) denotes the curve \(C\) with reversed orientation.




\begin{theorem}[Cauchy integral formula]
\label{thm:cauchy-integral-formula}
Let \(D\) be a simply connected open set and \(f\) holomorphic on \(D\).
Then, for any closed curve \(C\) in \(D\), we have 
\begin{equation}\label{eq:cauchy-integral-formula}
% f(z_0) \, \operatorname{Ind}_C(z_0) = \frac{1}{2 \pi \iu} \oint_C \frac{f(z)}{z - z_0} \dif z
{\color{red}\boxed{\color{black}
\oint_C \frac{f(z)}{z - z_0} \dif z
=
2 \pi \iu \, f(z_0) \, \operatorname{Ind}_C(z_0) 
}}
\end{equation}
for all \(z_0 \in D \setminus \operatorname{im}(C)\), where \(\operatorname{im}(C)\) denotes the image of \(C\).
\end{theorem}
\begin{proof}
Given a function \(f\) that is holomorphic on \(D\) and a fixed point \(z_0 \in D\), we define a function \(g\) on \(D\) by
\[
g(z) := \begin{cases}
\frac{f(z) - f(z_0)}{z - z_0} & z \neq z_0 \\
f'(z_0) & z = z_0
\end{cases}
\]
Clearly, \(g\) is continuous on \(D\) and holomorphic on \(D \setminus \{z_0\}\).

Now consider a closed curve \(C\) that circulates \(z_0\). % in positive orientation.
By \autoref{rem:cauchy-integral-theorem-ass}, we can apply \autoref{thm:cauchy-integral-theorem} to \(g\) to obtain
\[
\begin{aligned}
0 
=
\oint_C g(z) \dif z
&= \oint_C \frac{f(z) - f(z_0)}{z - z_0} \dif z \\
&= \oint_C \frac{f(z)}{z - z_0} \dif z - f(z_0) \oint_C \frac{1}{z - z_0} \dif z \\
&= \oint_C \frac{f(z)}{z - z_0} \dif z - f(z_0) \, 2 \pi \iu \, \operatorname{Ind}_C(z_0)
\end{aligned}
\]
where we used \eqref{eq:index} in the last step.
This can be rearranged to obtain \eqref{eq:cauchy-integral-formula}.
\end{proof}

\autoref{thm:cauchy-integral-formula} has important consequences.
For example, it implies that the value of a holomorphic function on the inside of a simply connected set is completely determined by the values on the boundary.
If we choose as our curve for example a circular path centered at \(z_0\) that circles \(z_0\) with some radius \(R\) once in positive orientation, we obtain 
\[\begin{aligned}
f(z_0) &= \frac{1}{2 \pi \iu} \oint_C \frac{f(z)}{z - z_0} \dif z \\
&= \frac{1}{2 \pi \iu} \int_0^{2 \pi} \frac{f(z_0 + R \eu^{\iu t})}{(z_0 + R \eu^{\iu t}) - z_0} \cdot \iu R \eu^{\iu t} \dif t \\
&= \frac{1}{2 \pi} \int_0^{2 \pi} f(z_0 + R \eu^{\iu t}) \dif t
\end{aligned}\]
which is known as the \emph{mean value equation}.
The function value at any point is therefore the arithmetic mean of the function values on any circle centered at that point.

Another important consequence of \autoref{thm:cauchy-integral-formula} is that an inductive proof yields a formula for the \(n\)-th derivative of a holomorphic function \(f\),
\begin{equation}\label{eq:cauchy-integral-formula-derivative}
{\color{red}\boxed{\color{black}
f^{(n)}(z_0) = \frac{n!}{2 \pi \iu} \oint_{|z - z_0| = R} \frac{f(z)}{(z - z_0)^{n+1}} \dif z
}}
\end{equation}
for all \(n \in \N\) and \(R > 0\) such that the circle \(\{z \in \C : |z - z_0| = R\}\) is contained in \(D\).
Even though this formula is not very practical for computing derivatives, it implies that a holomorphic function is infinitely often complex differentiable, 
which is radically different from the real case.


% here I forgot some concepts. some that are missing:
% - Existenz beliebig hoher Ableitungen -> jede holomorphe Funktion kann in eine Potenzreihe entwickelt werden
% - Konvergenzradius mindestens und höchstens bis zum Rand des Holomorphiegebiets / bis zur nächsten Singularität
% - Holomorphe Funktionen = (lokal) in Potenzreihen entwickelbare Funktionen
% - Identitätssatz
% - Beweis Funktionalgleichung, inkl. bsp sin^2 + cos^2 = 1
% - reelle Funktionen e^x, sin x usw. nur auf eine Art holomorph nach ganz C fortsetzbar
% - Maximumprinzip
% - Satz von Liouville
% - Beweis des Fundamentalsatzes der Algebra
% - konforme Abbildungen, Laplacegleichung, harmonische Funktionen, etc.

% \subsubsection{Holomorphic functions are analytic}

More precisely, if \(f\) is holomorphic on a domain \(D\), then for every \(z_0 \in D\) and every \(R>0\) with \(\overline{D_R(z_0)} \subset D\), we have \eqref{eq:cauchy-integral-formula-derivative}.

This implies that \(f\) admits a Taylor expansion around \(z_0\) with positive radius of convergence
\begin{equation}\label{eq:taylor-series}
% {\color{red}\boxed{\color{black}
f(z) = \sum_{n=0}^{\infty} \frac{f^{(n)}(z_0)}{n!} (z-z_0)^n
% }}
\end{equation}
Moreover, the coefficients can be written as contour integrals
\begin{equation}\label{eq:taylor-coefficients-cauchy}
% {\color{red}\boxed{\color{black}
\frac{f^{(n)}(z_0)}{n!}
=
\frac{1}{2\pi\iu}\oint_{| z - z_0 |=R} \frac{f(z)}{(z-z_0)^{n+1}}\,\dif z
% }}
\end{equation}
which is just \eqref{eq:cauchy-integral-formula-derivative} divided by \(n!\).

The Taylor series \eqref{eq:taylor-series} converges at least up to the boundary of the holomorphy domain, and it cannot cross a singularity.
If \(f\) is holomorphic on a domain \(D\) and \(z_0\in D\), then the convergence radius satisfies
% \begin{equation}\label{eq:taylor-radius-lower-bound}
\(
R \ge \operatorname{dist}(z_0,\partial D)
\).
% \end{equation}
More generally, if \(f\) extends holomorphically to a larger domain except for isolated singularities, then the maximal radius equals the distance from \(z_0\) to the nearest singularity.

Conversely, \emph{local power series representability} characterizes holomorphy: a function is holomorphic on \(D\) iff every point \(z_0\in D\) has a neighborhood on which \(f\) equals a convergent power series.




\subsubsection{Identity theorem and uniqueness of analytic continuation}

Holomorphic functions are extremely rigid.
\begin{theorem}[identity theorem]\label{thm:identity-theorem}
Let \(D\) be a domain and let \(f,g\) be holomorphic on \(D\).
If \(f(z)=g(z)\) on a set \(A\subset D\) that has a limit point in \(D\), then \(f\equiv g\) on \(D\).
Equivalently, if for some \(z_0\in D\) we have \(f^{(n)}(z_0)=g^{(n)}(z_0)\) for all \(n\in\N\), then \(f\equiv g\) on \(D\).
\end{theorem}


\begin{example}%[a typical use of \autoref{thm:identity-theorem}]
Define \(h(z):=\sin^2 z + \cos^2 z - 1\).
Since \(\sin z\) and \(\cos z\) are entire, so is \(h\).
On the real axis \(h(x)=0\) for all \(x\in\R\), hence \(h\equiv 0\) on \(\C\) by \autoref{thm:identity-theorem}.
Therefore
\[
\sin^2 z + \cos^2 z = 1
\]
for all \(z\in\C\).
\end{example}

In particular, real-analytic functions such as \(e^x,\sin x,\cos x\) have at most one holomorphic extension to a given connected domain in \(\C\).
This justifies defining the elementary functions by their power series: the complex extensions are then uniquely determined.



\subsubsection{Maximum modulus principle and Liouville}

Another rigidity phenomenon is that holomorphic functions cannot have genuine interior maxima of their modulus.
\begin{theorem}
Let \(D\) be a bounded domain and let \(f\) be holomorphic on \(D\) and continuous on \(\overline{D}\).
Then
\[
\max_{z\in\overline{D}} |f(z)|
=
\max_{z\in\partial D} |f(z)|
\]
In particular, \(|f|\) has no strict local maximum in \(D\) unless \(f\) is constant.
\end{theorem}

A central corollary is
\begin{theorem}[Liouville]\label{thm:liouville}
Every bounded entire function is constant.
\end{theorem}

A standard application is the 
\begin{theorem}[fundamental theorem of algebra]\label{thm:fta}
Every non-constant complex polynomial has a zero in \(\C\).
\end{theorem}
\begin{proof}
Let \(P\) be a polynomial of degree \(\ge 1\) and assume it has no zero.
Then \(1/P\) is entire.
Moreover, \(|P(z)|\to\infty\) as \(|z|\to\infty\), hence \(1/P\) is bounded.
By \autoref{thm:liouville} the function \(1/P\) must be constant, so \(P\) is constant, a contradiction. \textcolor{red}{\Lightning}
\end{proof}





\subsection{Power and Laurent series}

% A power series with center \(z_0 \in \C\) is a series of the form
% \begin{equation}\label{eq:power-series}
% \sum_{n=0}^{\infty} a_n (z - z_0)^n
% \end{equation}
% where \(z \in \C\) and \(a_n \in \C\) are the coefficients of the series.
% After the root test, \eqref{eq:power-series} converges if
% % \begin{equation}\label{eq:power-series-convergence}
% \[
% \limsup_{n \to \infty} \sqrt[n]{|a_n (z - z_0)^n|} = |z - z_0| \limsup_{n \to \infty} |a_n|^{1/n} < 1
% \]
% % \end{equation}
%  and diverges if it is larger than 1.
% If it is equal to 1, the root test is inconclusive.
% Solving the above inequality for \(z - z_0\) yields the \emph{radius of convergence} \(R\) of the power series, which is given by
% \begin{equation}\label{eq:power-series-radius}
% R = \frac{1}{\limsup_{n \to \infty} |a_n|^{1/n}}
% \end{equation}
% and the power series converges for all \(z\) with \(|z - z_0| < R\) and diverges for all \(z\) with \(|z - z_0| > R\).
% On the boundary \(\{z \in \C : |z - z_0| = R\}\), there may be convergence or divergence.


Recall: A power series around \(z_0 \in \C\) is of the form
\begin{equation}\label{eq:power-series}
% \(
\sum_{n=0}^{\infty} a_n (z-z_0)^n
% \)
\end{equation}
and its radius of convergence is
\begin{equation}\label{eq:power-series-radius}
R = \frac{1}{\limsup_{n\to\infty} |a_n|^{1/n}}
\end{equation}
where the latter can be derived by applying the root test to the power series.
The series converges absolutely for \(|z-z_0|<R\) and diverges for \(|z-z_0|>R\), while the boundary \(|z-z_0|=R\) must be checked separately.

Power series describe holomorphic functions on disks (in particular, near points where the function is holomorphic). To treat functions that are holomorphic only on a punctured neighborhood (i.e. near isolated singularities) and to compute contour integrals around such points, we use Laurent series, which converge on annuli.

A Laurent series with center \(z_0 \in \C\) is a series of the form
\begin{equation}\label{eq:laurent-series}
\sum_{n=-\infty}^{\infty} a_n (z - z_0)^n
= \underbrace{\sum_{n=1}^{\infty} a_{-n} (z - z_0)^{-n}}_{\text{principal part}}
+ \underbrace{\sum_{n=0}^{\infty} a_n (z - z_0)^n}_{\text{regular part}} 
\end{equation}
It converges if both parts converge, so we apply the root test to both parts separately to obtain an inner and an outer radius of convergence:
\begin{itemize}
\item principal part:
\[
\limsup_{n \to \infty} \sqrt[n]{|a_{-n} (z - z_0)^{-n}|} < 1
\quad \Leftrightarrow \quad
|z - z_0| > \frac{1}{\limsup_{n \to \infty} |a_{-n}|^{1/n}} =: R_1 
\]
\item regular part:
\[
\limsup_{n \to \infty} \sqrt[n]{|a_n (z - z_0)^n|} < 1
\quad \Leftrightarrow \quad
|z - z_0| < \frac{1}{\limsup_{n \to \infty} |a_n|^{1/n}} =: R_2 
\]
\end{itemize}

The Laurent series thus converges on the annulus
\begin{equation}\label{eq:laurent-series-convergence}
D_{R_1, R_2} := \{z \in \C : R_1 < |z - z_0| < R_2\}
\end{equation}
and diverges outside of it.
The inner and outer boundary must be considered separately, as usual.
On \(D_{R_1, R_2}\), the Laurent series defines a holomorphic function.

Conversely, every function that is holomorphic on an annulus can be represented by a Laurent series.
If
\begin{equation}\label{eq:laurent-series-holomorphic-function}
f(z) = \sum_{n=-\infty}^{\infty} a_n (z - z_0)^n
\end{equation}
is the Laurent series representation of a holomorphic function \(f\) on an annulus \(D_{R_1, R_2}\), then the coefficients \(a_n\) can be computed by
\begin{equation}\label{eq:laurent-series-coefficients}
a_n = \frac{1}{2 \pi \iu} \oint_{C} \frac{f(z)}{(z - z_0)^{n+1}} \dif z 
\end{equation}
for any closed curve \(C\) in \(D_{R_1, R_2}\) that circulates \(z_0\) once in positive orientation.
This can be seen by plugging \eqref{eq:laurent-series-holomorphic-function} into the right-hand side of \eqref{eq:laurent-series-coefficients}:
\[
\begin{aligned}
\frac{1}{2 \pi \iu} \oint_{C} \frac{f(z)}{(z - z_0)^{n+1}} \dif z
&= \frac{1}{2 \pi \iu} \oint_{C} \frac{\sum_{m=-\infty}^{\infty} a_m (z - z_0)^m}{(z - z_0)^{n+1}} \dif z \\
&= \frac{1}{2 \pi \iu} \sum_{m=-\infty}^{\infty} a_m \oint_{C} (z - z_0)^{m - n - 1} \dif z \\
&= a_n
\end{aligned}
\]
where we used \eqref{eq:central-integral} in the last step.
For practical purposes, \eqref{eq:laurent-series-coefficients} is not very useful.
In most cases of interest, the coefficients can be inferred from a known power series or with the help of standard sum formulas (e.g.\ geometric series).



\subsubsection{Isolated singularities}

Let \(f\) be holomorphic on a punctured disk \(0<|z-z_0|<\varepsilon\).
Then \(f\) has a Laurent expansion
\[
f(z)=\sum_{n=-\infty}^{\infty} a_n (z-z_0)^n
\]
and we classify the isolated singularity \(z_0\) by the principal part.

\begin{definition}[classification of isolated singularities]\label{def:isolated-singularities}
Let \(f\) be holomorphic on \(0<|z-z_0|<\varepsilon\) with Laurent coefficients \(a_n\)
\begin{itemize}
\item \(z_0\) is \emph{removable} iff \(a_{-n}=0\) for all \(n\ge 1\)
\item \(z_0\) is a \emph{pole of order \(m\)} iff \(a_{-m}\neq 0\) and \(a_{-n}=0\) for all \(n>m\)
\item \(z_0\) is an \emph{essential singularity} iff \(a_{-n}\neq 0\) for infinitely many \(n\ge 1\)
\qedhere
\end{itemize}
\end{definition}

The following examples are representative and should be compared by their principal parts.
\begin{example}[removable singularity via cancellation]
If \(f\) is entire and \(f(0)=0\), then \(\frac{f(z)}{z}\) is holomorphic on \(\C\setminus\{0\}\) and has a removable singularity at \(0\).
In particular, \(\frac{\sin z}{z}\) has a removable singularity at \(0\) and extends holomorphically by defining the value \(1\) at \(0\).
\end{example}

\begin{example}[a pole]
The function \(\frac{1}{z-z_0}\) has a simple pole at \(z_0\).
More generally, \((z-z_0)^{-m}\) has a pole of order \(m\) at \(z_0\).
\end{example}

\begin{example}[an essential singularity]
The function \(e^{1/(z-z_0)}\) has an essential singularity at \(z_0\), since its Laurent series contains infinitely many negative powers.
\end{example}

A function that is holomorphic on a domain except for poles is called \emph{meromorphic} on that domain.
\begin{example}[rational and meromorphic functions]\label{def:meromorphic}
Rational functions provide the basic class of meromorphic functions.
A \emph{rational function} has the form
\[
R(z)=\frac{P(z)}{Q(z)}
\]
where \(P,Q\) are polynomials and \(Q\not\equiv 0\).
On the domain \(\C\setminus\{Q=0\}\), the function \(R\) is holomorphic, and its only isolated singularities are poles located at the zeros of \(Q\), with pole order equal to the multiplicity of the corresponding zero of \(Q\) after cancellation of common factors.
\end{example}



\subsubsection{Residues}

The coefficient \(a_{-1}\) of a Laurent expansion plays a special role and therefore receives its own name.
\begin{definition}[residue]\label{def:residue}
Let \(f\) be holomorphic on \(0<|z-z_0|<\varepsilon\) with Laurent expansion \(f(z)=\sum_{n=-\infty}^{\infty} a_n (z-z_0)^n\).
Then
\[
\operatorname{Res}(f;z_0):=a_{-1}
\qedhere
\]
\end{definition}

Residues are exactly the quantities that govern contour integrals around isolated singularities.
\begin{theorem}[residue theorem]\label{thm:residue-theorem}
Let \(G\) be a simply connected domain and let \(f\) be holomorphic on \(G\setminus\{z_1,\dots,z_N\}\), where \(z_1,\dots,z_N\in G\) are isolated singularities.
Let \(C\) be a closed piecewise \(C^1\) curve in \(G\) that avoids all \(z_j\).
Then
\begin{equation}\label{eq:residue-theorem}
{\color{red}\boxed{\color{black}
\oint_C f(z)\,\dif z
=
2\pi\iu \sum_{j=1}^{N} \operatorname{Ind}_C(z_j)\,\operatorname{Res}(f;z_j)
}}
\end{equation}
\end{theorem}
\begin{proof}
% Let $z_1,\dots,z_N\in G$ be the isolated singularities of $f$
Choose radii $\varepsilon_j>0$ such that the closed disks $\overline{D_{\varepsilon_j}(z_j)}$ are contained in $G$ and are pairwise disjoint and disjoint from $C$.
Denote by $C_j$ the positively oriented circle given by $|z-z_j|=\varepsilon_j$.

Since $f$ is holomorphic on $G\setminus\{z_1,\dots,z_N\}$ we may deform the curve $C$ inside this set.
Using \autoref{thm:cauchy-integral-theorem} this deformation yields the identity
\[
\oint_C f(z)\,\dif z
=
\sum_{j=1}^{N} \operatorname{Ind}_C(z_j)\,\oint_{C_j} f(z)\,\dif z
\]
which expresses the integral over $C$ as a sum of integrals over the small circles $C_j$.

Fix an index $j$.
On the punctured disk $0<|z-z_j|<\varepsilon_j$ the function $f$ admits a Laurent expansion of the form
\[
f(z)=\sum_{n=-\infty}^{\infty} a_{j,n}(z-z_j)^n
\]
and by definition of the residue we have $\operatorname{Res}(f;z_j)=a_{j,-1}$.

Integrating the Laurent series termwise over $C_j$ gives
\[
\oint_{C_j} f(z)\,\dif z
=
\sum_{n=-\infty}^{\infty} a_{j,n}\oint_{C_j}(z-z_j)^n\,\dif z
\]
and the central integral formula \eqref{eq:central-integral} implies that all terms vanish except the one corresponding to $n=-1$.

Applying \eqref{eq:central-integral} therefore yields
\[
\oint_{C_j} f(z)\,\dif z
=
2\pi\iu\,a_{j,-1}
\]
which equals $2\pi\iu\,\operatorname{Res}(f;z_j)$ by definition of the residue.

Substituting this expression into the deformation identity proves \eqref{eq:residue-theorem}.
\end{proof}     

For poles, residues can often be computed without explicitly writing the full Laurent series.
If \(z_0\) is a pole of order \(m\) of \(f\), then
\[
\operatorname{Res}(f;z_0)
=
\frac{1}{(m-1)!}\lim_{z\to z_0}\frac{\dif^{m-1}}{\dif z^{m-1}}\Bigl((z-z_0)^m f(z)\Bigr)
\]
where we used \autoref{def:isolated-singularities} to conclude that \autoref{eq:laurent-series-holomorphic-function} has a finite principal part.
In the simple pole case \(m=1\) this reduces to
\(
\operatorname{Res}(f;z_0)=\lim_{z\to z_0}(z-z_0)f(z)
\).




















\clearpage
\section{Fourier Series}

We write $f \in L^2([-\pi, \pi])$ if $f: \R \rightarrow \mathbb{C}$ is $2 \pi$-periodic and
\begin{equation}\label{eq:L2-norm}
{\color{red}
% \setlength{\fboxrule}{1pt}
\boxed{ 
\color{black}
\|f\|_2=\left(\frac{1}{2 \pi} \int_{-\pi}^\pi|f(x)|^2 \dif x\right)^{\frac{1}{2}}
}}
\end{equation}
is finite, i.e. \(\|f\|_2 < \infty\).

Functions in $L^2([-\pi, \pi])$ are called finite-energy signals, the norm $\|f\|_2$ is the energy of $f$. 
Observe that we normalized the integral in \eqref{eq:L2-norm} by means of the factor $\frac{1}{2 \pi}$. 
% In this way, we normalized the length of $[-\pi, \pi]$:
% $$
% \operatorname{length}([-\pi, \pi])=\frac{1}{2 \pi} \int_{-\pi}^\pi 1 \dif x
% =
% 1
% $$
Loosely speaking, we change the unit of measurement so that the length of $[-\pi, \pi]$ becomes $1$.

% see Figure 2.1.
% Differently from the other $L^p$ spaces, the space $L^2([-\pi, \pi])$ has the additional structure provided by the (sesquilinear) inner product:

% $$
% \langle f, g\rangle_{L^2}=\frac{1}{2 \pi} \int_{-\pi}^\pi f(x) \overline{g(x)} d x, \quad f, g \in L^2([-\pi, \pi]) .
% $$


% For the benefit of the reader, we recall the definition of inner product, that reflects the usual definition of inner product on $\R^d$, with the necessary precaution needed when working in the complex framework.




% Here, we used the component-wise continuity of $\langle\cdot, \cdot\rangle_{L^2}$ to intertwine series and inner products. 
For every $f \in L^2([-\pi, \pi])$,
\begin{equation}\label{eq:fourier-series}
  f(x) \stackrel{L^2}{=} \sum_{n=-\infty}^{\infty} \hat{f}(n) \eu^{\iu n x}
\end{equation}
where, for \(n \in \Z\),
\begin{equation}\label{eq:fourier-coefficient}
{\color{red}\boxed{\color{black}
\hat{f}(n)
=
\left\langle f, e_n\right\rangle_{L^2}
=
\frac{1}{2 \pi} \int_{-\pi}^\pi f(x) \eu^{-\iu n x} \dif x
}}
\end{equation}
is the $n$-th Fourier coefficient of $f$. 
\eqref{eq:fourier-series} is the Fourier series of $f$.


For every \(f \in L^2([-\pi,\pi])\), we have 
\begin{equation}\label{eq:parseval}
\|f\|_2^2=\sum_{n=-\infty}^{\infty}|\hat{f}(n)|^2
\end{equation}
which is known as \hl{Parseval's theorem}.




\clearpage

\section{Fourier Transform}



For $f \in L^1(\R)$ the Fourier transform
\begin{equation}\label{eq:fourier-transform}
{\color{red}\boxed{\color{black}
\hat{f}(\xi)=\int_{-\infty}^{\infty} f(x) \eu^{-2 \pi \iu \xi x} \dif x
}}
\end{equation}
converges for every $\xi \in \R$. 
Moreover,
\[
|\hat{f}(\xi)| \leq \int_{-\infty}^{\infty}|f(x)| \dif x=\|f\|_1
\]
for any \(\xi \in \R\),
\hl[2]{implying that $\hat{f}$ is in $L^{\infty}(\R)$} and the operator 
\[
\mathcal{F}: f \in L^1(\R) \mapsto \hat{f} \in L^{\infty}(\R)
\]
is bounded with
\(
\|\hat{f}\|_{\infty} \leq\|f\|_1 
\).

\begin{definition}
The operator $\mathcal{F}$ defined above is called Fourier transform, and the function $\hat{f}$ is the Fourier transform of $f$.
\end{definition}

Recall that a function $f: \R \rightarrow \mathbb{C}$ is uniformly continuous on $\R$ if for every $\varepsilon>0$ there exists $\delta>0$ depending only on $\varepsilon$ such that
$$
|x-y|<\delta \quad \Longrightarrow \quad|f(x)-f(y)|<\varepsilon 
$$


\begin{fact}
The Fourier transform is a bounded operator from $L^1(\R)$ to $L^{\infty}(\R)$. Moreover, $\hat{f}$ is a uniformly continuous function on $\R$ for every $f \in L^1(\R)$.
\end{fact}








% So for \( \xi \neq 0\),
% \[
% \hat{f}(\xi) = \frac{1}{2 \pi \xi} 2 \sin( \pi \xi A) = \frac{\sin( \pi \xi A)}{\pi \xi}
% \]
% In conclusion
% \[
% \hat{f}(\xi) =
% \begin{cases}
%     \frac{\sin( \pi \xi A)}{\pi \xi}, & \xi \neq 0, \\
%     A, & \xi = 0.
% \end{cases}
% \]

% Oberserve that \(\hat{f}\) is indeed in \(\mathcal{C}_b(\R)\) 

% \(\frac{\sin( \pi \xi A)}{\pi \xi}\) is a continuous at every \(\xi \neq 0\) and
% \[
% \lim_{\xi \to 0} \frac{\sin( \pi \xi A)}{\pi \xi} = A = \hat{f}(0)
% \]
% by definition.
\begin{example}[Characteristic function]
Let 
\begin{equation}\label{eq:characteristic-function}
f(x) = \chi_{[-A/2, A/2]}(x):=
\begin{cases}
1, & x \in [-A/2, A/2] \\
0, & \text{otherwise}
\end{cases}
\end{equation}
be the characteristic function of the interval \([-A/2, A/2]\), with \(A > 0\).
Then for every \(\xi \ne 0\),
\[
\hat{f}(\xi) 
=
\frac{\sin( \pi A \xi)}{\pi \xi}
\]
and \(\hat{f}(0) = A\).
\[
\begin{tikzpicture}[scale=1.2, font=\footnotesize]
  \tikzset{>=latex}
  \colorlet{myblue}{green!80!black}
  \colorlet{mydarkblue}{myblue!80!black}
  \tikzstyle{xline}=[myblue,thick]
  \def\tick#1#2{\draw[thick] (#1) ++ (#2:0.1) --++ (#2-180:0.2)}

  % -------- parameters (feel free to tweak axes) ----------
  \def\xmax{4.5}
  \def\ymin{-0.4}
  \def\ymax{1.4}
  \def\Aparam{1.0} % this is the "A" in sin(pi*A*xi)/(pi*xi)
  % --------------------------------------------------------

  \draw[->,thick] (0,\ymin) -- (0,\ymax);% node[left] {$\hat f(\xi)$};
  \draw[->,thick] (-\xmax,0) -- (\xmax+0.1,0) node[below=1,right=0.05] {$\xi$};

  % plot: sin(pi*A*xi)/(pi*xi), trig in pgf is degrees -> use deg(...)
  \draw[xline,samples=200,smooth,variable=\t,domain=-0.94*\xmax:0.94*\xmax]  plot(\t,{ (abs(\t)<0.01) ? (\Aparam) : (sin(deg(pi*\Aparam*\t))/(pi*\t)) });

  % zeros at xi = k/A
  \tick{-4/\Aparam,0}{90} node[below=0,scale=1] {$-\frac{4}{A}$};
  \tick{-3/\Aparam,0}{90} node[below=0,scale=1] {$-\frac{3}{A}$};
  \tick{-2/\Aparam,0}{90} node[below=0,scale=1] {$-\frac{2}{A}$};
  \tick{-1/\Aparam,0}{90} node[below=0,scale=1] {$-\frac{1}{A}$};
  \tick{ 1/\Aparam,0}{90} node[below=0,scale=1] {$ \frac{1}{A}$};
  \tick{ 2/\Aparam,0}{90} node[below=0,scale=1] {$ \frac{2}{A}$};
  \tick{ 3/\Aparam,0}{90} node[below=0,scale=1] {$ \frac{3}{A}$};
  \tick{ 4/\Aparam,0}{90} node[below=0,scale=1] {$ \frac{4}{A}$};

  % peak value
  \tick{0,\Aparam}{0} node[left=0.05,scale=0.9] {$A$};

  \node[mydarkblue,right,scale=0.95] at (0.15*\xmax,0.85*\Aparam) {$\displaystyle \hat{f}(\xi)=\frac{\sin(\pi A \xi)}{\pi \xi}$};
\end{tikzpicture}
\]
Thus we conclude that \(\hat{f}(\xi) = \frac{\sin( \pi A \xi)}{\pi \xi}\), where we implicitly consider its continuous continuation.
In particular, for \(A = 1\), we have
\[
\hat{f}(\xi) = \frac{\sin( \pi \xi)}{\pi \xi}
=:
\operatorname{sinc}(\xi)
\]
which is called \emph{cardinal sine} or \emph{sinc function}.
\end{example}


\begin{example}[Gaussian]
The Fourier transform of \(f(x) = \eu^{-\pi x^2}\) is \(\hat{f}(\xi) = \eu^{-\pi \xi^2}\).
\end{example}
\begin{example}
The Fourier transform of \(f(x) = \eu^{-2 |x|}\) is \(\hat{f}(\xi) = \frac{1}{1 + \pi^2 \xi^2}\).
\end{example}


% --- operators (as in the notes) ---
\subsection{Important operators in harmonic analysis}
\begin{definition}\label{def:harmonic-operators}
\begin{align}
% {\color{red}\boxed{\color{black}
&\qquad& &\text{\textcolor{red}{translation}:}& T_{x_0} f(x) &= f(x-x_0),                 & x_0 &\in \R \quad &\quad&\\
&\qquad& &\text{\textcolor{red}{modulation}:}& M_{\xi_0} f(x) &= \eu^{2\pi \iu \xi_0 x} f(x), & \xi_0 &\in \R \quad &\quad&\\
&\qquad& &\text{\textcolor{red}{upper dilation}:}& f^{\lambda}(x) &= f(\lambda x),     & \lambda &> 0 \quad &\quad&\\
&\qquad& &\text{\textcolor{red}{lower dilation}:}& f_{\lambda}(x) &= \frac{1}{\lambda} f(x/\lambda), & \lambda &> 0 \quad &\quad&%%
\end{align}
% }}
\end{definition}


% \begin{theorem}[Duality]\label{thm:duality-dilations}
\subsubsection{Duality}
Let \(f \in L^1(\R)\) and \(\lambda > 0\).

Translation/Modulation:
% {\color{red}\boxed{\color{black}
\begin{align}
\widehat{(T_{x_0} f)}(\xi)
&= M_{-x_0}\hat f(\xi)\\
% = \eu^{-2\pi \iu x_0 \xi}\hat f(\xi)\\[0.4em]
\widehat{(M_{\xi_0} f)}(\xi)
&= T_{\xi_0}\hat f(\xi)
% = \hat f(\xi-\xi_0)
\end{align}
% }}


Dilation:
% {\color{red}\boxed{\color{black}
\begin{align}
\widehat{(f^{\lambda})}(\xi)
&= (\hat f)_{\lambda}(\xi)\\
% = \frac{1}{\lambda}\hat f(\xi/\lambda)\\[0.4em]
\widehat{(f_{\lambda})}(\xi)
&= (\hat f)^{\lambda}(\xi)
% = \hat f(\lambda \xi)
\end{align}
% }}

% \end{itemize}
% \end{theorem}

Compressing a function in the time domain causes its Fourier transform to expand in the frequency domain and vice versa.



\subsection{Convolution}
\begin{equation}\label{eq:convolution}\
  {\color{red}\boxed{\color{black}
  (f * g)(x) := \int_{-\infty}^{\infty} f(y) g(x - y) \dif y
  }}
\end{equation}

If \(f, g \in L^1(\R)\), then \(f * g \in L^1(\R)\) and \(\|f * g\|_1 \leq \|f\|_1 \|g\|_1\).

The Fourier transform turns convolution into pointwise multiplication:
\begin{equation}\label{eq:convolution-theorem}
% {\color{red}\boxed{\color{black}
\widehat{(f * g)}(\xi) = \hat{f}(\xi) \cdot \hat{g}(\xi)
% }}
\end{equation}


\subsection{Riemann-Lebesgue lemma}
\begin{theorem}[Riemann-Lebesgue lemma]
If \(f \in L^1(\R)\), then \(\hat{f}(\xi) \in C_0(\R)\), i.e., \(\hat{f}\) is continuous and vanishes outside a certain interval.
\end{theorem}

\subsection{Inversion formula}
\begin{theorem}\label{thm:inversion}
Let \(f \in L^1(\R)\).
Assume that \(\hat{f} \in L^1(\R)\).
Then,
\begin{equation}\label{eq:inversion}
{\color{red}\boxed{\color{black}
f(x) = \int_{-\infty}^{\infty} \hat{f}(\xi) \eu^{2 \pi \iu \xi x} \dif \xi
}}
\end{equation}
for every \(x \in \R\), outside a set of measure zero.
If \(f \in C(\R)\), then \eqref{eq:inversion} holds for every \(x \in \R\).
\end{theorem}




\clearpage

\section{PDE}

Let \(f \in L^1(\R)\) and assume that \(f'\) exists and is in \(L^1(\R)\).
Then,
\begin{equation}\label{eq:fourier-derivative}
{\color{red}\boxed{\color{black}
\widehat{(f')}(\xi) = 2\pi \iu \xi \cdot \hat{f}(\xi)
}}
\end{equation}





\end{document}
